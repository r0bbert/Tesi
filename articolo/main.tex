\documentclass{article}
\usepackage{graphicx, amssymb,amsthm, amsmath, tikz-cd}

\usepackage[english]{babel}

\theoremstyle{definition}
\newtheorem{definition}{Definizione}

\theoremstyle{definition}
\newtheorem{example}{Esempio}

\newcommand{\id}[1]{id_{#1}}

\begin{document}

\section{Concetti di di base}

Questa sezione presenta i concetti di base di teoria delle cateogrie [...] 

\subsection{Categorie}

	\begin{definition}[Categorie]
		Una \emph{categoria $\mathcal{C}$} è composta da:
		\begin{enumerate}
			\item una collezione di \emph{oggetti} $Ob(\mathcal{C})$;
			\item una collezione di \emph{morfismi} $Mor(\mathcal{C})$;
			\item due operazioni che assegnano ad ogni morfismo $f$ un oggetto $dom\ f$, detto \emph{dominio} e un oggetto $cod\ f$, detto \emph{codominio}, denotando con $f: A \rightarrow B$ il fatto che $dom\ f = A \text{ e } cod\ f = B$; la collezione di tutti i morfismi da un oggetto $A$ ad un oggetto $B$ in $\mathcal{C}$ è denotato con $\mathcal{C}(A, B)$;
			\item un operatore di composizione che assegna ad ogni coppia di morfismi $f$, $g$ tali che $dom\ g = cod\ f$, un morfismo \emph{composto} $g \circ f: dom\ f \rightarrow cod\ g$, tale da soddisfare
				\[
					h \circ (g \circ f) = (h \circ g) \circ f
				\]
				per ogni $f$, $g$, $h$ tali che $dom\ g = cod\ f \text{ e } dom\ h = cod\ f$;
			\item per ogni oggetto $A$, un morfismo \emph{identità} $\id{A}: A \rightarrow A$ tale che:
				\[
					\id{B} \circ f = f = f \circ \id{A}
				\]
				per ogni $f: A \rightarrow B$.
		\end{enumerate}
	\end{definition}

	[puntualizzazione sull'utilizzo improprio del temine "collezione" nella definizione].
	
	Un primo esempio interessante è il seguente, nonché fonte di intuizioni per buona parte di questa presentazione della teoria delle categorie.

	\begin{example}
		La categoria $\textbf{Set}$ è la categoria degli insiemi e delle funzioni totali tra essi. La composizione di morfismi corrisponde alla composizione di funzioni, e i morfismi identità corrispondono alle funzioni identità.
	\end{example}



\end{document}
