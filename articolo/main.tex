\documentclass{article}
\usepackage{graphicx, amssymb,amsthm, amsmath, tikz-cd}

\usepackage[english]{babel}

\theoremstyle{plain}
\newtheorem{theorem}{Teorema}[section]
\newtheorem{obs}[theorem]{Osservazione}
\newtheorem{prop}[theorem]{Proposizione}

\theoremstyle{definition}
\newtheorem{definition}[theorem]{Definizione}
\newtheorem{example}[theorem]{Esempio}

\newcommand{\id}[1]{id_{#1}}

\tikzcdset{row sep/normal=5em}
\tikzcdset{column sep/normal=5em}

\begin{document}

\section{Concetti di di base}

Questa sezione presenta i concetti di base di teoria delle cateogrie [...] 

\subsection{Categorie}

	\begin{definition}[Categorie, sottocategorie]
		Una \emph{categoria $\mathcal{C}$} è composta da:
		\begin{enumerate}
			\item una collezione di \emph{oggetti} $Ob(\mathcal{C})$;
			\item una collezione di \emph{morfismi} $Mor(\mathcal{C})$;
			\item due operazioni che assegnano ad ogni morfismo $f$ un 
				oggetto $dom\ f$, detto \emph{dominio} e un oggetto $cod\ f$, detto \emph{codominio}, 
				denotando con $f: A \rightarrow B$ il fatto che $dom\ f = A \text{ e } cod\ f = B$; 
				la collezione di tutti i morfismi da un oggetto $A$ ad un oggetto $B$ in $\mathcal{C}$ è 
				denotato con $\mathcal{C}(A, B)$;
			\item un operatore di composizione che assegna ad ogni coppia di morfismi $f$, $g$ tali che 
				$dom\ g = cod\ f$, un morfismo \emph{composto} $g \circ f: dom\ f \rightarrow cod\ g$, 
				tale da soddisfare
				\[
					h \circ (g \circ f) = (h \circ g) \circ f
				\]
				per ogni $f$, $g$, $h$ tali che $dom\ g = cod\ f \text{ e } dom\ h = cod\ f$;
			\item per ogni oggetto $A$, un morfismo \emph{identità} $\id{A}: A \rightarrow A$ tale che:
				\[
					\id{B} \circ f = f = f \circ \id{A}
				\]
				per ogni $f: A \rightarrow B$.
		\end{enumerate}

		Una categoria $\mathcal{B}$ è una \emph{sottocategoria} di $\mathcal{C}$ se
		\begin{enumerate}
			\item ogni oggetto di $\mathcal{B}$ è un oggetto di $\mathcal{C}$;
			\item se $A$ e $B$ sono oggetti di $\mathcal{B}$, $\mathcal{B}(A, B) \subseteq \mathcal{C}(A, B)$;
			\item morfismi composti e identità di $\mathcal{B}$ sono gli stessi di $\mathcal{C}$.
		\end{enumerate}
		
		Una sottocategoria $\mathcal{B}$ di $\mathcal{C}$ si dice \emph{completa} se, per ogni coppia di oggetti $A$, $B$,si ha
		$\mathcal{B}(A, B) = \mathcal{C}(A, B)$.

	\end{definition}

	[puntualizzazione sull'utilizzo improprio del temine "collezione" nella definizione].
	
	Un primo esempio interessante è il seguente, nonché fonte di intuizioni per buona parte 
	di questa presentazione della teoria delle categorie.

	\begin{example}
		La categoria $\textbf{Set}$ è la categoria degli insiemi e delle funzioni totali tra essi. 
		La composizione di morfismi corrisponde alla composizione di funzioni, 
		e i morfismi identità corrispondono alle funzioni identità.
	\end{example}

	Un modo di rappresentare una categoria $\mathcal{C}$ è mediante \emph{diagrammi}, come illustrato di seguito,
	i cui vertici sono etichettati con oggetti di $\mathcal{C}$, e le frecce con morfismi di $f$, dove una 
	freccia $f$ da $A$ a $B$ rappresenta il morfismo $f: A \rightarrow B$. 
	Un diagramma è detto \emph{commutativo} se, per ogni coppia di vertici $A$, $C$, 
	ogni percorso da $A$ a $C$ determina lo stesso morfismo:
	\[
		\begin{tikzcd}
			A \arrow[r, "f"] \arrow[rd, "h" swap] & B  \arrow[d, "g"] \\
			& C 
		\end{tikzcd}
	\]
	in questo caso, la commutatività di tale diagramma esprime l'uguaglianza $g \circ f = h$.

	\begin{definition}[Monomorfismi, epimorfismi, isomorfismi]
		Un morfismo $m:B \rightarrow C$ in una categoria $\mathcal{C}$ è un \emph{monomorfismo}
		(oppure è \emph{mono}) se, per ogni coppia di morfismi $f, g: A \rightarrow B$ in $\mathcal{C}$, 
		l'uguaglianaza $m \circ f = m \circ g$ implica $f = g$. \-
		Un morfismo $e: C \rightarrow D$ in $\mathcal{C}$ è un \emph{epimorfismo} (oppure è \emph{epi}) se
		l'uguaglianza $f \circ e = g \circ e$ implica $f = g$. \-
		Un morfismo $\phi: A \rightarrow B$ in $\mathcal{C}$ è un \emph{isomorfismo} se esiste in $\mathcal{C}$
		un morfismo $\psi: B \rightarrow A$ tale che $\psi \circ \phi = \id{A}$ e $\phi \circ \psi = \id B$.

	\end{definition}

	Le definizioni di monomorfismo, epimorfismo e isomorfismo sono una generalizzazione di un concetto ben noto
	in teoria degli insiemi, come illustra il seguente esempio.

	\begin{example}
		In $\textbf{Set}$, un monomorfismo $m: A \rightarrow B$ è una 
		funzione iniettiva dall'inieme $A$ all'insieme $B$,
		un epimorfismo $e: C \rightarrow D$ è una funzione surgettiva 
		da $C$ a $D$, mentre un isomorfismo $\phi: E \rightarrow F$
		è una corripondenza biunivoca tra gli insiemi $E$ ed $F$
	\end{example}

	[Non andrei troppo nel dettaglio con gli esempi]

	


\end{document}
