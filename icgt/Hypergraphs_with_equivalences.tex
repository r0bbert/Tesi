\section{Hypergraphs with equivalences}

\begin{definition}
	A \emph{hypergraph with equivalence} $\mathcal{G} = (E_\mathcal{G}, V_{\mathcal{G}}, C_\mathcal{G}, s_\mathcal{G}, t_\mathcal{G}, q_\mathcal{G})$ is a 6-tuple such that $\mathcal{G} = (E_\mathcal{G}, V_{\mathcal{G}}, s_\mathcal{G}, t_\mathcal{G})$ is a hypergraph, $C_\mathcal{G}$ is the object of \emph{equivalence classes} and $q_{\mathcal{G}}: V_{\mathcal{G}}\to C_{\mathcal{G}}$ is an epimorphism called \emph{quotient map}. A morphism $h:\mathcal{G\to H}$ is a triple $(h_E, h_V, h_C)$ such that the following diagrams commute.
	\[\xymatrix{
		{E_\mathcal{G}}\ar[r]^{s_\mathcal{G}}\ar[d]_{h_E} & {V_{\mathcal{G}}^\star}\ar[d]^{h_V^\star} & {E_\mathcal{G}}\ar[r]^{t_\mathcal{G}}\ar[d]_{h_E} & {V_{\mathcal{G}^\star}}\ar[d]^{h_V^\star} \\
		{E_\mathcal{H}}\ar[r]^{s_\mathcal{H}} & {V_{\mathcal{H}}^\star}	 			& {E_\mathcal{H}}\ar[r]^{t_\mathcal{H}} & {V_{\mathcal{H}^\star}} \\
						      & {V_\mathcal{G}}\ar[r]^{q_\mathcal{G}}\ar[d]_{h_V} & {C_{\mathcal{G}}} \ar[d]^{h_C} \\
						      & {V_\mathcal{H}}\ar[r]^{q_\mathcal{H}} & {C_\mathcal{H}}
	}\]
	The category of hypergraphs with equivalences and their morphisms is denoted $\EqHyp$.

\end{definition}

\begin{remark}\label{rem:eqhyp_morphs}
	Morphisms of hypergraphs with equivalences are uniquely determined by the first two components. That is, if $h_1 = (h_E, h_V, f)$ and $h_2 = (h_E, h_V, g)$ are two morphisms $\mathcal{G \to H}$, then we have
	\[\xymatrix{
			{V_\mathcal{G}} \ar[r]^{h_V}\ar[d]_{q_\mathcal{G}} & V_\mathcal{H} \ar[d]^{q_\mathcal{H}} & V_\mathcal{G}\ar[l]_{h_V}\ar[d]^{q_\mathcal{G}}\\
			C_{\mathcal{G}}\ar[r]_{f} & C_{\mathcal{H}} & C_{\mathcal{G}}\ar[l]^{g}
	}\]
	Hence,
	\begin{align*}
		f \circ q_\mathcal{G} &= q_\mathcal{H}\circ h_V \\ &=g\circ q_\mathcal{G}
	\end{align*}
	Since $q_\mathcal{G}$ is epi, we obtain $f = g$.
\end{remark}

$\EqHyp$ has a forgetful functor $U_{\EqHyp}:\EqHyp \to \Set$, which sends each $\mathcal{G} = (E_\mathcal{G}, V_{\mathcal{G}}, C_\mathcal{G}, s_\mathcal{G}, t_\mathcal{G}, q_\mathcal{G})$ into $V_\mathcal{G}$, and each $h = (h_E, h_V, h_C)$ onto $h_V$. 

\begin{proposition}
	$U_\EqHyp$ has a left adjoint $\Delta_{\EqHyp}: \Set \to \EqHyp$.
\end{proposition}

\begin{proof}
	For each set $X$, define $\Delta_\EqHyp(X):= (\emptyset, X, \{\bullet\}, ?_X, ?_X, !_X)$. Consider now $h: \Delta_{\EqHyp}(X) \to \mathcal{H}$.
	\[\xymatrix@C=2.3cm{
			\Delta_{\EqHyp}(X) \ar@{.>}[d]_{\Delta_\EqHyp(f)} \ar[dr]^{h} & \\
			\Delta_\EqHyp(U_{\EqHyp}(\mathcal{H})) \ar[r]_{\epsilon_{\mathcal{H}}} & \mathcal{H}
	}\]
	Where $\Delta_\EqHyp(U_{\EqHyp}(\mathcal{H})) = (\emptyset, V_\mathcal{H}, \{\bullet\}, ?_{V_\mathcal{H}}, ?_{V_\mathcal{H}}, !_{V_\mathcal{H}})$ and $\epsilon_{\mathcal{H}} = (?_{E_\mathcal{H}}, id_{V_\mathcal{H}}, g)$.
	Note that, since $\Delta_{\EqHyp}(X)$ has the empty set as object of edges, $h_E = ?_{E_\mathcal{H}}$, then, the unique arrow that fits in the diagram is $\Delta_{\EqHyp}(f) = (?_{E_\mathcal{H}}, h_V, id_{\{\bullet\}})$.

\end{proof}



