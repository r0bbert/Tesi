\section{Hypergraphs and term graphs with equivalences}


\begin{definition}
	A \emph{hypergraph with equivalence} $\mathcal{G} = (E_\mathcal{G}, V_{\mathcal{G}}, C_\mathcal{G}, s_\mathcal{G}, t_\mathcal{G}, q_\mathcal{G})$ is a 6-tuple such that $\mathcal{G} = (E_\mathcal{G}, V_{\mathcal{G}}, s_\mathcal{G}, t_\mathcal{G})$ is a hypergraph, $C_\mathcal{G}$ is an object and $q_{\mathcal{G}}: V_{\mathcal{G}}\to C_{\mathcal{G}}$ is a regular epimorphism called \emph{quotient map}. 
	
	A morphism $h:\mathcal{G\to H}$ is a triple $(h_E, h_V, h_C)$ such that the following diagrams commute.
	\[\xymatrix{
		{E_\mathcal{G}}\ar[r]^{s_\mathcal{G}}\ar[d]_{h_E} & {V_{\mathcal{G}}^\star}\ar[d]^{h_V^\star} & {E_\mathcal{G}}\ar[r]^{t_\mathcal{G}}\ar[d]_{h_E} & {V_{\mathcal{G}^\star}}\ar[d]^{h_V^\star} & {V_\mathcal{G}}\ar[r]^{q_\mathcal{G}}\ar[d]_{h_V} & {C_{\mathcal{G}}} \ar[d]^{h_C} \\
		{E_\mathcal{H}}\ar[r]_{s_\mathcal{H}} & {V_{\mathcal{H}}^\star}	& {E_\mathcal{H}}\ar[r]_{t_\mathcal{H}} & {V_{\mathcal{H}^\star}}& {V_\mathcal{H}}\ar[r]_{q_\mathcal{H}} & {C_\mathcal{H}}
	}\]
	The category of hypergraphs with equivalences and their morphisms is denoted $\EqHyp$.

\end{definition}

\begin{remark}\label{rem:eqhyp_morphs}
	Morphisms of hypergraphs with equivalences are uniquely determined by the first two components. That is, if $h_1 = (h_E, h_V, f)$ and $h_2 = (h_E, h_V, g)$ are two morphisms $\mathcal{G \to H}$, then we have
	\[\xymatrix{
			{V_\mathcal{G}} \ar[r]^{h_V}\ar[d]_{q_\mathcal{G}} & V_\mathcal{H} \ar[d]^{q_\mathcal{H}} & V_\mathcal{G}\ar[l]_{h_V}\ar[d]^{q_\mathcal{G}}\\
			C_{\mathcal{G}}\ar[r]_{f} & C_{\mathcal{H}} & C_{\mathcal{G}}\ar[l]^{g}
	}\]
	Hence,
	\begin{align*}
		f \circ q_\mathcal{G} &= q_\mathcal{H}\circ h_V \\ &=g\circ q_\mathcal{G}
	\end{align*}
	Since $q_\mathcal{G}$ is epi, we obtain $f = g$.
\end{remark}

$\EqHyp$ has a forgetful functor $U_{\EqHyp}:\EqHyp \to \Set$, which sends each $\mathcal{G} = (E_\mathcal{G}, V_{\mathcal{G}}, C_\mathcal{G}, s_\mathcal{G}, t_\mathcal{G}, q_\mathcal{G})$ into $V_\mathcal{G}$, and each $h = (h_E, h_V, h_C)$ onto $h_V$. 

\begin{proposition}
	$U_\EqHyp$ has a left adjoint $\Delta_{\EqHyp}: \Set \to \EqHyp$.
\end{proposition}

\begin{proof}
	For each set $X$, define $\Delta_\EqHyp(X):= (\emptyset, X, \{\bullet\}, ?_X, ?_X, !_X)$. Consider now $h: \Delta_{\EqHyp}(X) \to \mathcal{H}$.
	\[\xymatrix@C=2.3cm{
			\Delta_{\EqHyp}(X) \ar@{.>}[d]_{\Delta_\EqHyp(f)} \ar[dr]^{h} & \\
			\Delta_\EqHyp(U_{\EqHyp}(\mathcal{H})) \ar[r]_{\epsilon_{\mathcal{H}}} & \mathcal{H}
	}\]
	Where $\Delta_\EqHyp(U_{\EqHyp}(\mathcal{H})) = (\emptyset, V_\mathcal{H}, \{\bullet\}, ?_{V_\mathcal{H}}, ?_{V_\mathcal{H}}, !_{V_\mathcal{H}})$ and $\epsilon_{\mathcal{H}} = (?_{E_\mathcal{H}}, \id{V_\mathcal{H}}, g)$.
	Note that, since $\Delta_{\EqHyp}(X)$ has the empty set as object of edges, $h_E = ?_{E_\mathcal{H}}$, then, the unique arrow that fits in the diagram is $\Delta_{\EqHyp}(f) = (?_{E_\mathcal{H}}, h_V, \id{\{\bullet\}})$.

\end{proof}

We now define another functor $T: \EqHyp \to \hyp$, which ``forgets'' the quotient part, mapping each hypergraph with equivalence $\mathcal{G} = (E_\mathcal{G}, V_{\mathcal{G}}, C_\mathcal{G}, s_\mathcal{G}, t_\mathcal{G}, q_\mathcal{G})$ onto $T(\mathcal{G})=(E_{\mathcal{G}}, V_{\mathcal{G}}, s_\mathcal{G}, t_{\mathcal{G}})$. Then, we have the following result.

\begin{proposition}
	$T$ has a left adjoint $L: \hyp \to \EqHyp$.
\end{proposition}

\begin{proof}
	Let $\mathcal{G}$ be a hypergraph, and define $L(\mathcal{G}) := (E_\mathcal{G}, V_{\mathcal{G}}, \{\bullet\}, s_\mathcal{G}, t_\mathcal{G}, !_{V_\mathcal{G}})$. Let now $h: L(\mathcal{G})\to \mathcal{H}$ be a morphism in $\EqHyp$, and consider the following situation.
	\[\xymatrix@C=2.3cm{
		L(\mathcal{G}) \ar@{.>}[d]_{L(f)} \ar[dr]^{h}\\ L(T(\mathcal{H})) \ar[r]_{\epsilon_{\mathcal{H}}} & \mathcal{H}
	}\]
	Where $L(T(\mathcal{H}))=(E_\mathcal{H}, V_{\mathcal{H}}, \{\bullet\}, s_\mathcal{H}, t_\mathcal{H}, !_{V_\mathcal{H}})$. Then, $\epsilon_\mathcal{H} = (\id{E_\mathcal{H}}, \id{V_\mathcal{H}}, h_C)$ (recall that by \Cref{rem:eqhyp_morphs}, the last component is uniquely determined by the first two), and $L(f)$ must be $(h_E, h_V, \id{\{\bullet\}})$.
\end{proof}

\begin{remark}
	$T$ is faithful. Indeed, consider two morphisms $h = (h_E, h_V, h_C)$ and $k = (k_E, k_V, k_C)$, and suppose $T(h) = T(k)$, that is, $(h_E, h_V) = (k_E, k_V)$.
	By \Cref{rem:eqhyp_morphs}, we can conclude also $h_C = k_C$, and hence the faithfulness of $T$.
\end{remark}

Let now $K: \EqHyp \to \Set$ be the functor which sends each hypergraph with equivalence $\mathcal{G} = (E, V, C, s, t, q)$ onto $K(\mathcal{G}) = C$, and each morphsism $(h_E, h_V, h_C)$ to $h_C$.

\begin{proposition}
	\EqHyp is complete and cocomplete, and $T$ preserves limits and colimits.
\end{proposition}

\begin{proof}
	Let $D: \cat{I}\to \EqHyp$ be a diagram, and, for each $i \in \cat{I}$, $D(i) = (E_i, V_i, C_i, s_i, t_i, q_i)$.
	Suppose now $(E, V, s, t)$, together with morphisms $(\pi_i^E, \pi_i^V)$, be the limit of $T \circ D$.
	Then, $V$, together with $(q_i\circ \pi_V^i)_{i\in \cat I}$, is a cone for $K \circ D$. Indeed, let $\alpha: i \to j$ be an arrow of $\cat I$, $D(\alpha) = (h_E,h_V, h_C)$.
	By definition of $T$, $(T \circ D)(\alpha) = (h_E, h_V)$, hence we have the following situation.
	\[\xymatrix{
		& V \ar[dl]_{\pi_V^i} \ar[dr]^{\pi_V^j} & \\V_i\ar[rr]^{h_V}\ar[d]_{q_i}&&V_j\ar[d]^{q_j}\\C_i\ar[rr]_{h_C}&&C_j
	}\]
	Suppose now that $L$, with morphisms $(l_i)_{i\in \cat I}$ be the limit of $K\circ D$. Hence, we have an arrow $l:V \to L$, which is not epi in general.
	Let then $l = m \circ q$ be the epi-mono factorization of it. Consider the following situation, where the outer rectangle commutes by definition, and the dotted arrow is yielded by ({\color{red}cite left lifting prop}).
	\[\xymatrix{
			V\ar[r]^{\pi_V^i}\ar[d]_{q}&V_i\ar[r]^{q_i}&C_i\ar[d]^{\id{C_i}}\\C\ar[r]_{m}\ar@{.>}[urr]^{\pi_C^i}&L\ar[r]_{l_i}&{C_i}
	}\]
	Thus, $(E, V, C, s, t, q)$, together with $(\pi_E^i, \pi_V^i, \pi_C^i)$ is a cone over $D$. {\color{red} remain to show that this cone is terminal}
	
	Suppose now $(E', V', s', t')$, together with $(\kappa_E^i, \kappa_V^i)_{i \in \cat I}$, be the colimit of $T \circ D$, and $C'$, with $(c_i)_{i \in \cat I}$ be the colimit of $K \circ D$.
	Then, we have the folliwing situation.
	\[\xymatrix{
								  &  {V'} &                             \\
		V_i \ar[ur]^{\kappa_i} \ar[rr]^{h_V} \ar[d]_{q_i} && V_j \ar[ul]_{\kappa_j} \ar[d]^{q_j}\\
		C_i \ar[dr]_{c_i}      \ar[rr]_{h_C}              &&C_j  \ar[dl]^{c_j}                  \\
								  &  {C'} &
	}\]
	Then, $C'$ with morphisms $(c_i \circ q_i)_{i\in \cat I}$ is a conone for $U \circ D$. Then, there exists a unique morphism $q': V' \to C'$ such that $q'\circ \kappa_V^i = c_i\circ q_i$.
	Such morphism is epi ({\color{red}cite Lemma 1.3.45 of the thesis}), and thus $(E', V', C', s', t', q')$, together with $(\kappa_E^i, \kappa_V^i, c_i)_{i\in \cat I}$ is the colimit of $D$.
	
\end{proof}

\begin{corollary}\label{cor:mono1}
	An arrow $h = (h_E, h_V, h_C): \mathcal{G \to H}$ is mono if and only if $T(h)$ is mono.
\end{corollary}

\begin{proof}
	The ``if'' part is given by the faithfulness of $T$.
	The ``only if'' part is given by \Cref{rem:eqhyp_morphs}.
\end{proof}

\begin{corollary}\label{cor:mono2}
	If $h = (h_E, h_V, h_C): \mathcal{G\to H}$ is a regular mono in $\EqHyp$, then $h_E, h_V$ and $h_C$ are all monos.
\end{corollary}

\begin{proof}
	If $h$ is mono, from \Cref{cor:mono1} we have that $h_E$ and $h_V$ are monos. Suppose now $f, g: \mathcal{H\rightrightarrows K}$ be the arrows equalized by $h$. Then, we have:
	\begin{align*} f_C \circ h_C \circ q_\mathcal{G} &= f_C\circ q_\mathcal{H} \circ h_V\\&=q_\mathcal{K}\circ g_V \circ h_V\\&=q_\mathcal{K}\circ f_V \circ h_V\\&=g_C \circ h_C \circ q_\mathcal{G} \end{align*}
	Since $q_\mathcal{G}$ is epi, we have $f_C \circ h_C = g_C \circ h_C$, hence $h_C$ is an equalizer for $f_C$ and $g_C$, and thus a monomorphism.
\end{proof}

\begin{proposition}
	Let $h = (h_E, h_V, h_C): \mathcal{G \to H}$ be a regular monomorphism in $\EqHyp$.
	Then, $h_E$ and $h_V$ are monos and $(K, \pi_1, \pi_2)$ is the kernel pair of $q_\mathcal{H}\circ h_V$ if and only if $(K, \pi_1, \pi_2)$ is the kernel pair of $q_\mathcal{G}$.
\end{proposition}

\begin{proof}
	By \Cref{cor:mono2}, we have that $h_E, h_V$ and $h_C$ are all monos.
	Hence, by \Cref{cor:kermono}, $(K, \pi_1, \pi_2)$ is the kernel pair of $q_\mathcal{G}$ if and only if it is the kernel pair also of $h_C \circ q_\mathcal{G}$, since $h_C$ is mono by hypothesis.
	The thesis follows from $h_C \circ q_\mathcal{G} = q_H \circ h_V$, and from the hypothesis of $h_E$ mono.
\end{proof}


\begin{remark}
    It is possible to restate the last proposition, by \Cref{ex:kernel_pairs_in_Set}, as 
    \begin{displayquote}
    \textit{$h_E$ and $h_V$ are mono and, for every $v, v'\in V_H$, $q_H(h_V(v))=q_H(h_V(v'))$ if and only if $q_G(v)=q_G(v')$}
    \end{displayquote}
    That is, a regular monomorphism in $\EqHyp$ is a morphism that reflects equivalences besides preserving them.
\end{remark}
