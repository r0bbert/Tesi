\documentclass[runningheads,envcountsect]{llncs}

\usepackage{stmaryrd}
\usepackage{bbding}
%\Envelope
\usepackage{amsmath,amsfonts,amssymb}
\newcommand{\cat}[1]{\ensuremath{\mathbf{#1}}}
\usepackage[all]{xy}

\usepackage[utf8]{inputenc} % Input encoding - per caratteri particolari
\usepackage[english]{babel} % Lingua principale inglese
\usepackage{graphicx} % Per includere immagini esterne
\usepackage[tickmarkheight=.5em,textwidth=\marginparwidth,textsize=small]{todonotes}
\usepackage{mathtools}
\usepackage{csquotes}


%\usepackage{amssymb,amsthm, amsmath, mathrsfs, dsfont} %simboli matematici
\usepackage{hyperref, cleveref}





\DeclareMathAlphabet{\mymathbb}{U}{BOONDOX-ds}{m}{n}

\newcommand{\Ob}{\mathcal{O}b}
\newcommand{\Hom}{\mathcal{H}om}
\newcommand{\Set}{\mathbf{Set}}
\newcommand{\Reg}{\mathcal{Reg}}
\newcommand{\Mono}{\mathcal{Mono}}
\newcommand{\initial}{\mymathbb{0}}
\newcommand{\terminal}{\mathds{1}}
\newcommand{\eg}[1]{\mathbf{EqGraph}_{\textbf {\textup{#1}}}}
\newcommand{\egg}[1]{\mathbf{EGG}_{\textbf {\textup{#1}}}}

\makeatletter
\def\@citecolor{blue}%
\def\@urlcolor{blue}%
\def\@linkcolor{blue}%
\def\UrlFont{\rmfamily}
\def\orcidID#1{\smash{\href{http://orcid.org/#1}{\protect\raisebox{-1.25pt}{\protect\includegraphics{orcid_color.eps}}}}}
\makeatother



\def\R{\mathsf{R}}
\def\B{\textbf {\textup{B}}}
\def\C{\textbf {\textup{C}}}
\def\D{\textbf {\textup{D}}}
\def\X{\textbf {\textup{X}}}
\def\Y{\textbf {\textup{Y}}}
\def\E{\textbf {\textup{E}}}


\bibliographystyle{abbrv}



\title{On the adhesivity properties of equivalence graphs\thanks{???????}}

\titlerunning{????} %TODO optional, please use if title is longer than one line

%% Author with single affiliation.
\author{Roberto Biondo \inst{1}{\small\Envelope}\and Davide Castelnovo\inst{1}\orcidID{0000-0002-5926-5615}
	\and \\Fabio Gadducci\inst{1}\orcidID{0000-0003-0690-3051}
}
\institute{Dept.~of Computer Science, University of Pisa, Italy. 
	\email{r.biondo@studenti.unipi.it},
	\email{castelnovod@gmail.com},
	 \email{fabio.gadducci@unipi.it}
}



%frecce
\newcommand{\mor}{\mathsf{Mor}}
\newcommand{\mon}{\mathsf{Mono}}
\newcommand{\reg}{\mathsf{Reg}}
\newcommand{\mto}{\rightarrowtail}
\newcommand{\eto}{\twoheadrightarrow}



\authorrunning{R.~Biondo, D.~Castelnovo, F.~Gadducci}


\begin{document}

	\maketitle \todo{Scegliere un titolo vero e mettere i ringraziamenti}
	\begin{abstract}
\todo{a very nice abstract}
	\end{abstract}


\section{Introduction}
\todo{A very nice introduction}
\section{$\mathcal{M}$-adhesive categories}

This  section recalls the basic theory of \emph{$\mathcal{M}$-adhesive categories} \cite{azzi2019essence,ehrig2012,ehrig2014adhesive,lack2005adhesive,heindel2009category}. Given a category $\X$ we will not distinguish notationally between $\X$ and its class of objects, so
``$X\in \X$'' means that $X$ is an object of $\X$. We let $\mor(\X)$, $\mon(\X)$ and $\reg(\X)$ denote the class of all arrows, monos and regular monos of $\X$, respectively.  Given an integer $n\in \mathbb{Z}$, $[0,n]$ denotes the set of natural numbers less than or equal to $n$; in particular, $[0,n]=\emptyset$ if $n<0$.

\subsection{$\mathcal{M}$-adhesivity}\label{subsec:ade}
The key property of $\mathcal{M}$-adhesive categories is the \emph{Van Kampen condition}~\cite{brown1997van,johnstone2007quasitoposes,lack2005adhesive}. In order to define it we need to introduce some terminology.  Let  $\X$ be a category. A subclass $\mathcal{A}$ of
$\mor(\X)$ is called
\begin{itemize}
	\parbox{11cm}{\item
		\emph{stable under pushouts (pullbacks)} if for every pushout (pullback) square as the one on the right, 	if $m \in \mathcal{A}$ ($n\in \mathcal{A}$) then $n \in \mathcal{A}$ ($m \in \mathcal{A}$);
		\item \emph{closed under composition} if $h, k\in \mathcal{A}$ implies $h\circ k\in \mathcal{A}$ whenever $h$ and $k$ are composable.{\tiny }}\hfill
	\parbox{2cm}{$\xymatrix{A\ar[r]^f  \ar[d]_{m}& B \ar[d]^n \\ C \ar[r]_g & D}$}
	\parbox{11cm}{}\hfill
\end{itemize}

\begin{definition}[Van Kampen property]
	Let $\X$ be a category and consider the diagram\\
	\parbox{10cm}{
		aside.
		Given  a class of arrows $\mathcal{A}\subseteq \mor(\X)$, we say that the bottom square
		is an \emph{$\mathcal{A}$-Van Kampen square} if
		\begin{enumerate}
			\item it is a pushout square;
			\item 	whenever the cube above has pullbacks as back and left faces and the vertical arrows belong to $\mathcal{A}$, then its top face is a pushout 
			if and only if the front and right faces are pullbacks.
	\end{enumerate}}
	\parbox{2cm}{$\xymatrix@C=10pt@R=6pt{&A'\ar[dd]|\hole_(.65){a}\ar[rr]^{f'} \ar[dl]_{m'} && B' \ar[dd]^{b} \ar[dl]_{n'} \\ C'  \ar[dd]_{c}\ar[rr]^(.7){g'} & & D' \ar[dd]_(.3){d}\\&A\ar[rr]|\hole^(.65){f} \ar[dl]^{m} && B \ar[dl]^{n} \\C \ar[rr]_{g} & & D }$ }\\
	Pushout squares that enjoy only the ``if'' half of item (2) above are called \emph{$\mathcal{A}$-stable}.
	
	A $\mor(\X)$-Van Kampen square is called  \emph{Van
		Kampen} and a $\mor(\X)$-stable square  \emph{stable}.
\end{definition}

We can now define $\mathcal{M}$-adhesive categories.

\begin{definition}[$\mathcal{M}$-adhesive category]
	Let $\X$ be a category and $\mathcal{M}$ a subclass of
	$\mon(\X)$  including  all isomorphisms, closed under composition,  and stable under pullbacks and pushouts.  The category  $\X$ is said to be \emph{$\mathcal{M}$-adhesive} if
	\begin{enumerate}
		\item it has \emph{$\mathcal{M}$-pullbacks}, i.e.~pullbacks along arrows of $\mathcal{M}$;
		\item it has \emph{$\mathcal{M}$-pushouts}, i.e.~pushouts along arrows of $\mathcal{M}$;
		\item  $\mathcal{M}$-pushouts are $\mathcal{M}$-Van Kampen squares.
	\end{enumerate}
	
	A category $\X$ is said to be \emph{strictly $\mathcal{M}$-adhesive}
	if $\mathcal{M}$-pushouts are Van Kampen squares.
\end{definition}

We write $m\colon X\rightarrowtail Y$ to denote that an arrow $m\colon X\to Y$ belongs to $\mathcal{M}$.

\begin{remark}
	\label{rem:salva}
	\emph{Adhesivity} and \emph{quasiadhesivity} 
	\cite{lack2005adhesive,garner2012axioms} coincide with strict
	$\mon(\X) $-adhesivity and strict $\reg(\X)$-adhesivity,
	respectively.
\end{remark}


$\mathcal{M}$-adhesivity is well-behaved with respect to  the construction of slice and functor categories \cite{mac2013categories}, as shown by the following theorems~\cite{ehrig2006fundamentals,lack2005adhesive}.

\begin{theorem}\todo{controllare meglio}
	\label{thm:slice-functors}
	Let $\X$ be an $\mathcal{M}$-adhesive category. Given an object $X$
	the category $\X/X$ is $\mathcal{M}/X$-adhesive with
	$\mathcal{M}/X:=\{m\in \mor(\X/X) \mid m\in
	\mathcal{M}\}$. Similarly, $X/\X$ is $X/\mathcal{M}$-adhesive with
	$X/\mathcal{M}:=\{m\in \mor(X/\X) \mid m\in \mathcal{M}\}$.
	
	Moreover for every small category $\Y$, the category $\X^\Y$ of
	functors $\Y\to \X$ is $\mathcal{M}^{\Y}$-adhesive, where
	$\mathcal{M}^{\Y}:=\{\eta \in \mor(\X^\Y) \mid \eta_Y \in
	\mathcal{M} \text{ for every } Y\in \Y\}$.
\end{theorem}

We can list various examples of $\mathcal{M}$-adhesive categories (see
\cite{castelnovo2023thesis,CastelnovoGM22,lack2006toposes}).


\begin{example}
	\label{ex:adhesive}
	$\cat{Set}$ is adhesive, and, more generally, every topos is
	adhesive~\cite{lack2006toposes}. By the closure properties above, every presheaf $[\cat{X},\cat{Set}]$ is adhesive, thus the category
	$\cat{Graph} = [ E \rightrightarrows V, \cat{Set}]$ is adhesive
	where $E \rightrightarrows {V}$ is the two objects category with two
	morphisms $s,t \colon{E} \to {V}$. Similarly, various
	categories of hypergraphs can be shown to be adhesive, such as term
	graphs and hierarchical graphs~\cite{CastelnovoGM24}. Note that the category $\cat{sGraphs}$ of simple graphs, 
	i.e.~graphs without parallel edges, is
	$\reg{(\cat{sGraphs})}$-adhesive~\cite{BehrHK23} but not
	quasiadhesive.
\end{example}


\subsection{Kernel pairs}
\todo{Inserire parte della tesi su kp}
\subsection{Factorization systems}
\todo{Inserire qualche cosa su sist. di fattorizazzione e M-adesività - devo pensare a come mettere la questione senza prendere troppo spazio}


\section{Graphs with equivalences}

\subsection{Graph signatures}
\todo{Inserire def di graph signature}

\todo{Grafi su una segnatura sono adesivi (grafi)}

\todo{Inserire equivalenze (stesso discorso della tesi)}


\subsection{E-graphs}

\todo{Copincollare dalla tesi senza grandi modifiche}

\section{Conclusions and further works}
\todo{Some very nice conclusions}
\bibliography{biblio}

\appendix
\section{Comparison with Ghica}
\todo{Lo vogliamo fare?}


\end{document}

