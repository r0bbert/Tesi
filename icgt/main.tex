\documentclass[runningheads,envcountsect]{llncs}

\usepackage{stmaryrd}
\usepackage{bbding}
%\Envelope
\newcommand{\cat}[1]{\ensuremath{\mathbf{#1}}}
\usepackage{amsmath,amssymb,mathrsfs, dsfont} %simboli matematici
\usepackage[all, cmtip]{xy}

\usepackage[utf8]{inputenc} % Input encoding - per caratteri particolari
\usepackage[english]{babel} % Lingua principale inglese
\usepackage{graphicx} % Per includere immagini esterne
\usepackage[tickmarkheight=.5em,textwidth=\marginparwidth,textsize=small]{todonotes}
\usepackage{mathtools}
\usepackage{csquotes}


\usepackage{hyperref, cleveref}
\usepackage{tikz}
\usetikzlibrary{decorations.markings}

% General Setting diagrams
%\usepackage{tikz-cd} %diagrammi
%\tikzcdset{row sep/normal=5em}
%\tikzcdset{column sep/normal=5em}
%\tikzcdset{every label/.append style = {font = \small}}

\spnewtheorem*{notation}{Notation}{\bfseries}{}

%funtori
\usepackage[usestackEOL]{stackengine}
\newcommand\functorop[1][l]{\csname#1functor\endcsname}
\newcommand\lfunctorop[3]{%
	\setbox0=\hbox{$#2$}%
	\kern\wd0%
	\ensurestackMath{\Centerstack[c]{#1\\ \mathllap{#2\;\,}\mathclap{\DownArrow}\\#3}}%
}		
\newcommand\rfunctorop[3]{%
	\setbox0=\hbox{$#2$}%
	\ensurestackMath{\Centerstack[c]{#1\\\mathclap{\UpArrow}\mathrlap{\,\;#2}\\#3}}%
	\kern\wd0%
}
\newcommand\functoropmapsto{\mathrel{\ensurestackMath{\Centerstack[c]{\longmapsto\\ \\\longmapsto}}}}
\setstackgap{L}{1.3\normalbaselineskip}
\newcommand\UpArrow{\rotatebox[origin=c]{90}{$\longrightarrow$\,}}
\newcommand\DownArrow{\rotatebox[origin=c]{-90}{$\longrightarrow$\,}}
\newcommand\functor[1][l]{\csname#1functor\endcsname}
\newcommand\lfunctor[3]{%
	\setbox0=\hbox{$#2$}%
	\kern\wd0%
	\ensurestackMath{\Centerstack[c]{#1\\ \mathllap{#2\;\,}\mathclap{\DownArrow}\\#3}}%
}
\newcommand\rfunctor[3]{%
	\setbox0=\hbox{$#2$}%
	\ensurestackMath{\Centerstack[c]{#1\\\mathclap{\DownArrow}\mathrlap{\,\;#2}\\#3}}%
	\kern\wd0%
}
\newcommand\functormapsto{\mathrel{\ensurestackMath{\Centerstack[c]{\longmapsto\\ \\\longmapsto}}}}
\setstackgap{L}{1.3\normalbaselineskip}



\DeclareMathAlphabet{\mymathbb}{U}{BOONDOX-ds}{m}{n}

\newcommand{\Ob}{\mathcal{O}b}
\newcommand{\Hom}{\mathcal{H}om}
\newcommand{\Set}{\mathbf{Set}}
\newcommand{\Reg}{\mathcal{Reg}}
\newcommand{\Mono}{\mathcal{Mono}}
\newcommand{\initial}{\mymathbb{0}}
\newcommand{\terminal}{\mathds{1}}
\newcommand{\eg}[1]{\mathbf{EqGraph}_{\textbf {\textup{#1}}}}
\newcommand{\egg}[1]{\mathbf{EGG}_{\textbf {\textup{#1}}}}

\makeatletter
\def\@citecolor{blue}%
\def\@urlcolor{blue}%
\def\@linkcolor{blue}%
\def\UrlFont{\rmfamily}
\def\orcidID#1{\smash{\href{http://orcid.org/#1}{\protect\raisebox{-1.25pt}{\protect\includegraphics{orcid_color.eps}}}}}
\makeatother



\def\R{\mathsf{R}}
\def\B{\textbf {\textup{B}}}
\def\C{\textbf {\textup{C}}}
\def\D{\textbf {\textup{D}}}
\def\X{\textbf {\textup{X}}}
\def\Y{\textbf {\textup{Y}}}
\def\E{\textbf {\textup{E}}}

%categorie varie
\newcommand{\catname}[1]{\textbf{\textup{#1}}}
\newcommand{\lab}{\catname{LHyp}}
\newcommand{\hyp}{\catname{Hyp}}
\newcommand{\hyps}{\catname{Hyp}_{\Sigma}}
\newcommand{\gr}{\textbf{\textup{Graph}}}
\newcommand{\dgr}{\catname{SGraph}}
\newcommand{\dg}{\catname{DAG}}
\newcommand{\rt}{\mathsf{dcl_s}}
\newcommand{\rta}{\mathsf{dcl}}
\newcommand{\rtd}{\mathsf{dcl_{d}}}
\newcommand{\slice}[2]{(\catname{#1}\downarrow{#2})}
\newcommand{\tg}[0]{\catname{TG}_{\Sigma}}
\newcommand{\teg}[0]{\catname{TeGr}_{\Sigma}}
\newcommand{\sv}[0]{\mathsf{Sieves}}
\newcommand{\mono}[1]{\mathcal{M}({#1})}
\newcommand{\pro}{\mathsf{prod}}
\newcommand{\spro}{\mathsf{ps}}
\newcommand{\prol}{\mathsf{lprod}}
\newcommand{\pred}[1]{{\downarrow}#1}
\newcommand{\colim}[0]{\mathrm{colim}}
\newcommand{\cod}{\mathsf{cod}}
\renewcommand{\sp}{\mathsf{sp}}
\renewcommand{\sup}{\mathsf{sup}}
\newcommand{\cow}[1]{\mathsf{cwd}({#1})}
\renewcommand{\inf}{\mathsf{inf}}
\newcommand{\dom}{\mathsf{dom}}
\newcommand{\dwnarrow}{\downarrow \hspace{-2pt}}
\newcommand{\Dwnarrow}{\Downarrow \hspace{-2pt}}

\newcommand{\ari}{\mathsf{ar}}
\newcommand{\abs}[1]{\lvert #1\rvert}


\newcommand{\upstr}[1] { {#1}^{\uparrow }}
\newcommand{\comma}[2]{#1\hspace{1pt} {\downarrow}#2}
\newcommand{\cma}[2]{\mathcal{#1}\hspace{1pt} {\downarrow}\hspace{1pt} \mathcal{#2}}



\bibliographystyle{abbrv}

%frecce
\newcommand{\mor}{\mathsf{Mor}}
\newcommand{\mon}{\mathsf{Mono}}
\newcommand{\reg}{\mathsf{Reg}}
\newcommand{\mto}{\rightarrowtail}
\newcommand{\eto}{\twoheadrightarrow}
\newcommand{\id}[1]{\mathsf{id}_{#1}}



\title{On the adhesivity properties of equivalence graphs\thanks{???????}}

\titlerunning{????} %TODO optional, please use if title is longer than one line

%% Author with single affiliation.
\author{Roberto Biondo \inst{1}{\small\Envelope}\and Davide Castelnovo\inst{1}\orcidID{0000-0002-5926-5615}
	\and \\Fabio Gadducci\inst{1}\orcidID{0000-0003-0690-3051}
}
\institute{Dept.~of Computer Science, University of Pisa, Italy. 
	\email{r.biondo@studenti.unipi.it},
	\email{castelnovod@gmail.com},
	 \email{fabio.gadducci@unipi.it}
}






\authorrunning{R.~Biondo, D.~Castelnovo, F.~Gadducci}


\begin{document}


\tikzset{->-/.style={decoration={
			markings,
			mark=at position #1 with {\arrow{>}}},postaction={decorate}}}
	\maketitle \todo{Scegliere un titolo vero e mettere i ringraziamenti}
	\begin{abstract}
\todo{a very nice abstract}
	\end{abstract}


\section{Introduction}
\todo{A very nice introduction}
\section{$\mathcal{M}$-adhesive categories}

\todo{Va riscritta tutta con xy}

This  section recalls the basic theory of \emph{$\mathcal{M}$-adhesive categories} \cite{azzi2019essence,ehrig2012,ehrig2014adhesive,lack2005adhesive,heindel2009category}. 

\begin{notation}
	contenuto...
\end{notation}
Given a category $\X$ we will not distinguish notationally between $\X$ and its class of objects, so
``$X\in \X$'' means that $X$ is an object of $\X$. We let $\mor(\X)$, $\mon(\X)$ and $\reg(\X)$ denote the class of all arrows, monos and regular monos of $\X$, respectively.  Given an integer $n\in \mathbb{Z}$, $[0,n]$ denotes the set of natural numbers less than or equal to $n$; in particular, $[0,n]=\emptyset$ if $n<0$.

\subsection{$\mathcal{M}$-adhesivity}\label{subsec:ade}
The key property of $\mathcal{M}$-adhesive categories is the \emph{Van Kampen condition}~\cite{brown1997van,johnstone2007quasitoposes,lack2005adhesive}. In order to define it we need to introduce some terminology.  Let  $\X$ be a category. A subclass $\mathcal{A}$ of
$\mor(\X)$ is called
\begin{itemize}
	\parbox{11cm}{\item
		\emph{stable under pushouts (pullbacks)} if for every pushout (pullback) square as the one on the right, 	if $m \in \mathcal{A}$ ($n\in \mathcal{A}$) then $n \in \mathcal{A}$ ($m \in \mathcal{A}$);
		\item \emph{closed under composition} if $h, k\in \mathcal{A}$ implies $h\circ k\in \mathcal{A}$ whenever $h$ and $k$ are composable.{\tiny }}\hfill
	\parbox{2cm}{$\xymatrix{A\ar[r]^f  \ar[d]_{m}& B \ar[d]^n \\ C \ar[r]_g & D}$}
	\parbox{11cm}{}\hfill
\end{itemize}

\begin{definition}[Van Kampen property]
	Let $\X$ be a category and consider the diagram\\
	\parbox{10cm}{
		aside.
		Given  a class of arrows $\mathcal{A}\subseteq \mor(\X)$, we say that the bottom square
		is an \emph{$\mathcal{A}$-Van Kampen square} if
		\begin{enumerate}
			\item it is a pushout square;
			\item 	whenever the cube above has pullbacks as back and left faces and the vertical arrows belong to $\mathcal{A}$, then its top face is a pushout 
			if and only if the front and right faces are pullbacks.
	\end{enumerate}}
	\parbox{2cm}{$\xymatrix@C=10pt@R=6pt{&A'\ar[dd]|\hole_(.65){a}\ar[rr]^{f'} \ar[dl]_{m'} && B' \ar[dd]^{b} \ar[dl]_{n'} \\ C'  \ar[dd]_{c}\ar[rr]^(.7){g'} & & D' \ar[dd]_(.3){d}\\&A\ar[rr]|\hole^(.65){f} \ar[dl]^{m} && B \ar[dl]^{n} \\C \ar[rr]_{g} & & D }$ }\\
	Pushout squares that enjoy only the ``if'' half of item (2) above are called \emph{$\mathcal{A}$-stable}.
	
	A $\mor(\X)$-Van Kampen square is called  \emph{Van
		Kampen} and a $\mor(\X)$-stable square  \emph{stable}.
\end{definition}

We can now define $\mathcal{M}$-adhesive categories.

\begin{definition}[$\mathcal{M}$-adhesive category]
	Let $\X$ be a category and $\mathcal{M}$ a subclass of
	$\mon(\X)$  including  all isomorphisms, closed under composition,  and stable under pullbacks and pushouts.  The category  $\X$ is said to be \emph{$\mathcal{M}$-adhesive} if
	\begin{enumerate}
		\item it has \emph{$\mathcal{M}$-pullbacks}, i.e.~pullbacks along arrows of $\mathcal{M}$;
		\item it has \emph{$\mathcal{M}$-pushouts}, i.e.~pushouts along arrows of $\mathcal{M}$;
		\item  $\mathcal{M}$-pushouts are $\mathcal{M}$-Van Kampen squares.
	\end{enumerate}
	
	A category $\X$ is said to be \emph{strictly $\mathcal{M}$-adhesive}
	if $\mathcal{M}$-pushouts are Van Kampen squares.
\end{definition}

We write $m\colon X\rightarrowtail Y$ to denote that an arrow $m\colon X\to Y$ belongs to $\mathcal{M}$.

\begin{remark}
	\label{rem:salva}
	\emph{Adhesivity} and \emph{quasiadhesivity} 
	\cite{lack2005adhesive,garner2012axioms} coincide with strict
	$\mon(\X) $-adhesivity and strict $\reg(\X)$-adhesivity,
	respectively.
\end{remark}


$\mathcal{M}$-adhesivity is well-behaved with respect to  the construction of slice and functor categories \cite{mac2013categories}, as shown by the following theorems~\cite{ehrig2006fundamentals,lack2005adhesive}.

\begin{theorem}\todo{controllare meglio}
	\label{thm:slice-functors}
	Let $\X$ be an $\mathcal{M}$-adhesive category. Given an object $X$
	the category $\X/X$ is $\mathcal{M}/X$-adhesive with
	$\mathcal{M}/X:=\{m\in \mor(\X/X) \mid m\in
	\mathcal{M}\}$. Similarly, $X/\X$ is $X/\mathcal{M}$-adhesive with
	$X/\mathcal{M}:=\{m\in \mor(X/\X) \mid m\in \mathcal{M}\}$.
	
	Moreover for every small category $\Y$, the category $\X^\Y$ of
	functors $\Y\to \X$ is $\mathcal{M}^{\Y}$-adhesive, where
	$\mathcal{M}^{\Y}:=\{\eta \in \mor(\X^\Y) \mid \eta_Y \in
	\mathcal{M} \text{ for every } Y\in \Y\}$.
\end{theorem}

We can list various examples of $\mathcal{M}$-adhesive categories (see
\cite{castelnovo2023thesis,CastelnovoGM22,lack2006toposes}).


\begin{example}
	\label{ex:adhesive}
	$\cat{Set}$ is adhesive, and, more generally, every topos is
	adhesive~\cite{lack2006toposes}. By the closure properties above, every presheaf $[\cat{X},\cat{Set}]$ is adhesive, thus the category
	$\cat{Graph} = [ E \rightrightarrows V, \cat{Set}]$ is adhesive
	where $E \rightrightarrows {V}$ is the two objects category with two
	morphisms $s,t \colon{E} \to {V}$. Similarly, various
	categories of hypergraphs can be shown to be adhesive, such as term
	graphs and hierarchical graphs~\cite{CastelnovoGM24}. Note that the category $\cat{sGraphs}$ of simple graphs, 
	i.e.~graphs without parallel edges, is
	$\reg{(\cat{sGraphs})}$-adhesive~\cite{BehrHK23} but not
	quasiadhesive.
\end{example}


\todo{Inserire parte della tesi su kp}
\iffalse 
\subsection{Kernel Pairs and Regular Epimorphisms}

\begin{definition}[Kernel Pair]
    A \emph{kernel pair} for an arrow $f: A \to B$ is an object $K_f$ together with two arrows $\pi^1_f, \pi^2_f : K_f \to A$, denoted as $(K_f, \pi^1_f, \pi^2_f)$, such that the following square is a pullback.
    \[
        \begin{tikzcd}
            K_f \ar[r, "{\pi^1_f}"] \ar[d, "{\pi^2_f}" swap] & A \arrow[d, "f"] \\
            A \ar[r, "f"swap] & B
        \end{tikzcd}
    \]
\end{definition}

\begin{remark}
	If a category $\cat{C}$ has pullbacks then every arrow has a kernel pair.
\end{remark}

\begin{remark}
    Since a kernel pair is nothing more that a pullback, that is, a limit, by \Cref{rem:limits_are_unique_up_to_isomorphisms}, it make sense to refer to it as \emph{the} kernel pair for a morphism $f$.
\end{remark}

\begin{example}\label{ex:kernel_pairs_in_Set}
    In $\Set$, a kernel pair for a function $f: A\to B$ is the set
    \[
        K_f=\{(x, y) \in A \times A \mid f(x) = f(y)\}
    \]
    together with the canonical projection on the first and the second component of the pairs.
\end{example}

\begin{proposition}\label{prop:pairng_of_kernel_pairs_mono}
    Let $(K, p_1, p_2)$ be the kernel pair of $f: X \to Y$, and let $(X\times X, \pi_1, \pi_2)$ be the product of $X$ with itself. Then, the mediating arrow $\langle p_1, p_2\rangle : K \to X \times X$ is mono.
\end{proposition}

\begin{proof}
	Suppose $\langle p_1, p_2 \rangle \circ f = \langle p_1, p_2 \rangle \circ g$ for $f, g: Z \to K$. Then, we have
	\[
		\begin{split}
			\langle p_1, p_2 \rangle \circ f &=  \langle p_1, p_2 \rangle \circ g \\
			\pi_1 \circ \langle p_1, p_2 \rangle \circ f &=  \pi_1 \circ \langle p_1, p_2 \rangle \circ g \\
			p_1 \circ f &= p_1 \circ g
		\end{split}
		\qquad
		\begin{split}
			\langle p_1, p_2 \rangle \circ f &=  \langle p_1, p_2 \rangle \circ g \\
			\pi_2 \circ \langle p_1, p_2 \rangle \circ f &=  \pi_2 \circ \langle p_1, p_2 \rangle \circ g \\
			p_2 \circ f &= p_2 \circ g
		\end{split}
	\]
	Thus, from the universal property of the pullback, $f = g$.
\end{proof}

\begin{proposition}\label{prop:kermono}
	An arrow $m\colon M\to X$ is mono if and only if $(M, id_M, id_M)$ is a kernel pair for it.
\end{proposition}

\begin{proof}
    To prove the ``if'' part of the statement, let $f, g: A \to M$ be such that $m\circ f = m\circ g$, and consider the following situation.
    \[
        \begin{tikzcd}[row sep=26, column sep = 26]
        A \arrow[drr, bend left=30, "f"] \arrow[ddr, bend right=30,"g" swap] \arrow[dr, dashed, "u"] & & \\
        & M  \arrow[d, "{id_M}"] \arrow[r, "{id_M}"] & M \arrow[d, "m"] \\
        & M  \arrow[r, "m" swap] & X
        \end{tikzcd}
    \]
    For the universal property of pullbacks, we have that $$f  =  id_M \circ u =  g$$
    Hence, $m$ is mono.

    Conversely, if $m$ is mono, then, we have that
    \begin{align*}
        m \circ f = m \circ g   &\Rightarrow    f = g \\
                                &\Rightarrow    f \circ id_M = g\circ id_M
    \end{align*}
    Hence, $f$ is the unique arrow that makes the commutative square illustrated above a pushout.
\end{proof}

\begin{remark}\label{rem:monos_in_presh_cats}
    From characterization of monos via pullbacks in \Cref{prop:kermono} and \Cref{lemma:limits_of_presheaves}, we have that a mono in a category of presheaves is a natural transformation of which each component is mono.
\end{remark}

\begin{corollary}\label{cor:kermono}
	$(K_f, \pi_f^1, \pi_f^2)$ is a kernel pair for $f\colon X\to Y$ if and only if, for each mono $m\colon Y\to Z$, $(K_f, \pi_f^1, \pi_f^2)$ is a kernel pair also for $m\circ f$.
\end{corollary}
\begin{proof}
    It is enough to see that, by \Cref{lemma:pullback_lemma} and \Cref{prop:kermono} the outer boundary of the following square is a pullback.
        \[\begin{tikzcd}[row sep=13 pt, column sep=13 pt]
    	{K_f} && X && X \\
    	\\
    	X && Y && Y \\
    	\\
    	X && Y && Z
    	\arrow["{\pi_f^2}", from=1-1, to=1-3]
    	\arrow["{\pi_f^2}"', from=1-1, to=3-1]
    	\arrow["{id_X}", from=1-3, to=1-5]
    	\arrow["f", from=1-3, to=3-3]
    	\arrow["f", from=1-5, to=3-5]
    	\arrow["f", from=3-1, to=3-3]
    	\arrow["{id_X}"', from=3-1, to=5-1]
    	\arrow["{id_Y}", from=3-3, to=3-5]
    	\arrow["{id_Y}", from=3-3, to=5-3]
    	\arrow["m", from=3-5, to=5-5]
    	\arrow["f"', from=5-1, to=5-3]
    	\arrow["m"', from=5-3, to=5-5]
    \end{tikzcd}\]
    The leftward part of the statement follows by definition of monomorphism an \Cref{lemma:pullback_lemma}.
\end{proof}

\begin{lemma}\label{lemma:kern_pairs_pres_pullbacks}
    Suppose the following situation, and that $f: X \to Y$ and $g: Z \to W$ have kernel pairs.
    \[
        \begin{tikzcd}
            X \ar[r, "h"] \ar[d, "f"swap] & Z \ar[d, "g"] \\
            Y \ar[r, "t"swap] & W
        \end{tikzcd}
    \]
    
    Then, there exists a unique arrow $k_h: K_f \to K_g$ making the squares below commute.
    \[
        \begin{tikzcd}[row sep = 25 pt, column sep= 25 pt]
            K_f \ar[r, dashed, "{k_h}"] \ar[d, "{\pi_f^1}"swap] & K_g \ar[d, "{\pi_g^1}"] \\
            X \ar[r, "h"swap] & Z 
        \end{tikzcd}
        \qquad
        \begin{tikzcd}[row sep = 25 pt, column sep= 25 pt]
            K_f \ar[r, dashed, "{k_h}"] \ar[d, "{\pi_f^2}"swap] & K_g \ar[d, "{\pi_g^2}"] \\
            X \ar[r, "h"swap] & Z 
        \end{tikzcd}
    \]

    Moreover, if the beginning square is a pullback, then also the preceding ones are so.
\end{lemma}

\begin{proof}
    Computing, we have
    \begin{align*}
        g \circ h \circ \pi_f^1 &=  t \circ f \circ \pi_f^1     \\
                                &=  t \circ f \circ \pi_f^2     \\
                                &=  g \circ h \circ \pi_f^2
    \end{align*}
    By the universal property of $K_g$ as the pullback of $g$ along itself, such $k_h$ exists and it is unique.

    To prove the second half of the thesis, let us consider the two rectangles below, which, by \Cref{lemma:pullback_lemma} are pullbacks.
    \[\begin{tikzcd}[row sep= 25 pt, column sep= 25 pt]
	{K_f} & X & Z \\
	X & Y & W
	\arrow["{\pi_f^1}", from=1-1, to=1-2]
	\arrow["{\pi_f^2}"', from=1-1, to=2-1]
	\arrow["h", from=1-2, to=1-3]
	\arrow["f", from=1-2, to=2-2]
	\arrow["g", from=1-3, to=2-3]
	\arrow["f"', from=2-1, to=2-2]
	\arrow["t"', from=2-2, to=2-3]
    \end{tikzcd}
    \qquad
    \begin{tikzcd}[row sep = 25 pt, column sep=25 pt]
	{K_f} & X & Z \\
	X & Y & W
	\arrow["{\pi_f^2}", from=1-1, to=1-2]
	\arrow["{\pi_f^1}"', from=1-1, to=2-1]
	\arrow["h", from=1-2, to=1-3]
	\arrow["f", from=1-2, to=2-2]
	\arrow["g", from=1-3, to=2-3]
	\arrow["f"', from=2-1, to=2-2]
	\arrow["t"', from=2-2, to=2-3]
    \end{tikzcd}
    \]
    But then the following ones are pullbacks too.
    \[\begin{tikzcd}[row sep= 25 pt, column sep= 25 pt]
	{K_f} & {K_g} & Z \\
	X & Y & W
	\arrow["{k_h}"', from=1-1, to=1-2]
	\arrow["{\pi_f^1}"', from=1-1, to=2-1]
	\arrow["{\pi_g^2}"', from=1-2, to=1-3]
	\arrow["{\pi_g^1}", from=1-2, to=2-2]
	\arrow["g", from=1-3, to=2-3]
	\arrow["h", from=2-1, to=2-2]
	\arrow["g", from=2-2, to=2-3]
        \arrow["{h\circ \pi_f^2}", from=1-1, to=1-3, bend left = 30]
        \arrow["{t \circ f}"', from=2-1, to=2-3, bend right = 30]
    \end{tikzcd}
    \qquad
    \begin{tikzcd}[row sep = 25 pt, column sep=25 pt]
	{K_f} & {K_g} & Z \\
	X & Y & W
	\arrow["{k_h}"', from=1-1, to=1-2]
	\arrow["{\pi_f^2}"', from=1-1, to=2-1]
	\arrow["{\pi_g^1}"', from=1-2, to=1-3]
	\arrow["{\pi_g^2}", from=1-2, to=2-2]
	\arrow["g", from=1-3, to=2-3]
	\arrow["h", from=2-1, to=2-2]
	\arrow["g", from=2-2, to=2-3]
        \arrow["{h\circ \pi_f^1}", from=1-1, to=1-3, bend left = 30]
        \arrow["{t \circ f}"', from=2-1, to=2-3, bend right = 30]
    \end{tikzcd}
    \]

    The thesis follows again by \Cref{lemma:pullback_lemma}.
\end{proof}

\begin{proposition}\label{prop:reg_epi_coeq}
    Let $e: X \to Y$ be a regular epimorphism in a category $\cat C$ with a kernel pair $(K, \pi_1, \pi_2)$. Then, $e$ is the coequalizer of $\pi_1$ and $\pi_2$.
\end{proposition}

\begin{proof}
    By hypothesis, there exists a pair $f, g: Z \to X$ of which $e$ is the coequalizer. Since $e \circ f = e \circ g$, we have
    \[
        \begin{tikzcd}[row sep= 20, column sep = 13]
            Z \ar[drr, "f", bend left=30] \ar[ddr, "g"swap, bend right=30] \ar[dr, dashed, "k"] & & \\
            & K \ar[r, "{\pi_1}"] \ar[d, "{\pi_2}" swap] & X \ar[d, "e"] \\
            & X \ar[r, "e"swap] & Y
        \end{tikzcd}
    \]
    thus there exists the unique $k: Z \to K$. Let now $h: Z \to V$ be an arrow such that $h \circ \pi_1 = h \circ \pi_2$, then
    \begin{align*}
        h \circ f &= h \circ \pi_1 \circ k \\
                  &= h \circ \pi_2 \circ k \\
                  &= h \circ g
    \end{align*}
    and thus there exists a unique $l: Y \to V$ such that $l \circ e = h$.
\end{proof}

\begin{corollary}\label{cor:reg_epi_components_reg_epi_nat_trans}
    Let $\cat C$ be a category with pullbacks and $\phi : D \dot\to D'$ be a natural transformation between two functors $D, D': \cat{I \to C}$. If $\phi_i$ is a regular epi for every $i$, then $\phi$ is a regular epi.
\end{corollary}

\begin{proof}
    Let $(K_i, \pi_i^1, \pi_i^2)$ be the kernel pair of $\phi_i$ for each $i$. Given an arrow $\alpha: i \to j$ of $\cat I$, we have
    \begin{align*}
        \phi_j \circ D(\alpha) \circ \pi_i^1 &= D'(\alpha) \circ \phi_i \circ \pi_i^1 \\
                                             &= D'(\alpha) \circ \phi_i \circ \pi_i^2 \\
                                             &= \phi_j \circ D(\alpha) \circ \pi_i^2
    \end{align*}
    Thus, the outer boundary of the diagram below commutes, yielding the arrow $K(\alpha)$
    \[
        \begin{tikzcd}[row sep=26, column sep=26]
            {K_i} \ar[r, "{\pi_i^1}"] \ar[d, "{\pi_i^2}"swap] \ar[dr, dashed, "{K(\alpha)}"] & D(i) \ar[dr, "{D(\alpha)}"] \\
            D(i) \ar[dr, "{D(\alpha)}"swap] & K_j \ar[r, "{\pi_j^1}"] \ar[d, "{\pi_j^2}"swap] & D(j) \ar[d, "{\phi_j}"] \\
            & D(j) \ar[r, "{\phi_j}"swap] & D'(j)
        \end{tikzcd}
    \]

    In this way, we get a functor $K: \cat{I \to C}$, 
    which maps each $i$ onto $K_i$ and each arrow $\alpha$ onto $K(\alpha)$. 
	We have in fact $K(id_i) : K_i \to K_i$ is the arrow such that
        \[
            \begin{split}
                D(id_i) \circ \pi_i^1 &= \pi_i^1 \circ K(id_i) \\
                \pi_i^1 &= \pi_i^1 \circ K(id_i)
            \end{split}
                \qquad
            \begin{split}
                D(id_i) \circ \pi_i^2 &= \pi_i^2 \circ K(id_i) \\
                \pi_i^2 &= \pi_i^2 \circ K(id_i)
            \end{split}
        \]
	Thus, for the universal property of pullbacks, $K(id_i) = id_{K_i}$.

        Suppose now $\alpha : i \to j$ and $\beta: j \to k$. Computing, we have
        \[
            \begin{split}
                \pi_k^1 \circ K(\beta \circ \alpha) &= D(\beta \circ \alpha) \circ \pi_i^1 \\
                                                    &= D(\beta) \circ D(\alpha) \circ \pi_i^1 \\
                                                    &= D(\beta) \circ \pi_j^1 \circ K(\alpha) \\
                                                    &= \pi_k^1 \circ K(\beta) \circ K(\alpha)
            \end{split} \qquad
            \begin{split}
                \pi_k^2 \circ K(\beta \circ \alpha) &= D(\beta \circ \alpha) \circ \pi_i^2 \\
                                                    &= D(\beta) \circ D(\alpha) \circ \pi_i^2 \\
                                                    &= D(\beta) \circ \pi_j^2 \circ K(\alpha) \\
                                                    &= \pi_k^2 \circ K(\beta) \circ K(\alpha)
            \end{split}
        \]
        Again, for universal property of pullbacks, necessarily we have $K(\beta \circ \alpha) = K(\beta) \circ K(\alpha)$, proving functoriality of $K$.
   
    
     Hence, we have two natural transformations $\pi^1, \pi^2 : E \dot\to D$. By \Cref{prop:reg_epi_coeq}, every component $\phi_i$ is the coequalizer of $\pi_i^1, \pi_i^2: E \to D$, and so $\phi$ is the coequalizer of $\pi^1$ and $\pi^2$.
\end{proof}

\begin{lemma}\label{lemma:nat_trans_reg_epi_canonical_arrow_reg_epi}
    Let $D, D': \cat{I \to C}$ be two diagrams, and let $((c_i)_{i \in \cat I}, C)$ and $((c_i')_{i\in \cat I}, C')$ be, respectively, the colimit of $D$ and $D'$. If $\cat C$ has all colimits, for diagrams of shape $\cat I$ and $\phi: D \dot\to D'$ is a natural transformation in which all components are regular epimorphisms, then, the arrow induced by $\phi$ from $C$ to $C'$ (\Cref{prop:nat_tran_induces_unique_arrow_between_colimits}) is a regular epimorphism too.
\end{lemma}

\begin{proof}
    By \Cref{cor:reg_epi_components_reg_epi_nat_trans}, we know that $\phi: D \dot\to D'$ is a regular epimorphism, so that there is a functor $E: \cat{I \to C}$ and $\eta, \theta: E \dot\to D$ such that $\phi$ is the coequalizer of $\eta$ and $\theta$. Let now $((p_i)_{i \in \cat I}, P)$ be the colimit of $E$, by \Cref{prop:nat_tran_induces_unique_arrow_between_colimits}, we have $a, b: P \to C$ fitting in the diagram below.
    \[
        \begin{tikzcd}[row sep = 26, column sep = 26]
            E(i) \ar[r, "{p_i}"] \ar[d, "{\eta_i}"swap] & P \ar[d, dashed, "a"] \\
            D(i) \ar[r, "{c_i}"swap] & C
        \end{tikzcd}
        \qquad
        \begin{tikzcd}[row sep = 26, column sep = 26]
            E(i) \ar[r, "{p_i}"] \ar[d, "{\theta_i}"swap] & P \ar[d, dashed, "b"] \\
            D(i) \ar[r, "{c_i}"swap] & C
        \end{tikzcd}
    \]

    We want to show that $c$ coequalizes $\eta$ and $\theta$. Let thus $t: C \to T$ be an arrow such that $t \circ a = t \circ b$. Then, for every $i$, we have
    \begin{align*}
        t \circ c_i \circ \eta_i &= t \circ a \circ p_i \\
                                 &= t \circ b \circ p_i \\
                                 &= t \circ c_i \circ \theta_i
    \end{align*}    

    Thus, there is $t_i: D(i) \to T$ such that $t\circ c_i = t_i \circ \phi_i$. It is now easy to see that $((t_i)_{i \in \cat I}, T)$ is a cocone of $D'$: suppose $\alpha: i \to j$ be an arrow of $\cat I$, obtaining
    \begin{align*}
        t_i \circ \phi_i &= t \circ c_i \\
                         &= t \circ c_j \circ D(\alpha) \\
                         &= t_j \circ \phi_j \circ D(\alpha) \\
                         &= t_j \circ D'(\alpha) \circ \phi_i
    \end{align*}
    By the hypothesis that $\phi_i$ is regular epi for each $i$, therefore epi (by the dual of \Cref{prop:eq_are_mono}), we can conclude $t_i = t_j \circ D'(\alpha)$.
    
    Hence, we have an arrow $k: C' \to T$ such that $k \circ c_i' = t_i$. But then
    \begin{align*}
        c \circ c \circ c_i &= k \circ c_i' \circ \phi_i \\
                            &= t_i \circ \phi \\
                            &= t \circ c_i
    \end{align*}
    Showing that $k \circ c = t$.

    For the uniqueness, let $k': C' \to T$ be another arrow such that $k' \circ c = t$. Then we have
    \begin{align*}
        k' \circ c_i' \circ \phi_i &= k' \circ c \circ c_i \\
                                   &= t \circ c_i \\
                                   &= t_i \circ \phi_i
    \end{align*}    
    Since $\phi_i$ is a regular epimorphism, we have $k' \circ c_i' = t_i$, and, because $k \circ c_i' = t_i$ by construction, we can conclude that $k'=k$ since $((c_i')_{i \in \cat I}, C')$ is a colimit.
\end{proof}

\subsection{Factorization systems}
\todo{Inserire qualche cosa su sist. di fattorizazzione e M-adesività - devo pensare a come mettere la questione senza prendere troppo spazio}
\fi 



\section{Hypergraphical structures}
\todo{A very nice introduction}

\subsection{The category of hypergraphs}


We will start this section with the definition of hypergraphs and we will see how to label them with an algebraic signature. 

A pivotal role will be played by the Kleene star $(-)^\star\colon \Set\to \Set$ sending a set to the free monoid on it \cite{...}. In particular, given a set $X$, $X^\star$ is the set of functions with a natural number as domain. Then the neutral element is $?_X\colon \emptyset \to X$ and the composition of $f\colon n\to X$ with $g\colon m\to X$ is simply the arrow $n+m\to X$ induced by them. If $h\colon X\to Y$ is an arrow in $\Set$, then $h^\star \colon \X^\star\to Y^\star$ is the composition with $h$.


\begin{definition}A \emph{hypergraph} is a 4-uple $\mathcal{G}:=(E_\mathcal{G}, V_\mathcal{G}, s_\mathcal{G}, t_\mathcal{G})$ made by two sets $E_\mathcal{G}$ and $V_\mathcal{G}$, whose elements are called respectively \emph{hyperedges} and \emph{nodes}, plus a pair of \emph{source} and \emph{target} functions  $s_\mathcal{G}, t_\mathcal{G}\colon E_\mathcal{G}\rightrightarrows V_\mathcal{G}^\star$. A \emph{hypergraph morphism} $(E_\mathcal{G}, V_\mathcal{G}, s_\mathcal{G}, t_\mathcal{G})\to (E_\mathcal{H}, V_\mathcal{H}, s_\mathcal{H}, t_\mathcal{H})$ is a pair $(h,k)$ of functions $h\colon E_\mathcal{G}\to E_\mathcal{H}$, $k\colon V_\mathcal{G}\to V_\mathcal{H}$ such that the following diagrams commute.
	
	\[\xymatrix{ E_{\mathcal{G}} \ar[d]_{h} \ar[r]^{s_{\mathcal{G}}}& V^\star_{\mathcal{G}}  \ar[d]^{k^\star}& E_{\mathcal{G}} \ar[r]^{t_{\mathcal{G}}} \ar[d]_{h} & V^\star_{\mathcal{G}} \ar[d]^{k^\star}  \\ E_{\mathcal{G}} \ar[r]_{s_{\mathcal{H}}} & V^\star_{\mathcal{H}} & E_{\mathcal{G}} \ar[r]_{t_{\mathcal{H}}} & V^\star_{\mathcal{H}} }\]
	We define $\hyp$ to be the resulting category.
\end{definition}

Let $\pro^\star$ be the functor sending $X$ to $X^\star\times X^\star$, then we can present $\hyp$ as a comma category.
\begin{proposition}\label{prop:com}
	$\hyp$ is isomorphic to $\comma{\id{\Set}}{\pro^\star}$
\end{proposition}
\begin{proof}
	 Define two functors $F\colon  \hyp \to \comma{\id{\Set}}{\pro^\star} $
	and $G\colon \comma{\id{\Set}}{\pro^\star}\to \hyp$	 as follows 
	\begin{align*}
		\functor[l]{(E_\mathcal{{G}}, V_\mathcal{{G}}, s_\mathcal{{G}}, t_\mathcal{{G}})}{(f,g)}{\left(E_\mathcal{{H}}, V_\mathcal{{H}}, s_\mathcal{{H}}, t_\mathcal{{H}}\right)}
		\functormapsto
		\rfunctor{\left(E_\mathcal{{G}}, V_\mathcal{{G}}, 	\left(s_\mathcal{{G}}, t_\mathcal{{G}}\right)\right)}{(f, g) }{\left(E_\mathcal{{H}}, V_\mathcal{{H}}, \left(s_\mathcal{{H}}, t_\mathcal{{H}}\right)\right)}
	\end{align*}
	 \begin{align*}
		\functor[l]{(E_\mathcal{{G}}, V_\mathcal{{G}}, p_{\mathcal{G}})}{(f,g)}{(E_\mathcal{{H}}, V_\mathcal{{H}}, p_{\mathcal{H}})}
		\functormapsto
		\rfunctor{(E_\mathcal{{G}}, V_\mathcal{{G}}, 	\pi_1\circ  p_{\mathcal{G}}, \pi_2\circ  p_{\mathcal{G}})}{(f, g) }{(E_\mathcal{{H}}, V_\mathcal{{H}}, 	\pi_1\circ  p_{\mathcal{H}}, \pi_2\circ  p_{\mathcal{H}})}
	\end{align*} 
	Now it is immediate to notice that they are one the inverse of the other.
\end{proof}

We can show that the  Kleene star preserves pullbacks (see also \cite[Sec. 3]{carboni1995connected}  and \cite[Ch.4]{leinster2004higher} for details and a deeper and more conceptual approach).\index{Kleene star}

\begin{proposition}\label{prop:pb}
	The functor $(-)^\star$ preserves pullbacks.
\end{proposition}
\begin{proof}
	Suppose that a pullbacks square as the one below is given.
	\[\xymatrix{P \ar[r]^{p_1} \ar[d]_{p_2}& X \ar[d]^{f}\\ Y \ar[r]_{g} & Z}\]
	and consider the solid part of the following diagram.
	\[\xymatrix{Z\ar@{.>}[dr]^{t} \ar@/^.3cm/[drr]^{t_1} \ar@/_.3cm/[ddr]_{t_2}\\&P^\star \ar[r]^{p^\star_1} \ar[d]_{p^\star_2}& X^\star \ar[d]^{f^\star}\\ &Y^\star \ar[r]_{g^\star} & Z^\star}\]
	For every $z\in Z$ we have $t_1(z)\colon n\to X$ and $t_2(z)\colon m\to Y$ such that
	\[f^\star(t_1(x))=g^\star(t_2(z))\]
	In particular this entails that $n=m$ and that there is $t(z)\colon n\to P$ as in the diagram below 
	\[\xymatrix{n\ar[dr]^{t(z)} \ar@/^.3cm/[drr]^{t_1(z)} \ar@/_.3cm/[ddr]_{t_2(z)}\\&P \ar[r]^{p_1} \ar[d]_{p_2}& X \ar[d]^{f}\\ &Y \ar[r]_{g} & Z}\]
	But this is equivalent to say that the dotted $t\colon Z\to P$ exists, while its uniqueness follows at once from the universal property of the pullback with which we started.
\end{proof}

\begin{remark}\label{rem:mono}
	Preservation of pullbacks implies that $(-)^\star$ sends monos to monos.
\end{remark}

\begin{corollary}\label{prop:hypadh}
	$\hyp$ is an adhesive category.
\end{corollary}
\begin{proof}
	$(-)^\star$ preserves pullbacks by \Cref{prop:pb}, while $\pro$ is continuous by definition, thus the thesis follows from This follows from \Cref{comma,prop:com}.
\end{proof}



\Cref{prop:left,prop:com} allows us to deduce immediately the following.

\begin{proposition}\label{cor:left}
	The forgetful functor $U_{\catname{Hyp}}\colon \hyp \to \catname{Set}$ which sends an hypergraph $\mathcal{G}$ to its set of nodes has a left adjoint $\Delta_{\hyp}$.
\end{proposition}

\begin{remark}Since the initial object of $\catname{Set}$ is the empty set,  $\Delta_{\hyp}(X)$ is the hypergraph which has $X$ as set of nodes and $\emptyset$ as set of hyperedges and $?_X$ as both source and target function.
\end{remark}


Hypergraphs,  can be represented graphically. We will use dots to denote nodes and squares to denote hyperedges, the name of a node or of an hyperedge will be put near the corresponding dot or square. Sources and targets are represented by lines between dots and squares: the lines from the sources of an hyperedge will have an arrowhead in the middle pointing towards the hyperedge, while the lines to the targets will have arrowheads pointing to the target nodes.  We will decorate the arrow corresponding to the $i^{th}$ letter (i.e.~its value at $i-1$) of a target or a source with a label $i$.

\begin{example}Take $V_{\mathcal{G}}$ to be be $\{v_1, v_2, v_3, v_4, v_5\}$ and $E_{\mathcal{G}}$ to be $\{h_1, h_2, h_3\}$. Sources and targets are given by:
	\[\begin{matrix}
		s_{\mathcal{G}}(h_1)\colon 2\to V_{\mathcal{G}}  & \begin{matrix}
			0 \mapsto v_1\\
			1\mapsto v_2
		\end{matrix} && s_{\mathcal{G}}(h_2)\colon 2\to V_{\mathcal{G}} & \begin{matrix}
			0 \mapsto v_3\\
			1\mapsto v_4 
		\end{matrix} && s_{\mathcal{G}}(h_3)\colon 1\to V_{\mathcal{G}} & 
		0 \mapsto v_5\\
		t_{\mathcal{G}}(h_1)\colon 2\to V_{\mathcal{G}} & \begin{matrix}
			0 \mapsto v_3\\
			1\mapsto v_4
		\end{matrix} && t_{\mathcal{G}}(h_2)\colon 2\to V_{\mathcal{G}} & 0\mapsto v_5 && t_{\mathcal{G}}(h_3)\colon 0\to V_{\mathcal{G}} &  t_{\mathcal{G}}(h_3)=?_{ V_{\mathcal{G}}} 
	\end{matrix}\]
	
	We can draw the resulting $\mathcal{G}$ as follows:
	\begin{center}\begin{tikzpicture}
			\node[circle,fill=black,inner sep=0pt,minimum size=6pt,label=above:{$v_1$}] (A) at (0,0) {};
			\node[circle,fill=black,inner sep=0pt,minimum size=6pt,label=above:{$v_2$}] (B) at (0,-1.5) {};
			\node[circle,fill=black,inner sep=0pt,minimum size=6pt,label=above:{$v_3$}] (C) at (3,0) {};
			\node[circle,fill=black,inner sep=0pt,minimum size=6pt,label=above:{$v_4$}] (D) at (3,-1.5) {};
			\node[circle,fill=black,inner sep=0pt,minimum size=6pt,label=above:{$v_5$}] (E) at (6,-0.75) {};
			\draw[rounded corners] (1.25, -1) rectangle (1.75, -0.5) {};
			\draw[->-=.5](4.75,-0.75)--(E)node[pos=0.5, above,font=\fontsize{7}{0}\selectfont]{$1$};
			\draw[->-=.5](E)--(7,-0.75)node[pos=0.5, above,font=\fontsize{7}{0}\selectfont]{$1$};
			\draw[rounded corners] (4.25, -1) rectangle (4.75, -0.5) {};
			\node at (4.5, -0.3){$h_2$};
			\node at (1.5, -0.3){$h_1$};
			\node at (7.25, -0.3){$h_3$};
			\draw[rounded corners] (7, -1) rectangle (7.5, -0.5) {};
			\draw(A)[->-=.5]..controls(0.5,0)and(1.2,-0.2)..(1.25,-0.6)node[pos=0.5, above,font=\fontsize{7}{0}\selectfont]{$1$};
			\draw(B)[->-=.5]..controls(0.5,-1.5)and(1.2,-1.3)..(1.25,-0.9)node[pos=0.5, below,font=\fontsize{7}{0}\selectfont]{$2$};
			
			\draw(C)[->-=.5]..controls(3.5,0)and(4.2,-0.25)..(4.25,-0.6)node[pos=0.5, above,font=\fontsize{7}{0}\selectfont]{$1$};
			\draw(D)[->-=.5]..controls(3.5,-1.5)and(4.2,-1.3)..(4.25,-0.9)node[pos=0.5, below,font=\fontsize{7}{0}\selectfont]{$2$};
			
			\draw[->-=.5](1.75,-0.9)..controls(1.8,-1.3)and(2.5,-1.5)..(D)node[pos=0.5, below,font=\fontsize{7}{0}\selectfont]{$2$};
			\draw[->-=.5] (1.75,-0.6)..controls(1.8,-0.25)and(2.5,0)..(C) node[pos=0.5, above,font=\fontsize{7}{0}\selectfont]{$1$};
		\end{tikzpicture}
	\end{center}
\end{example}
\begin{example}\label{exa_2} Let $V_{\mathcal{G}}$ be as in the previous example and $E_{\mathcal{G}}=\{h_1, h_2, h_3\}$.	Then we define
	\[\begin{matrix}
		s_{\mathcal{G}}(h_1)\colon 0\to V_{\mathcal{G}}  & s_{\mathcal{G}}(h_1)=?_{V_\mathcal{G}} && s_{\mathcal{G}}(h_2)\colon 2\to V_{\mathcal{G}} & \begin{matrix}
			0 \mapsto v_1\\
			1\mapsto v_2
		\end{matrix}&& s_{\mathcal{G}}(h_3)\colon 2\to V_{\mathcal{G}} & \begin{matrix} 
			0 \mapsto v_1\\
			1\mapsto v_4	
		\end{matrix}\\
		t_{\mathcal{G}}(h_1)\colon 1\to V_{\mathcal{G}} & 
		0 \mapsto v_1 && 		t_{\mathcal{G}}(h_2)\colon 1\to V_{\mathcal{G}} & 0\mapsto v_3 &&   t_{\mathcal{G}}(h_3)\colon 1\to V_{\mathcal{G}} & 1\mapsto v_5
	\end{matrix}\]
	
	Now we can depict $\mathcal{G}$ as
	\begin{center}\begin{tikzpicture}
			\node[circle,fill=black,inner sep=0pt,minimum size=6pt,label=above:{$v_1$}] (A) at (0,0) {};
			\node[circle,fill=black,inner sep=0pt,minimum size=6pt,label=above:{$v_2$}] (B) at (0,-1.5) {};
			\node[circle,fill=black,inner sep=0pt,minimum size=6pt,label=above:{$v_3$}] (C) at (3,-0.75) {};
			\node[circle,fill=black,inner sep=0pt,minimum size=6pt,label=above:{$v_4$}] (D) at (3,-2.25) {};
			\node[circle,fill=black,inner sep=0pt,minimum size=6pt,label=right:{$v_5$}] (E) at (6,-1.5) {};
			\draw[->-=.5] (1.75,-0.75)--(C)node[pos=0.5, above,font=\fontsize{7}{0}\selectfont]{$1$};
			\draw[rounded corners] (1.25, -1) rectangle (1.75, -0.5) {};
			\draw[->-=.5] (4.75,-1.5)--(E)node[pos=0.5, above,font=\fontsize{7}{0}\selectfont]{$1$};
			\draw[->-=.5] (-1.5,0)--(A)node[pos=0.5, above,font=\fontsize{7}{0}\selectfont]{$1$};
			\draw[rounded corners] (4.25, -1.75) rectangle (4.75, -1.25) {};
			\node at (4.5, -1.05){$h_3$};
			\node at (1.5, -0.3){$h_2$};
			\node at (-1.75, 0.45){$h_1$};
			\draw[rounded corners] (-2, -0.25) rectangle (-1.5, 0.25) {};
			\draw[->-=.5] (A)..controls(0.5,0)and(1.2,-0.2)..(1.25,-0.6)node[pos=0.5, above,font=\fontsize{7}{0}\selectfont]{$1$};
			\draw[->-=.5] (B)..controls(0.5,-1.5)and(1.2,-1.3)..(1.25,-0.9)node[pos=0.5, below,font=\fontsize{7}{0}\selectfont]{$2$};
			
			\draw[->-=.5] (C)..controls(3.5,-0.75)and(4.2,-0.95)..(4.25,-1.35)node[pos=0.5, above,font=\fontsize{7}{0}\selectfont]{$1$};
			\draw[->-=.5] (D)..controls(3.5,-2.25)and(4.2,-2.05)..(4.25,-1.65)node[pos=0.5, below,font=\fontsize{7}{0}\selectfont]{$2$};
		\end{tikzpicture}
	\end{center}
\end{example}

\begin{example}\label{exa_3} Let $\Sigma=(O_\Sigma, \ari_\Sigma)$ be an algebraic signature, we can construct the hypergraph $\mathcal{G}^\Sigma$ taking $V_{\mathcal{G}^\Sigma}$ and $E_{\mathcal{G}^\Sigma}$ to be respectively the singleton $\{\heartsuit\}$ and the set $O_\Sigma$. We put
	\[\begin{matrix}
		s_{\mathcal{G}^\Sigma}\colon O_\Sigma\to \{\heartsuit\}^\star & o\mapsto \delta_\heartsuit^{\ari_\Sigma(o)}
		&&
		t_{\mathcal{G}^\Sigma}\colon O_\Sigma\to \{\heartsuit\}^\star & o\mapsto \delta_\heartsuit
	\end{matrix}\]
	For instance let $\Sigma_G$ be the signature of groups of \Cref{ex:g}, then $\mathcal{G}^{\Sigma_G}$ is depicted as:
	\begin{center}
		\begin{tikzpicture}
			\node[circle,fill=black,inner sep=0pt,minimum size=6pt,label=above:{$v$}] (V) at (0,0) {};
			\node(E)at(-2, 0.4){$e$};
			\node(M)at(0, 2.15){$\cdot$};
			\node(I)at(2, 0.5){$(-)^{-1}$};
			\draw[->-=.5](-1.75,0)--(V)node[pos=0.5, above,font=\fontsize{7}{0}\selectfont]{$1$};
			\draw[->-=.5](V)..controls(-0.5,0.5)and(-0.8,1)..(-0.25,1.6)node[pos=0.5, right,font=\fontsize{7}{0}\selectfont]{$2$};
			\draw[->-=.5](V)..controls(-1,0.6)and(-1,1.1)..(-0.25,1.9)node[pos=0.5, left,font=\fontsize{7}{0}\selectfont]{$1$};
			\draw[->-=.5](0.25,1.75)..controls(0.8,0.8)and(0.5,0.5)..(V)node[pos=0.5, right,font=\fontsize{7}{0}\selectfont]{$1$};
			\draw[->-=.5](V)--(1.75,0)node[pos=0.5, above,font=\fontsize{7}{0}\selectfont]{$1$};
			\draw[->-=.5](2.25,0)..controls(3.5,0)and(2.5,-2)..(V)node[pos=0.5, below,font=\fontsize{7}{0}\selectfont]{$2$};
			\draw[rounded corners] (-2.25, -0.25) rectangle (-1.75, 0.25) {};
			\draw[rounded corners] (-0.25, 1.5) rectangle (0.25, 2) {};
			\draw[rounded corners] (2.25, -0.25) rectangle (1.75, 0.25) {};
		\end{tikzpicture}
	\end{center}
\end{example}


\subsubsection{$\hyp$ as a topos of presheaves}

By \Cref{lim} we already know that $\hyp$ has all pullbacks and by \Cref{prop:hypadh} we know that it is adhesive. Actually more can be proved about it: we can realize $\hyp$ as a topos of presheaves \cite{bonchi2022string}.


\begin{definition}Let $\catname{H}$ be the category in which:
	\begin{itemize}
		\item the set of objects is given by $ (\mathbb{N}\times \mathbb{N}) \cup \{\bullet\}$
		\item arrows are given by the identities $\id{k,l}$ and $\id{\bullet}$ and exactly $k+l$ arrows $f_i\colon (k,l)\rightarrow \bullet$, where $i$ ranges from $0$ to $k+l-1$;
		\item composition is defined simply putting, for every $f_i\colon (k,l)\rightarrow \bullet$:
		\begin{equation*}
			f_i=f_i\circ \id{k,l} \qquad f_i = \id{\bullet}\circ f_i 
		\end{equation*}
	\end{itemize}
\end{definition}

Now, given $F\colon \catname{H}\to \catname{Set}$ we can define
\[E_F:=\sum_{k,l\in \mathbb{N}}F(k,l)\]
For every element $x$ of $F(k,l)$ we can put
\begin{align*}
	s^F_{k,l}(x)\colon k\to F(\bullet) \quad i \mapsto F(f_i)(x)\qquad 
	t^F_{k,l}(x)\colon l\to F(\bullet) \quad i \mapsto F(f_{i+k})(x)
\end{align*}
obtaining
$s_F, t_F\colon E_F\rightrightarrows \mathcal{F}(\bullet)^{\star}$. Let $\mathcal{G}_F$ be the resulting hypergraph. Now, every $\eta\colon F\rightarrow H$ in $\catname{Set}^{\catname{H}}$ has components $\eta_{k,l}\colon F(k,l)\to H(k,l)$, $\eta_{\bullet}\colon F(\bullet)\to H(\bullet)$, thus it induces a function $\hat{\eta}\colon E_F\rightarrow E_H$ such that the following squares commute
\[\xymatrix{ E_F \ar[r]^{s_F} \ar[d]_{\hat{\eta}}& F(\bullet)^\star \ar[d]^{\eta^\star_\bullet} & E_F \ar[r]^{t_F} \ar[d]_{\hat{\eta}} & F(\bullet)^\star \ar[d]^{\eta^\star_\bullet}\\ E_H \ar[r]_{s_H} & H(\bullet)^\star & E_H \ar[r]_{t_H}& H(\bullet)^\star}\]

This is equivalent to say that $\eta$ induces a morphism $(\hat{\eta}, \eta_{\bullet})\colon \mathcal{G}_F\to \mathcal{G}_H$. It is now clear that sending $F$ to $\mathcal{G}_F$ and $\eta$ to $(\hat{\eta}, \eta_{\bullet})$ defines a faithful functor $\mathcal{G}_{-}\colon \catname{Set}^{\catname{H}}\to \hyp$.

\begin{proposition}
	$\hyp$ is equivalent to the category $\catname{Set}^{\catname{H}}$.
\end{proposition}
\begin{proof}
	Let $X$ be a set, for every $n\in \mathbb{N}$ define
	\[X_{n}:= \left\{w\in X^{\star} \mid \dom(w)=n\right\}\]
	In particular, if $F\colon \catname{H}\to \catname{Set}$ then the image of the coprojection $\iota^{F}_{k,l}\colon F(k,l)\to E_F$ is the intersection
	\[s^{-1}_F\left(F(\bullet)_{k}\right)\cap t^{-1}_F\left(F(\bullet)_{l}\right) \]
	
	We are now ready to that $\mathcal{G}_{-}$ is full and essentially surjective.
	\begin{itemize}
		\item For fullness, let $(f,g)\colon \mathcal{G}_F\to \mathcal{G}_{H}$ be a morphism of hypergraphs and define $f_{k,l}$ to be $f\circ \iota^F_{k,l}$, the composition of $h$ with Now, if $x\in F(k,l)$ then 
		\[\begin{split}
			s_H (f_{k,l} (x))&=s_H \left(f \left(\iota^F_{k,l} (x)\right)\right)\\&=g^{\star}\hspace{-1pt}\left(s_{F} \hspace{-1pt}\left(\iota^F_{k,l} (x)\right)\right)
		\end{split}\qquad \begin{split}
			t_H (f_{k,l} (x))&=s_t\hspace{-1pt} \left(f \hspace{-1pt}\left(\iota^F_{k,l} (x)\right)\right)\\&=g^{\star}\hspace{-1pt} \left(t_{F}\hspace{-1pt} \left(\iota^F_{k,l} (x)\right)\right)
		\end{split}\]
		Therefore there exists $\eta_{k,l}\colon F(k,l)\to H(k,l)$ fitting in the diagram below
		\[\xymatrix{F(k,l) \ar[r]^-{\iota^{F}_{k,l}} \ar@{.>}[d]_{\eta_{k,l}}& E_F \ar[d]^{f_{k,l}}\\ H(k,l) \ar[r]_-{\iota^{H}_{k,l}}& E_H}\]
		Define $\eta_{\bullet}\colon F(\bullet)\to H(\bullet)$ simply as $g^\star$, then  the collection of all the $\eta_{k,l}$ and of $\eta_\bullet$ defines a natural transformation $\eta\colon F\to H$. Indeed, if $f_i\colon (k,l)\to \bullet$ we have:
		\[\xymatrix{F(k,l) \ar@/^.4cm/[rr]^{s^F_{k,l}} \ar[r]_-{\iota^{F}_{k,l}} \ar[d]_{\eta_{k,l}}& E_F \ar[d]_{f_{k,l}} \ar[r]_-{s_F}& F(\bullet)^\star \ar[d]^{g^\star}&F(k,l) \ar@/^.4cm/[rr]^{t^F_{k,l}}\ar[r]_-{\iota^{F}_{k,l}} \ar[d]_{\eta_{k,l}}& E_F \ar[d]_{f_{k,l}} \ar[r]_-{t_F}& F(\bullet)^\star \ar[d]^{g^\star}\\ H(k,l) \ar@/_.4cm/[rr]_{s^H_{k,l}} \ar[r]^-{\iota^{H}_{k,l}}& E_H \ar[r]^-{s_H}& H(\bullet)^\star& H(k,l) \ar@/_.4cm/[rr]_{t^H_{k,l}} \ar[r]^-{\iota^{H}_{k,l}}& E_H \ar[r]^-{t_H}& H(\bullet)^\star}\]
		Thus if $i<k$ then
		\begin{align*}
			\eta_\bullet(F(f_i)(x))&=g(F(f_i)(x))\\&=g\hspace{-1pt}\left(s^F_{k,l}(x)(i)\right)\\&=g^\star\hspace{-1pt}\left(s^F_{k,l}(x)\right)(i)\\&=s^H_{k,l}\hspace{-1pt}\left(\eta_{k,l}(x)\right)(i)\\&=F(f_i)(\eta_{k,l}(x))
		\end{align*}
		while, if $k\leq i < k+l-1$
		\begin{align*}
			\eta_\bullet(F(f_i)(x))&=g(F(f_i)(x))\\&=g\hspace{-1pt}\left(t^F_{k,l}(x)(i)\right)\\&=g^\star\hspace{-1pt}\left(t^F_{k,l}(x)\right)(i)\\&=t^H_{k,l}\hspace{-1pt}\left(\eta_{k,l}(x)\right)(i)\\&=F(f_i)(\eta_{k,l}(x))
		\end{align*}
		Finally, by contruction it is clear that $(\hat{\eta}, \eta_{\bullet})=(f,g)$. 
		\item Given an hypergraph $\mathcal{G}=(E_\mathcal{G}, V_\mathcal{G}, s_\mathcal{G}, t_\mathcal{G})$ we can define 
		\[F_{\mathcal{G}}(k,l):=s_\mathcal{G}^{-1}(V_k)\cap t_\mathcal{G}^{-1}(V_l) \qquad F_{\mathcal{G}}(\bullet):=V_\mathcal{G}\]
		Given $f_i\colon (k,l)\to \bullet$ we put
		\[F_{\mathcal{G}}(f_i)\colon F_{\mathcal{G}}(k,l)\to F_{\mathcal{G}}(\bullet) \qquad x\mapsto \begin{cases}
			s_\mathcal{G}(x)(i) & i<k\\
			t_\mathcal{G}(x) (i-k) &i\leq k < k+l-1
		\end{cases}  \]
		
		$F_{\mathcal{G}}$ so defined is a functor $\catname{H}\to \catname{Set}$ and for every $h\in E_\mathcal{G}$ there exists a unique pair $(k,l)$ such that $h\in F_{\mathcal{G}}(k,l) $, namely the pair $(\dom(s_\mathcal{G})(h), \dom(t_\mathcal{G})(h))$thus
		\[\sum_{k,l\in \mathbb{N}}F_{\mathcal{G}}(k,l)\simeq E\]
		Moreover, by construction $s_{F_{\mathcal{G}}}=s$ and $t_{F_{\mathcal{G}}}=t$, from which the thesis follows. 
	\end{itemize}
\end{proof}
As a corollary we get immediately the following.
\begin{corollary}
	$\hyp$ is a complete category.
\end{corollary}

\subsection{Labelling hypergraph with an algebraic signature}\index{signature!algebraic -}
Let $\Sigma=(O_\Sigma, \ari_\Sigma)$ be an algebraic signature, we are going to use the hypergraph $\mathcal{G}^{\Sigma}$ of \Cref{exa_3} in order to label hyperedges with operations.
\begin{definition}Let $\Sigma=(O, \ari)$ be an algebraic signature, the category $\hyps$ of \emph{algebraically labelled hypergraphs} is the slice category $\hyp/\mathcal{G}^\Sigma$.\index{hypergraph!algebraically labelled -}
\end{definition}
\Cref{cor:mono} and \Cref{cor:slice} give us immediately an adhesivity result for $\hyp_{\Sigma}$ and a characterization of monomorphisms in it.
\begin{proposition}\label{prop:mono}
	For every algebraic signature $\Sigma$, $\hyps$ is an adhesive category. Moreover a morphism $(h,k)$ between two object of $\hyp_{\Sigma}$ is a mono if and only if $h$ and $k$ are injective functions.
\end{proposition}


\begin{remark}\label{rem:label}	
	Let $\mathcal{H}=(E, V, s, t)$ be an hypergraph, since  $U_{\hyp}(\mathcal{G}^{\Sigma})$ is the singleton an arrow $\mathcal{H}\rightarrow \mathcal{G}^{\Sigma}$, is determined by a function $h\colon E_\mathcal{H}\to O_\Sigma$  such that, for every $e\in E_{\mathcal{H}}$
	\[\ari_\Sigma(h(e))=s_\mathcal{H}(e)\]
	
	On the other hand, if $\mathcal{H}$ has an hyperedge $e$ such that $t_{\mathcal{H}}(e)$ has a length different from $1$, then there is no morphism $\mathcal{H}\to \mathcal{G}^{\Sigma}$. Indeed, if such a morphism $(h,!_{V_\mathcal{H}})\colon \mathcal{H}\to \mathcal{G}^\Sigma$ exists, then, for every $e\in E_{\mathcal{H}}$ we have
	\begin{align*}
		f^{\star}(t_{\mathcal{H}}(h))&=t_{\mathcal{G}^{\Sigma}}(f(h))\\&=\delta_\heartsuit
	\end{align*}
	and so $\dom(t_{\mathcal{H}}(h))=1$.  
\end{remark}

$\hyp_{\Sigma}$, as any slice category, has a forgetful functor $U_{\Sigma}\colon \hyp_{\Sigma}\to \catname{Set}$ which sends $(h,k)\colon \mathcal{H}\to \mathcal{G}^{\Sigma}$ to $U_{\hyp}(\mathcal{H}$). Now, $U_{\hyp}(\mathcal{G}^{\Sigma})=\{v\}$ thus, for every set $X$, there is only one arrow $X\to U_{\hyp}(\mathcal{G}^{\Sigma})$. Define $\Delta_{\Sigma}(X)\colon \Delta_{\hyp}(X)\to \mathcal{G}^{\Sigma}$ to be the transpose of this arrow.

\begin{proposition} $U_\Sigma$
	has a left adjoint $\Delta_\Sigma$.
\end{proposition}
\begin{proof}Let $(h, !_{V_\mathcal{H}})\colon \mathcal{H}\to \mathcal{G}^{\Sigma}$ be an object of $\hyp_{\Sigma}$, and suppose that there exists $f\colon X\to U_{\Sigma}(\mathcal{H})$. Since $U_{\Sigma}(\mathcal{H})=U_{\hyp}(\mathcal{H})$ and $\id{\catname{Set}}$ is the unit of $\Delta_\hyp \dashv U_{\hyp}$, there exists a unique morphism $(k,f)\colon \Delta_{\hyp}(X)\to \mathcal{H}$ of $\hyp$. Since the set of hyperedges of $\Delta_{\hyp}(X)$ is empty, $k$ must be $?_{E_\mathcal{H}}$ and the commutativity of each of the two triangles below is equivalent to that of the other
	\[\xymatrix{\Delta_\hyp(X) \ar[rr]^{(?_{E_\mathcal{H}}, f)} \ar[dr]_{\Delta_\Sigma(X)} && \mathcal{H} \ar[dl]^{(h, !_{V_\mathcal{H}})} & U_\hyp(\Delta_\hyp(X)) \ar[rr]^{f} \ar[dr]_{U_\hyp(\Delta_\Sigma(X))\hspace{10pt}} && U_\hyp(X) \ar[dl]^{!_{V_\mathcal{H}}}\\& \mathcal{G}^\Sigma &&&U_\hyp\hspace{-1.5pt}\left(\mathcal{G}^\Sigma\right)}\]
	But the triangle on the right commutes because $U_{\hyp}(\mathcal{G}^{\Sigma})$ is terminal.
\end{proof}

We will extend our graphical notation of hypergraphs to labeled ones putting the label of an hyperedge $h$ inside its corresponding square.
\begin{example}\label{lab_1}
	The simplest example is given by the identity $\id{\mathcal{G}^\Sigma}\colon \mathcal{G}^\Sigma\rightarrow \mathcal{G}^{\Sigma}$. If $\Sigma$ is the signature of groups $\Sigma_G$we get 
	\begin{center}
		\begin{tikzpicture}
			\node[circle,fill=black,inner sep=0pt,minimum size=6pt,label=above:{$\heartsuit$}] (V) at (0,0) {};
			\node(E)at(-2, 0.4){$e$};
			\node(M)at(0, 2.15){$\cdot$};
			\node(I)at(2, 0.75){$(-)^{-1}$};
			
			\node(E')at(-2, 0){$e$};
			\node(M')at(0, 1.75){$\cdot$};
			\node(I')at(2, 0){$(-)^{-1}$};
			\draw[->-=.5](-1.75,0)--(V)node[pos=0.5, above,font=\fontsize{7}{0}\selectfont]{$1$};
			\draw[->-=.5](V)..controls(-0.5,0.5)and(-0.8,1)..(-0.25,1.6)node[pos=0.5, right,font=\fontsize{7}{0}\selectfont]{$2$};
			\draw[->-=.5](V)..controls(-1,0.6)and(-1,1.1)..(-0.25,1.9)node[pos=0.5, left,font=\fontsize{7}{0}\selectfont]{$1$};
			\draw[->-=.5](0.25,1.75)..controls(0.8,0.8)and(0.5,0.5)..(V)node[pos=0.5, right,font=\fontsize{7}{0}\selectfont]{$1$};
			\draw[->-=.5](V)--(1.5,0)node[pos=0.5, above,font=\fontsize{7}{0}\selectfont]{$1$};
			\draw[->-=.5](2.5,0)..controls(4,0)and(2.5,-2)..(V)node[pos=0.5, below,font=\fontsize{7}{0}\selectfont]{$2$};
			\draw[rounded corners] (-2.25, -0.25) rectangle (-1.75, 0.25) {};
			\draw[rounded corners] (-0.25, 1.5) rectangle (0.25, 2) {};
			\draw[rounded corners] (2.5, -0.5) rectangle (1.5, 0.5) {};
		\end{tikzpicture}
	\end{center}
\end{example}

\begin{example}\label{lab_2}
	Take again $\Sigma_G$ the signature of groups, then the hypergraph $\mathcal{G}$ of \Cref{exa_2} can be labeled defining
	\begin{align*}
		e=f(h_1) \quad \cdot=f(h_2)\quad \cdot=f(h_3)
	\end{align*}
	In this case we get the following picture
	
	\begin{center}\begin{tikzpicture}
			\node[circle,fill=black,inner sep=0pt,minimum size=6pt,label=above:{$v_1$}] (A) at (0,0) {};
			\node[circle,fill=black,inner sep=0pt,minimum size=6pt,label=above:{$v_2$}] (B) at (0,-1.5) {};
			\node[circle,fill=black,inner sep=0pt,minimum size=6pt,label=above:{$v_3$}] (C) at (3,-0.75) {};
			\node[circle,fill=black,inner sep=0pt,minimum size=6pt,label=above:{$v_4$}] (D) at (3,-2.25) {};
			\node[circle,fill=black,inner sep=0pt,minimum size=6pt,label=right:{$v_5$}] (E) at (6,-1.5) {};
			\draw[->-=.5] (1.75,-0.75)--(C)node[pos=0.5, above,font=\fontsize{7}{0}\selectfont]{$1$};
			\draw[rounded corners] (1.25, -1) rectangle (1.75, -0.5) {};
			\draw[->-=.5] (4.75,-1.5)--(E)node[pos=0.5, above,font=\fontsize{7}{0}\selectfont]{$1$};
			\draw[->-=.5] (-1.5,0)--(A)node[pos=0.5, above,font=\fontsize{7}{0}\selectfont]{$1$};
			\draw[rounded corners] (4.25, -1.75) rectangle (4.75, -1.25) {};
			\node at (4.5, -1.05){$h_3$};
			\node at (1.5, -0.3){$h_2$};
			\node at (-1.75, 0.45){$h_1$};
			\draw[rounded corners] (-2, -0.25) rectangle (-1.5, 0.25) {};
			\draw[->-=.5] (A)..controls(0.5,0)and(1.2,-0.2)..(1.25,-0.6)node[pos=0.5, above,font=\fontsize{7}{0}\selectfont]{$1$};
			\draw[->-=.5] (B)..controls(0.5,-1.5)and(1.2,-1.3)..(1.25,-0.9)node[pos=0.5, below,font=\fontsize{7}{0}\selectfont]{$2$};
			
			\draw[->-=.5] (C)..controls(3.5,-0.75)and(4.2,-0.95)..(4.25,-1.35)node[pos=0.5, above,font=\fontsize{7}{0}\selectfont]{$1$};
			\draw[->-=.5] (D)..controls(3.5,-2.25)and(4.2,-2.05)..(4.25,-1.65)node[pos=0.5, below,font=\fontsize{7}{0}\selectfont]{$2$};
			
			\node at (-1.75,0) {$e$};
			\node at (1.5,-0.75) {$\cdot$};
			\node at (4.5,-1.5) {$\cdot$};
		\end{tikzpicture}
	\end{center}
\end{example}



\subsection{Term Graphs}

\section{Hypergraphs with equivalences}

\subsection{title}

\section{title}

\section{EGGS}

\section{Conclusions and further works}
\todo{Some very nice conclusions}
\bibliography{biblio}

\appendix
\section{Comparison with Ghica}
\todo{Lo vogliamo fare?}


\end{document}

