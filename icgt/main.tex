\documentclass[runningheads,envcountsect]{llncs}

\usepackage{stmaryrd}
\usepackage{bbding}
%\Envelope
\newcommand{\cat}[1]{\ensuremath{\mathbf{#1}}}
\usepackage{amsmath,amsthm,amssymb,mathrsfs, dsfont} %simboli matematici
\usepackage[all, cmtip]{xy}

\usepackage[utf8]{inputenc} % Input encoding - per caratteri particolari
\usepackage[english]{babel} % Lingua principale inglese
\usepackage{graphicx} % Per includere immagini esterne
\usepackage[tickmarkheight=.5em,textwidth=\marginparwidth,textsize=small]{todonotes}
\usepackage{mathtools}
\usepackage{csquotes}


\usepackage{hyperref, cleveref}

% General Setting diagrams
\usepackage{tikz-cd} %diagrammi
\tikzcdset{row sep/normal=5em}
\tikzcdset{column sep/normal=5em}
\tikzcdset{every label/.append style = {font = \small}}

\theoremstyle{plain}
\newtheorem{theorem}{Theorem}[section]
\newtheorem{prop}[theorem]{Proposition}
\newtheorem{lemma}[theorem]{Lemma}
\newtheorem{cor}[theorem]{Corollary}

\theoremstyle{definition}
\newtheorem{definition}[theorem]{Definition}
\newtheorem{example}[theorem]{Example}
\newtheorem{remark}[theorem]{Remark}
\newtheorem{obs}[theorem]{Observation}

\DeclareMathAlphabet{\mymathbb}{U}{BOONDOX-ds}{m}{n}

\newcommand{\Ob}{\mathcal{O}b}
\newcommand{\Hom}{\mathcal{H}om}
\newcommand{\Set}{\mathbf{Set}}
\newcommand{\Reg}{\mathcal{Reg}}
\newcommand{\Mono}{\mathcal{Mono}}
\newcommand{\initial}{\mymathbb{0}}
\newcommand{\terminal}{\mathds{1}}
\newcommand{\eg}[1]{\mathbf{EqGraph}_{\textbf {\textup{#1}}}}
\newcommand{\egg}[1]{\mathbf{EGG}_{\textbf {\textup{#1}}}}

\makeatletter
\def\@citecolor{blue}%
\def\@urlcolor{blue}%
\def\@linkcolor{blue}%
\def\UrlFont{\rmfamily}
\def\orcidID#1{\smash{\href{http://orcid.org/#1}{\protect\raisebox{-1.25pt}{\protect\includegraphics{orcid_color.eps}}}}}
\makeatother



\def\R{\mathsf{R}}
\def\B{\textbf {\textup{B}}}
\def\C{\textbf {\textup{C}}}
\def\D{\textbf {\textup{D}}}
\def\X{\textbf {\textup{X}}}
\def\Y{\textbf {\textup{Y}}}
\def\E{\textbf {\textup{E}}}


\bibliographystyle{abbrv}



\title{On the adhesivity properties of equivalence graphs\thanks{???????}}

\titlerunning{????} %TODO optional, please use if title is longer than one line

%% Author with single affiliation.
\author{Roberto Biondo \inst{1}{\small\Envelope}\and Davide Castelnovo\inst{1}\orcidID{0000-0002-5926-5615}
	\and \\Fabio Gadducci\inst{1}\orcidID{0000-0003-0690-3051}
}
\institute{Dept.~of Computer Science, University of Pisa, Italy. 
	\email{r.biondo@studenti.unipi.it},
	\email{castelnovod@gmail.com},
	 \email{fabio.gadducci@unipi.it}
}



%frecce
\newcommand{\mor}{\mathsf{Mor}}
\newcommand{\mon}{\mathsf{Mono}}
\newcommand{\reg}{\mathsf{Reg}}
\newcommand{\mto}{\rightarrowtail}
\newcommand{\eto}{\twoheadrightarrow}



\authorrunning{R.~Biondo, D.~Castelnovo, F.~Gadducci}


\begin{document}

	\maketitle \todo{Scegliere un titolo vero e mettere i ringraziamenti}
	\begin{abstract}
\todo{a very nice abstract}
	\end{abstract}


\section{Introduction}
\todo{A very nice introduction}
\section{$\mathcal{M}$-adhesive categories}

This  section recalls the basic theory of \emph{$\mathcal{M}$-adhesive categories} \cite{azzi2019essence,ehrig2012,ehrig2014adhesive,lack2005adhesive,heindel2009category}. Given a category $\X$ we will not distinguish notationally between $\X$ and its class of objects, so
``$X\in \X$'' means that $X$ is an object of $\X$. We let $\mor(\X)$, $\mon(\X)$ and $\reg(\X)$ denote the class of all arrows, monos and regular monos of $\X$, respectively.  Given an integer $n\in \mathbb{Z}$, $[0,n]$ denotes the set of natural numbers less than or equal to $n$; in particular, $[0,n]=\emptyset$ if $n<0$.

\subsection{$\mathcal{M}$-adhesivity}\label{subsec:ade}
The key property of $\mathcal{M}$-adhesive categories is the \emph{Van Kampen condition}~\cite{brown1997van,johnstone2007quasitoposes,lack2005adhesive}. In order to define it we need to introduce some terminology.  Let  $\X$ be a category. A subclass $\mathcal{A}$ of
$\mor(\X)$ is called
\begin{itemize}
	\parbox{11cm}{\item
		\emph{stable under pushouts (pullbacks)} if for every pushout (pullback) square as the one on the right, 	if $m \in \mathcal{A}$ ($n\in \mathcal{A}$) then $n \in \mathcal{A}$ ($m \in \mathcal{A}$);
		\item \emph{closed under composition} if $h, k\in \mathcal{A}$ implies $h\circ k\in \mathcal{A}$ whenever $h$ and $k$ are composable.{\tiny }}\hfill
	\parbox{2cm}{$\xymatrix{A\ar[r]^f  \ar[d]_{m}& B \ar[d]^n \\ C \ar[r]_g & D}$}
	\parbox{11cm}{}\hfill
\end{itemize}

\begin{definition}[Van Kampen property]
	Let $\X$ be a category and consider the diagram\\
	\parbox{10cm}{
		aside.
		Given  a class of arrows $\mathcal{A}\subseteq \mor(\X)$, we say that the bottom square
		is an \emph{$\mathcal{A}$-Van Kampen square} if
		\begin{enumerate}
			\item it is a pushout square;
			\item 	whenever the cube above has pullbacks as back and left faces and the vertical arrows belong to $\mathcal{A}$, then its top face is a pushout 
			if and only if the front and right faces are pullbacks.
	\end{enumerate}}
	\parbox{2cm}{$\xymatrix@C=10pt@R=6pt{&A'\ar[dd]|\hole_(.65){a}\ar[rr]^{f'} \ar[dl]_{m'} && B' \ar[dd]^{b} \ar[dl]_{n'} \\ C'  \ar[dd]_{c}\ar[rr]^(.7){g'} & & D' \ar[dd]_(.3){d}\\&A\ar[rr]|\hole^(.65){f} \ar[dl]^{m} && B \ar[dl]^{n} \\C \ar[rr]_{g} & & D }$ }\\
	Pushout squares that enjoy only the ``if'' half of item (2) above are called \emph{$\mathcal{A}$-stable}.
	
	A $\mor(\X)$-Van Kampen square is called  \emph{Van
		Kampen} and a $\mor(\X)$-stable square  \emph{stable}.
\end{definition}

We can now define $\mathcal{M}$-adhesive categories.

\begin{definition}[$\mathcal{M}$-adhesive category]
	Let $\X$ be a category and $\mathcal{M}$ a subclass of
	$\mon(\X)$  including  all isomorphisms, closed under composition,  and stable under pullbacks and pushouts.  The category  $\X$ is said to be \emph{$\mathcal{M}$-adhesive} if
	\begin{enumerate}
		\item it has \emph{$\mathcal{M}$-pullbacks}, i.e.~pullbacks along arrows of $\mathcal{M}$;
		\item it has \emph{$\mathcal{M}$-pushouts}, i.e.~pushouts along arrows of $\mathcal{M}$;
		\item  $\mathcal{M}$-pushouts are $\mathcal{M}$-Van Kampen squares.
	\end{enumerate}
	
	A category $\X$ is said to be \emph{strictly $\mathcal{M}$-adhesive}
	if $\mathcal{M}$-pushouts are Van Kampen squares.
\end{definition}

We write $m\colon X\rightarrowtail Y$ to denote that an arrow $m\colon X\to Y$ belongs to $\mathcal{M}$.

\begin{remark}
	\label{rem:salva}
	\emph{Adhesivity} and \emph{quasiadhesivity} 
	\cite{lack2005adhesive,garner2012axioms} coincide with strict
	$\mon(\X) $-adhesivity and strict $\reg(\X)$-adhesivity,
	respectively.
\end{remark}


$\mathcal{M}$-adhesivity is well-behaved with respect to  the construction of slice and functor categories \cite{mac2013categories}, as shown by the following theorems~\cite{ehrig2006fundamentals,lack2005adhesive}.

\begin{theorem}\todo{controllare meglio}
	\label{thm:slice-functors}
	Let $\X$ be an $\mathcal{M}$-adhesive category. Given an object $X$
	the category $\X/X$ is $\mathcal{M}/X$-adhesive with
	$\mathcal{M}/X:=\{m\in \mor(\X/X) \mid m\in
	\mathcal{M}\}$. Similarly, $X/\X$ is $X/\mathcal{M}$-adhesive with
	$X/\mathcal{M}:=\{m\in \mor(X/\X) \mid m\in \mathcal{M}\}$.
	
	Moreover for every small category $\Y$, the category $\X^\Y$ of
	functors $\Y\to \X$ is $\mathcal{M}^{\Y}$-adhesive, where
	$\mathcal{M}^{\Y}:=\{\eta \in \mor(\X^\Y) \mid \eta_Y \in
	\mathcal{M} \text{ for every } Y\in \Y\}$.
\end{theorem}

We can list various examples of $\mathcal{M}$-adhesive categories (see
\cite{castelnovo2023thesis,CastelnovoGM22,lack2006toposes}).


\begin{example}
	\label{ex:adhesive}
	$\cat{Set}$ is adhesive, and, more generally, every topos is
	adhesive~\cite{lack2006toposes}. By the closure properties above, every presheaf $[\cat{X},\cat{Set}]$ is adhesive, thus the category
	$\cat{Graph} = [ E \rightrightarrows V, \cat{Set}]$ is adhesive
	where $E \rightrightarrows {V}$ is the two objects category with two
	morphisms $s,t \colon{E} \to {V}$. Similarly, various
	categories of hypergraphs can be shown to be adhesive, such as term
	graphs and hierarchical graphs~\cite{CastelnovoGM24}. Note that the category $\cat{sGraphs}$ of simple graphs, 
	i.e.~graphs without parallel edges, is
	$\reg{(\cat{sGraphs})}$-adhesive~\cite{BehrHK23} but not
	quasiadhesive.
\end{example}


\todo{Inserire parte della tesi su kp}

\subsection{Kernel Pairs and Regular Epimorphisms}

\begin{definition}[Kernel Pair]
    A \emph{kernel pair} for an arrow $f: A \to B$ is an object $K_f$ together with two arrows $\pi^1_f, \pi^2_f : K_f \to A$, denoted as $(K_f, \pi^1_f, \pi^2_f)$, such that the following square is a pullback.
    \[
        \begin{tikzcd}
            K_f \ar[r, "{\pi^1_f}"] \ar[d, "{\pi^2_f}" swap] & A \arrow[d, "f"] \\
            A \ar[r, "f"swap] & B
        \end{tikzcd}
    \]
\end{definition}

\begin{remark}
	If a category $\cat{C}$ has pullbacks then every arrow has a kernel pair.
\end{remark}

\begin{remark}
    Since a kernel pair is nothing more that a pullback, that is, a limit, by \Cref{rem:limits_are_unique_up_to_isomorphisms}, it make sense to refer to it as \emph{the} kernel pair for a morphism $f$.
\end{remark}

\begin{example}\label{ex:kernel_pairs_in_Set}
    In $\Set$, a kernel pair for a function $f: A\to B$ is the set
    \[
        K_f=\{(x, y) \in A \times A \mid f(x) = f(y)\}
    \]
    together with the canonical projection on the first and the second component of the pairs.
\end{example}

\begin{prop}\label{prop:pairng_of_kernel_pairs_mono}
    Let $(K, p_1, p_2)$ be the kernel pair of $f: X \to Y$, and let $(X\times X, \pi_1, \pi_2)$ be the product of $X$ with itself. Then, the mediating arrow $\langle p_1, p_2\rangle : K \to X \times X$ is mono.
\end{prop}

\begin{proof}
	Suppose $\langle p_1, p_2 \rangle \circ f = \langle p_1, p_2 \rangle \circ g$ for $f, g: Z \to K$. Then, we have
	\[
		\begin{split}
			\langle p_1, p_2 \rangle \circ f &=  \langle p_1, p_2 \rangle \circ g \\
			\pi_1 \circ \langle p_1, p_2 \rangle \circ f &=  \pi_1 \circ \langle p_1, p_2 \rangle \circ g \\
			p_1 \circ f &= p_1 \circ g
		\end{split}
		\qquad
		\begin{split}
			\langle p_1, p_2 \rangle \circ f &=  \langle p_1, p_2 \rangle \circ g \\
			\pi_2 \circ \langle p_1, p_2 \rangle \circ f &=  \pi_2 \circ \langle p_1, p_2 \rangle \circ g \\
			p_2 \circ f &= p_2 \circ g
		\end{split}
	\]
	Thus, from the universal property of the pullback, $f = g$.
\end{proof}

\begin{prop}\label{prop:kermono}
	An arrow $m\colon M\to X$ is mono if and only if $(M, id_M, id_M)$ is a kernel pair for it.
\end{prop}

\begin{proof}
    To prove the ``if'' part of the statement, let $f, g: A \to M$ be such that $m\circ f = m\circ g$, and consider the following situation.
    \[
        \begin{tikzcd}[row sep=26, column sep = 26]
        A \arrow[drr, bend left=30, "f"] \arrow[ddr, bend right=30,"g" swap] \arrow[dr, dashed, "u"] & & \\
        & M  \arrow[d, "{id_M}"] \arrow[r, "{id_M}"] & M \arrow[d, "m"] \\
        & M  \arrow[r, "m" swap] & X
        \end{tikzcd}
    \]
    For the universal property of pullbacks, we have that $$f  =  id_M \circ u =  g$$
    Hence, $m$ is mono.

    Conversely, if $m$ is mono, then, we have that
    \begin{align*}
        m \circ f = m \circ g   &\Rightarrow    f = g \\
                                &\Rightarrow    f \circ id_M = g\circ id_M
    \end{align*}
    Hence, $f$ is the unique arrow that makes the commutative square illustrated above a pushout.
\end{proof}

\begin{remark}\label{rem:monos_in_presh_cats}
    From characterization of monos via pullbacks in \Cref{prop:kermono} and \Cref{lemma:limits_of_presheaves}, we have that a mono in a category of presheaves is a natural transformation of which each component is mono.
\end{remark}

\begin{cor}\label{cor:kermono}
	$(K_f, \pi_f^1, \pi_f^2)$ is a kernel pair for $f\colon X\to Y$ if and only if, for each mono $m\colon Y\to Z$, $(K_f, \pi_f^1, \pi_f^2)$ is a kernel pair also for $m\circ f$.
\end{cor}
\begin{proof}
    It is enough to see that, by \Cref{lemma:pullback_lemma} and \Cref{prop:kermono} the outer boundary of the following square is a pullback.
        \[\begin{tikzcd}[row sep=13 pt, column sep=13 pt]
    	{K_f} && X && X \\
    	\\
    	X && Y && Y \\
    	\\
    	X && Y && Z
    	\arrow["{\pi_f^2}", from=1-1, to=1-3]
    	\arrow["{\pi_f^2}"', from=1-1, to=3-1]
    	\arrow["{id_X}", from=1-3, to=1-5]
    	\arrow["f", from=1-3, to=3-3]
    	\arrow["f", from=1-5, to=3-5]
    	\arrow["f", from=3-1, to=3-3]
    	\arrow["{id_X}"', from=3-1, to=5-1]
    	\arrow["{id_Y}", from=3-3, to=3-5]
    	\arrow["{id_Y}", from=3-3, to=5-3]
    	\arrow["m", from=3-5, to=5-5]
    	\arrow["f"', from=5-1, to=5-3]
    	\arrow["m"', from=5-3, to=5-5]
    \end{tikzcd}\]
    The leftward part of the statement follows by definition of monomorphism an \Cref{lemma:pullback_lemma}.
\end{proof}

\begin{lemma}\label{lemma:kern_pairs_pres_pullbacks}
    Suppose the following situation, and that $f: X \to Y$ and $g: Z \to W$ have kernel pairs.
    \[
        \begin{tikzcd}
            X \ar[r, "h"] \ar[d, "f"swap] & Z \ar[d, "g"] \\
            Y \ar[r, "t"swap] & W
        \end{tikzcd}
    \]
    
    Then, there exists a unique arrow $k_h: K_f \to K_g$ making the squares below commute.
    \[
        \begin{tikzcd}[row sep = 25 pt, column sep= 25 pt]
            K_f \ar[r, dashed, "{k_h}"] \ar[d, "{\pi_f^1}"swap] & K_g \ar[d, "{\pi_g^1}"] \\
            X \ar[r, "h"swap] & Z 
        \end{tikzcd}
        \qquad
        \begin{tikzcd}[row sep = 25 pt, column sep= 25 pt]
            K_f \ar[r, dashed, "{k_h}"] \ar[d, "{\pi_f^2}"swap] & K_g \ar[d, "{\pi_g^2}"] \\
            X \ar[r, "h"swap] & Z 
        \end{tikzcd}
    \]

    Moreover, if the beginning square is a pullback, then also the preceding ones are so.
\end{lemma}

\begin{proof}
    Computing, we have
    \begin{align*}
        g \circ h \circ \pi_f^1 &=  t \circ f \circ \pi_f^1     \\
                                &=  t \circ f \circ \pi_f^2     \\
                                &=  g \circ h \circ \pi_f^2
    \end{align*}
    By the universal property of $K_g$ as the pullback of $g$ along itself, such $k_h$ exists and it is unique.

    To prove the second half of the thesis, let us consider the two rectangles below, which, by \Cref{lemma:pullback_lemma} are pullbacks.
    \[\begin{tikzcd}[row sep= 25 pt, column sep= 25 pt]
	{K_f} & X & Z \\
	X & Y & W
	\arrow["{\pi_f^1}", from=1-1, to=1-2]
	\arrow["{\pi_f^2}"', from=1-1, to=2-1]
	\arrow["h", from=1-2, to=1-3]
	\arrow["f", from=1-2, to=2-2]
	\arrow["g", from=1-3, to=2-3]
	\arrow["f"', from=2-1, to=2-2]
	\arrow["t"', from=2-2, to=2-3]
    \end{tikzcd}
    \qquad
    \begin{tikzcd}[row sep = 25 pt, column sep=25 pt]
	{K_f} & X & Z \\
	X & Y & W
	\arrow["{\pi_f^2}", from=1-1, to=1-2]
	\arrow["{\pi_f^1}"', from=1-1, to=2-1]
	\arrow["h", from=1-2, to=1-3]
	\arrow["f", from=1-2, to=2-2]
	\arrow["g", from=1-3, to=2-3]
	\arrow["f"', from=2-1, to=2-2]
	\arrow["t"', from=2-2, to=2-3]
    \end{tikzcd}
    \]
    But then the following ones are pullbacks too.
    \[\begin{tikzcd}[row sep= 25 pt, column sep= 25 pt]
	{K_f} & {K_g} & Z \\
	X & Y & W
	\arrow["{k_h}"', from=1-1, to=1-2]
	\arrow["{\pi_f^1}"', from=1-1, to=2-1]
	\arrow["{\pi_g^2}"', from=1-2, to=1-3]
	\arrow["{\pi_g^1}", from=1-2, to=2-2]
	\arrow["g", from=1-3, to=2-3]
	\arrow["h", from=2-1, to=2-2]
	\arrow["g", from=2-2, to=2-3]
        \arrow["{h\circ \pi_f^2}", from=1-1, to=1-3, bend left = 30]
        \arrow["{t \circ f}"', from=2-1, to=2-3, bend right = 30]
    \end{tikzcd}
    \qquad
    \begin{tikzcd}[row sep = 25 pt, column sep=25 pt]
	{K_f} & {K_g} & Z \\
	X & Y & W
	\arrow["{k_h}"', from=1-1, to=1-2]
	\arrow["{\pi_f^2}"', from=1-1, to=2-1]
	\arrow["{\pi_g^1}"', from=1-2, to=1-3]
	\arrow["{\pi_g^2}", from=1-2, to=2-2]
	\arrow["g", from=1-3, to=2-3]
	\arrow["h", from=2-1, to=2-2]
	\arrow["g", from=2-2, to=2-3]
        \arrow["{h\circ \pi_f^1}", from=1-1, to=1-3, bend left = 30]
        \arrow["{t \circ f}"', from=2-1, to=2-3, bend right = 30]
    \end{tikzcd}
    \]

    The thesis follows again by \Cref{lemma:pullback_lemma}.
\end{proof}

\begin{prop}\label{prop:reg_epi_coeq}
    Let $e: X \to Y$ be a regular epimorphism in a category $\cat C$ with a kernel pair $(K, \pi_1, \pi_2)$. Then, $e$ is the coequalizer of $\pi_1$ and $\pi_2$.
\end{prop}

\begin{proof}
    By hypothesis, there exists a pair $f, g: Z \to X$ of which $e$ is the coequalizer. Since $e \circ f = e \circ g$, we have
    \[
        \begin{tikzcd}[row sep= 20, column sep = 13]
            Z \ar[drr, "f", bend left=30] \ar[ddr, "g"swap, bend right=30] \ar[dr, dashed, "k"] & & \\
            & K \ar[r, "{\pi_1}"] \ar[d, "{\pi_2}" swap] & X \ar[d, "e"] \\
            & X \ar[r, "e"swap] & Y
        \end{tikzcd}
    \]
    thus there exists the unique $k: Z \to K$. Let now $h: Z \to V$ be an arrow such that $h \circ \pi_1 = h \circ \pi_2$, then
    \begin{align*}
        h \circ f &= h \circ \pi_1 \circ k \\
                  &= h \circ \pi_2 \circ k \\
                  &= h \circ g
    \end{align*}
    and thus there exists a unique $l: Y \to V$ such that $l \circ e = h$.
\end{proof}

\begin{cor}\label{cor:reg_epi_components_reg_epi_nat_trans}
    Let $\cat C$ be a category with pullbacks and $\phi : D \dot\to D'$ be a natural transformation between two functors $D, D': \cat{I \to C}$. If $\phi_i$ is a regular epi for every $i$, then $\phi$ is a regular epi.
\end{cor}

\begin{proof}
    Let $(K_i, \pi_i^1, \pi_i^2)$ be the kernel pair of $\phi_i$ for each $i$. Given an arrow $\alpha: i \to j$ of $\cat I$, we have
    \begin{align*}
        \phi_j \circ D(\alpha) \circ \pi_i^1 &= D'(\alpha) \circ \phi_i \circ \pi_i^1 \\
                                             &= D'(\alpha) \circ \phi_i \circ \pi_i^2 \\
                                             &= \phi_j \circ D(\alpha) \circ \pi_i^2
    \end{align*}
    Thus, the outer boundary of the diagram below commutes, yielding the arrow $K(\alpha)$
    \[
        \begin{tikzcd}[row sep=26, column sep=26]
            {K_i} \ar[r, "{\pi_i^1}"] \ar[d, "{\pi_i^2}"swap] \ar[dr, dashed, "{K(\alpha)}"] & D(i) \ar[dr, "{D(\alpha)}"] \\
            D(i) \ar[dr, "{D(\alpha)}"swap] & K_j \ar[r, "{\pi_j^1}"] \ar[d, "{\pi_j^2}"swap] & D(j) \ar[d, "{\phi_j}"] \\
            & D(j) \ar[r, "{\phi_j}"swap] & D'(j)
        \end{tikzcd}
    \]

    In this way, we get a functor $K: \cat{I \to C}$, 
    which maps each $i$ onto $K_i$ and each arrow $\alpha$ onto $K(\alpha)$. 
	We have in fact $K(id_i) : K_i \to K_i$ is the arrow such that
        \[
            \begin{split}
                D(id_i) \circ \pi_i^1 &= \pi_i^1 \circ K(id_i) \\
                \pi_i^1 &= \pi_i^1 \circ K(id_i)
            \end{split}
                \qquad
            \begin{split}
                D(id_i) \circ \pi_i^2 &= \pi_i^2 \circ K(id_i) \\
                \pi_i^2 &= \pi_i^2 \circ K(id_i)
            \end{split}
        \]
	Thus, for the universal property of pullbacks, $K(id_i) = id_{K_i}$.

        Suppose now $\alpha : i \to j$ and $\beta: j \to k$. Computing, we have
        \[
            \begin{split}
                \pi_k^1 \circ K(\beta \circ \alpha) &= D(\beta \circ \alpha) \circ \pi_i^1 \\
                                                    &= D(\beta) \circ D(\alpha) \circ \pi_i^1 \\
                                                    &= D(\beta) \circ \pi_j^1 \circ K(\alpha) \\
                                                    &= \pi_k^1 \circ K(\beta) \circ K(\alpha)
            \end{split} \qquad
            \begin{split}
                \pi_k^2 \circ K(\beta \circ \alpha) &= D(\beta \circ \alpha) \circ \pi_i^2 \\
                                                    &= D(\beta) \circ D(\alpha) \circ \pi_i^2 \\
                                                    &= D(\beta) \circ \pi_j^2 \circ K(\alpha) \\
                                                    &= \pi_k^2 \circ K(\beta) \circ K(\alpha)
            \end{split}
        \]
        Again, for universal property of pullbacks, necessarily we have $K(\beta \circ \alpha) = K(\beta) \circ K(\alpha)$, proving functoriality of $K$.
   
    
     Hence, we have two natural transformations $\pi^1, \pi^2 : E \dot\to D$. By \Cref{prop:reg_epi_coeq}, every component $\phi_i$ is the coequalizer of $\pi_i^1, \pi_i^2: E \to D$, and so $\phi$ is the coequalizer of $\pi^1$ and $\pi^2$.
\end{proof}

\begin{lemma}\label{lemma:nat_trans_reg_epi_canonical_arrow_reg_epi}
    Let $D, D': \cat{I \to C}$ be two diagrams, and let $((c_i)_{i \in \cat I}, C)$ and $((c_i')_{i\in \cat I}, C')$ be, respectively, the colimit of $D$ and $D'$. If $\cat C$ has all colimits, for diagrams of shape $\cat I$ and $\phi: D \dot\to D'$ is a natural transformation in which all components are regular epimorphisms, then, the arrow induced by $\phi$ from $C$ to $C'$ (\Cref{prop:nat_tran_induces_unique_arrow_between_colimits}) is a regular epimorphism too.
\end{lemma}

\begin{proof}
    By \Cref{cor:reg_epi_components_reg_epi_nat_trans}, we know that $\phi: D \dot\to D'$ is a regular epimorphism, so that there is a functor $E: \cat{I \to C}$ and $\eta, \theta: E \dot\to D$ such that $\phi$ is the coequalizer of $\eta$ and $\theta$. Let now $((p_i)_{i \in \cat I}, P)$ be the colimit of $E$, by \Cref{prop:nat_tran_induces_unique_arrow_between_colimits}, we have $a, b: P \to C$ fitting in the diagram below.
    \[
        \begin{tikzcd}[row sep = 26, column sep = 26]
            E(i) \ar[r, "{p_i}"] \ar[d, "{\eta_i}"swap] & P \ar[d, dashed, "a"] \\
            D(i) \ar[r, "{c_i}"swap] & C
        \end{tikzcd}
        \qquad
        \begin{tikzcd}[row sep = 26, column sep = 26]
            E(i) \ar[r, "{p_i}"] \ar[d, "{\theta_i}"swap] & P \ar[d, dashed, "b"] \\
            D(i) \ar[r, "{c_i}"swap] & C
        \end{tikzcd}
    \]

    We want to show that $c$ coequalizes $\eta$ and $\theta$. Let thus $t: C \to T$ be an arrow such that $t \circ a = t \circ b$. Then, for every $i$, we have
    \begin{align*}
        t \circ c_i \circ \eta_i &= t \circ a \circ p_i \\
                                 &= t \circ b \circ p_i \\
                                 &= t \circ c_i \circ \theta_i
    \end{align*}    

    Thus, there is $t_i: D(i) \to T$ such that $t\circ c_i = t_i \circ \phi_i$. It is now easy to see that $((t_i)_{i \in \cat I}, T)$ is a cocone of $D'$: suppose $\alpha: i \to j$ be an arrow of $\cat I$, obtaining
    \begin{align*}
        t_i \circ \phi_i &= t \circ c_i \\
                         &= t \circ c_j \circ D(\alpha) \\
                         &= t_j \circ \phi_j \circ D(\alpha) \\
                         &= t_j \circ D'(\alpha) \circ \phi_i
    \end{align*}
    By the hypothesis that $\phi_i$ is regular epi for each $i$, therefore epi (by the dual of \Cref{prop:eq_are_mono}), we can conclude $t_i = t_j \circ D'(\alpha)$.
    
    Hence, we have an arrow $k: C' \to T$ such that $k \circ c_i' = t_i$. But then
    \begin{align*}
        c \circ c \circ c_i &= k \circ c_i' \circ \phi_i \\
                            &= t_i \circ \phi \\
                            &= t \circ c_i
    \end{align*}
    Showing that $k \circ c = t$.

    For the uniqueness, let $k': C' \to T$ be another arrow such that $k' \circ c = t$. Then we have
    \begin{align*}
        k' \circ c_i' \circ \phi_i &= k' \circ c \circ c_i \\
                                   &= t \circ c_i \\
                                   &= t_i \circ \phi_i
    \end{align*}    
    Since $\phi_i$ is a regular epimorphism, we have $k' \circ c_i' = t_i$, and, because $k \circ c_i' = t_i$ by construction, we can conclude that $k'=k$ since $((c_i')_{i \in \cat I}, C')$ is a colimit.
\end{proof}

\subsection{Factorization systems}
\todo{Inserire qualche cosa su sist. di fattorizazzione e M-adesività - devo pensare a come mettere la questione senza prendere troppo spazio}


\section{Graphs with equivalences}

\subsection{Graph signatures}
\todo{Inserire def di graph signature}

\todo{Grafi su una segnatura sono adesivi (grafi)}

\todo{Inserire equivalenze (stesso discorso della tesi)}


\subsection{E-graphs}

\todo{Copincollare dalla tesi senza grandi modifiche}

\section{Conclusions and further works}
\todo{Some very nice conclusions}
\bibliography{biblio}

\appendix
\section{Comparison with Ghica}
\todo{Lo vogliamo fare?}


\end{document}

