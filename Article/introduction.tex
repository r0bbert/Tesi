\chapter*{Introduction}

This work discusses the adhesivity of e-graphs, a data structure introduced in \cite{???}\todo{citations}, giving them a structure of category and then working on its algebraic properties. 

The first chapter is about basic concepts of category theory, such as categories, limits, constructions and adhesivity, that will be used to model the categories on which in the second chapter we will work on.
The second chapter is entirely focused on graphs, starting from a general notion of graphs, that is, a two sorted algebra consisting of a set of edges and a set of vertices and two operators, 
source and target, which assign to each edge a source vertex and a target one. The structure we will use to model graphs are presheaves, which not only provides a straightforward formalization of them but also makes in evidence particular properties proper of graphs. Then, we endow graphs with an equivalence relation among verices, to end up with an adhesivity result. The last part is about e-graphs, that are nothing but graphs with a certain type of equivalence relation, inheriting then almost all the properties of the category of graphs with equivalences.
