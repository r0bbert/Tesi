\chapter*{Conclusions}
\addcontentsline{toc}{chapter}{Conclusions}

\iffalse
In questo caso, una introduzione meno brusca, ricapitolando un po' la struttura e soprattutto il fine della tesi

\begin{itemize}
\item la tesi ha investigato le proprieta' di adesivita' di egss, ottenendo che ...
\item questo e' interessante perche' permette di sfruttare il formalismo. Ad esempio, per parlare di update concorrenti, al momento non considerati per eggs
\item i conti sono stati fatti su Set, ma senza sfruttare l;e proprieta' della categoria. Questo permettera' strutture piu' generali come term graphs, grafi gerarchici etc
\item altri modelli fromali di9 eggs non nen conosciamo, ma alemno con quello di Gjuica faremo un confronto, anche se e' molto piu' copmplesso del nostro.
\end{itemize}

The result we ended up with is then $\Reg(\EGG))$-adhesivity, where $\EGG$ is a category of functors which have $\Set$ as codomain category.
The main question is now, is it strictly necessary to have $\Set$ as codomain of this functors? The entire work is developed in a more general perspective, using the least possible concept directly connected with set theory, using instead categorial constructions. The property of $\Set$ we have made large use of is the equivalence between quotient and product subobjects, which holds in exact categories.
Hence, this work can be viewed as a starting point to a more general study on e-graphs, in which edges, nodes and equivalence classes can be not only sets but objects of any exact category.
\fi



In this work, we have defined graphs with equivalences and e-graphs in a more abstract manner than \cite{egg}, independent of specific implementations. This allowe us  to analyze the categorical properties of these structures. Building upon this framework, we have property, in particular, the adhesivivty of graphs with equivalences and e-graphs with respect to the class of regular monomorphisms. This result is potentially useful for advancing state-of-the-art techniques, such as the DPO approach for concurrent updates to the structure, which has not yet been explored.



Both graphs with equivalences and e-graphs can be realized as functors with values in $\Set$. One might ask whether it is strictly necessary to use $\Set$ as the codomain of these functors. Our framework is developed from a more general perspective, using minimal set-theoretic concepts and instead relying on categorical constructions. It only requires $\mathcal{M}$-adhesivity and exactness  \cite{barrexact}. This propert certainly hold in $\Set$ (and more generally every elementary topos \cite{maclane2012sheaves}). Roughly speaking, this is  the property that enabled the equivalence noted in \Cref{rem:eqgrph_set_eq}. Therefore, a potential direction for future work will be to provide a more general presentation of these structures, replacing $\Set$ with any adhesive and exact category.