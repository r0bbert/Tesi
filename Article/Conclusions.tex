\chapter*{Conclusions}
\addcontentsline{toc}{chapter}{Conclusions}

\iffalse
In questo caso, una introduzione meno brusca, ricapitolando un po' la struttura e soprattutto il fine della tesi

\begin{itemize}
\item la tesi ha investigato le proprieta' di adesivita' di egss, ottenendo che ...
\item questo e' interessante perche' permette di sfruttare il formalismo. Ad esempio, per parlare di update concorrenti, al momento non considerati per eggs
\item i conti sono stati fatti su Set, ma senza sfruttare l;e proprieta' della categoria. Questo permettera' strutture piu' generali come term graphs, grafi gerarchici etc
\item altri modelli fromali di9 eggs non nen conosciamo, ma alemno con quello di Gjuica faremo un confronto, anche se e' molto piu' copmplesso del nostro.
\end{itemize}

The result we ended up with is then $\Reg(\EGG))$-adhesivity, where $\EGG$ is a category of functors which have $\Set$ as codomain category.
The main question is now, is it strictly necessary to have $\Set$ as codomain of this functors? The entire work is developed in a more general perspective, using the least possible concept directly connected with set theory, using instead categorial constructions. The property of $\Set$ we have made large use of is the equivalence between quotient and product subobjects, which holds in exact categories.
Hence, this work can be viewed as a starting point to a more general study on e-graphs, in which edges, nodes and equivalence classes can be not only sets but objects of any exact category.
\fi


Retracing the work, we have defined graphs with equivalence and e-graphs in a more abstract way, untied from potential implementations, in order to analyze formal properties of this kind of structures.

Then, based upon this framework, we proved adhesive properties of graphs with equivalences and e-graphs, obtaining an interesting result of adhesivity
(both of theses structure are adhesive with respect to the class of their regular monomorphisms)
useful for a potential exploiting in state-of-the-art techniques, such as the DPO approach for concurrent updates to the structure, currently not explored yet.

Both graphs with equivalence and e-graphs are considered in this work as functors with $\Set$ as codomain category.
One may ask then, is it strictly necessary to have $\Set$ as codomain of this functors?
The entire work is developed in a more general perspective, using the least possible concept directly connected with set theory, using instead categorial construction, and requiring only $\mathcal{M}$-adhesivity and exactness (this last concept is quite implicit in $\Set$, hence is not explicitly mentioned, but is, intuitively, the one that permitted the equivalence noted in \Cref{rem:eqgrph_set_eq}).
Hence, a future work will be a more general presentation of this structures, substituting $\Set$ with any category with this properties.
