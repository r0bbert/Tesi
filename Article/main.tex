\documentclass[a4paper, twoside,openright]{report}
\usepackage[T1]{fontenc} % Font encoding, T1 = it
\usepackage[utf8]{inputenc} % Input encoding - per caratteri particolari
\usepackage[english]{babel} % Lingua principale inglese
\usepackage{graphicx} % Per includere immagini esterne
\usepackage[tickmarkheight=.5em,textwidth=\marginparwidth,textsize=small]{todonotes}
\usepackage[top=3cm,bottom=3cm,left=2cm,right=3cm]{geometry} %impaginazione e margini documento -- DA POROBLEMI!!!
\usepackage[fontsize=12pt]{scrextend} %dimensione font

\usepackage{titlesec} %indice
\usepackage[all]{xy}


\usepackage{amssymb,amsthm, amsmath, mathrsfs, dsfont} %simboli matematici
\usepackage{hyperref, cleveref}
\raggedbottom % Se la pagina non è completa, lascia lo spazio alla fine
\pagestyle{headings}

\theoremstyle{plain}
\newtheorem{theorem}{Theorem}[section]
\newtheorem{prop}[theorem]{Proposition}
\newtheorem{lemma}[theorem]{Lemma}
\newtheorem{cor}[theorem]{Corollary}

\theoremstyle{definition}
\newtheorem{definition}[theorem]{Definition}
\newtheorem{example}[theorem]{Example}
\newtheorem{remark}[theorem]{Remark}
\newtheorem{obs}[theorem]{Observation}

\DeclareMathAlphabet{\mymathbb}{U}{BOONDOX-ds}{m}{n}

% General Setting diagrams
\usepackage{tikz-cd} %diagrammi
\tikzcdset{row sep/normal=5em}
\tikzcdset{column sep/normal=5em}
\tikzcdset{every label/.append style = {font = \small}}

\begin{document}

\newcommand{\cat}[1]{\mathscr{#1}}
\newcommand{\Ob}{\mathcal{O}b}
\newcommand{\Hom}{\mathcal{H}om}
\newcommand{\Set}{\mathbf{Set}}
\newcommand{\Reg}{\mathcal{R}eg}
\newcommand{\Mono}{\mathcal{M}ono}
\newcommand{\initial}{\mymathbb{0}}
\newcommand{\terminal}{\mathds{1}}
\newcommand{\eqgraph}[1]{\mathbb{#1}}
\newcommand{\Egg}{\mathbf{EGG}}

% input file frontespizio.tex
% \input{frontespizio}

\tableofcontents

% \chapter*{Introduction}
% [TODO]

\chapter{Background}\label{chap:one}

In this chapter the building blocks for this work, almost entirely based on categories, will be defined.
The aim of what follows is not only to introduce concepts that will be used later, but also to understand how category theory is general enough to give the abstraction of known notions (mainly from set theory) to reuse them in different contexts. This is not a complete tutorial on categories, but instead a sufficient compendium of definitions to make clear what will be done in the next chapters.


\section{Basic Notions}\label{sect:basic_nots}

%The very fist concept to get familiar with is what is a category. We can see a category as a general construction that can provide any level of abstraction needed. In fact, the definition is not about what the category represents, but instead a more general scheme that \emph{may} represent something. The definition is about what a category is composed by, and how the objects in it interact between themselves.

This section is all about basic definitions and examples, to get familiar with the formalism of categories.

\subsection{Categories}\label{ssect:cats}

\begin{definition}[Category]\label{def:category}
    A \emph{category} $\cat{C}$ comprises:
    \begin{enumerate}
        \item A collection of \emph{objects} $\Ob(\cat{C})$;
        \item A collection of \emph{arrows} (or \emph{morphisms}) $\Hom(\cat{C})$, often called \emph{homset}.
    \end{enumerate}
    Two operators, $dom$ and $cod$, that map every morphism \-$f \in \Hom(\cat{C})$ to two objects, respectively, its \emph{domain} and its \emph{codomain}. In case $dom\ f = A$ and $cod\ f = B$, we will write $f: A \rightarrow B$. The collection of morphisms from an object $A$ to an object $B$ is denoted as $\cat{C}(A, B)$.
    An operator $\circ$ of \emph{composition} maps every couple of morphisms $f$, $g$ with $cod\ f = dom \ g$ (in this case $f$ and $g$ are said to be composable) to a morphism $g \circ f : dom\ f \rightarrow cod \ g$. The composition operator is associative, i.e., for each composable arrows $f$, $g$ and $h$, it holds that
    $$
        h \circ (g \circ f) = (h\circ g) \circ f
    $$
    For each object $A$, an \emph{identity} morphism $id_A : A \rightarrow A$ (or, when it is clear from the context, just denoted $A$)  such that, for each $f: A \rightarrow B$:
    \[
        id_B \circ f = f = f \circ id_A 
    \]
\end{definition}

The most important thing here is not the structure of the objects, but instead how this structure is preserved by the morphisms.

\begin{example}\label{ex:0_cat}
    A trivial example of category is the one with no objects, and hence no morphisms. Such category is denoted by $\textbf 0$ and is called \emph{empty category}.
\end{example}

\begin{example}\label{ex:1_cat}
    The category with just one object and just one arrow, the identity arrow on that object, is denoted $\textbf 1$. In particular, the only object of this category is $\bullet$, and the only arrow is $id_{\bullet}$.
\end{example}

Given an arrow $f A \rightarrow B$ in a category $\cat C$, we say that $f$ \emph{factors through} $g: C \rightarrow B$ if there exists an arrow $h: A \rightarrow C$ such that $f = h \circ g$.

\begin{definition}\label{def:dual_cat}[Dual Category]
    Given a category $\cat C$, there exist a category $\cat C^{op}$ such that:
    \begin{itemize}
        \item $Ob(\cat C^{op}) = Ob(\cat C)$;
        \item if $f: A \rightarrow B$ is a morphism in $\cat C$, then $f: B\rightarrow A$ is a morphism in $\cat C ^ {op}$.
    \end{itemize}
    Hence, given $f : A \rightarrow B$ and $g: B \rightarrow C$ arrows in $\cat C$, as $g \circ f: A \rightarrow C$ is an arrow in $\cat C$, then $f \circ g: C \rightarrow A$ is an arrow in $\cat C ^{op}$.
    Such category is called \emph{dual category} or \emph{opposite category}.
\end{definition}

Duality is a concept that we will encounter most of the time. Given a property $P$ valid for a category $\cat C$, we will refer to the same property in the opposite category $\cat C^{op}$ as the \emph{dual} of $P$, without explicitly constructing $\cat C^{op}$. There exist some properties that coincide exactly with their dual, and such properties are said to be \emph{self dual} properties.

%  \emph{dual category} of a category $\cat{C}$, denoted $\cat{C}^{op}$, in which the objects are the same of $\cat{C}$, and the arrows are the opposite of the arrows in $\cat{C}$, i.e., if $f: A \rightarrow B$ is an arrow of $\cat{C}$, then $f: B \rightarrow A$ is an arrow of $\cat{C}^{op}$.
% Each definition in category theory has a dual form. In general, if a statement $S$ is true in a category $\cat{C}$, then the opposite of the statement, $S^{op}$, obtained switching the words "domain" and "codomain" and replacing each composite $g \circ f$ into $f \circ g$, is still true in the category $\cat{C}^{op}$. Moreover, since every category is the opposite of its opposite, if a statement $S$ is true for every category, then $S^{op}$ is also true for every category ~\cite[pp8-9]{pierce91}. {\color{red} la discussione sul duale temo confonda}

To represent morphisms of a category $\cat{C}$ it is possible to use \emph{diagrams}, as the one below, in which the vertices are objects of $\cat{C}$, and the edges are morphisms of $\cat{C}$.
    \[
    \begin{tikzcd}
        X \arrow[d, "g'" swap] \arrow[r, "f'"] & Z \arrow[d, "g"] \\
        W \arrow[r, "f" swap] & Y        
    \end{tikzcd}
    \]
The diagram is said to commute whenever  $f \circ g' = g \circ f'$. Unique morphisms are represented with dashed arrows.
A more rigorous definition of what a diagram is will be given later (\Cref{def:diagram}).

\begin{example}
    It is easy to see that taking sets as objects and (total) functions as arrows, we obtain a category. In fact, given two functions $f: A \rightarrow B$ and $g: B \rightarrow C$, it is possible to compose them obtaining an arrow $g \circ f : A \rightarrow C$, and the composition is associative. For each set $A$ there exists an identity function $id_A: A \rightarrow A$ such that $id_A(a) = a$ for each $a\in A$.
    This category is denoted as $\Set$.
\end{example}

\begin{remark}\label{rem:small_cats}
    It is important to note that the \Cref{def:category} above does not specify what kind of collections
    %for a category $\cat{C}$, 
    $\Ob(\cat{C})$ and $\Hom(\cat{C})$ are.
    Taking $\Set$ as example, the collection $\Ob(\Set)$ cannot be a set itself, due to Russel's paradox. It would be more appropriate referring to a category $\cat{C}$ which $\Ob(\cat{C})$ and $\Hom(\cat{C})$ are both sets as a \emph{small category}, but it is assumed in this work, except where it is made explicit, for a category to be small.
    Another clarification must to be done, still considering $\Set$. Given two sets $A$ and $B$, it is possible to construct the set $B^A$ of all functions from $A$ to $B$. This is isomorphic to $\Set(A, B)$, for each pair of sets $A$ and $B$.
    A category $\cat C$ where, for each pair of objects $A$ and $B$, $\cat C (A, B)$ is a set is said to be \emph{locally small}.
\end{remark}

\subsection{Mono, Epi and Iso}\label{ssect:Mono_epi_iso}

Between the morphisms of a category, it is possible to distinguish some that have certain properties, as functions between sets can be surjective, injective or bijective.

\begin{definition}[Monomorphism]\label{def:mono}
    An arrow $f:B\rightarrow C$ in a category $\cat{C}$ is a \emph{monomorphism} if, for any pair of arrows of $\cat{C}$ $g:A \rightarrow B$, $h: A \rightarrow B$, the equality $f \circ g = f \circ h$ implies $g = h$. The class of monomorphisms of $\cat C$ is denoted $\Mono(\cat C)$.
\end{definition}

\begin{remark}\label{rem:fact_of_subobject_is_unique}
    For a morphism, from an algebraic point of view, being mono means being \emph{left cancelable}.
    %This fact can led us to define a particular kind of class of morphisms, which will reveal useful further.
    Let $A$ be an object in a category $\cat C$. Given two monomorphism $m: X \to A$ and $n: Y \to A$, then if  $h: X \to Y$ is a morphism such that $m = n \circ h$, then is the unique one: suppose $k$ is another morphism such that $m = n \circ k$. We can conclude $h = k$ observing that $n \circ h = n \circ k$ implies $h = k$ when $n$ is mono, which is by hypothesis.    
\end{remark}


\begin{definition}[Subobject]\label{def:subobj}
    Starting from this consideration, we can define a preorder on monomorphisms, placing $m \leq n$ if $m = n \circ h$ for some $h$. Such preorder induces an equivalence relation $\equiv$ on monomorphisms with codomain $A$, where $m \equiv n$ whenever $m \leq n$ and $n \leq m$, and the corresponding equivalence class is called \emph{subobject} of $A$.
\end{definition}


% \begin{definition}[Subobjects]
%     Let $C$ be an object in a category $\cat C$. Then, if $m: A \rightarrow C$ is mono\todo{sinceramente non trovo molto sensata la scelta di chiamare sottoggetto la coppia $(A, m)$. Lo standard, per quanto ne so, è usare sottoggetto per una classe di equivalenza di mono o, al massimo, per i mono}, $(A, m)$ is said to be a \emph{subobject} of $C$. Factorization of morphisms induces a preorder on subobjects of an object. $(A, m) \leq (B, n)$ whenever there exists a morphism $f : A\rightarrow B$ such that $m = n \circ f$.
%     \[
%         \begin{tikzcd}[row sep=9]
%             A \arrow[dr, "m"] \arrow[dd, dashed, "f"swap] & \\
%             & B \\
%             C \arrow[ur, "n"swap] &
%         \end{tikzcd}
%     \]
%     Two subobject $(A, m)$ and $(B, n)$ can are said to be \emph{equivalent subobjects}, written $(A, m) \approx (B, n)$ if $(A, m)\leq (B, n)$ and $(B, n) \leq (A, m)$. 
% \end{definition}

% An useful fact about subobjects is how factorization behaves. In particular $(A, m)$ and $(B, n)$ are subobjects of $C$. Then, if $(A, m)\leq (B, n)$, we have $m = n \circ h$ for some $h$. Suppose $k$ is another morphism such that $m = n \circ k$. We can conclude $h = k$ observing that $n \circ h = n \circ k$ implies $h = k$ when $n$ is mono, which is by hypothesis. This is to say what follows.

% \begin{prop}\label{prop:fact_of_subobjects_is_unique}
%     Let $(A, m)$ and $(B, n)$ be subobjects of $C$ in a category $\cat C$, with $(A, m)\leq (B, n)$. Then, the factorization of $m$ through $n$ is unique.
% \end{prop}



\begin{definition}[Epimorphism]\label{def:epi}
    An arrow $f: A\rightarrow B$ in a category $\cat{C}$ is an \emph{epimorphism} if, for any pair of arrows of $\cat{C}$ $g : B \rightarrow C$, $h: B \rightarrow C$, the equality $g \circ f = h \circ f$ implies $g = h$.
\end{definition}

\begin{definition}[Isomorphism]\label{def:iso}
    An arrow $f:A \rightarrow B$ is an \emph{isomorphism} if there is an arrow $f^{-1}:B \rightarrow A$, called the \emph{inverse} of $f$, such that $f^{-1}\circ f = id_{A}$ and $f \circ f^{-1} = id_{B}$. Two objects are said to be \emph{isomorphic} if there is an isomorphism between them.
\end{definition}

\begin{example}
    In $\Set$, monomorphisms are injective functions, epimorphisms are surjective functions and isomorphisms are bijections.
\end{example}

\begin{remark}
    Mono and epi are dual concepts. This fact is easily shown by considering how a monomorphism $m$ in a category $\cat C$ behaves in the dual category $\cat C^{op}$.
    In $\cat C$ we have that $m \circ f = m \circ g$ implies $f = g$. In $\cat C^{op}$, the we can state that $f \circ m = g \circ m$ implies $f = g$, obtaining the definition of epi. 
\end{remark}

\begin{prop}\label{prop:epi_mono_prop}
    The following statements hold for every pair of composable arrows $f$ and $g$ for any category $\cat C$:
    \begin{enumerate}
        \item if both $f$ and $g$ are mono, then $g \circ f$ is mono;
        \item if $g \circ f$ is mono, then $f$ is mono;
        \item if both $f$ and $g$ are epi, then $g \circ f$ is epi;
        \item if $g \circ f$ is epi, then $g$ is epi.
    \end{enumerate}
\end{prop}

\begin{obs}\label{obs:equiv_subobj_implies_isom_of_domains}
    Consider two monomorphisms $m: X \to A$ and $n: Y \to A$, and suppose $m \equiv n$ (in the sense of \Cref{def:subobj}). This corresponds to having a isomorphism between $X$ and $Y$. in fact, since $m \leq n$, there exists s unique $h: Y \to X$ such that $m = n \circ h$, and, since $n \leq m$, there exists a unique $k: X \to Y$ such that $n = m \circ k$. But then
    \[
        \begin{split}
            m &= n \circ h \\
              &= m \circ h \circ k
        \end{split}
        \qquad
        \begin{split}
            n &= m \circ k \\
              &= n \circ h \circ k
        \end{split}
    \]
    Since $m$ is mono, we obtain $id_X = h \circ k$, and, since $n$ is mono, $id_Y = k\circ h$, thus, an isomorphism. 
\end{obs}

The next proposition will be useful later.

\begin{prop}\label{prop:prop_epi_mono_Set}
    In $\Set$, for every commutative square as the one below, if $e: X \to Y$ is epi and $m: M \to Z$ is mono, then there exists a unique morphism $h: Y \to M$ making the whole diagram below commutative.
    \[
        \begin{tikzcd}
            X \ar[r, "f"] \ar[d, "e"swap] & M \ar[d, "m"] \\
            Y \ar[ur, dashed, "h"] \ar[r, "g"swap] & Z
        \end{tikzcd}
    \]
\end{prop}

%{\color{red}\begin{proof}
%    Before we start proving the statement, we have to note that, given a function $t: A \to B$, it is possible to decompose it as a composition of a injective function and a surjective function, considering the function $A \to t(A)$ sending each element onto its image along $t$, and then applying the inclusion $t(A) \to B$, and such functions are unique. Another way to factorize $t$ is via a composition of a surjective function and an injective function. Consider the equivalence relation $\sim$ defined on $A$, where $a \sim a'$ whenever $t(a) = t(a')$. This equivalence relation induces a map $A \to A/_\sim$, which is surjective. The function $A/_\sim \to B$, mapping each equivalence class onto its image along $t$ is then injective, and this factorization in unique too.

%    Let now be $f = f_i \circ f_s$ be the decomposition of $f$ with $f_s$ surjective (i.e., epi in $\Set$) and $f_i$ injective (i.e., mono in $\Set$), and $g = g_i \circ g_s$ be the decomposition of $g$ with $g_i$ injective and $g_s$ surjective, having the following situation.
%    \[
%        \begin{tikzcd}
%            X \ar[r, "{f_s}"] \ar[d, "e"] & f(X) \ar[r, "{f_i}"] & M \ar[d, "m"] \\
%            Y \ar[r, "{g_i}"swap] \ar[urr, "h"]& Y/_\sim \ar[r, "{g_s}"swap] & Z
%        \end{tikzcd}
%   \]
%    For the diagram above to commute, must be $f_i \circ f_s = h \circ e$ and $g_s \circ g_i = m \circ h$\todo{completare!!!}
%\end{proof}}
\begin{proof}
	Let $y \in Y$. Since $e$ is epi (i.e., surjective), there exists $x \in X$ such that $e(x) = y$. Then, we put $h(y) = f(x)$. $h$ is well defined, in fact, let $x' \in X$ be such that $e(x') = y$. In this case, we have $h(x) = f(x')$, and so
	\begin{align*}
		(m \circ f) (x')  &= (g \circ e) (x') \\
				&= g(y)		\\
				&= (g \circ e) (x)	\\
				&= (m \circ f) (x)
	\end{align*}
	since $m$ is mono by hypothesis, $f(x) = f(x')$.
\end{proof}
{\color{red}
\begin{cor}\label{cor:unique}
	La fattorizzazione di un morfismo è unica a meno di isomorfismo \todo{Scrivilo bene (l'isomorfismo dovrà rispettare certe coerenze) e dimostralo}
\end{cor}
}
\subsection{Categories from other categories}\label{ssect:cats_from_cats}

Starting from a category, it is possible to construct other categories with some interesting properties, as the following examples show.

The first notion to introduce is the one of subcategory.

\begin{definition}[Subcategory]
    A category $\cat{D}$ is a \emph{subcategory} of a category $\cat{C}$ if:
    \begin{enumerate}
        \item each object of $\cat{D}$ is an object of $\cat{C}$;
        \item \label{inc} each morphism between two objects of $\cat{D}$ is a morphism of $\cat{C}$; and
        \item \label{comp} composites and identities of $\cat{D}$ are the same of $\cat{C}$.
    \end{enumerate}

    If the inclusion at~\ref{inc} is an equality (i.e. $\cat{D}(A, B) = \cat{C}(A, B)$ for each couple of objects $A$, $B$ of $\cat{D}$), then $\cat{D}$ is said to be a \emph{full subcategory} of $\cat{C}$.
    Another way to express that composites are the same (point~\ref{comp}) is to say that if $f, g \in \Hom(\cat D)$ are composable, then $g \circ f \in \Hom(\cat D)$, i.e., $\Hom(\cat D)$ is \emph{closed under composition}.
\end{definition}

An object of a category marks out a category itself. This is the case of slice (and coslice) categories.

\begin{definition}[Slice Category]\label{def:slice_cat}
    Given a category $\cat{C}$ and an object $X \in \Ob(\cat{C})$, the \emph{slice category} $\cat{C}/X$ is the category that has pairs $(A, f)$ as objects, where $A$ is an object of $C$ and $f: A \rightarrow X$ is an arrow in $\cat{C}$, and arrows $\phi: (A, f) \rightarrow (B, g)$ are given by a morphism $\phi: A \rightarrow B$ of $\cat{C}$ such that the following diagram commutes:
    \[
        \begin{tikzcd}
            A \arrow[r, "{\phi}"] \arrow[dr, "f" swap] & B \arrow[d, "g"] \\
            & X
        \end{tikzcd}
    \]
    -- i.e, $g \circ \phi = f$.
    Composition between two arrows in $\cat{C}/X$ $\phi: (A, f) \rightarrow (B, g)$ and $\psi: (B, g) \rightarrow (C, h)$ is the arrow $\psi \circ \phi : (A, f) \rightarrow (C, h)$ obtained in the obvious way:
    \[
        \begin{tikzcd}
            A \arrow[bend left=30]{rr}{\psi \circ \phi}  \arrow[r, "{\phi}"] \arrow[dr, "f" swap] & B \arrow[r, "{\psi}"] \arrow[d, "g"] & C \arrow[dl, "h"] \\
            & X & 
        \end{tikzcd}
    \]

    The dual definition of \emph{coslice category}, noted $X/\cat{C}$ (where $X \in \Ob(\cat{C})$), is obtained by taking as objects the morhisms of $\cat{C}$ with domain $X$ and as arrows the morphisms $\phi: (A, f) \rightarrow (B, g)$ such that $f:X\rightarrow A, g:B \rightarrow X \text{ of }\cat{C}$ and $g = \phi \circ f$. 
\end{definition}

Furthermore, it is possible to raise a new category from two old ones by taking their product, as the following definition shows.

\begin{definition}[Product category]
    Given two categories $\cat C, \cat D$, the \emph{product category} $\cat{C\times D}$ has as objects pairs of objects $(A, B)$, where $A \in \Ob(\cat C), B \in \Ob(\cat D)$, and as arrows pairs of morphisms $(f, g)$, where $f$ is an arrow in $\cat C$ and $g$ is an arrow in $\cat D$. Composition and identities are defined pairwise: $(f, g) \circ (h, k) = (f \circ h, g \circ k)$, and $id_{(A, B)} = (id_A, id_B)$.
\end{definition}


\section{Functors, Natural Transformations, Adjoints}\label{sect:funct_nats}

\subsection{Functors and Natural Transformations}

A functor is a structure preserving map between categories. 
\begin{definition}[Functor]
    Let $\cat{C}$ and $\cat{D}$ be categories. A \emph{functor} $F:\cat{C \rightarrow D}$ is a map taking each object of $A \in \Ob(\cat{C})$ to an object $F(A) \in \Ob(\cat{D})$ and each arrow $f:A\rightarrow B$ of $\cat C$ to a arrow $F(f): F(A) \rightarrow F(B)$ of $\cat D$, such that, for all objects $A \in \Ob(\cat C)$ and composable arrows $f$ and $g$ of $\cat C$:
    \begin{itemize}
        \item $F(id_{A}) = id_{F(A)}$;
        \item $F(g \circ f) = F(g) \circ F(f)$.
    \end{itemize}

    In this case, $\cat C$ is called \emph{domain} and $\cat D$ is called \emph{codomain} of the functor $F$.
\end{definition}

\begin{example}
    A first example of functor is the \emph{identity functor}. Given a category $\cat C$, the identity functor $Id_\cat C :\cat{C \rightarrow C}$ is the functor that maps each object on itself and each arrow onto itself. 
\end{example}

Once defined what a functor is, we can give a more rigorous definition of diagram. Although this may seem extremely technical, it will be useful, especially in the definition of limits (\Cref{def:limit}).

\begin{definition}[Diagram]\label{def:diagram}
    A \emph{diagram in a category $\cat C$ of shape $\cat I$} is a functor $D: \cat{I \rightarrow C}$.
    The category $\cat I$ can be considered as the category indexing the objects and the morphisms of $\cat C$ shaped in $\cat I$.
\end{definition}

\begin{example}\label{ex: span}
    A diagram of shape $\Lambda = (L \xleftarrow{l} X \xrightarrow{r} R)$ is said to be a \emph{span}, and is denoted by $(l, X, r): L \rightharpoonup R$.
    A span can be viewed as the generalization of relations between sets. In fact, in $\Set$, a relation $R \subseteq A \times B$ is a span, with the projections $\pi_A : R \rightarrow A$ and $\pi_B : R \rightarrow B$ as arrows.
    
    The dual notion of span is a \emph{cospan}, namely, a diagram of shape $\Lambda^{op} = (L \xrightarrow{l} X \xleftarrow{r} R)$, and is denoted by $(l, X, r): L \rightharpoondown R$.
\end{example}

Functor are often used to generalize some structural behaviour that constructions in categories have. An important example of this fact is the universal property. The definition is not straightforward, but it gives the abstraction of a property that will be useful in further definitions%~\cite{Herrlich_Strecker_1979}.

\begin{definition}[Universal property]\label{def:univ_prop}
    Let $F: \cat{C \rightarrow D}$ be a functor, and let $B \in \Ob(\cat D)$. A pair $(u, A)$, with $A \in \Ob(\cat C)$ and $u: B \rightarrow F(A)$ is said to be an \emph{universal map for $B$ with respect to $F$} if for each $A' \in \Ob(\cat C)$ and each $f: B \rightarrow F(A')$ there exists a unique morphism $h \in \cat C(A, A')$ such that the following triangle commute:
    \[
        \begin{tikzcd}
            B \arrow[r, "u"] \arrow[dr, "f" swap] & F(A) \arrow[d, dashed, "F(h)"] & A \arrow[d, dashed, "h"]\\
            & F(A') & A'
        \end{tikzcd}
    \]

    -- i.e. there exists a unique $h$ such that $F(h) \circ u = f$. In this case, $(u, A)$ is said to have the \emph{universal property}.

    Dually, if $G: \cat C \rightarrow \cat D$ is a functor and $B \in \Ob(\cat D)$, then a pair $(A, u)$ is a \emph{co-universal map for $B$ with respect to $G$} if $u:G(A) \rightarrow B$ and for each $A' \in \Ob(\cat C)$ and each $f: G(A') \rightarrow B$ there exists a unique morphism $h \in \cat{C}(A', A)$ such that the following diagram commutes:
    \[
        \begin{tikzcd}
            A' \arrow[d, dashed, "h" swap] & G(A') \arrow[d, dashed, "G(h)" swap] \arrow[dr, "f"] \\
            A & G(A) \arrow[r, "u" swap] & B
        \end{tikzcd}
    \]
\end{definition}

Some interesting properties of certain functors depend strictly on how they behave on the homsets of the domain and the codomain categories. The following definitions are about this particular type of functors.

\begin{definition}[Full functor, faithful functor, fully faithful functor]
    Let $F: \cat C \rightarrow \cat D$ be a functor, and consider the inducted function
    $$
        F_{A, B} : {\cat C}(A, B) \rightarrow {\cat D}(F(A), F(B))
    $$
    If, for each $A$, $B$ objects of $\cat C$, $F_{A, B}$ is surjective, then $F$ is said to be \emph{full}, if it is injective, $F$ is said to be \emph{faithful}, if it is both injective and surjective, $F$ is said to be \emph{fully faithful}.
\end{definition}

\begin{obs}
        Properties such as fullness and faithfulness are so called \emph{self-dual}, because the dual notion coincide with the same notion. This fact can be advantageous because if for example the faithfulness implies the preservation of some property, then the dual property is implied at the same way.
\end{obs}
    

\begin{example}\label{ex:full_subc_inc_fully_faith}
    Let $\cat C$ be a category and $\cat D$ a subcategory. The inclusion functor $I: \cat{D \rightarrow C}$, mapping each object and each arrow onto itself. $I$ is a faithful functor, because, given any pair of objects $A$ and $B$ of $\cat D$, $I_{A, B}$ is injective. If $\cat D$ is a full subcategory, then $I$ is fully faithful.
\end{example}

Having such classification among functors turns out to be useful in many contexts. For example, consider $F(m): F(B) \rightarrow F(C)$ be a monomorphism in a category $\cat D$, where $F: \cat C \rightarrow \cat D$ is a faithful functor. Then, if $f, g: A \rightarrow B$ are two morphisms in $\cat C$ such that $m \circ f = m \circ g$, then $F(m \circ f) = F(m) \circ F(f) = F(m) \circ F(g) = F(m\circ g)$. Since $F(m)$ is mono, then $F(f) = F(g)$, and, since $F_{A, B}$ is injective, $f = g$. Together with the fact that faithfulness is a self-dual concept, we have a proof for what follows~\cite{Herrlich_Strecker_1979}.

\begin{prop}
    Let $F: \cat{C \rightarrow D}$ be a faithful functor. Then $F$ reflects monomorphisms and epimorphisms.
\end{prop}


%\subsection{Natural Transformations}

Given two functors that share domain and codomain categories, it is possible to define a transformation between them, taking each object of the domain of the functors to an arrow in the codomain of the functors that represent the action of ``changing the functor acting on that object''.

\begin{definition}[Natural transformation]
    Let $F,G : \cat {C \rightarrow D}$ be two functors. A \emph{natural transformation} $\eta$ between them, denoted $\eta: F \dot\rightarrow G$, is a function $\eta: \Ob(\cat C) \rightarrow \Hom(\cat D)$ taking each $A \in \Ob(\cat C)$ to a morphism $\eta_A:F(A) \rightarrow G(A)$ in $\cat{D}$, such that, for each morphism $f: A \rightarrow B$ of $\cat C$, the following diagram commutes:
    \[
        \begin{tikzcd}
            F(A) \arrow[d, "F(f)" swap] \arrow[r, "{\eta_A}"] & G(A) \arrow[d, "G(f)"] \\
            F(B) \arrow[r, "{\eta_B}"swap] & G(B)
        \end{tikzcd}
    \]
    -- i.e., such that $G(f) \circ \eta_A = \eta_B \circ F(f)$.

    We say that $\eta: F \dot\rightarrow G$ is a \emph{natural isomorphism} if, for each $A \in \Ob(\cat C) \text{, } \eta_A$ is an isomorphism in $\cat D$. In this case, $F$ and $G$ are said to be \emph{naturally isomorphic}, and is denoted $F \cong G$.    
\end{definition}

\begin{obs}\label{obs:comp_assoc_nat_tran}
It is easy to see that, given two natural transformations $\eta:F\dot\rightarrow G$, $\theta: G \dot\rightarrow H$, it is possible to compose them obtaining a new natural transformation $\xi = \theta \circ \eta : F \dot \rightarrow H$. This follows by the fact that the diagram
\[
        \begin{tikzcd}
            F(A) \arrow[d, "F(f)" swap] \arrow[r, "{\eta_A}"] 
                & G(A) \arrow[d, "G(f)"] \arrow[r, "{\theta_A}"]
                & H(A) \arrow[d, "H(f)"] \\
            F(B) \arrow[r, "{\eta_B}"swap]
                & G(B) \arrow[r, "{\theta_B}" swap]
                & H(B)
        \end{tikzcd}
\]
commutes because the two inner squares do. Sticking another diagram on the right of the one above, it is possible to show associativity of composition of natural transformations.
\end{obs}

\subsubsection{Functor Categories}

The \Cref{obs:comp_assoc_nat_tran} shows that natural transformations recreate on the functors the same structure that morphisms in a category have on objects. This leads us to define a particular kind of category, in which objects are functors between two categories, and arrow are natural transformations.


\begin{definition}[Functor Category]\label{def:functor_category}
    Let $\cat C$ and $\cat D$ be categories. The category whose objects are functors between $\cat C$ and $\cat D$ and whose arrows are natural transformations between them is said to be a \emph{functor category}, and it is denoted by $[\cat{C, D}]$.
\end{definition}

\begin{lemma}\label{lemma:canonical_equiv}
    Let $\cat{C, D, I}$ be categories. Then, it holds that
    \[
        \cat{[I, [C, D]] \cong [I \times C, D]}
    \]
\end{lemma}

A functor with a small category as domain (\Cref{rem:small_cats}) and $\Set$ as codomain is said to be a \emph{presheaf} on that category. Given a category $\cat C$, it is possible to construct the functor category of the presheaves on $\cat C$, i.e. $[\cat C, \Set]$.

\begin{remark}
    What we are calling here a presheaf is not totally accurate, because technically a presheaf on a small category $\cat C$ is a functor $F: \cat C ^{op} \rightarrow \Set$. This technicality would bring more complexity, and it is beyond the scope of this work, so we will continue adopting the definition given above.
\end{remark}

\iffalse
\begin{lemma}\label{lemma:limits_of_presheaves}
    Let $\cat C$ be a category, and let $[\cat C, \Set]$ be the category of presheaves on $\cat C$. Let $D: \cat I \to [\cat C, \Set]$ be a diagram of shape $\cat I$ on the presheaves category early mentioned. Then,
    \begin{enumerate}
        \item The limit of $D$ exists, and it is the presheaf $L: \cat C \to \Set$ such that, for each object $A$ of $\cat C$, $L(A)$ is the limit in $\Set$ of the values of the presheaves $D(i)(A)$ for each $i$.
        \item The colimit of $D$ exists, and it is the presheaf $C: \cat C \to \Set$ such that, for each object $A$ of $\cat C$, $C(A)$ is the colimit in $\Set$ of the values of the presheaves $D(i)(A)$, for each $i$.
    \end{enumerate}
\end{lemma}
\fi

\iffalse
\subsection{Comma Categories}

Functor constructions allow us to generalise basic concepts already seen for categories. An important example of this fact are comma categories, a more general notion of slice categories (\Cref{def:slice_cat}).

% \begin{definition}[Comma category]
%     Given two functors $F: \cat{C} \rightarrow \cat{E}$, $G: \cat{D} \rightarrow \cat{E}$, the \emph{comma category $(F \downarrow G)$} is the category whose objects are triples $(A, f, B)$, with $A \in \Ob(\cat C)$, $B \in \Ob(\cat D)$ and $f \in \cat{E}(F(A), G(B))$, and whose morphisms are the pairs $(a, b) : (A, f, B) \rightarrow (C, g, D)$ where $a : A \rightarrow C$, $b: B \rightarrow D$ and such that
%     \[
%         \begin{tikzcd}
%             F(A) \arrow[r, "f"] \arrow[d, "F(a)" swap] & G(B) \arrow[d, "G(b)"] \\
%             F(C) \arrow[r, "g"] & G(B)
%         \end{tikzcd}
%     \]
%     commutes; composition of morphisms is obtained via pairwise composition, i.e., $(a, b) \circ (c, d) = (a \circ c, b \circ d)$.
% \end{definition}

\begin{definition}[Comma category]\label{def:comma_category}
    Let $\cat{C \text{, } D \text{ and } E}$ be categories, and let $S: \cat{C \rightarrow E}$, $T:\cat{D \rightarrow E}$ be functors (source and target):
    \[
        \begin{tikzcd}
            \cat{C} \arrow[r, "S"] & \cat{E} & \cat{D} \arrow[l, "T" swap]
        \end{tikzcd}
    \]
    Then, the \emph{comma category $(S \downarrow T)$} is the category in which: 
    \begin{itemize}
        \item the objects are triples $(A, f, B)$, where $A \in \Ob(\cat{C})$, $B \in \Ob(\cat D)$ and $f: S(A) \rightarrow T(B)$ is an arrow of $\cat E$;
        \item the arrows are pairs $(c, d): (A, f, B) \rightarrow (C, g, D)$, where $c \in \Hom(\cat C)$ and $d \in \Hom(\cat D)$, such that the square below commutes;
        \[
            \begin{tikzcd}
            S(A) \arrow[r, "f"] \arrow[d, "S(c)" swap] & G(B) \arrow[d, "T(d)"] \\
            T(C) \arrow[r, "g" swap] & T(B)
            \end{tikzcd}
        \]
        \item composition of morphisms is obtained via pairwise composition, i.e., $(a, b) \circ (c, d) = (a \circ c, b \circ d)$.
    \end{itemize}
\end{definition}

Thus, the slice category $\cat C / X$ is the comma category given by the two functors $Id_{\cat C}$ (the identity functor), and the functor $!_X: \textbf{1} \rightarrow \cat C$, where $\textbf 1$ is the one-object category defined in \Cref{ex:1_cat}, and $!_X$ sends the only object of $\textbf{1}$ to $X$ (then the only morphism of $\textbf{1}$ to $id_X$ of $\cat C$):
\[
    \begin{tikzcd}
        \cat C \arrow[r, "{Id_\cat C}"] & \cat C & \textbf{1} \arrow[l, "{!_X}"swap]
    \end{tikzcd}
\]
It is easy to see that $(Id_\cat C \downarrow !_X)$ is exactly the same of $\cat C / X$.

In the same way, it is possible to define coslice categories in terms of comma categories: the category $(!_X \downarrow Id_\cat C)$ is exactly the coslice $X / \cat C $.

\fi

\subsection{Adjoints}

% \begin{definition}[Adjoint Functor]\label{def:adjoint}
%     Let $\cat C$ and $\cat D$ be categories, and let $R:\cat{C \rightarrow D}$ and $L : \cat{D \rightarrow C}$ be functors. $L$ is called \emph{left adjoint} of $R$ if there exists a natural transformation $\eta: Id_{\cat C} \dot\rightarrow (L \circ R)$ such that, for each object $A$ and each arrow $f: A \rightarrow L(B)$ of $\cat C$, there exists a unique arrow $g: R(A) \rightarrow B$ of $\cat D$ such that the following diagrams commutes:
%     \[
%         \begin{tikzcd}
%             A \arrow[r, "{\eta_A}"] \arrow[dr, "f" swap] & L(R(A)) \arrow[d, dashed, "L(g)"] \\
%             & L(B)
%         \end{tikzcd}
%     \] -- i.e., if $L(g) \circ \eta_{A} = f$. In this case $\eta$ is called the \emph{unit} of the adjoint.

%     Analogously, $R$ is called \emph{right adjoint} of $L$ provided that there exists a natural transformation $\epsilon : (R \circ L) \dot\rightarrow Id_{\cat D}$ such that, for each arrow $g: R(A) \rightarrow B$ of $\cat D$ there exists a unique arrow $f: A \rightarrow L(B)$ of $\cat C$ such that the following diagram commutes:
%     \[
%         \begin{tikzcd}
%             R(L(B)) \arrow[r, "{\epsilon_{B}}"] & B \\
%             R(A) \arrow[u, dashed, "R(f)"] \arrow[ur, "g"swap] 
%         \end{tikzcd}
%     \] -- i.e., $\epsilon_{B} \circ R(f) = g$. In this case, $\epsilon$ is called \emph{co-unit} of the adjoint.
% \end{definition}
%\todo{sistemare la definizione}

\begin{definition}[Right Adjoint]\label{def:right_adjoint}
    Let $R: \cat{C \rightarrow D}$ be a functor. $R$ is said \emph{right adjoint} if, for each object $A$ of $\cat D$, there exists an object $L(A)$ and an arrow $\eta_A:A \rightarrow R(L(A))$ in $\cat C$ such that, for each arrow $f: A \rightarrow R(B)$ of $\cat D$, there is a unique arrow $g:L(A) \rightarrow B$ such that the following diagram commutes.
    \[
        \begin{tikzcd}
            A \arrow[r, "{\eta_A}"] \arrow[dr, "f" swap] & R(L(A)) \arrow[d, dashed, "R(g)"] \\
            & R(B)
        \end{tikzcd}
    \]
    --i.e., $R(g) \circ \eta_A = f$.
\end{definition}

\begin{prop}\label{prop:ext_left_to_funct}
    In \Cref{def:right_adjoint}, the map that takes an object $A$ to an object $L(A)$ can be extended to a functor $L: \cat{D \rightarrow C}$. Moreover, there exists a natural transformation $id_{\cat{D}}\to R\circ L$.
\end{prop}
\begin{proof}
    % Let $R$ be the right adjoint as in ~\Cref{def:right_adjoint}. It is possible to define, for $f: X \rightarrow Y$, $L(f) : L(X) \rightarrow L(Y)$  such that: \[R(L(f)) \circ \eta_X = \eta_Y \circ f\] \todo{finire}
    Let $R$ be the right adjoint as in ~\Cref{def:right_adjoint}. Given $f\colon X \rightarrow Y$, we can define $L(f)$ as the unique arrow  $L(X) \rightarrow L(Y)$ whose image through $R$ fits in the diagram below.
    \[
    \begin{tikzcd}
    	X \arrow[r, "{\eta_X}"] \arrow[d, "f" swap] & R(L(X)) \arrow[d, dashed, "R(L(f))"] \\
    	Y\arrow[r, "\eta_Y" swap]& R(L(Y))
    \end{tikzcd}
    \]
    
    To see that in this way we get a functor it is now enough to notice the commutativity of the following diagrams.
    \[
    \begin{tikzcd}
    	X \arrow[r, "{\eta_X}"] \arrow[d, "{id_X}" swap] & R(L(X)) \arrow[d, "{R(id_{L(X)})}"]   \\
    	Y\arrow[r, "{\eta_Y}" swap] & R(L(Y))
    \end{tikzcd}
    \]    \[
    \begin{tikzcd}
    	   X \arrow[d, "{\eta_X}" swap]  \arrow[r, "{f}"]& Y \arrow[d, "{\eta_Y}" swap] \arrow[r, "{g}"]& Z\arrow[d, "{\eta_Z}"]\\
    	R(L(X))  \arrow[r, "{R(L(f))}" swap] & R(L(Y)) \arrow[r, "{R(L(g))}" swap] & R(L(Z))
    \end{tikzcd}
    \]
    
    Finally, by construction the family given by all the $\eta_A\colon A\to R(L(A))$ is natural and we can conclude. 
\end{proof}

\begin{remark}
    The family above mentioned is called \emph{unit} of the adjunction.
\end{remark}

\begin{definition}[Left Adjoint]\label{def:left_adjoint}
    Let $L: \cat{D \rightarrow C}$ be a functor. $L$ is a \emph{left adjoint} if, for each object $B$ of $\cat C$, there exists an object $R(B)$ and an arrow $\epsilon_B : L(R(B)) \rightarrow B$ in $\cat D$ such that, for each arrow $g: L(A) \rightarrow B$ of $\cat C$, there exists a unique arrow $f:A \rightarrow R(B)$ such that the following diagram commutes.
    \[
        \begin{tikzcd}
            L(R(B)) \arrow[r, "{\epsilon_B}"] & B \\
            L(A) \arrow[u, dashed, "L(f)"] \arrow[ur, "g"swap]
        \end{tikzcd}
    \]
    -- i.e., $\epsilon_B \circ L(f) = g$.
\end{definition}

As we have shown before, it is possible to extend the mapping $A \rightarrow R(B)$ to a functor $R$, whose functoriality follows placing $\epsilon_X \circ L(R(f)) = f \circ \epsilon_Y$ for each $f: X \rightarrow Y$. The family $\epsilon_B : L(R(B)) \rightarrow B$ is natural and it is called \emph{counit} of the adjunction.

The connection between left and right adjoints is expressed in the following proposition.

\begin{prop}
    Let $L$ be the functor of \Cref{prop:ext_left_to_funct}. Then, $L$ is a left adjoint.
\end{prop}

\begin{proof}
	Given an object $B$ in $\cat{C}$, we can consider the solid part of the diagram below. 
	Since $R$ is a right adjoint, we get a unique arrow whose image through $R$ make the triangle commutative.
	    \[
	\begin{tikzcd}
		R(B)\arrow[r, "{\eta_{R(B)}}"] \arrow[dr, "id_{R(B)}" swap] & R(L(R(B))) \arrow[d, dashed, "R(\epsilon_B)"] \\
		& R(B)
	\end{tikzcd}
	\]
	
	Let now $A$ be an object of $\cat{D}$ and $g\colon L(A)\to B$ an arrow in $\cat{C}$. We can consider the composite $R(g)\circ \eta_A\colon A\to R(B)$. Then we have
	\begin{align*}
		R(\epsilon_B)\circ R(L(R(g)))\circ R(L(\eta_A))\circ \eta_A&=R(\epsilon_B)\circ R(L(R(g)))\circ \eta_{R(L(A))}\circ \eta_A\\&=R(\epsilon_B)\circ \eta_{R(B)}\circ R(g)\circ \eta_A\\&=R(g)\circ \eta_A
	\end{align*}
	Since $R$ is a right adjoint and $\eta$ its unit, it follows that $\epsilon_B\circ L(R(g)\circ \eta_A)$ coincides with $g$ as wanted.
 \end{proof}

 % Il concetto di functore aggiunto è molto più complesso di come viene presentato qui. In questo lavoro, avere una definizione di funtore aggiunto serve soltanto per una certa proprietà che verrà presentata nel capitolo sui limiti.

\section{Limits and Universal Constructions}\label{sect:limits_univ_constr}
\subsection{Limits and Colimits}

% ---- Coni, Limiti ----

\begin{definition}[Cones]
    Let $D:\cat {I \rightarrow C}$ be a diagram in $\cat C$ of shape $\cat I$. A \emph{cone} for $D$ is an object $X$ of $\cat C$, together with arrows $f_i : X \rightarrow D(i)$ indexed by $\cat I$ (i.e. one for each object $i$ of $\cat I$), such that, for each morphism $\alpha: i \rightarrow j$ of $\cat I$, the following diagram commutes:
    \[
        \begin{tikzcd}
            & X \arrow[dl, "{f_i}"swap] \arrow[dr, "{f_j}"] & \\
            D(i) \arrow[rr, "{D(\alpha)}" swap] & & D(j)
        \end{tikzcd}
    \]
    -- i.e., $D(\alpha) \circ f_i = f_j$.
    We denote such cone as $\{f_i: X \rightarrow D(i)\}$.

\end{definition}

\begin{obs}\label{obs:category_of_cones}
    Given a diagram $D$, the category of the cones for $D$, denoted $\textbf{Cone}(D)$, is defined to have cones for $D$ as objects and cone morphisms as arrows, where a cone morphism $\phi: C \rightarrow C'$ from $C = \{f_i: X \rightarrow D(i)\}$ to $C' = \{f_i':X' \rightarrow D(i)\}$ is a morphism $\phi: X \rightarrow X'$ such that the following diagram commutes for each $i$:
    \[
        \begin{tikzcd}
            X \arrow[rr, "{\phi}"] \arrow[dr, "{f_i}" swap] & & X' \arrow[dl, "{f_i'}"] \\
            & D(i) &
        \end{tikzcd}
    \]
\end{obs}

\begin{definition}[Limits]\label{def:limit}
    Let $D:\cat {I \rightarrow C}$ be a diagram in $\cat C$ of shape $\cat I$. A cone $\{f_i: X \rightarrow D(i)\}$ is a \emph{limit} provided that, for any other cone $\{f_{i}': X' \rightarrow D(i)\}$ for $D$, then there exists a unique morphism $k: X' \rightarrow X$ such that the following diagram commutes for each object $i$ of $\cat I$:
    \[
        \begin{tikzcd}
            X' \arrow[rr, dashed, "k"] \arrow[dr, "{f_i'}" swap] & & X \arrow[dl, "{f_i}"] \\
            & D(i) &
        \end{tikzcd}
    \]
    -- i.e., $f_i \circ k = f_i'$ for each object $i$ of $\cat I$. Such limit is denoted as $(X, f_i)_{i \in \cat I}$
\end{definition}

\begin{obs}
    Given a diagram $D$, a limit for $D$ is exactly the terminal object of the category $\textbf{Cone}(D)$, defined in \Cref{obs:category_of_cones}.
\end{obs}

The dual notions of cones and limits are that of cocones and colimits.

\begin{definition}(Cocones, Colimits)
    A \emph{cocone} for a diagram $D: \cat{I \rightarrow C}$ is an object $Y$ of $\cat C$ together with arrows $f_i : D(i) \rightarrow Y$ such that, for each $g: D(i) \rightarrow D(j)$ of $\cat C$, $f_j \circ g = f_i$. A cocone is denoted $\{f_i: D(i) \rightarrow Y \}$.
    A \emph{colimit} for $D$ is a cocone $C = \{f_i: D(i) \rightarrow Y \}$ with the universal property -- i.e., if $C' = \{ f_i' : D(i) \rightarrow Y'\}$ is another cone for $D$, then there exists a unique arrow $h:Y \rightarrow Y'$ such that, for each $i$, $h \circ f_i = f_i'$.
\end{definition}

\begin{remark}
    It makes sense to refer to a (co)limit as \emph{the} (co)limit. Suppose $(P, p_i)_{i\in \cat I}$ and $(Q, q_i)_{i \in \cat I}$ be limits for a diagram $D: \cat{I \rightarrow C}$. Then, there exists a unique morphism $h: Q \rightarrow P$ such that $p_i \circ k = q_i$ for each $i$. At the same way, there exists a unique morphisms $k: P \rightarrow Q$ such that $q_i \circ k = p_i$ for each $i$. From the existence of the identity, must be $k \circ h = id_{Q}$ and $h \circ k = id_P$, that is, $P$ and $Q$ are isomorphic.
\end{remark}

Notion such limits and colimits are generalization of more particular cases that will be now introduced, that we will often call \emph{universal constructions}.


% ---- Oggetti Iniziali & Terminali ----

\begin{definition}[Initial Object, Terminal Object]
    Consider the empty diagram (i.e., a diagram $D: \textbf{0} \rightarrow \cat C$ where $\textbf{0}$ is the empty category \Cref{ex:0_cat}). Then, the limit of $D$ is called \emph{terminal object} and the colimit of $D$ is called \emph{initial object}, denoted, respectively, $\terminal_\cat{C}$ and $\initial_{\cat C}$. (Subscripts are omitted when they are clear from the context).
\end{definition}

\begin{example}\label{ex:set_init_term}
    In $\Set$, the initial object is the empty set $\varnothing$, because, for each set $S$, there exists the empty function from $\varnothing$ to $S$. The terminal object of $\Set$ is the singleton $\{ \bullet \}$, because there is exactly one function from a set $S$ to $\{ \bullet \}$, namely, the function which sends each $s \in S$ to $\bullet$.
    % It is possible to visualize the \Cref{obs:terminal_are_isomorph}: given two singletons $\{ \bullet \}$ and $\{ \star \}$, the function between them is bijective, while there exists a unique initial object.
\end{example}

We now illustrate a result on functor categories (\Cref{def:functor_category}) that will be useful later.

\begin{prop}
    Let $\cat D$ be a category. If $\cat D$ has an initial object, then, for any category $\cat C$, $[\cat{C, D}]$ has an initial object. If $\cat D$ has a terminal object, then, for any category $\cat C$, $[\cat{C, D}]$ has a terminal object.
\end{prop}

\begin{proof}
    Let $\initial_{\cat{D}}$ be the initial object of $\cat D$,  and consider the constant functor $I(f) = id_{\initial_{\cat D}}$ for all $f \in \Hom(\cat C)$. Then, for any $G: \cat{C \rightarrow D}$, $\eta: I \rightarrow G$, defining $\eta_A$ as the \emph{unique morphism from $\initial_{\cat D}$ to $G(A)$} for each $A \in \Ob(\cat C)$, is a natural transformation $I \dot\rightarrow G$, as the diagram below shows:
    \[
        \begin{tikzcd}
            I(A)=\initial_{\cat D} 
                    \arrow[r, "{\eta_A}"]
                    \arrow[d, "{I(f) = id_{\initial_{\cat D}}}" swap] &
            G(A)
                    \arrow[d, "G(f)"] \\
            I(A') = \initial_{\cat D}
                    \arrow[r, "{\eta_{A'}}" swap] &
            G(A')
        \end{tikzcd}
    \]
    for each $f: A \rightarrow A'$, the square above must commute, since there is only one morphism from $\initial_{\cat D}$ to $G(A')$. For the same reason, $\eta$ is the only natural transformation from $I$ to $G$, being indeed the initial object of $[\cat{C, D}]$.
    
    Defining $T(f) = id_{\terminal_{\cat D}}$ for each $f \in \Hom(\cat C)$. Then, $\theta:F\rightarrow T$, for any $F: \cat{C \rightarrow D}$, defining $\theta_A$ as the \emph{unique morphism from $F(A)$ to $\terminal_{\cat{D}}$} is a natural transformation due to the commutativity of the following diagram for each $f: A \rightarrow A'$:
    \[
        \begin{tikzcd}
            F(A) \arrow[r, "{\theta_A}"] \arrow[d, "F(f)" swap] &
            T(A) = \terminal_{\cat D} \arrow[d, "{T(f) = id_{\terminal_{\cat D}}}"] \\
            F(A') \arrow[r, "{\theta_{A'}}" swap] & T(A') = \terminal_{\cat D}
        \end{tikzcd}
    \]
    Hence, $\theta$ is the unique natural transformation from $F$ to $T$, and $T$ is the terminal object of $[\cat{C, D}]$.
\end{proof}

In particular, every presheaf has an initial and a terminal object, because $\Set$ does (\Cref{ex:set_init_term}).

% ----- Prodotti, Coprodotti ----

\begin{definition}[Product, Coproduct]
        Let $D$ be the following diagram:
    \[
        \begin{tikzcd}
            A & & B
        \end{tikzcd}
    \]
    Then, a cone for $D$ is an object $X$ and two arrows $f: X \rightarrow A$, $g: X \rightarrow B$ (i.e., a span, defined in \Cref{ex: span}):
    \[
        \begin{tikzcd}
            A & X \arrow[l, "f" swap] \arrow[r, "g"] & B
        \end{tikzcd}
    \]
    If it exists, a limit for $D$ is called \emph{product} of $A$ and $B$, usually denoted as $(A\times B, \pi_A, \pi_B)$, while whose arrows are called \emph{projections}.
    The colimit of $D$ is called \emph{coproduct} of $A$ and $B$, usually denoted as $(\iota_A, \iota_B, A + B)$.
\end{definition}

\begin{example}
    $\Set$ has both products and coproduts. Given two sets $A$ and $B$, the categorical product is the set-theoretic cartesian product $A \times B$, together with the two projections $\pi_A$ and $\pi_B$, while the coproduct is the disjoint sum $A \amalg B = \{ (x, 0) \mid x \in A\} \cup \{(y, 1) \mid y \in B \}$, together with the two canonical injections $\iota_A$ and $\iota B$, where $\iota_A(a) = (a, 0)$ and $\iota_B(b) = (b, 1)$. 
\end{example}

The notions of product and coproduct can be easily generalized, extending the definition to the product (and coproduct) of a family of objects, together with appropriate arrows (e.g., the projection arrows for each object in the product). We will denote the product of a collection of objects indexed by a (finite) category $\cat I$ as $\big(\prod_{i \in \Ob(\cat I)} X_i, (\pi_i)_{i \in \Ob(\cat I)}\big)$, and the coproduct as \- $\big((\iota_i)_{i \in \Ob(\cat I)}, \coprod_{i \in \Ob(\cat I)} X_i\big)$.

% ---- Equalizzatori, Coequalizzatori ----

\begin{definition}[Equalizer, Coequalizer]\label{def:equalizer_coequalizer}
    Let $D$ be the diagram below.
    \[
        \begin{tikzcd}
            A \arrow[r, shift left, "f"] \arrow[r, shift right,"g"swap] & B
        \end{tikzcd}
    \]
    The limit of $D$ is called \emph{equalizer}, and its colimit is called \emph{coequalizer}.
\end{definition}

\begin{prop}\label{prop:eq_are_mono}
    Let $e: E \rightarrow A$ be the arrow that equalizes $f, g : A \rightarrow B$ in a category $\cat C$. Then, $e$ is a monomorphism.
\end{prop}

\begin{proof}
    Suppose $X$ be an object and $x, y: X \rightarrow E$ be two morphisms in $\cat C$ such that $e \circ x = e \circ y$, and let $z = e \circ x$. Then, since $e$ is an equalizer, $f \circ e = g \circ e$, and $f \circ z = g \circ z$. But, for the universal property of limits, there must be exactly one $u: Z \rightarrow E$ such that $z = e \circ u$. It follow that $x = u$ and $y = u$, hence $x = y$.
\end{proof}

Of all monomorphisms, an interesting subclass of them is the one that contains only the equalizers.

\begin{definition}[Regular Monomorphism]\label{def:reg_mono}
    A monomorphism that is an equalizer for a pair of arrows is said \emph{regular monomorphism}. The class of all regular monomorphisms of a category $\cat C$ is denoted $\Reg(\cat C)$.
\end{definition}

\begin{obs}
    Given two composable regular monos $m$ and $n$, suppose that $n$ equalizes two arrows $f$ and $g$. Then, we have
    \begin{align*}
        g \circ (n \circ m) &= (g \circ n) \circ m &&\\
                            &= (f \circ n) \circ m  && \textit{$n$ equalizer} \\
                            &= f \circ (n \circ m)
    \end{align*}
    Since $n \circ m$ is mono (\Cref{prop:epi_mono_prop}), we have shown that, given a category $\cat C$, $\Reg(\cat C)$ is closed under composition. 
\end{obs}


% ---- Pullback, Pushout -----


\begin{definition}[Pullback, Pushout]\label{def:pullback_pushout}
    Let $D$ be the cospan $(f, C, g) : A \rightharpoondown B$. % $$(A \xrightarrow{f} C \xleftarrow{g} B)$.
     A cone for $D$ is an object $P$ and three arrows $\phi:P \rightarrow A$, $\psi: P \rightarrow B$, and $h: P \rightarrow C$, but the latter is uniquely determined by the other ones ($f \circ \phi = h = g \circ \psi$).
    Thus, the following diagram is a cone:
    \[
        \begin{tikzcd}
            P \arrow[r, "{\psi}"] \arrow[d, "{\phi}" swap] & B \arrow[d, "g"] \\
            A \arrow[r, "f" swap] & C
        \end{tikzcd}
    \]
     Then, the limit of $D$ is called \emph{pullback} of $f$ and $g$.
     Given a span $S = (l, X, r): L \rightharpoonup R$, shown in the diagram below,
    \[
        \begin{tikzcd}
            L & X \arrow[l, "l"swap] \arrow[r, "r"]& R
        \end{tikzcd}
    \]
    a cocone for $S$ is any commutative square of the form
    \[
        \begin{tikzcd}
            & C & \\
            L \arrow[ur, "f"] &
            X \arrow[l, "l"swap] \arrow[r, "r"]&
            R \arrow[ul, "g" swap]
        \end{tikzcd}
    \]
    (the morphism $X \rightarrow C$ is uniquely determined by the relation $f \circ l = g \circ r$).
    The colimit for $S$ is called \emph{pushout} of $l$ and $r$.
\end{definition}

\begin{example}
    In $\Set$, given two functions $f: A \rightarrow C$ and $g: B \rightarrow C$, a pullback of $f$ and $g$ exists and is exactly the set $P = \{(x, y) \in A \times B \mid f(x) = g(y)\}$, with $\pi_f : P \rightarrow B$ and $\pi_g : P \rightarrow C$ defined, respectively, by $\pi_f((x, y)) = y$ and $\pi_g((x, y)) = x$. In this way, we have then, $\forall (x, y) \in P$:
    \begin{align*}
        (f \circ \pi_g) ((x, y))
                    &= f(\pi_g((x, y)))     &&  \\
                    & = f(x)                &&  \textit{Definition of $\pi_g$} \\
                    & = g(y)                &&  (x, y) \in P \\
                    & = g(\pi_f((x, y)))    &&  \textit{Definition of $\pi_f$} \\
                    & = (g \circ \pi_f) ((x, y)) && 
    \end{align*}
    thus, $f \circ \pi_g = g \circ \pi_f$.
\end{example}

Another important example to our aims is a concrete definition of what is a pushout in the category of sets, and why morally we can regard a pushout as \textit{the way to identify part of an object with a part of another}~\cite{Barr_Wells_1995}.

\begin{example}\label{ex:po_in_set}
    \color{blue}
    In $\Set$, given two functions $f: A \rightarrow B$ and $g: A \rightarrow C$, the pushout of them is the set $X = (B \amalg C) /_\sim$, where $\sim$ is the least equivalence relation such that $f(a) \sim g(a)$ for each $a \in A$, with $\iota_g:B \rightarrow X$ and $\iota_f : C \rightarrow X$ as arrows, sending each element of the domain in the corresponding equivalence class in $X$. In particular, for each $a \in A$:
    \begin{align*}
        (\iota_g \circ f) (a)
                        &= \iota_g(f(a))    &&\\
                        &= [(f(a), 0)]           && \textit{Definition of $\iota_g$} \\
                        &= [(g(a), 1)]           && f(a) \sim g(a) \\
                        &= \iota_f(g(a))    && \textit{Definition of $\iota_f$} \\
                        &= (\iota_f \circ g) (a) &&
    \end{align*}
    % This is not sure!!
    When both $f$ and $g$ are monos (that is, injections), then we can construct the pushout in the same way we have done above, with $(f(a), 0) \sim (g(a), 1)$ when such $a$ exists and $(b, 0) \sim (c, 1)$ on each $b$ and $c$ with no preimage in $A$, with $\iota_f$ and $\iota_g$ injective.
    An easy way to see this fact is considering the following situation: let $f: A \rightarrow A \cup B$ and $g: A \rightarrow A \cup C$, with $A$ disjoint from $B$ and $C$, $f(a)= a$ and $g(a) = a$. Then the pushout is the object $A \cup B \cup C$, with the inclusions as arrows, that are also injective.
    A more general case is what happens considering functions $f: A \rightarrow B$ and $g: A \rightarrow C$ injective. Differently from the previous example, in this case is not possible to take just the union of codomains as the pushout, but rather the disjoint union of them and then identify the elements $f(a)$ with $g(a)$, as we have done above. In the category of sets and functions, we have the certainty that the pullback arrows are injective. In fact, taking the equivalence relation $\sim$, we have that $f(a) \sim f(a')$ if and only if $a = a'$ by hypothesis, and then $x \sim x'$ if and only if $x = x'$, then the pushout morphisms sends each element in an equivalence class composed only by the element itself, thus are injective.
    This is an interesting property that in other categories may do not hold, and will be recalled later.
\end{example}

Given a subclass of morphisms of a category, an important property is \emph{stability} under certain type of constructions. In our case, we are interested in stability under pullbacks and under pushouts.
\[
    \begin{tikzcd}\label{diag:p_square}\tag{$\ast$}
        A \arrow[r, "f"] \arrow[d, "m"swap] & B \arrow[d, "n"] \\
        C \arrow[r, "g" swap] & D
    \end{tikzcd}
\]

\begin{definition}[Stability under pullbacks, pushouts]\label{def:stab_under_pb_po}
    Given a category $\cat C$, a subclass $\mathcal{A} \subseteq \Hom(\cat C)$ is said to be \emph{stable under pullbacks} if, for every pullback square as the one in \eqref{diag:p_square}, if $n \in \mathcal{A}$, then $m \in \mathcal{A}$.
    $\mathcal A$ is said to be \emph{stable under pushouts} if, for every pushout square as the one in \eqref{diag:p_square}, if $m \in \mathcal{A}$, then $n \in \mathcal{A}$.
\end{definition}

\begin{prop}\label{prop:monos_pres_by_pullback}
    Let $f: A \rightarrow C$, $g: B \rightarrow C$ be arrows in any category $\cat C$, and consider the following pullback square:
    \[
        \begin{tikzcd}
            P \arrow[r, "{\pi_f}"] \arrow[d, "{\pi_g}"swap] & B \arrow[d, "g"] \\
            A \arrow[r, "f"swap] & C
        \end{tikzcd}
    \]
    If $g$ is mono, then so is $\pi_g$.
\end{prop}

The proposition above can be dualised stating that pushouts preserves epimorphisms.

The connection between constructions as products and equalizers and limits is made clear by the following theorem. The idea behind the proof is the fact that, given a diagram $D : \cat I \rightarrow \cat C$, if each subset of objects $X = \{D(i) \mid i \in \Ob(\cat I)\} \subseteq \Ob(\cat C)$ has a product $(\prod_{i \in I} D(i), (\pi_i)_{i \in \Ob( \cat I)})$ and each pair of arrows $f, g \in \cat C (D(i), D(j))$ has an equalizer $Eq(f, g)$, then one can construct the cone taking the equalizer of the arrows that has as domain the product of the objects of the diagram, and as codomain the product of the codomains of the arrows of the diagram. This construction has the universal property because equalizers and products do. A detailed proof is in the appendix.

\begin{theorem}[Limit theorem]\label{th:limit}
    Let $\cat C$ be a category. Then $\cat C$ has all finite limits if and only if $\cat C$ has all finite products and all finite equalizers.
\end{theorem}

\begin{remark}
    The theorem above (and its relative proof) can be stated in its dual form leading to a theorem on existence of colimits, and a relative criterion to calculate them (taking the dual of the proof).
\end{remark}

\begin{example}\label{ex:lim_of_sets}
    Limit theorem gives us an easy way to calculate limits. An example of this fact is how limits are computed in $\Set$. Given a diagram $D: \cat{I} \rightarrow \Set$, where $\cat I$ is a small category and $I = Ob(\cat I)$, its limit is the set $L$ defined as follows:
    $$
        L = \{ (d_i)_{i \in I} \in \prod_{i \in I}D(i) \mid \forall \phi \in \cat I(i, i'), D(\phi)(d_i) = d_{i'} \}
    $$
    with projections as arrows.
\end{example}

\begin{example}\label{ex:colm_of_sets}
    As we have done in \Cref{ex:lim_of_sets}, we illustrate how to construct colimits in the category of sets. Given a small category $\cat I$, $ I = Ob(\cat I)$, and a diagram $D: \cat I \rightarrow \Set$, consider the equivalence relation $\sim$ defined on $\coprod_{i\in I} D(i)$ such that $d_i \sim d_{i'}$ if $d_i \in D(i), d_{i'} \in D(i')$ and there exists some $\phi \in \cat I(i, i')$ such that $D(\phi)(d_i) = d_{i'}$. Then, a colimit for $D$ is the set
    $$
        C = \big ( \coprod_{i \in I} D(i) \big ) / \sim
    $$
    with the inclusions as arrows.
\end{example}

\begin{remark}
    Since a diagram is nothing more than a functor from a ``shape'' category to another, it makes sense to talk about limits of functors in general, even when they are not intended to be diagrams.
\end{remark}

\begin{obs}\label{obs:limits_in_presh}
    So far we introduced categories of presheaves. In these categories, an interesting fact is that limits and colimits are computed pointwise -- i.e., the limit of a diagram in a category of presheaves is exactly the limit on each of its components.
\end{obs}

In the next sections, we will work on a special kind of diagrams with certain properties. In particular, we are interested in how a functor behaves with respect to the constructions defined so far.

\begin{definition}
    Let $D : \cat{I \rightarrow C}$ be a diagram, and $F: \cat{C \rightarrow D}$ a functor. We say that $F$:
    \begin{enumerate}
        \item \emph{preserves limits} of $D$ if, given a limit $(L, l_i)_{i \in \cat I}$ for $D$, then $(F(L), F(l_i))_{i \in \cat I}$ is a limit for $F \circ D$.
        \item \emph{reflects limits} of $D$ if a cone $(L, l_i)_{i \in \cat I}$ is a limit for $D$ whenever $(F(L), F(l_i))_{i \in \cat I}$ is a limit for $F \circ D$.
        \item \emph{lifts limits (uniquely)} of $D$ if, given a limit $(L, l_i)_{i \in \cat I}$ for $F \circ D$, there exists a (unique) limit $(L', l_i')_{i \in \cat I}$ for $D$ such that $(F(L'), F(l_i'))_{i \in \cat I} = (L, l_i)_{i \in \cat I}$.
        \item \emph{creates limits} of $D$ if $D$ has a limit and $F$ preserves and reflects limits along it.
    \end{enumerate}
    The dual notions are obtained in the obvious way, namely, substituting the words ``limits'' and ``cones'' with ``colimits'' and ``cocones'', respectively
\end{definition}

\begin{obs}\label{obs:funct_creat_lim_then_lift}
    It holds that if a functor creates limits, then lifts uniquely limits~\cite{Adamek_Herrlich_Strecker_2009}.
\end{obs}

\begin{prop}\label{prop:inc_funct_reflects_so_limits}
    A fully faithful functor reflects all limits and colimits.
\end{prop}

The next theorem is about a particular property that adjoint functors have.

\begin{theorem}\label{th:adjoints_preserves_lim}
    Let $F: \cat{C \rightarrow D}$ be a functor, and $G: \cat{D \rightarrow C}$ its right adjoint. Then, $G$ preserves limits.
\end{theorem}

\begin{remark}
    The dual of the theorem above states that, if $G$ is a functor and $F$ is a left adjoint, then $F$ preserves colimits.
\end{remark}



\section{Adhesivity}\label{sect:adh}

The next section is about adhesivity.
An adhesive category is intuitively a category in which pushouts of (some) monomorphisms exist and they behave more or less as they do among sets. 

\begin{definition}(Van Kampen property)
    Let $\mathcal A$ be a subclass of $\Hom(\cat C)$, and consider the diagram below:
    % \[
    % \begin{tikzcd}
    %     A \arrow[r, "f"] \arrow[d, "m" swap] & B \arrow[d, "n"] \\
    %     C \arrow[r, "g" swap]                & D
    % \end{tikzcd}
    % \]
	\iffalse
    \[
    \begin{tikzcd}[row sep = scriptsize]
        A' \arrow[ddd, "m'"swap] \arrow[rrr, "f'"] \arrow[dr, "a"swap] & & & B \arrow[ddd, "n'"] \arrow[dl, "b"] \\
        & A \arrow[r, "f"] \arrow[d, "m" swap] & B \arrow[d, "n"]   & \\
        & C \arrow[r, "g" swap]                & D                  & \\
        C' \arrow[rrr, "g'"swap] \arrow[ur, "c"] & & & D' \arrow[ul, "d" swap] 
    \end{tikzcd}
    \]
	\fi
    \[\begin{tikzcd}[row sep=25, column sep=25]
	& {A'} && {B'} \\
	{C'} && {D'} \\
	& A && B \\
	C && D
	\arrow["{f'}", from=1-2, to=1-4]
	\arrow["{m'}"', from=1-2, to=2-1]
	\arrow["a"'{pos=0.7}, from=1-2, to=3-2]
	\arrow["{n'}"', from=1-4, to=2-3]
	\arrow["b", from=1-4, to=3-4]
	\arrow["{g'}"{pos=0.7}, from=2-1, to=2-3, crossing over]
	\arrow["c"', from=2-1, to=4-1]
	\arrow["f"'{pos=0.3}, from=3-2, to=3-4]
	\arrow["d"{pos=0.3}, from=2-3, to=4-3, crossing over]
	\arrow["m", from=3-2, to=4-1]
	\arrow["n", from=3-4, to=4-3]
	\arrow["g"', from=4-1, to=4-3]
    \end{tikzcd}\]
    we say that the inner square is an \emph{$\mathcal A$-Van Kampen} square if:
    \begin{itemize}
        \item it is a pushout;
        \item $a, b, c, d \in \mathcal{A}$;
%        \item whenever the top and the left squares are pullbacks then the outer square is a pushout if and only of the right and the bottom squares are pullbacks.
	\item whenever the back and the left squares of the cube are pullbacks, then the top square is a pullback if and only if the front and the right squares are pullbacks.
    \end{itemize}
	when in the last requirement holds only the \textit{if} part, we say that the category is $\mathcal{A}$-stable. Moreover, $\Hom(\cat C)$-Van Kampen squares are said to be \emph{Van Kampen}, and $\Hom(\cat C)$-stable squares are called \emph{stable}.
\end{definition}

We are now ready to give the notion of $\mathcal M$-adhesivity. % CITE: https://web3.arxiv.org/pdf/2407.06181

\begin{definition}[$\mathcal{M}$-adhesivity]\label{def:adh}
    Let $\cat C$ be a category and $\mathcal M \subseteq \Mono(\cat C)$ containing all isomorphisms, closed under composition and stable under pullbacks and pushouts (\Cref{def:stab_under_pb_po}).
    Then $\cat C$ is \emph{$\mathcal M$-adhesive} if
    \begin{enumerate}
        \item every cospan $C \xrightarrow[]{g} D \xleftarrow[]{m} B$ with $m \in \mathcal M$ can be completed to a pullback (such pullbacks are called $\mathcal M$-pullbacks);
        \item every span $C \xleftarrow{m} A \xrightarrow{f} B$ with $ m \in \mathcal M$ can be completed to a pushout (such pushouts are called $\mathcal M$-pushouts);
        \item pushouts along $\mathcal M$-arrows are $\mathcal M$-Van Kampen squares.
    \end{enumerate}
	$\cat C$ is said to be \emph{strictly $\mathcal{M}$-adhesive} if $\mathcal{M}$-pushouts are Van Kampen sqaures.
	We also say that $\cat C$ is \emph{adhesive} when it is strictly $\Mono(\cat C)$-adhesive, and \emph{quasiadhesive} when it is strictly $\Reg(\cat C)$-adhesive.% \todo{NO, questo è sbagliato: bisogna introdurre la nozione di adesività forte (cioè VK vale per tutti i cubi)}
\end{definition}



    %We also say that $\cat C$ is \emph{adhesive} when it is $\Mono(\cat C)$-adhesive, and \emph{quasiadhesive} when it is $\Reg(\cat C)$-adhesive. \todo{NO, questo è sbagliato: bisogna introdurre la nozione di adesività forte (cioè VK vale per tutti i cubi)}
\begin{obs}
    $\Set$ is adhesive.
    % It is easy to check the first two points of the \Cref{def:adh} while for the Van Kampen property of pushouts along monos [...]
\end{obs}

Here it follows an interesting property of adhesive categories~\cite{lack2011embeddingtheoremadhesivecategories}.

\begin{prop}\label{prop:monos_are_preserved_by_pullbacks_in_adh_cats}
    In any adhesive category, the pushout of a monomorphism along any morphism is a monomorphism, and the resulting square is also a pullback. %\todo{almeno la seconda parte devi provarla o citarla per le M-adesive}
	In $\mathcal{M}$-adhesive categories, $\mathcal{M}$-pushouts are pullbacks.
\end{prop}

\begin{proof}
	Let the following square be a pushout, with $m$ mono.
	\[\begin{tikzcd}[row sep = 26, column sep = 26]
		A \ar[r, "f"] \ar[d, "m"swap] & B \ar[d, "n"] \\
		C \ar[r, "g"swap] & D
	\end{tikzcd}\]
	Consider now the following cube.
	\[\begin{tikzcd}[row sep = 20, column sep = 20]
	& A && B \\
	A && B \\
	& A && B \\
	C && D \\
	& {}
	\arrow["f", from=1-2, to=1-4]
	\arrow["{id_A}"', from=1-2, to=2-1]
	\arrow["{id_A}"'{pos=0.7}, from=1-2, to=3-2]
	\arrow["{id_B}"', from=1-4, to=2-3]
	\arrow["{id_B}", from=1-4, to=3-4]
	\arrow["m"', from=2-1, to=4-1]
	\arrow["f"{pos=0.7}, from=3-2, to=3-4]
	\arrow["m"', from=3-2, to=4-1]
	\arrow["n", from=3-4, to=4-3]
	\arrow["g"', from=4-1, to=4-3]
	\arrow["f"{pos=0.7}, from=2-1, to=2-3, crossing over]
	\arrow["n"'{pos=0.7}, from=2-3, to=4-3, crossing over]
	\end{tikzcd}\]

	We have then that the top face is a pushout, and by \Cref{prop:kermono}, the left face is a pullback. Since the bottom face is a pushout by hypothesis, we have that the front face and the right one are pullbacks (by adhesivity). Hence, if $\cat C$ is adhesive, we can conclude that the starting square (which is the front and the bottom face of the cube) is a pushout, and by \Cref{prop:kermono}, since the left face is a pullback, $n$ is mono. If $\cat C$ is $\mathcal M$-adhesive, with $m \in \mathcal{M}$ (that is, the square is a $\mathcal{M}$-pushout), $n \in \mathcal M$, and the vertical arrows then in $\mathcal{M}$. Hence, the front face is a pullback.
\iffalse
	Suppose now that $\cat C$ is $\mathcal{M}$-adhesive, and let the following square be a $\mathcal{M}$-pushout, so $m, n \in \mathcal{M}$.
	\[\begin{tikzcd}[row sep = 26, column sep = 26]
		A \ar[r, "f"] \ar[d, "m"swap] & B \ar[d, "n"] \\
		C \ar[r, "g"swap] & D
	\end{tikzcd}\]
	Consider now the following cube.
	\[\begin{tikzcd}[row sep = 20, column sep = 20]
	& A && B \\
	A && B \\
	& A && B \\
	C && D \\
	& {}
	\arrow["f", from=1-2, to=1-4]
	\arrow["{id_A}"', from=1-2, to=2-1]
	\arrow["{id_A}"'{pos=0.7}, from=1-2, to=3-2]
	\arrow["{id_B}"', from=1-4, to=2-3]
	\arrow["{id_B}", from=1-4, to=3-4]
	\arrow["m"', from=2-1, to=4-1]
	\arrow["f"{pos=0.7}, from=3-2, to=3-4]
	\arrow["m"', from=3-2, to=4-1]
	\arrow["n", from=3-4, to=4-3]
	\arrow["g"', from=4-1, to=4-3]
	\arrow["f"{pos=0.7}, from=2-1, to=2-3, crossing over]
	\arrow["n"'{pos=0.7}, from=2-3, to=4-3, crossing over]
	\end{tikzcd}\]
	By the same argument as before, we can conclude that the starting square is a pullback.\fi
\end{proof}

Verifying $\mathcal M$-adhesivity using the definition above may turn out to be very complex, so we can make use of the following result~\cite{castelnovo2022newcriterionmathcalmmathcalnadhesivity}. 

\begin{theorem}\label{th:crit_for_adh}
    Let $\cat C$ be a category, $\mathcal M \subseteq \Mono(\cat C)$ containing all isomorphisms, closed under composition and stable under pullbacks and pushouts. Let now $F: \cat{C \rightarrow D}$ be a functor with $\cat D$ $\mathcal{N}$-adhesive for some $\mathcal{N} \subseteq \Mono(\cat D)$.
    If $F$ is such that $F(\mathcal{M}) \subseteq \mathcal N$ and creates pullbacks and $\mathcal{M}$-pushout, then $\cat C$ is $\mathcal M$-adhesive.
\end{theorem}

The idea behind this theorem is to simplify calculations to show that a certain category is adhesive for some subclass of monomorphisms, considering a functor from the category of which we want to prove adhesivity to a category we know it is adhesive, requiring that such functor has some properties.

\begin{proof}
    In order to prove $\mathcal M$-adhesivity of $\cat C$, we have to verify the condition in \Cref{def:adh}.
    \begin{itemize}
        \item Let $C \xrightarrow[]{g} D \xleftarrow[]{m} B$ with $m \in \mathcal M$ be a cospan in $\cat C$. Applying $F$, we obtain $F(C) \xrightarrow[]{F(g)} F(D) \xleftarrow[]{F(m)} B$, with $F(m) \in \mathcal{N}$ by hypothesis. Then, there exists a pullback $(P_F, p_{F(B)}, p_{F(D)})$ in $\cat D$, which is an $\mathcal N$-pullback (\Cref{def:pullback_pushout}). Since $F$ creates pullbacks, hence lifts them (\Cref{obs:funct_creat_lim_then_lift}), there exist a pullback $(P, p_B, p_D)$ in $\cat C$.
        \item Let $C \xleftarrow{m} A \xrightarrow{f} B$ with $ m \in \mathcal M$ be a cospan in $\cat C$. Analogously to the previous point, applying the functor $F$ we obtain $F(C) \xleftarrow{F(m)} F(A) \xrightarrow{F(f)} F(B)$ with $ F(m) \in \mathcal N$, and there exists a $\mathcal N$-pushout $(q_{F(C)}, q_{F(B)}, F(Q))$ in $\cat D$. Since $F$ reflects pushouts, $(q_C, q_B, Q)$ is a $\mathcal{M}$-pushout in $\cat C$.
        \item the Van Kampen property of $\mathcal M$-pullbacks follows from the closure under pullbacks and pushouts of $\mathcal M$ and from the fact that $F$ reflects pullbacks.
    \end{itemize}
    
\end{proof}

\begin{cor}\label{cor:adhesivity_functor_categories}
    Let $\cat A$ be a $\mathcal M$-adhesive category for some $\mathcal M \subseteq \Hom(\cat A)$. Then, if $\cat A$ is a category whith all the pullbacks, the functor category $[\cat C, \cat A]$ is $\mathcal M^{\cat C}$-adhesive, where
    \[
        \mathcal{M}^\cat C = \{\eta \in \Hom([\cat C, \cat A]) \mid \eta_C \in \mathcal M \textit{ for each object $C$ of $\cat C$ }\}
    \]
\end{cor}

Since $\Set$ is adhesive, we can conclude what follows.

\begin{cor}\label{cor:presh_adhesive}
    Every category of presheaves is adhesive.
\end{cor}



\begin{lemma}\label{lemma:pushouts_kernel_pairs}
	Let $\cat C$ be a strict $\mathcal{M}$-adhesive category with all pullbacks, and suppose that in the cube below the top face is an $\mathcal{M}$-pushout.
	\[\begin{tikzcd}[row sep=25, column sep=25]
		& {A'} && {B'} \\
		{C'} && {D'} \\
		& A && B \\
		C && D
		\arrow["{f'}", from=1-2, to=1-4]
		\arrow["{m'}"', from=1-2, to=2-1]
		\arrow["a"'{pos=0.7}, from=1-2, to=3-2]
		\arrow["{n'}"', from=1-4, to=2-3]
		\arrow["b", from=1-4, to=3-4]
		\arrow["{g'}"{pos=0.7}, from=2-1, to=2-3, crossing over]
		\arrow["c"', from=2-1, to=4-1]
		\arrow["f"'{pos=0.3}, from=3-2, to=3-4]
		\arrow["d"{pos=0.3}, from=2-3, to=4-3, crossing over]
		\arrow["m", from=3-2, to=4-1]
		\arrow["n", from=3-4, to=4-3]
		\arrow["g"', from=4-1, to=4-3]
	\end{tikzcd}\]
	
	Then, the square below is a pushout.
	\[
	\begin{tikzcd}%[row sep=25, column sep=25]
		{K_a} \ar[r, "{k_{f'}}"] \ar[d, "{k_{m'}}"'] & {K_b} \ar[d, "{k_{n'}}"] \\
		{K_c} \ar[r, "{k_{g'}}"'] & {K_d}
	\end{tikzcd}
	\]
\end{lemma}

\begin{proof}
	Since $f'$, being the pullback of $f$, is in $\mathcal{M}$ then by  \Cref{prop:monos_are_preserved_by_pullbacks_in_adh_cats} the bottom face of the cube is a pullback. Thus \Cref{lemma:kern_pairs_pres_pullbacks} entails that in the following cube the vertical faces are pullbacks.
	\[\begin{tikzcd}[row sep = 20, column sep = 20]
		& {K_a} && {K_b} \\
		{K_c} && {K_d} \\
		& {A'} && {B'} \\
		{C'} && {D'}
		\arrow["{k_{f'}}", from=1-2, to=1-4]
		\arrow["{k_{m'}}"', from=1-2, to=2-1]
		\arrow["{\pi_a^1}"{pos=0.7}, from=1-2, to=3-2]
		\arrow["{k_{n'}}", from=1-4, to=2-3]
		\arrow["{\pi_b^1}", from=1-4, to=3-4]
		\arrow["{k_{g'}}"{pos=0.7}, from=2-1, to=2-3, crossing over]
		\arrow["{\pi_c^1}"', from=2-1, to=4-1]
		\arrow["{f'}"'{pos=0.3}, from=3-2, to=3-4]
		\arrow["{\pi_d^1}"{pos=0.7}, from=2-3, to=4-3, crossing over]
		\arrow["{m'}", from=3-2, to=4-1]
		\arrow["{n'}", from=3-4, to=4-3]
		\arrow["{g'}"', from=4-1, to=4-3]
	\end{tikzcd}\]
	
Now the thesis follows from strong $\mathcal{M}$-adhesivity.
\end{proof}




\renewcommand{\graph}[1]{\mathcal{#1}}
\newcommand{\Graph}{\mathbf{Graph}}
\newcommand{\EqGrph}{\mathbf{EqGrph}}
\newcommand{\Q}{Q} % Quotient functor

\chapter{Categories of Graphs}
This chapter is about graphs, and how it is possible to formalize them using categories in order to point out their properties from an abstract point of view. Starting from the set-theoretical definition of graphs, we will give an abstraction via functor categories, in which a graph is nothing but a functor between a categories.

%\section{Graphs}\label{sect:graphs}

A (directed graph) $\graph{G}$ is a mathematical structure consisting of a set of edge, a set of nodes and two functions, one assigning a \emph{source} node and one assigning a \emph{target} node to an edge. Formally, $\graph{G}$ is a quadruple $(V_\graph{G}, E_\graph{G}, s_\graph{G}, t_\graph{G})$, where $V_\graph{G}$ is the set of nodes, $E_\graph{G}$ is the set of edges, and $s_\graph{G}, t_\graph{G}: E_\graph{G} \rightarrow V_\graph{G}$ are the source and the target functions.

A \emph{graph homomorphism} $h: \graph{G \rightarrow H}$ is then a pair of functions $h = (h_V: V_\graph{G} \rightarrow V_\graph{H}, h_E: E_\graph{G} \rightarrow E_\graph{H})$ such that
    \[
        h_V \circ s_\graph{G} = s_\graph{H} \circ h_E
    \]
    and
    \[
        h_V \circ t_\graph{G} = t_\graph{H} \circ h_E
    \]
that is, a structure preserving map.

We can then generalize such notion to something more abstract, considering a graph to be nothing more than a presheaf from the category $(E \rightrightarrows V)$ to the category of sets.
Having two of such presheaves, a natural transformation from one to another encapsulates the behavior of a graph morphism due to naturality. We can now define the category of graphs.

\begin{definition}[Category of Graphs]\label{def:cat_of_graph}
    We denote as $\Graph$ the category $$[E \mathrel{\mathop{\rightrightarrows}^{s}_{t}} V , \Set]$$
\end{definition}

\todo{Dimostrare come si calcolano i limiti nelle categorie di prefasci (conponente per componente) e poi dare qualche esempio, porendendolo dalla versione precedente. Dimostrare anche come sono fatti i mono (mono sulle componenti).}
\begin{remark}
    TODO: Si può generalizzare a tutte le categorie regolari per evitare di perdere le proprietà che usiamo (da eq.rel. a quot.).
\end{remark}
%\color{blue}

\section{Graphs with equivalences - notes and results}

\subsection{Some results on kernel pair and regular epis}
\todo{Questi risultati vanno completati e distribuiti lungo la tesi: quello sui kernel pair nella parte sui pullback, quello sugli epi regolari nella parte sui coequalizzatori, quelli che riguardano l'adesività nella relativa sezione}

\todo{Sicuramente devi anche mettere dei riferimenti ai risultati della sez. sull'adesività (tipo alla proposizione che gli M-pushout sono pullback)}

\begin{lemma}\label{lem:pb1}
	Suppose that the following diagram is given and its right half is a pullback. Then the whole rectangle is a pullback if and only if its left half is a pullback.
	\[\xymatrix{X\ar[r]^{f} \ar[d]_{t} & Y \ar[d]^{k}\ar[r]^{g} &Z \ar[d]^{h}\\A \ar[r]_{a}& B \ar[r]_{b} & C}\]
\end{lemma}
\begin{proof}
	\todo{Esercizio per te}
\end{proof}

\begin{cor}\label{cor:cube} Let $\cat{C}$ be a category and suppose that  the solid part of the following cube is given
	\[\xymatrix@C=13pt@R=13pt{&Y'\ar[dd]|\hole_(.65){y}\ar[rr]^{g'} \ar@{.>}[dl]_{q'} && Z' \ar[dd]^{z} \ar[dl]_{r'} \\ B'  \ar[dd]_{b}\ar[rr]^(.65){k'} & & C' \ar[dd]_(.3){c}\\&Y\ar[rr]|\hole^(.65){g} \ar[dl]^{q} && Z \ar[dl]^{r} \\B \ar[rr]_{k} & & C}\]
	If the front face is a pullback then there is a unique $q'\colon Y'\to B'$ filling the diagram. If, moreover, the other two vertical faces are also pullbacks, then the following square is a pullback too.
	\[\xymatrix{Y' \ar[r]^{q'} \ar[d]_{y}& B'\ar[d]^{b}\\ Y \ar[r]_{q} & B}\]
\end{cor}
\begin{proof}
	Let us compute:
	\begin{align*}
		c\circ r'\circ g'&=r\circ z \circ g'\\&=r\circ g \circ y\\&=k\circ q \circ y
	\end{align*}
	Since the front face is a pullback, this guarantees the existence of $q'$.  The second half of the thesis follows applying \cref{lem:pb1} to the following rectangle.
	\[\xymatrix{Y'\ar@/^.4cm/[rr]^{r'\circ g'} \ar[r]_{q'} \ar[d]_{y}& B'\ar[d]_{b}\ar[r]_{k'}& C'\ar[d]^{c}\\ Y\ar@/_.4cm/[rr]_{r\circ g} \ar[r]^{q} & B \ar[r]^{k}& C}\]
\end{proof}


\iffalse 
In an $\mathcal{M}$-adhesive category, pullbacks also enjoy a kind of left cancellation property.

\begin{lemma}\label{lem:pb2}
	Let $\cat{C}$ be a category with pullbacks, given the following diagrams:
	\[
	\xymatrix{Y\ar[r]^{f_2} \ar[d]_{f_1} & X_2 \ar[d]^{r_2} & Z_1 \ar[d]_{x_1}\ar[r]^{z_1} & W \ar[r]^{w} \ar[d]_{r} & Q'\ar[d]^{q} & Z_2 \ar[d]_{x_2} \ar[r]^{z_2}  & W  \ar[r]^{w} \ar[d]_{r}  & Q' \ar[d]^{q}\\ X_1 \ar[r]_{r_1} &R  & X_1 \ar[r]_{r_1} & R \ar[r]_{s}  & Q& X_2 \ar[r]_{r_2} & R \ar[r]_{s} & Q}\]
	if the first square is a stable pushout and the whole rectangles and their left halves are pullbacks, then their common right half is a pullback too.
\end{lemma}
\begin{proof}Pulling back  $q$ along $s$ we get a square 
	\[\xymatrix{U \ar[r]^{u} \ar[d]_{h}& Q' \ar[d]^{q}\\ R \ar[r]_s & S}\]
	Notice that
	\[
	q\circ w\circ z_1=s\circ r_1\circ x_1 \qquad 
	q\circ w\circ z_2=s\circ r_2\circ x_2 \]
	Thus we get $u_1\colon Z_1\to U$ and $u_2\colon Z_2\to U$ fitting in the rectangles
	\[\xymatrix{  Z_1  \ar@/^.4cm/[rr]^{w\circ z_1}\ar[r]_{u_1}\ar[d]_{x_1}&  U \ar[d]_h \ar[r]_{u} & Q'  \ar[d]^{q}&  Z_2 \ar@/^.4cm/[rr]^{w\circ z_2} \ar[r]_{u_2}\ar[d]_{x_2} &  U \ar[d]_h \ar[r]_{u}& Q' \ar[d]^{q} \\  X_1 \ar[r]_{r_1} & R \ar[r]_s& Q & X_2 \ar[r]_{r_2}& R \ar[r]_s & Q}\]
	which, by hypothesis and  \cref{lem:pb1} have left halves which are pullbacks. Now,
	\[s\circ r_1\circ f_1 =s\circ r_2\circ f_2\]
	Pulling back $q$ along this arrow we get another square
	\[\xymatrix@C=40pt{Z'_0 \ar[r]^{t} \ar[d]_{y}& Q' \ar[d]^{q}\\ R \ar[r]_{s\circ r_1\circ f_1} & S}\]
	In particular, we obtain the dotted $b_1\colon Z'_0\to Z_1$ and $b_2\colon Z'_0\to Z_2$ in
	\[\xymatrix@C=30pt{ Z'_0 \ar@/^.4cm/[rrr]^{t}\ar[d]_{y} \ar@{.>}[r]_{b_1} & Z_1  \ar[r]_{u_1}\ar[d]_{x_1}&  U \ar[d]_h \ar[r]_{u} & Q'  \ar[d]^{q} &Z'_0 \ar@/^.4cm/[rrr]^{t} \ar[d]_{y}\ar@{.>}[r]_{b_2}  & Z_2  \ar[r]_{u_2}\ar[d]_{x_2} &  U \ar[d]_h \ar[r]_{u}& Q' \ar[d]^{q} \\ Y \ar[r]_{f_1}& X_1 \ar[r]_{r_1} & R \ar[r]_s& Q & Y \ar[r]_{f_2} & X_2 \ar[r]_{r_2}& R \ar[r]_s & Q}\]
	in which, using again \cref{lem:pb1}, all of the squares on the bottom rows are pullbacks. 
	
	We are going to construct another row above these two rectangles. By hypothesis 
	\[q\circ w = s\circ r\]
	Thus there exists a unique $g\colon W\to U$ such that
	\[r=h\circ g \qquad w=u\circ g\]
	Moreover, we also have that
	\[\begin{split}h\circ g \circ z_1&=r \circ z_1\\&= r_1\circ x_1 \\&=h\circ u_1
	\end{split}\qquad\begin{split}h\circ g \circ z_2&=r \circ z_2\\&= r_2\circ x_2 \\&=h\circ u_2
	\end{split}\]
	and 
	\[\begin{split}u\circ g \circ z_1&=w \circ z_1\\&= u\circ u_1 
	\end{split} \qquad \begin{split}u\circ g \circ z_1&=w \circ z_2\\&= u\circ u_2
	\end{split}\]
	which together show that
	\[g\circ z_1=u_1 \qquad g\circ z_2=u_2\]
	
	Summing up, we can depict all the arrows we have constucted so far in the following diagrams
	\[\xymatrix{Z'_0 \ar[r]^{b_1} \ar[d]_{id_{Z'}}& Z_1 \ar[r]^{z_1} \ar[d]_{id_{Z_1}} &  W \ar[d]_g \ar[r]^{w} & Q' \ar[d]^{id_{Q'}}&& Z'_0 \ar[r]^{b_2}  \ar[d]_{id_{Z'}}& Z_2 \ar[d]_{id_{Z_2}} \ar[r]^{z_2}&  W \ar[d]_g\ar[r]^{w}& Q' \ar[d]^{id_{Q'}}\\ Z'_0 \ar[d]_{y} \ar[r]^{b_1} & Z_1  \ar[r]^{u_1}\ar[d]_{x_1}&  U \ar[d]_h \ar[r]^{u} & Q'  \ar[d]^{q}& &Z'_0  \ar[d]_{y}\ar[r]^{b_2}  & Z_2  \ar[r]^{u_2}\ar[d]_{x_2} &  U \ar[d]_h \ar[r]^{u}& Q' \ar[d]^{q} \\ Y \ar[r]_{f_1}& X_1 \ar[r]_{r_1} & R \ar[r]_s& Q & &Y \ar[r]_{f_2} & X_2 \ar[r]_{r_2}& R \ar[r]_s & Q}\]
	If we show that $g$ is an isomorphism we are done. Consider the cubes
	\[\xymatrix@C=13pt@R=13pt{&Z_0'\ar[dd]|\hole_(.7){y}\ar[rr]^{b_2} \ar[dl]_{b_1} && Z_2 \ar[dd]^{x_2} \ar[dl]_{z_2}  &&Z'_0\ar[dd]|\hole_(.7){y}\ar[rr]^{b_2} \ar[dl]_{b_1} && Z_2 \ar[dd]^{F_j(b)} \ar[dl]_{u_2}\\Z_1  \ar[dd]_{x_1}\ar[rr]^(.65){z_1} & & W \ar[dd]_(.3){r}&& Z_1  \ar[dd]_{x_1}\ar[rr]^(.65){u_1} & &U \ar[dd]_(.3){h}\\&Y\ar[rr]|\hole^(.65){f_2} \ar[dl]_{f_1} && X_2 \ar[dl]^{r_2} && Y\ar[rr]|\hole^(.65){f_2} \ar[dl]_{f_1} && X_2 \ar[dl]^{r_2}\\X_1 \ar[rr]_{r_1} & & R && X_1 \ar[rr]_{r_1} & & R}\]
	in which the vertical faces are pullbacks. Since the bottom face is a stable pushout we can deduce that
	\[\xymatrix{Z'_0 \ar[d]_{b_1} \ar[r]^{b_2}&  Z_2 \ar[d]^{z_2} & Z'_0 \ar[d]_{b_1} \ar[r]^{b_2}&  Z_2 \ar[d]^{u_2}\\ Z_1 \ar[r]_{z_1}& W & Z_1 \ar[r]_{u_1}& U }\]
	are pushout squares too. The arrow $g$ fits in the following  diagram
	\[\xymatrix{Z'_0\ar[r]^{b_2} \ar[d] _{b_1}& Z_2 \ar@/^.3cm/[ddr]^{u_2}\ar[d]^{z_2} \\ Z_1 \ar[r]_{z_1}  \ar@/_.3cm/[drr]_{u_1}& W \ar[dr]^{g} \\ &&U}\]
	and thus it is an isomorphism.
\end{proof} 
\fi 

\begin{definition}A \emph{kernel pair} $(K_f, p_{f, 1}, p_{f,2})$ for an arrow $f\colon X\to Y$ is an object $K_f$ with two arrows $p_{f,1}, p_{f, 2}\colon K_f\to X$ making the following square a pullback.
	\[\xymatrix{K_f \ar[r]^{p_{f,1}} \ar[d]_{p_{f,2}}& X \ar[d]^{f}\\ X \ar[r]_{f} & Y}\]
\end{definition}

\begin{remark}
	If a category $\cat{C}$ has pullbacks then every arrow has a kernel pair.
\end{remark}


\begin{prop}\label{prop:kermono}
	An arrow $m\colon M\to X$ is mono if and only if $(M, id_M, id_M)$ is a kernel pair for it.
\end{prop}
\begin{proof}\todo{esercizio}
\end{proof}

\begin{cor}\label{cor:kermono}
	Let $(K_f, p_{f,1}, p_{f,2})$ be a kernel pair for $f\colon X\to Y$. Then for every mono $m\colon Y\to Z$, $(K_f, p_{f,1}, p_{f,2})$ is a kernel pair also for $m\circ f$.
\end{cor}
\begin{proof}
	It is enough to see that, by \Cref{lem:pb1,prop:kermono} the outer boundary of the following square is a pullback.
	\[\xymatrix{K_f\ar[r]^{p_{f,1}}  \ar[d]_{p_{f,2}}& X \ar[r]^{id_X} \ar[d]_{f}&X\ar[d]^{f} \\X \ar[d]_{id_X} \ar[r]_{f} & Y \ar[d]_{id_Y} \ar[r]_{id_Y}& Y \ar[d]^{m}\\X \ar[r]_{f} & Y \ar[r]_{m} & Z}\]
	
\end{proof}

\begin{lemma}\label{lem:salvavita1}
Suppose that the following square is given and that $f\colon X\to Y$ and $g\colon Z\to W$ have kernel pairs.
\[\xymatrix{X \ar[d]_{f}\ar[r]^{h}& Z \ar[d]^{g} \\ Y \ar[r]_{t}& W}\]

Then there exists a unique arrow $k_h\colon K_f\to K_g$ making the squares below commutes.

\[\xymatrix{K_f \ar[d]_{p_{f, 1}}\ar@{.>}[r]^{k_h}& K_g \ar[d]^{p_{g,1}} & K_f \ar[d]_{p_{f,2}}\ar@{.>}[r]^{k_h}& K_g \ar[d]^{p_{g,2}} \\ X \ar[r]_{h}& Z & X \ar[r]_{h}& Z}\]

Moreover, if the beginning square is a pullback, then also the preceding ones are so.
\end{lemma}
\begin{proof}
	Computing we have
	\begin{align*}
		g\circ h\circ p_{f,1}&=t\circ f\circ p_{f,1}\\&=t\circ f\circ p_{f,2}\\&=g\circ h\circ p_{f,2}
	\end{align*}
	So that the wanted $k_h$ exists, and it is unique, by the universal property of $K_g$ as the pullback of $g$ along itself. 
	
	To prove the second half of the thesis, let us consider the  two rectangles below, which, by \Cref{lem:pb1} are pullbacks.
\[\xymatrix{K_f\ar[r]^{p_{f,1}}  \ar[d]_{p_{f,2}}& X \ar[r]^{h} \ar[d]^{f} & Z \ar[d]^{g} & K_f \ar[r]^{p_{f,2}}  \ar[d]_{p_{f,1}} & X \ar[d]^{f} \ar[r]^{h}& Z \ar[d]^{g}\\
	X \ar[r]_{f}& Y \ar[r]_{t}& W&  X \ar[r]_f & Y\ar[r]_{t} & W}\]
	
	But then the following ones are pullbacks too.
 	\[\xymatrix{K_f \ar@/^.4cm/[rr]^{h\circ p_{f,2}} \ar[r]_{k_h} \ar[d]_{p_{f,1}}& K_g  \ar[r]_{p_{g,2}} \ar[d]_{p_{g,1}} & Z \ar[d]^{g} & K_f \ar@/^.4cm/[rr]^{h\circ p_{f,1}} \ar[r]_{k_h} \ar[d]_{p_{f,2}}& K_g \ar[r]_{p_{g,1}} \ar[d]_{p_{g,2}} & Z \ar[d]^{g}\\X \ar@/_.4cm/[rr]_{t\circ f}\ar[r]^{h}& Z \ar[r]^{g} & W & X \ar@/_.4cm/[rr]_{t\circ f} \ar[r]^{h}& Z \ar[r]^{g}& W}\]
 	
 	The thesis now follows again by \Cref{lem:pb1}.
\end{proof}


\begin{lemma}\label{lem:salvavita2}
Let $\cat{C}$ be an $\mathcal{M}$-adhesive category with all pullbacks and suppose that the cube below is given, in which every face is a pullback and the bottom one is an $\mathcal{M}$-pushout.

 	\[\xymatrix@C=10pt@R=10pt{&A'\ar[dd]|\hole_(.65){a}\ar[rr]^{f'} \ar[dl]_{m'} && B' \ar[dd]^{b} \ar[dl]_{n'} \\ C'  \ar[dd]_{c}\ar[rr]^(.7){g'} & & D' \ar[dd]_(.3){d}\\&A\ar[rr]|\hole^(.65){f} \ar[dl]^{m} && B \ar[dl]^{n} \\C \ar[rr]_{g} & & D}\]
Then the square below is a pushout.
\[\xymatrix{K_{a} \ar[r]^{k_{f'}}  \ar[d]_{k_{m'}}& K_b \ar[d]^{k_{n'}}\\ K_c \ar[r]_{k_{g'}} & K_d}\]
\end{lemma}
\begin{proof} By \Cref{lem:salvavita1} in the following cube the vertical faces are all pullbacks. 
		\[\xymatrix@C=10pt@R=10pt{&K_a\ar[dd]|\hole_(.65){p_{a,1}}\ar[rr]^{k_{f'}} \ar[dl]_{k_{m'}} && K_b \ar[dd]^{p_{b,1}} \ar[dl]_{k_{n'}} \\ K_c  \ar[dd]_{p_{c,1}}\ar[rr]^(.7){k_{g'}} & & K_d \ar[dd]_(.3){p_{d,1}}\\&A'\ar[rr]|\hole^(.65){f'} \ar[dl]^{m'} && B' \ar[dl]^{n'} \\C' \ar[rr]_{g'} & & D'}\]
	$f'$ is in $\mathcal{M}$ as it is the pullback of $\mathcal{M}$, thus the bottom face of the cube is a Van Kampen pushout and the thesis follows.
\end{proof}


\begin{prop}\label{prop:regepi}
	Let $e\colon X\to Y$ be a regular epi in a category $\cat{C}$ with a kernel pair $p_1, p_2\colon P\rightrightarrows X$, then $e$ is the coequalizer of $p_1$ and $p_2$.
\end{prop}
\begin{proof}
	By hypothesis there exists a pair $f, g\colon Z\rightrightarrows X$ of which $e$ is the coequalizer, since $e\circ f=e\circ g$ we have a diagram
	\[\xymatrix{	Z \ar@/^.5cm/[drr]^{f} \ar@/_.5cm/[ddr]_{g} \ar@{.>}[dr]^{k}& &	\\ &P\ar[r]^{p_1} \ar[d]_{p_2} & X \ar[d]^{e} \\& X \ar[r]_{e}  & Y}\]
	and thus there exists  the dotted $k\colon Z\to P$. Let $h\colon Z\to V$ be an arrow such that $h\circ p_1=h\circ p_2$, then
	\begin{align*}h\circ f &= h \circ p_1\circ k \\&= h \circ p_2\circ k \\&=h\circ g
	\end{align*}
	and thus there exists a unique $l\colon Y\to V$ such that $l\circ e=h$.
\end{proof}


\begin{cor}\label{cor:regepi1}
Let $\cat{C}$ be a category with pullbacks and $\phi\colon D\to D'$ be a natural transformation between two functor $D, D'\colon \cat{I}\to \cat{C}$. If $\phi_i$ is a regular epi for every $i$, then $\phi$ is a regular epi.
\end{cor}
\begin{proof}
	Let $K_i$ be the kernel pair of $\phi_i$, with projections $p_{1,i}, p_{2,i}\colon K_i\rightrightarrows D(i)$. Given an arrow $f\colon i\to j$ in $\cat{I}$, we have 
	\begin{align*}
		\phi_j\circ D(f)\circ p_{1,i}&=D'(f)\circ \phi_i\circ p_{1,i}\\&=D'(f)\circ \phi_i\circ p_{2,i}\\&=\phi_j\circ D(f)\circ p_{2,i}
	\end{align*}
	
	
	Thus the outer boundary of the diagram below commutes, yielding the dotted arrows $K(f)$.
	\[\xymatrix{K_i \ar@{.>}[dr]^{K(f)}\ar[r]^{p_{1,i}} \ar[d]_{p_{2,i}}& D(i) \ar[dr]^{D(f)}\\D(i) \ar[dr]_{D(f)}&K_j \ar[r]^{p_{1,j}} \ar[d]_{p_{2,j}} & D(j) \ar[d]^{\phi_j}\\ &D(j) \ar[r]_{\phi_j} & D'(j)}\]
	
	In this way \todo{esercizio per te}we get a functor $E\colon \cat{I}\to \cat{C}$ with two natural transformations $p_{1}, p_2\colon E\rightrightarrows D$. By \Cref{prop:regepi} every component  $\phi_i$ of $\phi$ is the coequalizer of $p_{1,i}, p_{2,i}\colon E\rightrightarrows D$ and so $\phi$ is the coequalizer of $p_1$ and $p_2$.
\end{proof}


\todo{COSE DA FARE: nel capitolo 1 metti una proposizione in cui mostri che da ogni trasf naturale $D\to D'$ puoi ricavare una freccia tra i colimiti, così in questa e nelle altre proposizioni puoi citarla.}

\begin{lemma}\label{lem:regepi}
	Let $D, D'\colon \cat{I}\rightrightarrows  \cat{C}$ be two diagrams diagram with colimiting cocone $(C, \{c_i\}_{i\in \cat{I}})$ and   $(Q, \{q_i\}_{i\in \cat{I}})$. If $\cat{C}$ has all colimits for diagrams of shape $\cat{I}$ and $\phi\colon D\to D'$ is a natural transformation in which all components are regular epis, then the canonical arrow\todo{richiama prop precedente}  $c\colon C\to Q$ is a regular epi to.
\end{lemma}
\begin{proof}
	By \Cref{cor:regepi1} we know that $\phi\colon D\to D'$ is a regular epi, so that there is a functor $E\colon \cat{I}\to \cat{C}$ and $\alpha, \beta\colon E\rightrightarrows D$ such that $\phi$ is a coequalizer for $\alpha$ and $\beta$. Let $(P, \{p_i\}_{i\in \cat{I}})$ be the colimit of $E$, by ???\todo{richiama prop sulla freccia tra i colimiti} we have arrows $a, b\colon P\rightrightarrows  C$ fitting in the diagram belows:
	\[\xymatrix{E(i) \ar[r]^{p_i} \ar[d]_{\alpha_i}& P \ar@{.>}[d]^{a} & E(i) \ar[r]^{p_i} \ar[d]_{\alpha_i} & P \ar@{.>}[d]^{b}\\ D(i) \ar[r]_{c_i} & C & D(i) \ar[r]_{c_i} & C}\]
	
	We want to show that $c$ coequalizes $a$ and $b$. Let thus $t\colon C\to T$ be an arrow such that $t\circ a=t\circ b$. Then for every $i\in I$ we have 
	\begin{align*}
		t\circ c_i\circ \alpha_i&=t\circ a\circ p_i\\&=t\circ b\circ p_i\\&=t\circ c_i\circ \beta_i
	\end{align*}
	
	Thus there is $t_i\colon D'(i)\to T$ such that $t\circ c_i=t_i\circ \phi_i$. It is now easy to see that $(T, \{t_i\}_{i\in \cat{I}})$ is a cocone on $D'$ \todo{Verificalo!}.  Thus we have an arrow $k\colon Q\to T$ such that $k\circ q_i=t_i$. But then we have
	\begin{align*}
		k\circ c\circ c_i&=k\circ q_i\circ \phi_i\\&=t_i\circ \phi\\&=t\circ c_i
	\end{align*}
	Showing that $k\circ c=t$. 
	
	For uniqueness, let $k'$ be another arrow $Q\to T$ such that $k'\circ c=t$, then we have
	\begin{align*}
		k'\circ q_i\circ \phi_i&=k'\circ c\circ c_i\\&=t\circ c_i\\&=t_i\circ \phi_i
	\end{align*}
	Since $\phi_i$ is a regular epi, by ???\todo{mettere nel capitolo 1 una proposizione che mostra che epi regolari sono epi} this entails $k'\circ q_i=t_i$. By construction $k\circ q_i=t_i$ and  so $k=k'$ since $(Q, \{q_i\}_{i\in \cat{I}})$ is a colimiting cocone.
\end{proof}




\begin{definition}
	A \emph{graph with equivalence} is a 6-uple $(A, B, C, s, t, q)$ where $A, B$ and $C$ are set, $s,t\colon A\rightrightarrows B$ are functions and $q\colon B\to C$ is another surjective function.
	
	A morphism  $(A, B, C, s, t, q)\to (A', B', C', s', t', q')$ is a triple $(h_1, h_2, h_3)$ of functions $h_1\colon A\to A'$, $h_2\colon B\to B'$ and $h_3\colon C\to C'$ making the following diagrams commute.
	
	\[\xymatrix{A \ar[r]^{s} \ar[d]_{h_1} & B \ar[d]^{h_2} & A \ar[r]^{s} \ar[d]_{h_1}& B \ar[d]^{h_2} & B \ar[d]_{h_2} \ar[r]^{q} & C \ar[d]^{h_3}\\A' \ar[r]_{s'} & B' & A ' \ar[r]_{t'}& B' & B' \ar[r]_{q'} & C'}\]
	
	In this way, defining the composition componentwise, we get a category $\EqGrph$.
\end{definition}

\begin{remark}\label{rem:fedele}
	There is a faithful functor $U\colon \EqGrph\to \Graph$, forgetting the quotient part. \todo{Questo è per te da dimostrare (forse meglio come proposizione che come remark).}
\end{remark}

\begin{remark}
There is another functor $V\colon \EqGrph\to \Set$ sending $(A, B,C, s,t, q)$ to $C$ and a morphism to its last component.	
\end{remark}


\begin{prop}\label{prop:limits}
	$\EqGrph$ is complete, cocomplete and $U$ preserves limits and colimits.
\end{prop}

\begin{remark}\label{rem:ima}\todo{Dimostrala}In $\Set$ we have the following property: for every square as the one below, if $e\colon X\to Y$ is epi and $m\colon M\to Z$ is mono, then there exists a unique dotted arrow $Y\to M$ making the diagram below commutative.
	\[\xymatrix{X \ar[r]^{f} \ar[d]_{e}& M \ar[d]^{m}\\ Y \ar[r]_{g} \ar@{.>}[ur]^{h}& Z}\] 
\end{remark}


\begin{proof}\todo{Per generalizzare ad altre categorie: serve poter fattorizzare con un epi regolare.} 
	NOTATION: $D(i)$ is $(A_i, B_i, Q_i, s_i, t_i, q_i)$.	Let $(A, B, s, t)$ be the limit of $U\circ D$, with projections $(p_{1,i}, p_{2,i})\colon (A,B, s,t)\to (A_i, B_i, s_i, t_i)$. Let $(L, \{l_i\}_{i\in \cat{I}})$ be a limiting cone for $V\circ D$. 
	
	Now, notice that $(B, \{q_i\circ p_{2,i}\}_{i\in \cat{I}})$ is a cone over $V\circ D$, \todo{Dimostralo}, so that we have an arrow $l\colon B\to L$. This arrow is not epi in general, let $Q$ be its image and $q\colon B\to Q$ be the resulting epi and $m\colon Q\to L$ the corresponding mono.
	By definition the external square in the diagram below commutes, so for every $i\in \cat{I}$, \Cref{rem:ima} yields the dotted arrow $p_{3,i}$.
	\[\xymatrix{B \ar[r]^{p_{2,i}} \ar[d]_{q}& B_i \ar[r]^{q_i} & Q_i \ar[d]^{id_{Q_i}}\\ Q \ar[r]_{m} \ar@{.>}[urr]_{p_{3,i}}& L \ar[r]_{l_i} & Q_i}\]
	
	We have to show that in this way we get a cone over the diagram $D$. Let $f\colon i\to j$ be an arrow of $\cat{I}$, then we have:
	\begin{align*}
	U(D(f)\circ (p_{1,i}, p_{2, i}, p_{3,i}))  &=  U(D(f))\circ(p_{1,i}, p_{2, i})\\
                                                   &=  (p_{1,j}, p_{2, j})\\
                                                   &=  U(D(f)\circ (p_{1,j}, p_{2, j}, p_{3,j}))
	\end{align*}
	 And faithfulness of $U$ yields the thesis.
	 
	 To see that this cone is terminal, let $(E, F, G, a, b, c)$ be another graph with the vertex of a cone with sides $(t_{1,i},t_{2, i}, t_{3,i})$. By construction, we have an arrow $(t_1, t_2)\colon (E, F, a, b)\to (A, B, s, t)$ such that
	 \[\xymatrix{&E \ar@{.>}[dl]_{t_1} \ar[dr]^{t_{1,i}}&&&F \ar@{.>}[dl]_{t_2} \ar[dr]^{t_{2,i}}\\ A \ar[rr]_{p_{1,i}} & & A_i & B \ar[rr]_{p_{2,i}}&& B_i }\]
	
        Moreover $(G, \{t_{3,i}\}_{i\in \cat{I}})$ is a cone over $V\circ D$, \todo{Verificalo}, thus there exists an arrow $t\colon G\to L$ such that $l_i\circ t =t_{3,i}$. Now, precomposing with $c$ we get
	\begin{align*}
		l_i\circ t\circ c&=t_{3,i}\circ c\\&=q_i\circ t_{2,i}\\&=q_i\circ p_{2,i}\circ t_2\\&=l_i\circ l\circ t_2
	\end{align*} 
	
	Therefore the solid part of the diagram below commutes and \Cref{rem:ima} yields the dotted arrow $t_3\colon G\to Q$.
	
	\[\xymatrix{F \ar[r]^{t_2} \ar[d]_{c}& B \ar[r]^{q} & Q \ar[d]^{m}\\G \ar[rr]_{t} \ar@{.>}[urr]^{t_3}&& L}\]
	
	Faithfulness \todo{esercizio per te: scrivere la dimostrazione di queste ultime due righe}of $U$ now guarantees that $(t_1, t_2, t_3)$ is the unique arrow such that $(p_{1,i}, p_{2,i}, p_{3,i})\circ(t_1, t_2, t_3)=(t_{1,i},t_{2, i}, t_{3,i})$.	\todo{esercizio per te: fare i colimiti. Hint: la dimostrazione è diversa ma più semplice. Se $(A, B, s,t)$ è il colimite di $U\circ D$ puoi considerare il colimite $Q$ dei vari $Q_i$. Per la proprietà del colimite hai una freccia $B\to Q$, mostra che è epi (segue in una riga dai risultati che abbiamo).}
\end{proof}




\begin{cor}
	An arrow $(h_1, h_2, h_3)\colon (A, B, C, s,t, q)\to (E, F, G, a,b,c)$ in $\EqGrph$ is mono if and only if $h_1$ and $h_2$ are mono in $\Set$.
\end{cor}
\begin{proof}
	\todo{esercizio per te, usa fedeltà e continuità}
\end{proof}



\begin{cor}\label{cor:regmono}Let $(h_1, h_2, h_3)\colon (A, B, C, s,t, q)\to (E, F, G, a,b,c)$ be a morphism of $\EqGrph$, then the following are equivalent:
	\begin{enumerate}
		\item $(h_1, h_2, h_3)$ is a regular mono;
		\item $h_1$, $h_2$, $h_3$ are all monos;
		\item $h_1$ and $h_2$ are mono and for every $f, f'\in F$, $c(h_2(f))=c(h_2(f'))$ if and only if $q(f)=q(f')$.
	\end{enumerate}
\end{cor}
\begin{proof}
	$1\Rightarrow 2.$ \todo{Esercizio} 
	
	\smallskip \noindent
	$2\Rightarrow 3.$ \todo{Esercizio}
	
	\smallskip \noindent 
	$3\Rightarrow 1.$\todo{Esercizio}
\end{proof}

Let us turn to another functor $\EqGrph\to \Graph$.

\begin{definition}
The \emph{quotient functor} $Q:\EqGrph\to \Graph $ sends $(A, B, C, s,t, q)$ to $(A, C, q\circ s, q\circ t)$ and an arrow $(h_1, h_2, h_3) \colon (A, B, C, s,t, q)\to (E, F, G, a,b, c)$ to $(h_1, h_3)$.
\end{definition}

\todo{Verifica che questa cosa funziona $(h_1, h_3)$ è un morfismo di grafi?}

\begin{lemma}
	$Q$ is a left  adjoint.
\end{lemma}
\begin{proof} Let us start proving that $Q$ is a left adjoint. Let $R(A,B, s, t)$ be $(A, B, B, s,t, id_{B})$ , so that $Q(R(A,B,s,t))=(A,B,s,t)$. Now, suppose that an arrow $(h_1, h_2)\colon Q(E,F,G, a,b,c)\to (A,B, s,t)$ is given. Consider the triple $(h_1, h_2, h_2\circ c)$. Notice that, since $(h_1, h_2)$ is an arrow in $\Graph$:
	\[h_2\circ c\circ a= s\circ h_1 \qquad  h_2\circ c\circ b= t\circ h_1\]
	
	 Then we have three squares:	
	

\[\xymatrix{E \ar[r]^{a} \ar[d]_{h_1} & F \ar[d]^{h_2\circ c} & E \ar[r]^{b} \ar[d]_{h_1}& F \ar[d]_{h_2\circ c} & B \ar[d]_{h_2\circ c} \ar[r]^{c} & C \ar[d]^{h_2}\\A \ar[r]_{s} & B & A  \ar[r]_{t}& B & B \ar[r]_{id_B} & B}\]
	
	 We have therefore found a morphism $(E,F,G, a,b,c)\to R(A,B,s,t)$ whose image through $Q$ fits in the diagram below.
	\[\xymatrix{(A, B, s,t)\ar[r]^{(id_A, id_B)} &(A, B, s,t)\\ (E, G, q\circ s, q\circ t) \ar[ur]_{(h_1, h_2)} \ar[u]^{Q(h_1, h_2\circ q, h_2)}}\]
	
\todo{Esercizio: prova unicità (è facile)}
\end{proof}



\begin{prop}\label{prop:colimit}
	$Q$ creates colimits.
\end{prop}
\begin{proof}
	\todo{esercizio per te (devi verificare solo la riflessione: ma dato un diagramma in grafi fatto di quozienti, come costruisci un insieme di vertici per il colimite?)}
\end{proof}

\begin{example}
	\todo{Q non preserva i limiti}
\end{example}

\subsection{Adhesivity of $\EqGrph$}

\begin{lemma}\label{lem:stab}
In $\EqGrph$ pushouts along regular monos are stable.
\end{lemma}
\begin{proof}
	Suppose that the cube below is given, in which all the vertical faces are pullbacks and the bottom face is a pushout, with $(h_1, h_2, h_3)\colon (A_1,B_1,C_1, s_1,t_1,q_1)\to (A_2,B_2,C_2, s_2,t_2,q_2)$ a regular mono\todo{Questo diagramma va sistemato per farlo stare nella pagina (val la pena magari dire "sia $\mathcal{G}_1$ il grafo....")}.
	
		\[\xymatrix@C=20pt@R=20pt{&(A'_1,B'_1,C'_1, s'_1,t'_1,q'_1)\ar[dd]|\hole_(.65){(a_1, a_2, a_3)}\ar[rr]^{(h'_1, h'_2, h'_3)} \ar[dl]_{(k'_1, k'_2, k'_3)} && (A'_2,B'_2,C'_2, s'_2,t'_2,q'_2) \ar[dd]^{(b_1, b_2, b_3)} \ar[dl]_{(t'_1, t'_2, t'_3)} \\ (A'_3,B'_3,C'_3, s'_3,t'_3,q'_3) \ar[dd]_{(c_1, c_2, c_3)}\ar[rr]^(.65){(p'_1,p'_2,p'_3)} & & (A'_4,B'_4,C'_4, s'_4,t'_4,q'_4) \ar[dd]_(.3){(d_1, d_2, d_3)}\\&(A_1,B_1,C_1, s_1,t_1,q_1)\ar[rr]|\hole^(.65){(h_1, h_2, h_3)} \ar[dl]^{(k_1, k_2, k_3)} && (A_2,B_2,C_2, s_2,t_2,q_2) \ar[dl]^{(t_1, t_2, t_3)} \\(A_3,B_3,C_3, s_3,t_3,q_3) \ar[rr]_{(p_1, p_2, p_3)} & & (A_4,B_4,C_4, s_4,t_4,q_4)}\]

By \Cref{prop:limits,cor:regmono} the following two cubes have $\mathcal{M}$-pushouts as bottom faces and pullbacks as vertical faces, thus their top faces are $\mathcal{M}$-pushouts.

\[\xymatrix@C=10pt@R=10pt{&A'_1\ar[dd]|\hole_(.65){a_1}\ar[rr]^{h'_1} \ar[dl]_{k'_1} && A'_2 \ar[dd]^{b_1} \ar[dl]_{t'_1} && B'_1\ar[dd]|\hole_(.65){a_2}\ar[rr]^{h'_2} \ar[dl]_{k'_2} && B'_2 \ar[dd]^{b_2} \ar[dl]_{t'_2}\\ A'_3  \ar[dd]_{c_1}\ar[rr]^(.7){p'_1} & & A'_4 \ar[dd]_(.3){d_1} &&B'_3  \ar[dd]_{c_2}\ar[rr]^(.7){p'_2} & & B'_4 \ar[dd]_(.3){d_2}\\&A_1\ar[rr]|\hole^(.65){h_1} \ar[dl]^{k_1} && A_2 \ar[dl]^{t_1} && B_1\ar[rr]|\hole^(.65){h_2} \ar[dl]^{k_2} && B_2 \ar[dl]^{t_2} \\A_3 \ar[rr]_{p_1} & & A_4&&B_3 \ar[rr]_{p_2} & & B_4}\]

Now,  using \Cref{cor:cube}, we can consider a third cube, which, by \Cref{prop:colimit}, has a bottom face an $\mathcal{M}$-pushout and pullbacks as vertical faces, so that its top face is an $\mathcal{M}$-pushout too.

\[\xymatrix@C=10pt@R=10pt{&T\ar[dd]|\hole_(.65){x_2}\ar[rr]^{x_1} \ar[dl]_{w} && U \ar[dd]^{u_2} \ar[dl]_{u_1} \\ Y  \ar[dd]_{y_2}\ar[rr]^(.7){y_1} & & C'_4 \ar[dd]_(.3){d_3}\\&C_1\ar[rr]|\hole^(.65){h_3} \ar[dl]^{k_3} && C_2 \ar[dl]^{t_3} \\C_3 \ar[rr]_{p_3} & & C_4}\]

Moreover, by the proof of \Cref{prop:limits} we know that there are monos $m_2\colon C'_2\to U $ and $m_3\colon C'_3\to Y$ fitting in the diagrams
\[\xymatrix{B'_3\ar[r]^{p'_2} \ar[d]^{q'_3} \ar@/_.4cm/[dd]_{c_2}& B'_4\ar@/^.2cm/[dr]^{q'_4} && B'_2 \ar@/_.4cm/[dd]_{b_2} \ar[r]^{t'_2} \ar[d]^{q'_2}& B'_4 \ar@/^.2cm/[dr]^{q'_4} \\C'_3  \ar[r]^{m_3}& Y \ar[r]^{y_1} \ar[d]_{y_2} & C'_4 \ar[d]^{d_3} &C'_2 \ar[r]^{m_2}& U\ar[r]^{u_1} \ar[d]_{u_2} & C'_4 \ar[d]^{d_3}\\  B_3  \ar[r]_{q_2}& C_3 \ar[r]_{p_3}& C_4 &B_2 \ar[r]_{q_2}&  C_2 \ar[r]_{t_3}& C_4}\]

For $C'_1$, the we can make a similar argument, let $S$ be the pullback of $m_2$ along $x_1$, using \Cref{lem:pb1} and, again, the proof of \Cref{prop:limits} we know that $q'_1$ arise as the factorization of the arrow $B'_1\to S$ induced by $q'_2\circ h'_2$ and $a_2$ so that we have a diagram.
	\[\xymatrix{ B'_1 \ar@/_.4cm/[dd]_{a_2} \ar[r]^{h'_2} \ar[d]^{q'_2}& B'_2 \ar@/^.2cm/[dr]^{q'_2} \\C'_1 \ar[r]^{m_1}& S\ar[r]^{s_1} \ar[d]_{s_2}&C'_2 \ar[d]^{m_2}\\  B_2 \ar@/_.2cm/[dr]_{q_2}&  T \ar[r]^{x_1} \ar[d]_{x_2}& U \ar[d]_{u_2} \\ & C_1 \ar[r]_{h_3}&  C_2}\]

%Notice that $s_2\colon S\to T$, being the pullback of a mono is mono by ????\todo{metti questo lemma da qualche parte: il pullback di un mono è mono}
Moreover, notice that \todo{Esercizio (basta comporre prima e dopo con gli opportuni epi e mono)}

\[s_1\circ m_1 = h'_3 \quad w\circ s_2\circ m_1=  m_3\circ k'_3 \quad t'_3=u_1\circ m_2 \quad p_3=y_1\circ m_3\]

Let now $(z_1, z_2, z_3)\colon (A'_2, B'_2, C'_2)\to (E, F, G, e, f, g)$ and $(w_1, w_2, w_3)\colon (A'_3, B'_3, C'_3)\to (E, F, G, e, f, g)$ be two morphisms such that 
\[(z_1, z_2, z_3)\circ (h'_1, h'_2, h'_3)=(w_1, w_2, w_3)\circ (k'_1, k'_2, k'_3)\] 
Let $z\colon B'_4\to F$ be the arrow induced by $z_2$ and $w_2$, we want to construct the dotted arrow $v\colon C'_4 \to G$ in the diagram below
 
 \todo{Sistema il diagramma mettendo gli opportuni buchi}
\[\xymatrix@C=35pt@R=15pt{& B'_1 \ar[dd]|(.5)\hole_(.62){q'_1}\ar[rr]^{h'_2} \ar[dl]_{k'_2} && B'_2 \ar[ddd]|(.33)\hole^{q'_2} \ar[dl]_{t'_2} \ar@/^.2cm/[dr]^{z_2}\\B'_3 \ar[ddd]_{q'_3} \ar[rr]^(.65){p'_2} && B'_4  \ar[rr]^(.65){z}&& F \ar[ddddd]^{g}\\& C'_1 \ar@/^.2cm/[drr]^{h'_3} \ar@/_.2cm/[ddl]_{k'_3} \ar[d]^{m_1}\\& S \ar[rr]^{s_1} \ar[dd]_{s_2} && C'_2 \ar[dd]^{m_2} \ar@/^.2cm/[dddr]^{z_3} \ar[lddd]_{t'_3}\\ C'_3 \ar@/_.5cm/[ddrr]^(.65){p'_3} \ar[ddrrrr]_{w_3} \ar[dd]_{m_3}\\&T\ar[dd]|(.25)\hole|\hole_(.65){x_2}\ar[rr]^{x_1} \ar[dl]|(.25)\hole_{w} && U \ar[dd]^(.3){u_2} \ar[dl]_{u_1} \\ Y  \ar[dd]_{y_2}\ar[rr]_(.65){y_1} & & C'_4 \ar@{.>}[rr]_(.65){v} \ar[dd]_(.3){d_3} && G\\&C_1\ar[rr]|\hole^(.65){h_3} \ar[dl]^{k_3} && C_2 \ar[dl]^{t_3} \\C_3 \ar[rr]_{p_3} & & C_4}\]


Now by \Cref{prop:regepi} $d_3$ is the coequalizer of its kernel pair. On the other hand, by \Cref{lem:salvavita2}  we know that the top face of the cube below is a pushout.
\[\xymatrix@C=15pt@R=15pt{&K_{s_2\circ m_1\circ q'_1}\ar[dd]|\hole_(.65){p_{s_2\circ m_1\circ q'_1,1}}\ar[rr]^{k_{h'_2}} \ar[dl]_{k_{k'_2}} && K_{m_2\circ q'_2} \ar[dd]^{p_{m_2\circ q'_2,1}} \ar[dl]_{k_{t'_2}} \\ K_{m_3\circ q'_3}  \ar[dd]_{p_{m_3\circ q'_3,1}}\ar[rr]^(.7){k_{p'_2}} & & K_{q'_4} \ar[dd]_(.3){p_{q'_4,1}}\\&B'_1\ar[rr]|\hole^(.65){h'_2} \ar[dl]^{k'_2} && B'_2 \ar[dl]^{t'_2} \\B'_3 \ar[rr]_{p'_2} & & C_4}\]
Moreover, since $m_3$ and $m_2$ are monos, or by \Cref{cor:kermono} we also know that
\[q'_3\circ p_{m_3\circ q'_3, 1}  = q'_3\circ p_{m_3\circ q'_3,2} \qquad q'_2\circ p_{m_2\circ q'_2, 1}  = q'_2\circ p_{m_2\circ q'_2,2}\]

Now, we have
\[
\begin{split}
	g\circ z\circ p_{q'_4,1} \circ k_{p'_2}&=g\circ z\circ p'_2\circ p_{m_3\circ q'_3, 1}\\&=g\circ w_2\circ \circ  p_{m_3\circ q'_3, 1}\\&=w_3\circ q'_3\circ p_{m_3\circ q'_3, 1}\\&=w_3\circ q'_3\circ p_{m_3\circ q'_3, 2}\\&=g\circ w_2\circ \circ  p_{m_3\circ q'_3, 2}\\&=g\circ z\circ p'_2\circ p_{m_3\circ q'_3, 2}\\&=g\circ z\circ p_{q'_4,2} \circ k_{p'_2}
\end{split} \qquad 
\begin{split}
g\circ z\circ p_{q'_4,1} \circ k_{t'_2}&=g\circ z\circ t'_2\circ p_{m_2\circ q'_2, 1}\\&=g\circ z_2\circ \circ  p_{m_2\circ q'_2, 1}\\&=z_3\circ q'_2\circ p_{m_2\circ q'_2, 1}\\&=z_3\circ q'_2\circ p_{m_2\circ q'_2, 2}\\&=g\circ z_2\circ \circ  p_{m_2\circ q'_2, 2}\\&=g\circ z\circ t'_2\circ p_{m_2\circ q'_2, 2}\\&=g\circ z\circ p_{q'_4,2} \circ k_{t'_2}
\end{split}
\]

The thesis now follows \todo{Scrivi perché}

\todo{Esercizio: prova unicità}






\end{proof}


\begin{lemma}\label{lem:vk}
In $\EqGrph$ pushouts along regular monos are Van Kampen.
\end{lemma}
\begin{proof}
	contenuto...
\end{proof}

From \Cref{prop:limits} and \Cref{lem:stab,lem:vk} we deduce at once the following.

\begin{cor}\label{cor:equi}
	$\EqGrph$ is quasiadhesive.
\end{cor}

\color{black}

\section{Graphs with Equivalences}\label{sect:eq_graphs}

A (directed graph) $\graph{G}$ is a structure consisting of a set of edge, a set of nodes and two functions, one assigning a \emph{source} node and one assigning a \emph{target} node to an edge. Formally, $\graph{G}$ is a quadruple $(V_\graph{G}, E_\graph{G}, s_\graph{G}, t_\graph{G})$, where $V_\graph{G}$ is the set of nodes, $E_\graph{G}$ is the set of edges, and $s_\graph{G}, t_\graph{G}: E_\graph{G} \rightarrow V_\graph{G}$ are the source and the target functions.

A \emph{graph homomorphism} $h: \graph{G \rightarrow H}$ is then a pair of functions $h = (h_V: V_\graph{G} \rightarrow V_\graph{H}, h_E: E_\graph{G} \rightarrow E_\graph{H})$ such that
    \[
        h_V \circ s_\graph{G} = s_\graph{H} \circ h_E
    \]
    and
    \[
        h_V \circ t_\graph{G} = t_\graph{H} \circ h_E
    \]
that is, a structure preserving map.

We can then generalize such notion to something more abstract, considering a graph to be nothing more than a presheaf from the category $(E \rightrightarrows V)$ to the category of sets.
Having two of such presheaves, a natural transformation from one to another encapsulates the behavior of a graph morphism due to naturality. We can now define the category of graphs.

\begin{definition}[Category of Graphs]\label{def:cat_of_graph}
    We denote as $\Graph$ the category $$[E \mathrel{\mathop{\rightrightarrows}^{s}_{t}} V , \Set]$$
\end{definition}

Since $\Graph$ is a category of presheaves, \Cref{lemma:limits_of_presheaves} guarantees the existence of limits and colimits, and gives us an easy way to compute them.

\begin{cor}\label{cor:graph_has_co_limits}
    $\Graph$ has all limits and colimits.
\end{cor}

A graph with equivalence is a 6-tuple $\eqgraph{G} =  (E, V, C, s, t, q)$, where $E$ and $V$ are, respectively, the edges and the vertices sets, and $C$ is the set of the equivalence classes among vertices, $s,t : E \to V$ are the source and target functions and $q: V \to C$ is the \emph{quotient} function, that is, the map from a vertex to its equivalence class. For this definition to make sense, $q$ needs to be surjective. A morphisms $h$ from a graph with equivalence $\eqgraph{G} =  (E, V, C, s, t, q)$ to another $\eqgraph{H} = (E', V', C', s', t',  q')$ is a triple $h = (h_E, h_V, h_C)$ of functions $h_V : V \to V'$, $h_E : E \to E'$ and $h_C : C \to C'$ such that
\[
    h_E \circ s = s' \circ h_V \qquad h_E \circ t = t' \circ h_V'\qquad h_C \circ q = q' \circ h_V
\]

\begin{remark}\label{rem:eqgrph_set_eq}
     A graph with equivalence can be viewed as a graph endowed with an equivalence relation over its set of vertices, $(\graph G, \sim_\graph{G})$. An homomorphism between two graphs with equivalences $h :\eqgraph{G} = (\graph{G}, \sim_\graph{G})\rightarrow \eqgraph{H} = (\graph{H}, \sim_\graph{H})$ is a graph homomorphism $h = (h_V, h_E):\graph{G} \rightarrow \graph{H}$ such that if $v_1 \sim_\graph{G} v_2$ then $h_V(v_1) \sim_\graph{H} h_V(v_2)$. In $\Set$, it is possible to formalize an equivalence relation $\sim$ over $X$ as a surjective function sending each element $x$ on its equivalence class $[x]_{\sim}$, and this justify our formalization via surjective functions (i.e., epimorphisms).
\end{remark}

As we have done for graphs, we can think to a graph with equivalence as a presheaf, this time from a category $E \rightrightarrows V \rightarrow C$, where the image of $C$ along the presheaf is the set of the equivalence classes, requiring that the morphism $V\rightarrow C$ is an epimorphism (that is, a surjective function).

\begin{definition}[Category of Graphs with Equivalences]\label{def:eq_grphs}
    The category $\EqGrph$ is the subcategory of $$[E \mathrel{\mathop{\rightrightarrows}^{s}_{t}} V \xrightarrow{q} C, \Set]$$ such that, for each $\eqgraph{G} \in \Ob(\EqGrph)$, $\eqgraph{G}(q)$ is an epimorphism. 
\end{definition}

\begin{obs}\label{obs:eq_grph_morph_det_by_first_two_comp}
    Morphisms of graphs with equivalences are uniquely determined by the first two components. That is, if $h_1 = (h_E, h_V, \phi)$ and $h_2 = (h_E, h_V, \psi)$, then $\phi = \psi$. Indeed, consider two arrows $h_1, h_2 : \eqgraph{G \to H}$, where $\eqgraph{G} = (E_G, V_G, C_G, s_G, t_G, q_G)$ and $\eqgraph{H} = (E_H, V_H, C_H, s_H, t_H, q_H)$. Then, we have the following situation
    \[
        \begin{tikzcd}
            V_G \arrow[r, "{h_V}"] \arrow[d, "{q_G}"swap] & V_H \arrow[d, "{q_H}"] & V_G \arrow[l, "{h_V}"swap] \arrow[d, "{q_G}"] \\
            C_G \arrow[r, "{\phi}"swap] & C_H & \arrow[l, "{\psi}"] C_G
        \end{tikzcd}
     \]
     Then, we have:
     \begin{align*}
         \psi \circ q_G &= q_H \circ h_V \\
                        &= \phi \circ q_G
     \end{align*}
     From the fact that $q_G$ is epi, we can conclude $\phi = \psi$.
\end{obs}

A graph with equivalence is then a graph with an extra structure, the quotient map. Hence, it is possible to get the underlying graph by forgetting it. Such action is described by the \emph{forgetful functor} $U : \EqGrph \to \Graph$, that maps each graph with equivalence $\eqgraph{G} =  (E, V, C, s, t, q)$ onto $U(\eqgraph{G}) = (E, V, s, t)$, and each morphisms $h = (h_E, h_V, h_C)$ onto $U(h) = (h_E, h_V)$. $U$ is effectively a functor, since, on identities, $U((id_E, id_V, id_C)) = (id_E, id_V)$, and on compositions 
\[\begin{split}
	U(h \circ k) &= U((h_E \circ k_E, h_V \circ k_V, h_C \circ k_C))\\ &= (h_E \circ k_E, h_V \circ k_V) \\&= (h_E, h_V) \circ (k_E \circ k_V) \\&= U(h) \circ U(k)
\end{split}\]

\begin{prop}\label{prop:U_is_faithf}
    The forgetful functor $U: \EqGrph \to \Graph$ is faithful.
\end{prop}

\begin{proof}
    Let $\eqgraph{G} = (E_G, V_G, C_G, s_G, t_G, q_G)$ and $\eqgraph{H} = (E_H, V_H, C_H, s_H, t_H, q_H)$  be two graphs with equivalences, and let $h, k : \eqgraph{G \to H}$.
    If $U(h) = U(k)$ (i.e., the first two component of $h$ and $k$ are the same), from \Cref{obs:eq_grph_morph_det_by_first_two_comp}, we can conclude that $h = k$. Then, the restriction $U_{\eqgraph{G, H}} : \EqGrph(\eqgraph{G, H}) \to \Graph(U(\eqgraph{G}), U(\eqgraph{H}))$ is injective, therefore $U$ is faithful.
\end{proof}

Another functor that will be useful later is $V: \EqGrph \to \Set$, sending $(E_G, V_G, C_G, s_G, t_G, q_G)$ to $C_G$ and $h = (h_E, h_V, h_C)$ to $h_C$.

\begin{prop}\label{prop:eqgrph_complete}
    $\EqGrph$ has all limits, colimits and $U$ preserves limits and colimits.
\end{prop}

\begin{proof}
    Let $D : \cat I \to \EqGrph$ be a diagram. In the following, we will denote the graph with equivalence $D(i)$ as $(E_i, V_i, C_i, s_i, t_i, q_i)$.
    Let now be the graph $(A, B, s, t)$ the limit of $U \circ D$, with projections $(\pi_E^i, \pi_V^i):(A, B, s, t) \to (E_i, V_i, s_i, t_i)$. Notice now that $(B, (q_i\circ \pi_V^i)_{i \in \cat I})$ is a cone for $V \circ D$. To see this, let $\alpha : i \to j$ be an arrow of $\cat I$, $D(\alpha) = (h_E, h_V, h_C)$, $U \circ D (\alpha) = (h_E, h_V)$. From the definition of cone, we have that $U \circ D (\alpha) \circ (\pi_E^i, \pi_V^i) = (\pi_E^j, \pi_V^j)$, hence $h_V \circ \pi_V^i = \pi_V^j$. 
    Consider now the following diagram in $\Set$
    \[
        \begin{tikzcd}[row sep=25 pt, column sep = 25 pt]
            & B \arrow[dl, "{\pi_V^i}"swap] \arrow[dr, "{\pi_V^j}"] & \\
            V_i \arrow[rr, "{h_V}"] \arrow[d, "{q_i}" swap] & & V_j \arrow[d, "{q_j}"] \\
            C_i \arrow[rr, "{h_C}" swap] & & C_j 
        \end{tikzcd}
    \]
    So we have $q_j \circ h_V \circ \pi_V^i = q_j \circ \pi_V^j$, by definition of graph with equivalence, $h_C \circ q_i \circ \pi_V^i = q_j$, and, by definition of $V$, $V \circ D (\alpha) \circ q_i \circ \pi_V^i = q_j \circ \pi_V^j$.
    Suppose now $(L, (l_i)_{i \in \cat I})$ be a limit for $V \circ D$, so that we have an arrow $l: B \to L$. This arrow is not epi in general, so let $Q$ be its image, $q: Q \to B$ be the resulting epi and $m: Q \to L$ the corresponding mono, as the diagram below shows. By definition, the external rectangle commutes, so, for each $i$ object of $\cat I$, \Cref{prop:prop_epi_mono_Set} yields the dotted arrow $\pi_C^i$.
    \[
        \begin{tikzcd}
            B \arrow[r, "{\pi_V^i}"] \arrow[d, "q" swap] & B_ i \arrow[r, "{q_i}"] & Q_i \arrow[d, "{id_{Q_i}}"] \\
            Q \arrow[urr, dashed, "\pi_C^i"] \arrow[r, "m" swap] &L \arrow[r, "{l_i}"swap] & Q_i
        \end{tikzcd}
    \]
    We have to show that in this way we get a cone over the diagram $D$. Let $\alpha : i\to j$ be an arrow of $\cat{I}$, then we have:
    \begin{align*}
    U(D(\alpha)\circ (\pi_E^i, \pi_V^i, \pi_C^i))  &=  U(D(\alpha))\circ(\pi_E^i, \pi_V^i)\\
                                                   &=  (\pi_E^j, \pi_V^j)\\
                                                   &=  U(D(\alpha)\circ (\pi_E^j, \pi_V^j, \pi_C^j))
    \end{align*}
    And faithfulness of $U$ yields the thesis.

	To see that this cone is terminal, let $((E, F, G, a, b, c), (\tau_i)_{i \in \cat I})$, where $\tau_i = (\tau_E^i, \tau_V^i, \tau_C^i)$, be another cone. By construction, we have an arrow $(\tau_E, \tau_V):(E, F, a, b) \to (A, B, s, t)$ such that
    \[
        \begin{tikzcd}[row sep = 26, column sep = 26]
            & E \arrow[dl, dashed, "{\tau_E}" swap] \arrow[dr, "{\tau_E^i}"] & \\
            A \arrow[rr, "{\pi_E^i}" swap] & & A_i 
        \end{tikzcd}
        %
        \qquad
        %
        \begin{tikzcd}[row sep = 26, column sep = 26]
            & F \arrow[dl, dashed, "{\tau_V}" swap] \arrow[dr, "{\tau_V^i}"] & \\
            B \arrow[rr, "{\pi_V^i}" swap] & & B_i 
        \end{tikzcd}
    \]

    For the same reason as before, $(G, (\tau_C^i)_{i\in \cat I})$ is a cone over $V \circ D$, thus there exists an arrow $\tau : G \to L$ such that $l_i \circ \tau = \tau_C^i$. At this point, we get
    \begin{align*}
        l_i\circ \tau \circ c 
                        &= \tau_C^i\circ c              && \\
                        &=q_i\circ \tau_V^i             && \textit{$\tau_i$ is a morphism in $\EqGrph$}\\
                        &=q_i\circ \pi_V^i\circ \tau_V  && \textit{Diagram above} \\
                        &=l_i\circ l\circ \tau_V        && \textit{$(B, (q_i\circ \pi_V^i)_{i \in \cat I})$ cone} 
    \end{align*} 

	Therefore, the outer part of the rectangle below commutes, and by \Cref{prop:prop_epi_mono_Set} there exists a unique $\tau_C: G \to Q$
    \[
        \begin{tikzcd}
            F \arrow[r, "{\tau_V}"] \arrow[d, "c"swap] & B \arrow[r, "q"] & Q \arrow[d, "m"] \\
            G \arrow[urr, dashed, "{\tau_C}"] \arrow[rr, "{\tau}"swap] & & L
        \end{tikzcd}
    \]
     Faithfulness of $U$ and \Cref{obs:eq_grph_morph_det_by_first_two_comp} guarantees that $(\tau_E, \tau_V, \tau_C)$ is the unique arrow such that $(\pi_E^i, \pi_V^i, \pi_C^i) \circ (\tau_E, \tau_V, \tau_C) = (\tau_E^i, \tau_V^i, \tau_C^i)$.

     For colimits, let $(A, B, s, t)$ be the colimit of $U \circ D$, together with morphisms $(\kappa_i)_{i \in \cat I} = (\kappa_E^i, \kappa_V^i)_{i \in \cat I}$, and suppose $((c_i)_{i \in \cat I}, C)$ be the colimit of $V \circ D$. Then, we have the following situation
     \[
	     \begin{tikzcd}[row sep = 25, column sep=25]
			& B & \\
			B_i \ar[rr, "{h_V}"] \ar[ur, "{\kappa_V^i}"] \ar[d, "{q_i}"swap] & & B_j \ar[ul, "{\kappa_V^j}"swap] \ar[d, "q_j"] \\
			C_i \ar[rr, "{h_C}" swap] \ar[dr, "{c_i}"swap] & & C_j \ar[dl, "{c_j}"] \\
			& C &
		\end{tikzcd}
     \]
     Then, $((c_i \circ q_i)_{i \in \cat I}, C)$ is a cocone for all $B_i$, and, since $((\kappa_i)_{i \in \cat I}, B)$ is the colimit of all $B_i$ (\Cref{lemma:limits_of_presheaves}), there exists a unique arrow $q: B \to C$ such that, for each $i$, $q \circ \kappa_V^i = c_i \circ q_i$. Such $q$ is epi, by application of \Cref{lemma:nat_trans_reg_epi_canonical_arrow_reg_epi}, and thus $(A, B, C, s, t, q)$, together with arrows $(\kappa_E^i, \kappa_V^i, c_i)_{i \in \cat I}$ is the colimit of $D$.

\end{proof}

\begin{cor}\label{cor:mono_in_EqGrph}
    Let $\eqgraph{G} = (E_G, V_G, C_G, s_G, t_G, q_G)$ and $\eqgraph{H} = (E_H, V_H, C_H, s_H, t_H, q_H)$ be two graphs with equivalences. Then, an arrow $h = (h_E, h_V, h_C): \eqgraph{G} \to \eqgraph{H}$ in $\EqGrph$ is mono if and only if $h_E$ and $h_V$ are mono in $\Set$.
\end{cor}

\begin{proof}
    The ``if'' part is given by the fact that $U$ is faithful, and hence reflects monomorphisms. Since a morphism in a category of presheaves is mono if and only if it is injective on each component, we have that, if  $U(h)$ is mono, that is, $h_E$ and $h_V$ are injective in $\Set$, then $h$ is mono.
	For the ``only if'' part, suppose $f = (f_E, f_V, f_C)$, $g=(g_E, g_V, g_C)$, $f, g : \eqgraph{H \to K}$, where $\eqgraph{K} = (E_K, V_K, C_K, s_K, t_K, q_K)$, be such that $h \circ f = h \circ g$. Then, we have
    \begin{align*}
        h \circ f   &= (h_E \circ f_E, h_V \circ f_V, h_C \circ f_C) \\
                    &= (h_E \circ f_E, h_V \circ f_V, h_V \circ f_V \circ q_K) \\
                    &= (h_E \circ g_E, h_V \circ g_V, h_V \circ g_V \circ q_K)
    \end{align*}    
    Since $q_K$ is epi, we have, on the third component, that $h_V \circ f_V \circ q_K = h_V \circ g_V \circ q_K$ implies $f_C = g_C$, and hence $f = g$    
\end{proof}


\begin{cor}\label{cor:regmono}
	Let $\eqgraph{G} = (E_G, V_G, C_G, s_G, t_G, q_G)$ and $\eqgraph{H} = (E_H, V_H, C_H, s_H, t_H, q_H)$ be two graphs with equivalences,
	and let $ h = (h_E, h_V, h_C): \eqgraph{G}\to \eqgraph{H} $ be a regular monomorphism of $\EqGrph$, then $h_E$, $h_V$, $h_C$ are all monos.
\end{cor}
\begin{proof}
	If $h$ is mono, from \Cref{cor:mono_in_EqGrph} we have that $h_E$ and $h_V$ are monos. To derive $h_C$ mono, suppose $f, g :  \eqgraph{H \to K}$, where $\eqgraph{K} = (E_K, V_K, C_K, s_K, t_K, q_K)$to be the arrows equalized by $h$. Then we have
    \begin{align*}
        f_C \circ h_C \circ q_G  &=  f_C \circ q_H \circ h_V \\
                                            &=  q_K \circ f_V \circ h_V \\
                                            &=  q_K \circ g_V \circ h_V \\
                                            &=  g_C \circ h_C \circ q_G
    \end{align*}
    since $q_G$ is epi, we have that $f_C \circ h_C = g_C \circ h_C$, hence $h_C$ is an equalizer for $f_C$ and $g_C$, thus a monomorphism.
	\iffalse

	\smallskip \noindent
	% $2\Rightarrow 3.$ The leftward side of the statement is satisfied by the definition of morphism of graphs with equivalences. For the remaining part, we have
 %    \begin{align*}
 %        (\eqgraph{H}(q) \circ h_V) (v)  &=  (\eqgraph{H}(q) \circ h_V)(v') \\
 %        (h_C \circ \eqgraph{G}(q))(v)   &=  (h_C \circ \eqgraph{G}(q))(v')  
 %    \end{align*}
 %        since $h_C$ is mono, we can conclude $\eqgraph{G}(q)(v)= \eqgraph{G}(q)(v')$.
        \smallskip \noindent
	$2\Rightarrow 3.$ We note that, by \Cref{cor:kermono}, $(K, \pi_1, \pi_2)$ is the kernel pair of $q_G$ if and only if it is the kernel pair also of $h_C \circ q_G$, since $h_C$ is mono by hypothesis. The thesis follows from $h_C \circ q_G = q_H \circ h_V$, and from the hypothesis of $h_E$ mono.
	
	\smallskip \noindent 
	$3\Rightarrow 1.$\todo{Esercizio} \color{green}{idea: force the comm. of the diagram on the last two components to obtain the two arrows that are equalized, and show that the condition in 3 is sufficient to conclude reg. mono} \color{black}
	\fi
\end{proof}

\begin{prop}
	Let $\eqgraph{G} = (E_G, V_G, C_G, s_G, t_G, q_G)$ and $\eqgraph{H} = (E_H, V_H, C_H, s_H, t_H, q_H)$ be two graphs with equivalences,
	and let $ h = (h_E, h_V, h_C): \eqgraph{G}\to \eqgraph{H} $ be a regular monomorphism of $\EqGrph$. Then, $h_E$ and $h_V$ are mono and $(K, \pi_1, \pi_2)$ is the kernel pair of $q_H \circ h_V$ if and only if $(K, \pi_1, \pi_2)$ is the kernel pair of $q_G$.
\end{prop}

\begin{proof}
	By \Cref{cor:regmono}, we have that $h_E$, $h_V$ and $h_E$ are all monos. Hence, by \Cref{cor:kermono}, $(K, \pi_1, \pi_2)$ is the kernel pair of $q_G$ if and only if it is the kernel pair also of $h_C \circ q_G$, since $h_C$ is mono by hypothesis. The thesis follows from $h_C \circ q_G = q_H \circ h_V$, and from the hypothesis of $h_E$ mono.

\end{proof}

\begin{remark}
    It is possible to restate the last proposition, by \Cref{ex:kernel_pairs_in_Set}, as 
    \begin{displayquote}
    \textit{$h_E$ and $h_V$ are mono and, for every $v, v'\in V_H$, $q_H(h_V(v))=q_H(h_V(v'))$ if and only if $q_G(v)=q_G(v')$}
    \end{displayquote}
    That is, a regular monomorphism in $\EqGrph$ is a morphism that reflects equivalences besides preserving them.
\end{remark}

Let us turn to another functor $\EqGrph\to \Graph$.

\begin{definition}
The \emph{quotient functor} $Q:\EqGrph\to \Graph $ is defined as the one sending $(E_G, V_G, C_G, s_G, t_G, q_G)$ to $(E_G, C_G, q_G\circ s_G, q_G\circ t_G)$ and an arrow $(h_E, h_V, h_C) \colon (E_G, V_G, C_G, s_G, t_G, q_G)\to (E_H, V_H, C_H, s_H, t_H, q_H)$ to $(h_E, h_C)$.
\end{definition}

\begin{remark}
    The action of the functor on a morphism of graphs with equivalences gives a morphism of graphs, in fact $q_H \circ s_H \circ h_E = q_H \circ h_V \circ s_G = h_C \circ q_G \circ s_G$. The same is valid for $t_H$ and $t_G$. 
\end{remark}

\begin{lemma}\label{lemma:quot_funct_left_adj}
    $Q$ is a left adjoint.
\end{lemma}

\begin{proof}
    Let $R((A, B, s, t))$ be $(A, B, B, s, t, id_B)$, so that $Q(R((A, B, s, t))) = (A, B, s, t)$. Now, suppose that $h = (h_E, h_V): Q((E, V, C, s', t', q)) \to (A, B, s, t)$  is an arrow in $\Graph$, and consider the triple $(h_E, h_V, h_V \circ q)$. Since $h$ is a morphism of $\Graph$, 
    \[h_V\circ q\circ s'= s\circ h_E \qquad  h_V\circ q\circ t' = t\circ h_E\]
    Then we have the following squares:
    \[
        \begin{tikzcd}
            E \arrow[r, "{h_E}"] \arrow[d, "{s_G}"swap] & A \arrow[d, "s"] \\
            V \arrow[r, "{h_V \circ q}"swap] & B
        \end{tikzcd}
        \qquad
        \begin{tikzcd}
            E \arrow[r, "{h_E}"] \arrow[d, "{t_G}"swap] & A \arrow[d, "t"] \\
            V \arrow[r, "{h_V \circ q}"swap] & B
        \end{tikzcd}
        \qquad
        \begin{tikzcd}
            V \arrow[r, "{h_V\circ q}"] \arrow[d, "q" swap] & B \arrow[d, "{id_B}"] \\
            C \arrow[r, "{h_V}"swap] & B
        \end{tikzcd}
    \]

    We have therefore found a morphism $(E, V, C, s', t', q) \to R((A, B, s, t))$ whose image through $Q$ fits in the diagram below.
    \[
        \begin{tikzcd}
            (A, B, s, t) \arrow[r, "{id_A, id_B}"] & (A, B, s, t)\\
            (E, C, q\circ s', q\circ t') \arrow[u, "{Q((h_E, h_V \circ q, h_V))}"] \arrow[ur, "{(h_E, h_V)}" swap] 
        \end{tikzcd}
    \]
    Such arrow is unique. Suppose $f = (f_E, f_V, f_C)$ to be another arrow wit such property. Then, it must be $(id_A, id_B) \circ Q(f) = (f_E, f_C) = (h_E, h_C)$. Finally, $f_C = f_V \circ q = h_V \circ q$. 
\end{proof}

\begin{prop}\label{prop:quot_creat_colims}
    $Q$ creates colimits.
\end{prop}


\begin{proof}
    Preserve from \Cref{th:adjoints_preserves_lim}. Remain to see Reflect.
	Let $D: \cat I \to \EqGrph$ be a diagram, and let $((c_i)_{i\in \cat I}, \eqgraph{C})$ be the colimit of $Q \circ D$, where $\eqgraph{C} = (A, C, q\circ s, q\circ t)$, and $D(i)$ is $(A_i, B_i, C_i, s_i, t_i, q_i)$.
	Let now $T: \EqGrph \to \Set$ be the functor mapping each graph with equivalence onto its second component, $T((X, Y, Z, x, y, z)) = Y$, and each morphims onto its second component.
	Let then $((b_i)_{i\in \cat I}, B)$ be the colimit of $T \circ D$.
	Consider the following situation.
	\[\begin{tikzcd}[row sep = 24, column sep = 24]
		& B & \\
		B_i \ar[ur, "{b_i}"] \ar[rr, "h_V"] \ar[d, "{q_i}"] & & B_j \ar[ul, "{b_j}"swap] \ar[d, "{q_j}" swap] \\
		C_i \ar[dr, "{c_C^i}" swap] \ar[rr, "{h_C}"swap] & & C_j \ar[dl, "{c_C^j}"] \\
		& C &
	\end{tikzcd}\]

	Now, since $((c^i_C \circ q_i)_{i \in \cat I}, C)$ is a cocone for $T \circ D$, there exists a unique $q: B \to C$, which is epi by \Cref{lemma:nat_trans_reg_epi_canonical_arrow_reg_epi}.
	Consider now the functor $W: \EqGrph \to \Set$ mapping each $(X, Y, Z, x, y, z)$ onto $X$, and each morphism on its first component. By \Cref{prop:eqgrph_complete} and \Cref{lemma:limits_of_presheaves}, we have that $((c_E^i)_{i \in \cat I}, A)$ is the colimit of $W \circ D$.
	Notice that $((b_i \circ s_i)_{i \in \cat{I}}, B)$ and $((b_i \circ t_i)_{i \in \cat I}, B)$ are cocones for $W \circ D$, so let $s$ and $t$ be, respectively, the mediating arrow for the first one and the mediating arrow for the second one. It remains now to show that $(A, B, C, s, t, q)$, together with $(c_E^i, b_i, c_C^i)_{i \in \cat I}$, is a colimit for $D$, but this follows by the proof of \Cref{prop:eqgrph_complete}.
\end{proof}

\begin{example}
	$Q$ does not preserve limits. Indeed, let $\eqgraph{G}_1 = (E_1, A, A, s_1, t_1, id_A)$, $\eqgraph{G}_2 = (E_2, B, B, s_2, t_2, id_B)$ and $\eqgraph{G}_3 = (E_3, A + B, \terminal, s_3, t_3, !_{A + B})$, and let $h = (h_E, \iota_A, !_A): \eqgraph{G}_1 \to \eqgraph{G}_3$, $k = (k_E, \iota_B, !_B): \eqgraph{G}_2 \to \eqgraph{G}_3$, where $(\iota_A, \iota_B, A + B)$ is the coproduct of $A$ and $B$, $\terminal$ is the intial object (in $\Set$, the singleton set as shown in \Cref{ex:set_init_term}), and $!_X$ the unique arrow $X \to \terminal$.
	The following two diagrams show the pullback of $h$ and $k$ and the pullback of $Q(h)$ and $Q(k)$, on the second component (the vertices of the graphs)
	\[\begin{tikzcd}[row sep = 20, column sep = 20]
		\initial \ar[r, "{p_1}"] \ar[d, "{p_2}" swap] & A \ar[d, "{\iota_A}"] \\
		B \ar[r, "{\iota_B}"swap] & A+B
	\end{tikzcd}\
	\qquad
	\begin{tikzcd}[row sep = 20, column sep = 20]
		{A \times B} \ar[r, "{\pi_A}"] \ar[d, "{\pi_B}"swap] & A \ar[d, "{!_A}"] \\
		B \ar[r, "{!_B}"swap] & \terminal
	\end{tikzcd}\]

	But the arrow $\initial \to A\times B$ is not epi in general (this is easy to see taking $\Set$ as example), hence such pullback is not preserved by $Q$.
\end{example}


\subsection{Adhesivity of $\EqGrph$}

\begin{lemma}\label{lemma:stab}
	In $\EqGrph$, pushouts along regular monos are stable.
\end{lemma}

\begin{proof}
	Let $\graph{G}_i = (A_i, B_i, C_i, s_i, t_i, q_i)$, $\graph{G}'=(A_i', B_i', C_i', s_i', t_i', q_i')$, for $i = 1, 2, 3, 4$, and, in the diagram above, suppose all the vertical faces are pullbacks, the bottom face is a pushout and $h$ is regular mono.
	\[\begin{tikzcd}[row sep=23, column sep =23]
	& {\mathcal{G}_1'} && {\mathcal{G}_2'} \\
	{\mathcal{G}_3'} && {\mathcal{G}_4'} \\
	& {\mathcal{G}_1} && {\mathcal{G}_2} \\
	{\mathcal{G}_3} && {\mathcal{G}_4}
	\arrow["{h'}", from=1-2, to=1-4]
	\arrow["{k'}"', from=1-2, to=2-1]
	\arrow["b"'{pos=0.7}, from=1-2, to=3-2]
	\arrow["{t'}", from=1-4, to=2-3]
	\arrow["c", from=1-4, to=3-4]
	\arrow["{p'}"{pos=0.8}, from=2-1, to=2-3, crossing over]
	\arrow["a"', from=2-1, to=4-1]
	\arrow["h"{pos=0.7}, from=3-2, to=3-4]
	\arrow["d"'{pos=0.7}, from=2-3, to=4-3, crossing over]
	\arrow["k"', from=3-2, to=4-1]
	\arrow["t", from=3-4, to=4-3]
	\arrow["p"', from=4-1, to=4-3]
	\end{tikzcd}\]
	By \Cref{prop:eqgrph_complete} and \Cref{cor:mono_in_EqGrph}, the following two cubes have $\mathcal{M}$-pushouts as bottom faces and pullbacks as vertical faces, thus their top faces are $\mathcal{M}$-pushouts.
	\[\begin{tikzcd}[row sep=20, column sep = 20]
	& {A_1'} && {A_2'} \\
	{A_3'} && {A_4'} \\
	& {A_1} && {A_2} \\
	{A_3} && {A_4}
	\arrow["{h'_V}", from=1-2, to=1-4]
	\arrow["{k'_V}"', from=1-2, to=2-1]
	\arrow["{b_V}"'{pos=0.7}, from=1-2, to=3-2]
	\arrow["{t'_V}", from=1-4, to=2-3]
	\arrow["{c_V}", from=1-4, to=3-4]
	\arrow["{p'_V}"{pos=0.8}, from=2-1, to=2-3, crossing over]
	\arrow["{a_V}"', from=2-1, to=4-1]
	\arrow["{h_V}"{pos=0.7}, from=3-2, to=3-4]
	\arrow["{d_V}"'{pos=0.7}, from=2-3, to=4-3, crossing over]
	\arrow["{k_V}"', from=3-2, to=4-1]
	\arrow["{t_V}", from=3-4, to=4-3]
	\arrow["{p_V}"', from=4-1, to=4-3]
	\end{tikzcd}\qquad
	\begin{tikzcd}[row sep=20, column sep = 20]
	& {B_1'} && {B_2'} \\
	{B_3'} && {B_4'} \\
	& {B_1} && {B_2} \\
	{B_3} && {B_4}
	\arrow["{h'_E}", from=1-2, to=1-4]
	\arrow["{k'_E}"', from=1-2, to=2-1]
	\arrow["{b_E}"'{pos=0.7}, from=1-2, to=3-2]
	\arrow["{t'_E}", from=1-4, to=2-3]
	\arrow["{c_E}"', from=1-4, to=3-4]
	\arrow["{p'_E}"{pos=0.8}, from=2-1, to=2-3, crossing over]
	\arrow["{a_E}"', from=2-1, to=4-1]
	\arrow["{h_E}"{pos=0.7}, from=3-2, to=3-4]
	\arrow["{d_E}"'{pos=0.7}, from=2-3, to=4-3, crossing over]
	\arrow["{k_E}"', from=3-2, to=4-1]
	\arrow["{t_E}", from=3-4, to=4-3]
	\arrow["{p_E}"', from=4-1, to=4-3]
	\end{tikzcd}
	\]
	Now, using \Cref{cor:cube}, we can consider a third cube, which, by \Cref{prop:quot_creat_colims}, has a $\mathcal{M}$-pushout as bottom face and pullbacks as vertical faces, so that its top face is a $\mathcal M$-pushout too.
	\[\begin{tikzcd}[row sep=20, column sep = 20]
	& {T} && {U} \\
	{Y} && {C_4'} \\
	& {C_1} && {C_2} \\
	{C_3} && {C_4}
	\arrow["{x_1}", from=1-2, to=1-4]
	\arrow["{w}"', from=1-2, to=2-1]
	\arrow["{x_2}"'{pos=0.7}, from=1-2, to=3-2]
	\arrow["{u_1}", from=1-4, to=2-3]
	\arrow["{u_2}", from=1-4, to=3-4]
	\arrow["{y_1}"{pos=0.8}, from=2-1, to=2-3, crossing over]
	\arrow["{y_2}"', from=2-1, to=4-1]
	\arrow["{h_C}"{pos=0.7}, from=3-2, to=3-4]
	\arrow["{d_C}"'{pos=0.7}, from=2-3, to=4-3, crossing over]
	\arrow["{k_C}"', from=3-2, to=4-1]
	\arrow["{t_C}", from=3-4, to=4-3]
	\arrow["{p_C}"', from=4-1, to=4-3]
	\end{tikzcd}\]

	Moreover, by the proof of \Cref{prop:eqgrph_complete}, we know that there are monos $m_2:C_2'\to U$ and $m_3:C_3'\to Y$ fitting in the diagrams
	\[\begin{tikzcd}[row sep=18, column sep = 18]
	{B'_3} && {B'_4} \\
	\\
	{C'_3} && Y && {C'_4} \\
	\\
	{B_3} && {C_3} && {C_4}
	\arrow["{p'_V}", from=1-1, to=1-3]
	\arrow["{q'_3}", from=1-1, to=3-1]
	\arrow["{q'_4}", from=1-3, to=3-5, bend left = 20]
	\arrow["{m_3}", from=3-1, to=3-3]
	\arrow["{y_1}", from=3-3, to=3-5]
	\arrow["{y_2}"', from=3-3, to=5-3]
	\arrow["{d_C}", from=3-5, to=5-5]
	\arrow["{q_3}"', from=5-1, to=5-3]
	\arrow["{p_C}"', from=5-3, to=5-5]
        \arrow["{c_V}"', from=1-1, to=5-1, bend right = 20, shift right=1]
	\end{tikzcd}\qquad\begin{tikzcd}[row sep=18, column sep = 18]
	{B'_2} && {B'_4} \\
	\\
	{C'_2} && U && {C'_4} \\
	\\
	{B_2} && {C_2} && {C_4}
	\arrow["{t'_V}", from=1-1, to=1-3]
	\arrow["{q'_2}", from=1-1, to=3-1]
	\arrow["{q'_4}", from=1-3, to=3-5, bend left = 20]
	\arrow["{m_2}", from=3-1, to=3-3]
	\arrow["{u_1}", from=3-3, to=3-5]
	\arrow["{u_2}"', from=3-3, to=5-3]
	\arrow["{d_C}", from=3-5, to=5-5]
	\arrow["{q_2}"', from=5-1, to=5-3]
	\arrow["{t_C}"', from=5-3, to=5-5]
        \arrow["{b_V}"', from=1-1, to=5-1, bend right = 20, shift right = 1]
	\end{tikzcd}\]

	For $C'_1$, we can make a similar argument, let $S$ be the pullback of $m_2$ along $x_1$, using \Cref{lemma:pullback_lemma} and, again, the proof of \Cref{prop:eqgrph_complete} we know that $q'_1$ arise as the factorization of the arrow $B'_1\to S$ induced by $q'_2\circ h'_2$ and $a_2$ so that we have a diagram.
	\[\begin{tikzcd}[row sep = 16, column sep =16]
	{B'_1} && {B'_2} \\
	\\
	{C'_1} && S && {C'_2} \\
	\\
	{B_1} && T && U \\
	\\
	&& {C_1} && {C_2}
	\arrow["{h'_V}", from=1-1, to=1-3]
	\arrow["{q'_1}", from=1-1, to=3-1]
	\arrow["{q'_2}", from=1-3, to=3-5, bend left = 20]
	\arrow["{m_1}", from=3-1, to=3-3]
	\arrow["{s_1}", from=3-3, to=3-5]
	\arrow["{s_2}"', from=3-3, to=5-3]
	\arrow["{m_2}", from=3-5, to=5-5]
	\arrow["{q_1}"', from=5-1, to=7-3, bend right = 20, shift right = 1]
	\arrow["{x_1}", from=5-3, to=5-5]
	\arrow["{x_2}"', from=5-3, to=7-3]
	\arrow["{u_2}", from=5-5, to=7-5]
	\arrow["{h_C}"', from=7-3, to=7-5]
        \arrow["{a_V}"', from=1-1, to=5-1, bend right = 20, shift right = 1]
	\end{tikzcd}\]
	
	Moreover, we have that
	\[
		\begin{split}
			s_1 \circ m_1 \circ q_1 &= q'_2 \circ h'_V \\ &= h'_C \circ q'_1
		\end{split}\qquad\begin{split}
			q'_4 \circ t'_V &= t'_C \circ q'_2\\ &= u_1\circ m_2 \circ q'_2
		\end{split}
	\]\[
		\begin{split}
			y_1 \circ m_3 \circ q'_3 &= p'_V \circ q'_4\\ &= p'_C \circ q'_3
		\end{split}%\qquad\begin{split}TODO:\ w \circ s_2 \circ m_1 = m_3 \circ k_C'\end{split}
	\]
	Hence, $s_1\circ m_1 = h'_C$, $w \circ s_2 \circ m_1 = m_3 \circ k_C'$, $t'_C = u_1 \circ m_2$ and $p'_C = y_1 \circ m_3$.
	Let now $z: \graph{G}'_2 \to \graph H = (E, V, Q, s, t, q)$, $w: \graph G_3^1 \to \graph H$ be two morphisms such that $z \circ h' = w \circ k'$, and let $\phi:B'_4 \to V$ be the arrow induced by $z_V$ and $w_V$. We want to construct the arrow $v: C'_4 \to Q$ in the diagram below.
	\[\begin{tikzcd}[row sep = 20, column sep = 20]
	& {B_1'} && {B_2'} \\
	{B_3'} && {B_4'} && V \\
	& T && U \\
	Y && {C_4'} && Q \\
	& {C_1} && {C_2} \\
	{C_3} && {C_4}
	\arrow["{h_V'}", from=1-2, to=1-4]
	\arrow["{k_V'}"', from=1-2, to=2-1]
	\arrow["{s_2\circ m_1 \circ q_1'}"{description, pos=0.7}, from=1-2, to=3-2]
	\arrow["{t_V'}", from=1-4, to=2-3]
	\arrow["{z_V}", from=1-4, to=2-5]
	\arrow["{m_2\circ q'_2}"{pos=0.7}, from=1-4, to=3-4]
	\arrow["{m_3 \circ q'_3}"', from=2-1, to=4-1]
	\arrow["q", from=2-5, to=4-5]
	\arrow["{x_1}"{pos=0.8}, from=3-2, to=3-4]
	\arrow["w"', from=3-2, to=4-1]
	\arrow["{x_2}"'{pos=0.7}, from=3-2, to=5-2]
	\arrow["{u_1}", from=3-4, to=4-3]
	\arrow["{u_2}"{pos=0.7}, from=3-4, to=5-4]
	\arrow["{y_2}"', from=4-1, to=6-1]
	\arrow["{h_C}"{pos=0.8}, from=5-2, to=5-4]
	\arrow["{k_C}", from=5-2, to=6-1]
	\arrow["{t_C}", from=5-4, to=6-3]
	\arrow["{p_C}"', from=6-1, to=6-3]
	\arrow["{p_V'}"{pos=0.8}, from=2-1, to=2-3, crossing over]
	\arrow["\phi"{pos=0.8}, from=2-3, to=2-5, crossing over]
	\arrow["{q'_4}"'{pos=0.7}, from=2-3, to=4-3, crossing over]
	\arrow["{y_1}"{pos=0.7}, from=4-1, to=4-3, crossing over]
	\arrow["{d_C}"'{pos=0.7}, from=4-3, to=6-3, crossing over]
	\arrow["v"{pos=0.8}, dashed, from=4-3, to=4-5, crossing over]
	\end{tikzcd}\]
	
	By \Cref{prop:reg_epi_coeq}, $d_C$ is the coequalizer of its kernel pair. On the other hand, by \Cref{lemma:pushouts_kernel_pairs}, we know that the top face of the cube below is a pushout.
	\[\begin{tikzcd}[row sep=20, column sep = 20]
	& {K_{s_2\circ m_1 \circ q_1'}} && {K_{m_2\circ q_2'}} \\
	{K_{m_3 \circ q_3'}} && {K_{q_4'}} \\
	& {B'_1} && {B'_2} \\
	{B'_3} && {B'_4}
	\arrow["{k_{h'_V}}", from=1-2, to=1-4]
	\arrow["{k_{k_V'}}"', from=1-2, to=2-1]
	\arrow["{\pi_{s_2\circ m_1 \circ q_1'}^1}"'{pos=0.7}, from=1-2, to=3-2]
	\arrow["{k_{t_V'}}", from=1-4, to=2-3]
	\arrow["{\pi_{m_2\circ q_2'}^1}", from=1-4, to=3-4]
	\arrow["{k_{p_V'}}"{pos=0.8}, from=2-1, to=2-3, crossing over]
	\arrow["{\pi_{m_3 \circ q_3'}^1}"', from=2-1, to=4-1]
	\arrow["{h_V'}"{pos=0.7}, from=3-2, to=3-4]
	\arrow["{\pi_{q_4'}^1}"'{pos=0.7}, from=2-3, to=4-3, crossing over]
	\arrow["{k_V'}"', from=3-2, to=4-1]
	\arrow["{t_V'}", from=3-4, to=4-3]
	\arrow["{p_V'}"', from=4-1, to=4-3]
	\end{tikzcd}\]
	And, since $m_3$ and $m_2$ are monos,
	\[\begin{split}
		q'_3 \circ \pi_{m_3 \circ q'_3}^1 = q'_3 \circ \pi_{m_3\circ q_3'}^2
	\end{split}\qquad\begin{split}
		q_2' \circ \pi_{m_2 \circ q_2'}^1 = q'_2 \circ \pi_{m_2 \circ q_2'}^2
	\end{split}\]
	
	Computing, we obtain
	\[\begin{split}
		q \circ \phi \circ \pi_{q_4'}^1 \circ k_{p_V'} &= q \circ \phi \circ p_V' \circ \pi_{m_3 \circ q_3'}^1 \\ &= q \circ w_V \circ \pi_{m_3 \circ q_3'}^1 \\ &= w_C \circ q_3' \circ \pi_{m_3 \circ q_3'}^1 \\ &= w_C \circ q_3' \circ \pi_{m_3 \circ q_3}^2 \\ &= q \circ w_V \circ \pi_{m_3 \circ q_3'} \\ &= q \circ \phi \circ p_V' \circ \pi_{m_3 \circ q_3'}^2 \\ &= q \circ \phi \circ \pi_{q'_4}^2 \circ k_{p'_V}
	\end{split}\qquad\begin{split}
		q \circ \phi \circ \pi_{q'_4}^1 \circ k_{t_V'} &= q \circ \phi \circ t_V' \circ \pi_{m_2 \circ q_2'}^1 \\ &= q \circ z_V \circ \pi_{m_2 \circ q_2'}^1 \\ &= z_C \circ q_2' \circ \pi_{m_2 \circ q_2'}^1 \\ &= z_C \circ q_2' \circ \pi_{m_2 \circ q_2}^2 \\ &= q \circ z_V \circ \pi_{m_2 \circ q_2'} \\ &= q \circ \phi \circ t_V' \circ \pi_{m_2 \circ q_2'}^2 \\ &= q \circ \phi \circ \pi_{q'_4}^2 \circ k_{t'_V}
	\end{split}\]

	Since the previous cube has a pushout as top face, by universal property, we have
	\[
		q \circ \phi \circ \pi_{q'_4}^1 = q \circ \phi \circ \pi_{q'_4}^2
	\]
	hence, $v$ is the mediating arrow.
	\[
		v \circ q'_4 \circ \pi_{q'_4}^1 = v \circ q'_4 \circ \pi_{q'_4}^2
	\]
\end{proof}


\begin{lemma}\label{lemma:van_kampen}
	In $\EqGrph$, pushouts along regular monos are $\Reg(\EqGrph)$-Van Kampen.
\end{lemma}

\begin{proof}
	In lieu of \Cref{lemma:stab}, it is enough to proof that, given a cube as the one below, with pullbacks as back faces, pushouts as bottom and top faces and such that $h$ is a regular mono, the front faces are pullbacks too, where $\graph{G}_i = (A_i, B_i, C_i, s_i, t_i, q_i)$, $\graph{G}'=(A_i', B_i', C_i', s_i', t_i', q_i')$, for $i = 1, 2, 3, 4$.
        \[\begin{tikzcd}[row sep=23, column sep =23]
        & {\mathcal{G}_1'} && {\mathcal{G}_2'} \\
        {\mathcal{G}_3'} && {\mathcal{G}_4'} \\
        & {\mathcal{G}_1} && {\mathcal{G}_2} \\
        {\mathcal{G}_3} && {\mathcal{G}_4}
        \arrow["{h'}", from=1-2, to=1-4]
        \arrow["{k'}"', from=1-2, to=2-1]
        \arrow["b"'{pos=0.7}, from=1-2, to=3-2]
        \arrow["{t'}", from=1-4, to=2-3]
        \arrow["c", from=1-4, to=3-4]
        \arrow["{p'}"{pos=0.8}, from=2-1, to=2-3, crossing over]
        \arrow["a"', from=2-1, to=4-1]
        \arrow["h"{pos=0.7}, from=3-2, to=3-4]
        \arrow["d"'{pos=0.7}, from=2-3, to=4-3, crossing over]
        \arrow["k"', from=3-2, to=4-1]
        \arrow["t", from=3-4, to=4-3]
        \arrow["p"', from=4-1, to=4-3]
        \end{tikzcd}\]
        By \Cref{prop:eqgrph_complete} and \Cref{cor:mono_in_EqGrph}, the following two cubes have $\mathcal{M}$-pushouts as bottom faces and pullbacks as back faces, thus their front faces are pullbacks too.
        \[\begin{tikzcd}[row sep=20, column sep = 20]
        & {A_1'} && {A_2'} \\
        {A_3'} && {A_4'} \\
        & {A_1} && {A_2} \\
        {A_3} && {A_4}
        \arrow["{h'_V}", from=1-2, to=1-4]
        \arrow["{k'_V}"', from=1-2, to=2-1]
        \arrow["{b_V}"'{pos=0.7}, from=1-2, to=3-2]
        \arrow["{t'_V}", from=1-4, to=2-3]
        \arrow["{c_V}", from=1-4, to=3-4]
        \arrow["{p'_V}"{pos=0.8}, from=2-1, to=2-3, crossing over]
        \arrow["{a_V}"', from=2-1, to=4-1]
        \arrow["{h_V}"{pos=0.7}, from=3-2, to=3-4]
        \arrow["{d_V}"'{pos=0.7}, from=2-3, to=4-3, crossing over]
        \arrow["{k_V}"', from=3-2, to=4-1]
        \arrow["{t_V}", from=3-4, to=4-3]
        \arrow["{p_V}"', from=4-1, to=4-3]
        \end{tikzcd}\qquad
        \begin{tikzcd}[row sep=20, column sep = 20]
        & {B_1'} && {B_2'} \\
        {B_3'} && {B_4'} \\
        & {B_1} && {B_2} \\
        {B_3} && {B_4}
        \arrow["{h'_E}", from=1-2, to=1-4]
        \arrow["{k'_E}"', from=1-2, to=2-1]
        \arrow["{b_E}"'{pos=0.7}, from=1-2, to=3-2]
        \arrow["{t'_E}", from=1-4, to=2-3]
        \arrow["{c_E}"', from=1-4, to=3-4]
        \arrow["{p'_E}"{pos=0.8}, from=2-1, to=2-3, crossing over]
        \arrow["{a_E}"', from=2-1, to=4-1]
        \arrow["{h_E}"{pos=0.7}, from=3-2, to=3-4]
        \arrow["{d_E}"'{pos=0.7}, from=2-3, to=4-3, crossing over]
        \arrow["{k_E}"', from=3-2, to=4-1]
        \arrow["{t_E}", from=3-4, to=4-3]
        \arrow["{p_E}"', from=4-1, to=4-3]
        \end{tikzcd}
        \]


	On the other hand we can consider the diagrams below, in which the inner squares are pullbacks. Since the outer diagrams commute, by definition of porphism of $\EqGrph$, then we have the existence of $m_2\colon C'_2\to U$, $m_3\colon C'_3\to Y $, $a_3\colon B'_3\to Y$ and $a_2\colon B'_2\to Y$.

	\[\begin{tikzcd}[row sep = 20, column sep = 20]
	{C'_3} &&& {C_2'} \\
	& Y & {C'_4} & {} & U & {C_4'} \\
	& {C_3} & {C_4} && {C_2} & {C_4}
	\arrow["{m_3}", dashed, from=1-1, to=2-2]
	\arrow["{p'_C}", from=1-1, to=2-3, bend left = 20]
	\arrow["{c_C}"', from=1-1, to=3-2, bend right = 20]
	\arrow["{m_2}", dashed, from=1-4, to=2-5]
	\arrow["{t'_C}", from=1-4, to=2-6, bend left = 20]
	\arrow["{b_C}"', from=1-4, to=3-5, bend right = 20]
	\arrow["{y_1}", from=2-2, to=2-3]
	\arrow["{y_2}"', from=2-2, to=3-2]
	\arrow["{d_C}", from=2-3, to=3-3]
	\arrow["{u_1}", from=2-5, to=2-6]
	\arrow["{u_2}"', from=2-5, to=3-5]
	\arrow["{d_C}", from=2-6, to=3-6]
	\arrow["{p_C}"', from=3-2, to=3-3]
	\arrow["{t_C}"', from=3-5, to=3-6]
	\end{tikzcd}\]
	\[\begin{tikzcd}[row sep = 20, column sep = 20]
	{B_3'} & {B'_4} && {B'_2} & {B'_4} \\
	{C'_3} & Y & {C'_4} & {C_2'} & U & {C_4'} \\
	& {C_3} & {C_4} && {C_2} & {C_4}
	\arrow["{p'_V}", from=1-1, to=1-2]
	\arrow["{q'_3}"', from=1-1, to=2-1]
	\arrow["f", dashed, from=1-1, to=2-2]
	\arrow["{q'_4}", from=1-2, to=2-3]
	\arrow["{t'_V}", from=1-4, to=1-5]
	\arrow["{q'_2}"', from=1-4, to=2-4]
	\arrow["g", dashed, from=1-4, to=2-5]
	\arrow["{q'_4}", from=1-5, to=2-6]
	\arrow["{c_C}"', from=2-1, to=3-2]
	\arrow["{y_1}", from=2-2, to=2-3]
	\arrow["{y_2}"', from=2-2, to=3-2]
	\arrow["{d_C}", from=2-3, to=3-3]
	\arrow["{b_C}"', from=2-4, to=3-5]
	\arrow["{u_1}", from=2-5, to=2-6]
	\arrow["{u_2}"', from=2-5, to=3-5]
	\arrow["{d_C}", from=2-6, to=3-6]
	\arrow["{p_C}"', from=3-2, to=3-3]
	\arrow["{t_C}"', from=3-5, to=3-6]
\end{tikzcd}\]

	Now, notice that $m_3$ and $m_2$ are monos because $c_C$ and $b_C$ are monos (\Cref{cor:regmono}). By the proof of \Cref{prop:eqgrph_complete}, to conclude it is enough to show that
	\[\begin{split}
		m_3 \circ q'_3 = f
	\end{split}
	\qquad
	\begin{split}
		m_2\circ q'_2 = g
	\end{split}\]
	Indeed, if the previous equations hold, then $C'_3$ and $C'_2$ are epi-mono factorizations of $f$ and $g$, and the thesis follows from \Cref{cor:unique} and the proof of \Cref{prop:eqgrph_complete}.
	
	Now, if we compute, we obtain:
	\[\begin{split}
		y_1 \circ f &= q'_4 \circ p'_V \\ &= p'_C \circ q'_3 \\ &= y_1 \circ m_3 \circ q'_3
	\end{split}
	\qquad
	\begin{split}
		u_1 \circ g &= q'_4 \circ t'_V \\ &= t'_C \circ q'_3 \\ &= u_1 \circ m_2 \circ q'_2
	\end{split}\]
	\[\begin{split}
		y_2 \circ f &= d_C \circ q'_3 \\ &= y_2 \circ m_3 \circ q'_3
	\end{split}
	\qquad
	\begin{split}
		u_2 \circ g &= d_V \circ q'_2 \\ &= u_2 \circ m_2 \circ q'_2
	\end{split}\]
	Concluding the proof.
\end{proof}

From \Cref{prop:eqgrph_complete} and \Cref{lemma:stab,lemma:van_kampen}, by \Cref{th:crit_for_adh}, we can deduce at once the following.

\begin{cor}\label{cor:eqgrph_reg_adh}
	$\EqGrph$ is $\Reg(\EqGrph)$-adhesive.
\end{cor}

\section{E-Graphs}\label{sect:eggs}

\newcommand{\EGG}{\mathbf{EGG}}


%\todo[color=green!40]{
%    COSE DA FARE:
%Questa sezione ha più o meno gli stessi problemi della precedente. L'ordine da rispettare imho è il seguente:
%>Definizione
%>Calcolo dei limiti e certi colimiti (si fanno come in EqGrph)
%> cor del calcolo dei limiti: caratterizzare mono regolari
%>I crea limiti e i giusti pushout
%> e-graph sono quasiadesivi
%}

\iffalse

E-Graphs are a particular type of graphs with equivalences, in which holds that
$$
    \frac{s(e) \sim s(e')}{t(e) \sim t(e')}
$$
for each pair of edges $e$, $e'$ of $\eqgraph{G} = (G, \sim)$.
In a more general case, considering a graph with equivalence as a functor $\eqgraph{G} : (E \rightrightarrows V \rightarrow Q) \rightarrow \Set$, the inference rule above rewrites as
\[
    \frac{\eqgraph{G}(q \circ s)(e) = \eqgraph{G}(q \circ s)(e')}{\eqgraph{G}(q \circ t)(e) = \eqgraph{G}(q \circ t)(e')}
\] for each $e$, $e' \in \eqgraph{G}(E)$.
But this is to say that, given the two set $S = \{ (e, e') \in \eqgraph{G}(E) \times \eqgraph{G}(E) \mid \eqgraph{G}(q \circ s)(e) = \eqgraph{G}(q \circ s)(e') \}$ and $T =\{ (e, e') \in \eqgraph{G}(E) \times \eqgraph{G}(E) \mid \eqgraph{G}(q \circ t)(e) = \eqgraph{G}(q \circ t)(e') \} $, $S \subseteq T$. But $S$ (with the projection arrows $p_1$ and $p_2$) is exactly the kernel pair of $\eqgraph{G}(q \circ s)$, and $T$ (together with the projections $q_1, q_2$) is the kernel pair of $\eqgraph{G}(q \circ t)$ (\Cref{ex:kernel_pairs_in_Set}). Then, a more general way to express that $\eqgraph{G}$ is an e-graph is by saying that $\eqgraph{G}$ is such that there exists a monomorphism, which is the canonical inclusion, in $\Set$ from $S$ to $T$.
To find a structure to express this fact, we have to consider a more general case.

Since both $S$ and $T$ are subobjects of $\eqgraph{G}(E) \times \eqgraph{G}(E)$, we can make use of the \Cref{prop:fact_of_subobjects_is_unique}. Specifically, we want, in the following situation, $i$ to be mono and unique.
\[
    \begin{tikzcd}
        S \arrow[rr, "i"] \arrow[dr, "{\langle p_1, p_2 \rangle}"swap] & & T \arrow[dl, "{\langle q_1, q_2\rangle}"] \\
        & \eqgraph{G}(E) \times \eqgraph{G}(E)
    \end{tikzcd}
\]
% We have then that $\langle p_1, p_2 \rangle$ is mono, then so is $\langle q_1, q_2 \rangle \circ i$. From \Cref{prop:epi_mono_prop}, we can conclude that $i$ is mono too. The uniqueness follows from \Cref{prop:fact_of_subobjects_is_unique}. If such $i$ exists, then $\eqgraph{G}$ is an e-graph.

\begin{definition}[Category of E-Graphs]\label{def:cat_of_eggs}
    $\EGG$ is the subcategory of $\EqGrph$ such that
	\begin{itemize}
		\item $\eqgraph{G} \in \EGG$ if, being $(S, p_1, p_2)$ and $(T, q_1, q_2)$, respectively, the kernel pair of $\eqgraph{G}(q \circ s)$ and the kernel pair of $\eqgraph{G}(q \circ t)$, it holds that $\langle p_1, p_2 \rangle \leq \langle q_1, q_2 \rangle$ (in the sense of \Cref{def:subobj});
		\item given two e-graphs $\eqgraph{G, H}$, $\EGG(\eqgraph{G, H}) =\EqGrph(\eqgraph{G, H})$ (i.e., a full subcategory).
	\end{itemize}
\end{definition}

\fi

E-Graphs are a particular type of graphs with equivalences.

\begin{definition}[E-Graph]\label{def:e_graph}
	Given a graph with equivalence $\eqgraph{G} = (E, V, C, s, t, q)$, let $(S, p_1, p_2)$ be the kernel pair of $q \circ s$. Then, $\eqgraph G$ is an \emph{e-graph} if it holds that
	\[
		q \circ t \circ p_1 = q \circ t \circ p_2
	\]
\end{definition}

\begin{remark}
	From a set-theoretic point of view, $\eqgraph{G} = (\graph{G}, \sim)$ is a graph with equivalence, where $\graph{G} = (E, V, s, t)$ in which holds that, for each pair of edges $e$, $e'$,
	\[
		\frac{s(e)\sim s(e')}{t(e)\sim t(e')}
	\]
	As we have considered in the section on graphs with equivalences, we can formalize the equivalence relation with a surjective function $q: V \to V/_\sim = C$, rewriting the inference rule above as an inclusion.
	Let $S = \{(e, e') \in E\times E \mid q(s(e)) = q(s(e'))\}$ and $T = \{(e, e') \in E \times E \mid q(t(e)) = q(t(e'))\}$. Then, $\eqgraph{G}$ is an e-graph if $S \subseteq T$. As we noted in \Cref{ex:kernel_pairs_in_Set}, $S$ is the kernel pair of $q \circ s$ and $T$ is the kernel pair of $q \circ t$, if endowed with canonical projections. Moreover, we notice that the pairing of such projections yields a subobject of the product $E \times E$ \Cref{prop:pairng_of_kernel_pairs_mono}. Hence, by \Cref{rem:fact_of_subobject_is_unique}, the \Cref{def:e_graph} express this fact using categorial constructions.
\end{remark}

\begin{definition}[Category of E-Graphs]\label{def:cat_of_eggs}
	$\EGG$ is the subcategory of $\EqGrph$ in which objects are e-graphs, and, given two e-graphs $\eqgraph{G, H}$, $\EGG(\eqgraph{G, H}) = \EqGrph(\eqgraph{G, H})$.
\end{definition}

Since $\EGG$ is a full subcategory of $\EqGrph$, there exists a fully faithful functor $I: \EGG \to \EqGrph$ (\Cref{ex:full_subc_inc_fully_faith}).

\begin{lemma}
	$\EGG$ has all limits and $I$ preserves them.
\end{lemma}

\begin{proof}
	Let $D: \cat{I} \to \EGG$ be a diagram, with $D(i) = (A_i, B_i, C_i, s_i, t_i, q_i)$, let $(U, u_1^i, u_2^i)$ be the kernel pair of $q_i\circ s_i$. Let now be $(A, B, C, s, t, q)$, togehter with projections $(\pi_E^i, \pi_V^i, \pi_C^i)_{i \in \cat I}$ the limit of $I \circ D$, let $(U, u_1, u_2)$ be the kernel pair of $q\circ s$ and let $(L, (l_i)_{i \in \cat I})$ be the limit of $V\circ I\circ D$.
	By construction (proof of \Cref{???}\todo{citazione}), there exists a monomorphism $m: Q \to L$ such that $\pi_C^i = l_i \circ m$. Notice that
	\begin{align*}
		q_i\circ s_i\circ \pi^i_E\circ u_1 	&= q_i\circ \pi^i_V\circ s\circ u_1\\
							&= \pi_C^i\circ q\circ s\circ u_1\\
							&=\pi_C^i\circ q\circ s\circ u_2\\
							&= q_i \circ s_i \circ \pi_E^i \circ u_2
	\end{align*}
	Then, for each $i$, there exists an arrow $a_i:U\to U_i$ making the following diagram to commute
	\[
		\begin{tikzcd}[row sep = 25, column sep = 25]
			U \ar[r, "{u_1}"] \ar[d, "{u_2}"swap] \ar[dr, dashed, "{a_i}"] & A \ar[drr, "{\pi_E^i}"] && \\
			A \ar[ddr, "{\pi_E^i}"swap] & U_i \ar[rr, "{u_1^i}"] \ar[dd, "{u_2^i}"swap]&& A_i \ar[d, "{s_i}"] \\
			& & & B_i \ar[d, "{q_i}"]\\
			& A_i \ar[r, "{s_i}"swap] & B_i \ar[r, "{q_i}"swap] & C_i
		\end{tikzcd}
	\]
	We have then
	\begin{align*}
		l_i\circ m \circ q \circ t \circ u_1	&= q_i \circ \pi_V^i \circ t \circ u_1 \\
							&= q_i \circ t_i \circ \pi_E^i \circ u_1 \\
							&= q_i \circ t_i \circ u_1^i \circ a_i \\
							&= q_i \circ t_i \circ u_2^i \circ a_i \\
							&= q_i \circ t_i \circ \pi_E^i \circ u_2 \\
							&= q_i \circ \pi_V^i \circ t \circ u_2 \\
							&= l_i \circ m \circ q \circ t \circ u_2
	\end{align*}
	By universal property of limits, we have that \- $m\circ q \circ t \circ u_1 = m \circ q \circ t \circ u_2$, and, since $m$ is mono, $q \circ t \circ u_1 = q \circ t \circ u_2$, hence the thesis.
\end{proof}

\begin{cor}\label{cor:I_creates_limits}
	$I$ creates limits.
\end{cor}

\begin{cor}\label{cor:reg_mono_EGG}
	$h: \eqgraph{G \to H}$ is a regular mono in $EGG$ if and only if it is a regular mono in $\EqGrph$.
\end{cor}

\begin{lemma}\label{lem:I_pres_pushouts_of_monos}
	$I$ preserves pushouts along regular monomorphisms.	
\end{lemma}

\begin{proof}
	Let $\eqgraph{G}_1 = (A_1, B_1, C_1, s_1, t_1, q_1)$, $\eqgraph{G}_2 = (A_2, B_2, C_2, s_2, t_2, q_2)$ and $\eqgraph{G}_3 = (A_3, B_3, C_3, s_3, t_3, q_3)$ be e-graphs,  $h: \eqgraph{G}_1 \to \eqgraph{G}_2$ and $m: \eqgraph{G}_1 \to \eqgraph{G}_3$ be two morphisms with $m$ mono. Let now $\eqgraph{P} = (A, B, C, s, t, q)$, together with $k: \eqgraph{G}_3 \to \eqgraph{P}$ and $n: \eqgraph{G}_2 \to \eqgraph{P}$, the pushout of $h$ and $m$.
	Let $(K_i, \pi_i^1, \pi_i^2)$ the kernel pair of $q_i\circ s_i$, for $i = 1, 2, 3$, and let $(U, u_1, u_2)$ be the kernel pair of $q \circ s$.
	Consider the following cube in $\Set$
	\[\begin{tikzcd}[row sep=25, column sep = 25]
	& {A_1} && {A_2} \\
	{A_3} && A \\
	& {C_1} && {C_2} \\
	{C_3} && C
	\arrow["{h_E}", from=1-2, to=1-4]
	\arrow["{m_E}"', from=1-2, to=2-1]
	\arrow["{q_1\circ s_1}"'{pos=0.7}, from=1-2, to=3-2]
	\arrow["{n_E}", from=1-4, to=2-3]
	\arrow["{q_2\circ s_2}", from=1-4, to=3-4]
	\arrow["{k_E}"{pos=0.7}, from=2-1, to=2-3, crossing over]
	\arrow["{q_3\circ s_3}"', from=2-1, to=4-1]
	\arrow["{h_C}"{pos=0.8}, from=3-2, to=3-4]
	\arrow["{m_C}"', from=3-2, to=4-1]
	\arrow["{n_C}", from=3-4, to=4-3]
	\arrow["{k_C}"', from=4-1, to=4-3]
	\arrow["{q\circ s}"'{pos=0.2}, from=2-3, to=4-3, crossing over]
	\end{tikzcd}\]
	By adhesivity of $\Set$, and since $m$ is regular mono, the cube above is Van Kampen, and, by \Cref{prop:monos_are_preserved_by_pullbacks_in_adh_cats}, all the faces are pullbacks. Then, applying \Cref{lemma:pushouts_kernel_pairs}, the square below is a pushout
	\[\begin{tikzcd}[row sep=27, column sep = 27]
		K_1 \ar[r, "{f_h}"] \ar[d, "{f_m}"swap] & K_2 \ar[d, "{f_n}"] \\
		K_3 \ar[r, "{f_k}"swap] & U
	\end{tikzcd}\]

	Hence, we end up with the following situation
	\[
		\begin{tikzcd}[row sep = 20, column sep = 20]
	& {K_1} && {K_2} \\
	{K_3} && U \\
	& {A_1} && {A_2} \\
	{A_3} && A \\
	& {C_1} && {C_2} \\
	{C_3} && C
	\arrow["{f_h}", from=1-2, to=1-4]
	\arrow["{f_m}"', from=1-2, to=2-1]
	\arrow["{\pi_1^1}"'{pos=0.7}, from=1-2, to=3-2]
	\arrow["{f_n}", from=1-4, to=2-3]
	\arrow["{\pi^1_2}", from=1-4, to=3-4]
	\arrow["{f_k}"{pos=0.8}, from=2-1, to=2-3, crossing over]
	\arrow["{\pi_3^1}"', from=2-1, to=4-1]
	\arrow["{u_1}"'{pos=0.3}, from=2-3, to=4-3, crossing over]
	\arrow["{h_E}"{pos=0.8}, from=3-2, to=3-4]
	\arrow["{m_E}"', from=3-2, to=4-1]
	\arrow["{q_1\circ s_1}"'{pos=0.7}, from=3-2, to=5-2]
	\arrow["{n_E}", from=3-4, to=4-3]
	\arrow["{q_2\circ s_2}", from=3-4, to=5-4]
	\arrow["{k_E}"{pos=0.7}, from=4-1, to=4-3, crossing over]
	\arrow["{q_3\circ s_3}"', from=4-1, to=6-1]
	\arrow["{h_C}"{pos=0.8}, from=5-2, to=5-4]
	\arrow["{q\circ s}"'{pos=0.2}, from=4-3, to=6-3, crossing over]
	\arrow["{m_C}"', from=5-2, to=6-1]
	\arrow["{n_C}", from=5-4, to=6-3]
	\arrow["{k_C}"', from=6-1, to=6-3]
	\end{tikzcd}
		\qquad
		\begin{tikzcd}[row sep = 20, column sep = 20]
	& {K_1} && {K_2} \\
	{K_3} && U \\
	& {A_1} && {A_2} \\
	{A_3} && A \\
	& {C_1} && {C_2} \\
	{C_3} && C
	\arrow["{f_h}", from=1-2, to=1-4]
	\arrow["{f_m}"', from=1-2, to=2-1]
	\arrow["{\pi_1^2}"'{pos=0.7}, from=1-2, to=3-2]
	\arrow["{f_n}", from=1-4, to=2-3]
	\arrow["{\pi^2_2}", from=1-4, to=3-4]
	\arrow["{f_k}"{pos=0.8}, from=2-1, to=2-3, crossing over]
	\arrow["{\pi_3^2}"', from=2-1, to=4-1]
	\arrow["{u_2}"'{pos=0.3}, from=2-3, to=4-3, crossing over]
	\arrow["{h_E}"{pos=0.8}, from=3-2, to=3-4]
	\arrow["{m_E}"', from=3-2, to=4-1]
	\arrow["{q_1\circ s_1}"'{pos=0.7}, from=3-2, to=5-2]
	\arrow["{n_E}", from=3-4, to=4-3]
	\arrow["{q_2\circ s_2}", from=3-4, to=5-4]
	\arrow["{k_E}"{pos=0.7}, from=4-1, to=4-3, crossing over]
	\arrow["{q_3\circ s_3}"', from=4-1, to=6-1]
	\arrow["{h_C}"{pos=0.8}, from=5-2, to=5-4]
	\arrow["{q\circ s}"'{pos=0.2}, from=4-3, to=6-3, crossing over]
	\arrow["{m_C}"', from=5-2, to=6-1]
	\arrow["{n_C}", from=5-4, to=6-3]
	\arrow["{k_C}"', from=6-1, to=6-3]
		\end{tikzcd}\]

	Computing, we have
	\[
		\begin{split}
			q \circ t \circ u_1 \circ f_n &= q \circ t \circ n_E \circ \pi_2^1 \\ &= n_C \circ q_2 \circ s_2 \circ \pi_2^1 \\&= n_C \circ q_2 \circ s_2 \circ \pi_2^2 \\&=q \circ t \circ u_2 \circ f_n 
			\end{split}
			\qquad
			\begin{split} q \circ t \circ u_1 \circ f_k &= q \circ t \circ k_E \circ \pi_3^1 \\ &= k_C \circ q_3 \circ s_3 \circ \pi_3^1 \\&= k_C \circ q_3 \circ s_3 \circ \pi_3^2 \\&=q \circ t \circ u_2 \circ f_k
			\end{split}
	\]
	By universal property of pushouts, we can conclude $q \circ t \circ u_1 = q \circ t \circ u_2$, hence the thesis.
\end{proof}


\begin{cor}
	$I$ creates pushouts along regular monos.
\end{cor}

By direct application of \Cref{th:crit_for_adh}, we can conclude what follows.

\begin{cor}
	$\EGG$ is $\Reg(\EGG)-adhesive$.
\end{cor}

% APPUNTI
%Chiamiamo $(A_i, B_i, Q_i, s_i, t_i, q_i)$ le componenti del diagramma. Chiamiamo $(U_i, u_{1, i} , u_{2,i})$ il kernel pair di $q_i\circ s_i$.  Sia $(A, B, Q, s,t,q )$ il limite in $\EqGrph$ e $(U, u_1, u_2)$ il kernel pair di $q\circ s$. Vogliamo far vedere che $q\circ t \circ u_1= q\circ t\circ u_2$. Sia anche $(L, l_i)$ il limite del diagramma dato dai $Q_i$. In particolare abbiamo $m\colon Q\to L$ mono.

%Osserviamo che 
%\begin{align*}
%q_i\circ s_i\circ \pi^i_E\circ u_1 &= q_i\circ \pi^i_V\circ s\circ u_1\\&= \pi^i\circ q\circ s\circ u_1\\&=\pi^i\circ q\circ s\circ u_2=\dots (percorso al contrario)
%\end{align*}

%Dunque per ogni $i$ esiste una freccia $a_i\colon U\to U_{i}$ tale che il diagramma sotto commuta
%\[
%\xymatrix{U \ar@{.>}[dr]^{a_i}\ar[r]^{u_1} \ar[d]_{u_2}& A \ar[drr]^{\pi^i_E}\\A \ar[ddr]_{\pi^i_E}&U_i \ar[dd]_{u_{2,i}} \ar[rr]^{u_{1,i}}& & A_i \ar[d]^{s_i}\\
%& & &B_i \ar[d]^{q_i}\\&A_i \ar[r]_{s_i} & B_i \ar[r]_{s_i} & Q_i}
%\]


%\begin{align*}
%	l_i\circ m\circ q\circ t \circ u_1 &= q_i \circ \pi^{i}_V\circ t\circ u_1 \\&=q_i\circ t_i\circ \pi^i_E\circ u_1\\&=q_i\circ t_i\circ u_{1,i}\circ a_i\\&=q_i\circ t_i\circ u_{2,i}\circ a_i\\=\dots (uguaglianze al contrario)
%\end{align*}

%Dunque le frecce $m\circ q\circ t \circ u_1, m\circ q\circ t \circ u_2\colon U\rightrightarrows L$ sono uguali. Ma $m$ è mono per costruzione e allora $q\circ t \circ u_1=q\circ t \circ u_2$.

%FINE APPUNTI 




% ----------------- Appendix ------------------ %
\appendix

\chapter{Omitted Proofs}

\paragraph{\Cref{lemma:pullback_lemma}.}
    Suppose that the following diagram is given and its right half is a pullback. Then the whole rectangle is a pullback if and only if its left half is a pullback.
    \[
        \begin{tikzcd}
            X \arrow[r, "f"] \arrow[d, "t"swap] & Y \arrow[r, "g"] \arrow[d, "k"] & Z \arrow[d, "h"] \\
            A \arrow[r, "a"swap] & B \arrow[r, "b"swap] & C
        \end{tikzcd}
    \]
\begin{proof}
    For the ``only if'' part, suppose the left square to be a pullback. To verify the outer rectangle is a pullback, consider the following situation:
    \[
        \begin{tikzcd}
            P \arrow[bend left=30]{rrrd}{p_1} \arrow[bend right=30, swap]{ddr}{p_2} & & \\
            & & & Z \arrow[d, "h"] \\
            & A \arrow[r, "a"swap] & B \arrow[r, "b"swap] & C
        \end{tikzcd}
    \]
    But the right square is a pullback implies that there exists a unique $u$ such that
    \[
        \begin{tikzcd}
            P \arrow[bend left=30]{rrrd}{p_1} \arrow[bend right=30, swap]{ddr}{p_2} \arrow[bend left=15, dashed]{drr}{u}& & \\
            & &  Y \arrow[r, "g"] \arrow[d, "k"]& Z \arrow[d, "h"] \\
            & A \arrow[r, "a"swap] & B \arrow[r, "b"swap] & C
        \end{tikzcd}
    \]
    And, since the left square is a pullback, there exists a unique $v$ such that
    \[
        \begin{tikzcd}
            P \arrow[bend left=30]{rrrd}{p_1} \arrow[bend right=30, swap]{ddr}{p_2} \arrow[bend left=15]{drr}{u} \arrow[dr, "v", dashed]& & \\
            & X \arrow[r, "f"] \arrow[d, "t"swap] &  Y \arrow[r, "g"] \arrow[d, "k"]& Z \arrow[d, "h"] \\
            & A \arrow[r, "a"swap] & B \arrow[r, "b"swap] & C
        \end{tikzcd}
    \]
    Hence, the whole rectangle is a pullback.

    For the ``if'' part, consider the following situation.
    \[
        \begin{tikzcd}
            P \arrow[bend left=30]{rrd}{p_1} \arrow[bend right=30, swap]{ddr}{p_2} & \\
            & & Y \arrow[d, "k"] \\
            & B \arrow[r, "b"swap] & C
        \end{tikzcd}
    \]
    We have now to show that the unique arrow $v: P \to X$ (of the outer rectangle) is such that $f \circ v = p_1$, but this follows from the fact that the right square is a pullback, having the following situation.
    \[
        \begin{tikzcd}
            P \arrow[bend left=30]{rrrd}{g \circ p_1} \arrow[bend right=30, swap]{ddr}{p_2} \arrow[bend left=15, dashed]{drr}{u}& & \\
            & &  Y \arrow[r, "g"] \arrow[d, "k"]& Z \arrow[d, "h"] \\
            & A \arrow[r, "a"swap] & B \arrow[r, "b"swap] & C
        \end{tikzcd}
    \]
    \todo{Rivedere questa dimostrazione}
\end{proof}    

\paragraph{\Cref{th:limit}.}
 Let $\cat C$ be a category. Then $\cat C$ has all finite limits if and only if $\cat C$ has all finite products and all equalizers.
\begin{proof}
    Let $D: \cat{I \rightarrow C}$ a diagram, with $\cat I$ finite.
    
    The \emph{if} statement follows from definitions of products and equalizers (\Cref{def:pullback_pushout}, \Cref{def:equalizer_coequalizer})

    To satisfy the \emph{only if} statement, we want an object $L$ together with morphisms $p_i : L \rightarrow D(j)$ such that:
    \begin{enumerate}
        \item\label{item:cone} $\{p_i: L \rightarrow D(i)\}$ is a cone -- i.e., for each morphism of $\cat I$ $\alpha : i \rightarrow j$, $D(\alpha) \circ p_i = p_j$; and
        \item\label{item:univ_prop} for each $E$ and $q_i : E \rightarrow D(j)$ in $\cat C$, with $D(\alpha) \circ q_i = q_j$ for each $\alpha : i \rightarrow j$ of $\cat I$, there exists a unique $f: E \rightarrow L$ such that $q_i = p_i \circ f$ for each $i \in \Ob(\cat I)$.
    \end{enumerate}

    Consider the two products (which exist by hypothesis) $\prod_{j \in \Ob(\cat I)} D(j)$, the product of the objects of the diagram, and $\prod_{\alpha \in \Hom(\cat I)}D(cod \ \alpha)$, the product of the codomains of the morphisms in $D$, where $\pi_x$ is the $x$-th projection of the product.
    Let now:
    \[
        \gamma, \varepsilon : \prod_{j \in \Ob(\cat I)}D(j) \ \longrightarrow \prod_{\alpha \in \Hom(\cat I)} D(cod \ \alpha)
    \]
    be defined by $\gamma_\alpha = \pi_{D(cod\ \alpha)}$ (the projection on the codomain of $\alpha$) and $\varepsilon_\alpha = D(\alpha) \circ \pi_{D(dom \ \alpha)}$.
    Let now $e: L \rightarrow \prod_{j \in \Ob(\cat I)}D(j)$ the equalizer of $\gamma$ and $\varepsilon$ (which exists by hypothesis), and, for each $j \in \Ob(\cat I)$, $p_j: L \rightarrow D(j)$, defined by $p_j = \pi_{D(j)} \circ e$.
    
    What we want now is to show that $(L, (p_i))_{i \in \cat I}$ is the limit of $D$, namely, to prove that the conditions given at the beginning are valid.

    For condition~\ref{item:cone}, we have to show that, for each $\alpha : i \rightarrow j$ of $\cat I$, we have $D(\alpha) \circ p_i = p_j$:
    \begin{align*}
        D(\alpha) \circ p_i 
            &= D(\alpha) \circ \pi_{D(i)} \circ e   && \textit{Definition of $p_j$} \\
            &= \varepsilon_{\alpha} \circ e         && \textit{Definition of $\varepsilon$}\\
            &= \gamma_\alpha \circ e                && \textit{$e$ is an equalizer of $\pi, \varepsilon$}\\
            &= \pi_{D(j)} \circ e                   && \textit{Definition of $\pi$} \\
            &= p_j                                  && \textit{Definition of $p_j$} \\
    \end{align*}

    For condition~\ref{item:univ_prop}, suppose that $(E, (q_i)_{i \in \Ob(\cat I)})$ has the properties stated. By definition of product, there exists a (unique) arrow $q: E \rightarrow \prod_{j \in \Ob(\cat I)}D(j)$. For each arrow $\alpha: i \rightarrow j$, we have:
    \begin{align*}
            \gamma_\alpha \circ q 
                &= \pi_{D(j)} \circ q           && \textit{Definition of $\pi$} \\
                &= q_j                          && \textit{Definition of $q_j$} \\
                &= D(\alpha) \circ q_i          && \textit{Assumption on $q_j$} \\
                &= D(\alpha) \circ \pi_{D(j)} \circ q   &&\textit{Definition of $q_i$} \\
                &= \varepsilon_{\alpha} \circ q && \textit{Definition of $\varepsilon$} \\
    \end{align*}
    Since $e$ equalizes $\pi$ and $\varepsilon$, there exists a unique $f: E \rightarrow L$ in $\cat C$ such that $q = e \circ f$. Then, for each $j \in \Ob(\cat I)$, we have $\pi_{D(j)} \circ q = \pi_{D(j)} \circ e \circ f$, hence, $q_i = p_i \circ f$.
\end{proof}
 		

% ---------------- Bibliography ---------------- %
\bibliographystyle{alpha}

\bibliography{bibliography}

\end{document}

% --------------- Commenti ------------------- %

{% \color{red}{ [DC]Alcuni commenti su PRIMO capitolo. Non sono soddisfatto dall'ordine della presentazione. Io  personalmente proporrei il seguente indice:

% Sec. 1 Categorie e funtori

% >def di categorie
% 	-menzionare che data una categoria c'è il duale
% 	-epi, mono e  iso

% >funtori e trasf naturali

% >categorie comma (con slice e coslice come sottocaso)

% Sec. 2: Limiti e colimiti
% Io sinceramente questa la riscriverei completamente. La definizione di (co)limite deve venire prima e solo dopo deve venire l'elenco degli esempi. Struttura che propongo:

% >definizione, esempi e se avanza tempo come si calcolano (co)limiti generali a partire da quelli "base" (tipo prodotti da pb e equalizzatori) 
% 	-epi e mono regolari

% >Limiti nelle categorie di funtori e comma


% Sec. 3: Adesività
% Unica osservazione che ho per ora è che terrei il cubo invece del diagramma che hai fatto}
}
