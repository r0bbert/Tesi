\section{Graphs}\label{sect:graphs}

A (directed graph) $\graph{G}$ is a structure consisting of a set of edge, a set of nodes and two functions, one assigning a \emph{source} node and one assigning a \emph{target} node to an edge. Formally, $\graph{G}$ is a quadruple $(V_\graph{G}, E_\graph{G}, s_\graph{G}, t_\graph{G})$, where $V_\graph{G}$ is the set of nodes, $E_\graph{G}$ is the set of edges, and $s_\graph{G}, t_\graph{G}: E_\graph{G} \rightarrow V_\graph{G}$ are the source and the target functions.

A \emph{graph homomorphism} $h: \graph{G \rightarrow H}$ is then a pair of functions $h = (h_V: V_\graph{G} \rightarrow V_\graph{H}, h_E: E_\graph{G} \rightarrow E_\graph{H})$ such that
    \[
        h_V \circ s_\graph{G} = s_\graph{H} \circ h_E
    \]
    and
    \[
        h_V \circ t_\graph{G} = t_\graph{H} \circ h_E
    \]
that is, a structure preserving map.

We can then generalize such notion to something more abstract, considering a graph to be nothing more than a presheaf from the category $(E \rightrightarrows V)$ to the category of sets.
Having two of such presheaves, a natural transformation from one to another encapsulates the behavior of a graph morphism due to naturality. We can now define the category of graphs.

\begin{definition}[Category of Graphs]\label{def:cat_of_graph}
    We denote as $\Graph$ the category $$[E \mathrel{\mathop{\rightrightarrows}^{s}_{t}} V , \Set]$$
\end{definition}

Since $\Graph$ is a category of presheaves, \Cref{lemma:limits_of_presheaves} guarantees the existence of limits and colimits, and gives us an easy way to compute them.

\begin{cor}\label{cor:graph_has_co_limits}
    $\Graph$ has all limits and colimits.
\end{cor}

\begin{example}\label{ex:in_term_in_graph}
    The initial object in $\mathbf{Graph}$ is the empty graph, i.e., the graph with an empty set of vertices and an empty set of edges. The initial object instead is the graph with exactly one node and a single edge from that node to itself.
\end{example}

\begin{example}
    Given two graphs $G = (V_G, E_G, s_G, t_G)$ and $H=(V_H, E_H, s_H, t_H)$, the graph $G \times H = (V_G\times V_H, E_G \times E_H, (s_G, s_H), (t_G, t_H))$, where $(s_G, s_H), (t_G, t_H):V_G\times V_H \rightarrow E_G \times E_H$ are the pairwise sources and targets, is the categorical product in $\mathbf{Graph}$, together with the two projections $\pi_G: G \times H \rightarrow G$, $\pi_H : G \times H \rightarrow H$ defined in the obvious way.
\end{example}

\begin{example}
    The equalizer of two morphisms $h, k: G \rightarrow H$ in $\mathbf{Graph}$ is defined as in $\Set$, that is,  a graph $Q$ together with a graph morphism $q$ that is the restriction of $G$ to all the vertices and all the arcs that are mapped on the same vertices and edges both from $h$ and $k$. Formally, $(Q, q)$ is an equalizer for $h, k: G \rightarrow H$, $h = (h_V, h_E), k = (k_V, k_E)$ where $V(Q) = \{ n \in V(G) \mid h_V(n) = k_V(n)\}$, $E(Q) = \{ e \in E(G) \mid h_E(e) = k_E(e)\}$, $s_Q = s_G \mid_{V(Q)}$, $t_E = t_G \mid_{V(Q)}$.
\end{example}

%\todo{Dimostrare come si calcolano i limiti nelle categorie di prefasci (conponente per componente) e poi dare qualche esempio, porendendolo dalla versione precedente. Dimostrare anche come sono fatti i mono (mono sulle componenti) (Nei prefasci dipende dalla caratterizzqazione dei mono via pullback (Vedi roba kernel pairs)).}
\begin{remark}
    TODO: Si può generalizzare a tutte le categorie regolari per evitare di perdere le proprietà che usiamo (da eq.rel. a quot.).
\end{remark}
