\section{E-Graphs}\label{sect:eggs}

\todo[color=green!40]{
    COSE DA FARE:
Questa sezione ha più o meno gli stessi problemi della precedente. L'ordine da rispettare imho è il seguente:
>Definizione
>Calcolo dei limiti e certi colimiti (si fanno come in EqGrph)
> cor del calcolo dei limiti: caratterizzare mono regolari
>I crea limiti e i giusti pushout
> e-graph sono quasiadesivi
}

E-Graphs are a particular type of graphs with equivalences, in which holds that
$$
    \frac{s(e) \sim s(e')}{t(e) \sim t(e')}
$$
for each pair of edges $e$, $e'$ of $\eqgraph{G} = (G, \sim)$.
In a more general case, considering a graph with equivalence as a functor $\eqgraph{G} : (E \rightrightarrows V \rightarrow Q) \rightarrow \Set$, the inference rule above rewrites as
\[
    \frac{\eqgraph{G}(q \circ s)(e) = \eqgraph{G}(q \circ s)(e')}{\eqgraph{G}(q \circ t)(e) = \eqgraph{G}(q \circ t)(e')}
\] for each $e$, $e' \in \eqgraph{G}(E)$.
But this is to say that, given the two set $S = \{ (e, e') \in \eqgraph{G}(E) \times \eqgraph{G}(E) \mid \eqgraph{G}(q \circ s)(e) = \eqgraph{G}(q \circ s)(e') \}$ and $T =\{ (e, e') \in \eqgraph{G}(E) \times \eqgraph{G}(E) \mid \eqgraph{G}(q \circ t)(e) = \eqgraph{G}(q \circ t)(e') \} $, $S \subseteq T$. But $S$ (with the projection arrows $p_1$ and $p_2$) is exactly the pullback of $(q \circ s, q \circ s)$, and $T$ (together with the projections $q_1, q_2$) is the pullback of $(q \circ t, q \circ t)$. Then, a more general way to express that $\eqgraph{G}$ is an e-graph is by saying that $\eqgraph{G}$ is such that there exists a monomorphism, which is the canonical inclusion, in $\Set$ from $S$ to $T$.
To find a structure to express this fact, we have to consider a more general case.

Consider an arrow $f: X \rightarrow Y$, and let $(K, \pi_1, \pi_2)$ be the pullback of $(f, f)$.
\[
    \begin{tikzcd}
        K \arrow[r, "{\pi_1}"] \arrow[d, "{\pi_2}"swap] & X \arrow[d, "f"] \\
        X \arrow[r, "f"swap] & Y
    \end{tikzcd}
\]
Such pullback induces an arrow $\langle \pi_1, \pi_2 \rangle : K \rightarrow X\times X$. Such arrow is mono and unique, because of the universal property of the pullback, indeed a subobject.
\todo{capire quanto andare nello specifico qui}

Since both $S$ and $T$ are subobjects of $\eqgraph{G}(E) \times \eqgraph{G}(E)$, we can make use of the \Cref{prop:fact_of_subobjects_is_unique}. Specifically, we want, in the following situation, $i$ to be mono and unique.
\[
    \begin{tikzcd}
        S \arrow[rr, "i"] \arrow[dr, "{\langle p_1, p_2 \rangle}"swap] & & T \arrow[dl, "{\langle q_1, q_2\rangle}"] \\
        & \eqgraph{G}(E) \times \eqgraph{G}(E)
    \end{tikzcd}
\]
We have then that $\langle p_1, p_2 \rangle$ is mono, then so is $\langle q_1, q_2 \rangle \circ i$. From \Cref{prop:epi_mono_prop}, we can conclude that $i$ is mono too. The uniqueness follows from \Cref{prop:fact_of_subobjects_is_unique}. If such $i$ exists, then $\eqgraph{G}$ is an e-graph.

\begin{definition}[Category of E-Graphs]\label{def:cat_of_eggs}
    The subcategory of $\mathbf{EqGrph}$ whose objects are this particular kind of graphs is denoted as $\mathbf{EGG}$.
\end{definition}

\begin{prop}\label{prop:prod_of_EGGs_is_EGG}
    The product of two e-graphs in $\mathbf{EqGrph}$ is an e-graph.
\end{prop}

\begin{proof}
    Let $\eqgraph{G}$, $\eqgraph{H}$ be two e-graphs in $\mathbf{EqGrph}$. Then, we want to to show that $\eqgraph{G \times H}$ is an e-graph too. The argument lies on the consideration that limits in presheaves categories are computed pointwise. In fact, we can 
\end{proof}

Consider now the inclusion functor $I: \mathbf{EGG} \rightarrow \mathbf{EqGrph}$. Since $\mathbf{EGG}$ is a full subcategory of $\mathbf{EqGrph}$, $I$ is full and faithful (\Cref{ex:full_subc_inc_fully_faith}), it reflects all limits (\Cref{prop:inc_funct_reflects_so_limits}). But limits are also preserved, since the limit in $\mathbf{EqGrph}$ in which objects are e-graphs is an e-graph together with morphisms that are also morphisms of $\mathbf{EGG}$ since it is a full subcategory. Then, we can conclude as follows.

\begin{lemma}
    The inclusion functor $I: \mathbf{EGG \rightarrow EqGrph}$ creates limits.
\end{lemma}

\begin{proof}
    To prove that $I$ creates limits, we have to show that both preserves and reflects limits.
    To see that $I$ preserves limits, it is sufficient to note that a limit of e-graphs in $\mathbf{EqGrph}$ is again an e-graph, together with morphisms. (Note that, since $\Egg$ is a full subcategory of $\mathbf{EqGrph}$, these morphisms in $\mathbf{EqGrph}$ are morphisms of $\Egg$ too).
\end{proof}

Since $I$ is faithful, monomorphisms in $\mathbf{EqGrph}$ between graphs that are e-graphs too are monomorphisms in $\Egg$ too. Regular monomorphisms in $\Egg$ are, as in $\mathbf{EqGrph}$, monomorphisms which reflect equivalences, hence natural transformations with all the three components mono (\Cref{prop:reg_mono_in_EG_are_mono_in_graph}). The following result follows by the fact that a full and faithful functor preserves equalizers. {\color{red}{???? Da dimostrare}}

\begin{prop}
    Let $I$ be the inclusion functor from $\Egg$ to $\mathbf{EqGrph}$. Then, $I(\Reg(\Egg)) \subseteq \Reg(\mathbf{EqGrph})$.
\end{prop}

At this point, by direct application of \Cref{th:crit_for_adh}, it is possible to state what follows.

\begin{cor}
    $\mathbf{EGG}$ is quasiadhesive.
\end{cor}
