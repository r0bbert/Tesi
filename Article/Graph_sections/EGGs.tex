\section{E-Graphs}\label{sect:eggs}

\newcommand{\EGG}{\mathbf{EGG}}


%\todo[color=green!40]{
%    COSE DA FARE:
%Questa sezione ha più o meno gli stessi problemi della precedente. L'ordine da rispettare imho è il seguente:
%>Definizione
%>Calcolo dei limiti e certi colimiti (si fanno come in EqGrph)
%> cor del calcolo dei limiti: caratterizzare mono regolari
%>I crea limiti e i giusti pushout
%> e-graph sono quasiadesivi
%}

\iffalse

E-Graphs are a particular type of graphs with equivalences, in which holds that
$$
    \frac{s(e) \sim s(e')}{t(e) \sim t(e')}
$$
for each pair of edges $e$, $e'$ of $\eqgraph{G} = (G, \sim)$.
In a more general case, considering a graph with equivalence as a functor $\eqgraph{G} : (E \rightrightarrows V \rightarrow Q) \rightarrow \Set$, the inference rule above rewrites as
\[
    \frac{\eqgraph{G}(q \circ s)(e) = \eqgraph{G}(q \circ s)(e')}{\eqgraph{G}(q \circ t)(e) = \eqgraph{G}(q \circ t)(e')}
\] for each $e$, $e' \in \eqgraph{G}(E)$.
But this is to say that, given the two set $S = \{ (e, e') \in \eqgraph{G}(E) \times \eqgraph{G}(E) \mid \eqgraph{G}(q \circ s)(e) = \eqgraph{G}(q \circ s)(e') \}$ and $T =\{ (e, e') \in \eqgraph{G}(E) \times \eqgraph{G}(E) \mid \eqgraph{G}(q \circ t)(e) = \eqgraph{G}(q \circ t)(e') \} $, $S \subseteq T$. But $S$ (with the projection arrows $p_1$ and $p_2$) is exactly the kernel pair of $\eqgraph{G}(q \circ s)$, and $T$ (together with the projections $q_1, q_2$) is the kernel pair of $\eqgraph{G}(q \circ t)$ (\Cref{ex:kernel_pairs_in_Set}). Then, a more general way to express that $\eqgraph{G}$ is an e-graph is by saying that $\eqgraph{G}$ is such that there exists a monomorphism, which is the canonical inclusion, in $\Set$ from $S$ to $T$.
To find a structure to express this fact, we have to consider a more general case.

Since both $S$ and $T$ are subobjects of $\eqgraph{G}(E) \times \eqgraph{G}(E)$, we can make use of the \Cref{prop:fact_of_subobjects_is_unique}. Specifically, we want, in the following situation, $i$ to be mono and unique.
\[
    \begin{tikzcd}
        S \arrow[rr, "i"] \arrow[dr, "{\langle p_1, p_2 \rangle}"swap] & & T \arrow[dl, "{\langle q_1, q_2\rangle}"] \\
        & \eqgraph{G}(E) \times \eqgraph{G}(E)
    \end{tikzcd}
\]
% We have then that $\langle p_1, p_2 \rangle$ is mono, then so is $\langle q_1, q_2 \rangle \circ i$. From \Cref{prop:epi_mono_prop}, we can conclude that $i$ is mono too. The uniqueness follows from \Cref{prop:fact_of_subobjects_is_unique}. If such $i$ exists, then $\eqgraph{G}$ is an e-graph.

\begin{definition}[Category of E-Graphs]\label{def:cat_of_eggs}
    $\EGG$ is the subcategory of $\EqGrph$ such that
	\begin{itemize}
		\item $\eqgraph{G} \in \EGG$ if, being $(S, p_1, p_2)$ and $(T, q_1, q_2)$, respectively, the kernel pair of $\eqgraph{G}(q \circ s)$ and the kernel pair of $\eqgraph{G}(q \circ t)$, it holds that $\langle p_1, p_2 \rangle \leq \langle q_1, q_2 \rangle$ (in the sense of \Cref{def:subobj});
		\item given two e-graphs $\eqgraph{G, H}$, $\EGG(\eqgraph{G, H}) =\EqGrph(\eqgraph{G, H})$ (i.e., a full subcategory).
	\end{itemize}
\end{definition}

\fi

E-Graphs are a particular type of graphs with equivalences.

\begin{definition}[E-Graph]\label{def:e_graph}
	Given a graph with equivalence $\eqgraph{G} = (E, V, C, s, t, q)$, let $(S, p_1, p_2)$ be the kernel pair of $q \circ s$. Then, $\eqgraph G$ is an \emph{e-graph} if it holds that
	\[
		q \circ t \circ p_1 = q \circ t \circ p_2
	\]
\end{definition}

\begin{remark}
	From a set-theoretic point of view, $\eqgraph{G} = (\graph{G}, \sim)$ is a graph with equivalence, where $\graph{G} = (E, V, s, t)$ in which holds that, for each pair of edges $e$, $e'$,
	\[
		\frac{s(e)\sim s(e')}{t(e)\sim t(e')}
	\]
	As we have considered in the section on graphs with equivalences, we can formalize the equivalence relation with a surjective function $q: V \to V/_\sim = C$, rewriting the inference rule above as an inclusion.
	Let $S = \{(e, e') \in E\times E \mid q(s(e)) = q(s(e'))\}$ and $T = \{(e, e') \in E \times E \mid q(t(e)) = q(t(e'))\}$. Then, $\eqgraph{G}$ is an e-graph if $S \subseteq T$. As we noted in \Cref{ex:kernel_pairs_in_Set}, $S$ is the kernel pair of $q \circ s$ and $T$ is the kernel pair of $q \circ t$, if endowed with canonical projections. Moreover, we notice that the pairing of such projections yields a subobject of the product $E \times E$ \Cref{prop:pairng_of_kernel_pairs_mono}. Hence, by \Cref{rem:fact_of_subobject_is_unique}, the \Cref{def:e_graph} express this fact using categorial constructions.
\end{remark}

\begin{definition}[Category of E-Graphs]\label{def:cat_of_eggs}
	$\EGG$ is the subcategory of $\EqGrph$ in which objects are e-graphs, and, given two e-graphs $\eqgraph{G, H}$, $\EGG(\eqgraph{G, H}) = \EqGrph(\eqgraph{G, H})$.
\end{definition}

Since $\EGG$ is a full subcategory of $\EqGrph$, there exists a fully faithful functor $I: \EGG \to \EqGrph$ (\Cref{ex:full_subc_inc_fully_faith}).

\begin{lemma}
	$\EGG$ has all limits and $I$ preserves them.
\end{lemma}

\begin{proof}
	Let $D: \cat{I} \to \EGG$ be a diagram, with $D(i) = (A_i, B_i, C_i, s_i, t_i, q_i)$, let $(U, u_1^i, u_2^i)$ be the kernel pair of $q_i\circ s_i$. Let now be $(A, B, C, s, t, q)$, togehter with projections $(\pi_E^i, \pi_V^i, \pi_C^i)_{i \in \cat I}$ the limit of $I \circ D$, let $(U, u_1, u_2)$ be the kernel pair of $q\circ s$ and let $(L, (l_i)_{i \in \cat I})$ be the limit of $V\circ I\circ D$.
	By construction (proof of \Cref{prop:eqgrph_complete}), there exists a monomorphism $m: Q \to L$ such that $\pi_C^i = l_i \circ m$. Notice that
	\begin{align*}
		q_i\circ s_i\circ \pi^i_E\circ u_1 	&= q_i\circ \pi^i_V\circ s\circ u_1\\
							&= \pi_C^i\circ q\circ s\circ u_1\\
							&=\pi_C^i\circ q\circ s\circ u_2\\
							&= q_i \circ s_i \circ \pi_E^i \circ u_2
	\end{align*}
	Then, for each $i$, there exists an arrow $a_i:U\to U_i$ making the following diagram to commute
	\[
		\begin{tikzcd}[row sep = 25, column sep = 25]
			U \ar[r, "{u_1}"] \ar[d, "{u_2}"swap] \ar[dr, dashed, "{a_i}"] & A \ar[drr, "{\pi_E^i}"] && \\
			A \ar[ddr, "{\pi_E^i}"swap] & U_i \ar[rr, "{u_1^i}"] \ar[dd, "{u_2^i}"swap]&& A_i \ar[d, "{s_i}"] \\
			& & & B_i \ar[d, "{q_i}"]\\
			& A_i \ar[r, "{s_i}"swap] & B_i \ar[r, "{q_i}"swap] & C_i
		\end{tikzcd}
	\]
	We have then
	\begin{align*}
		l_i\circ m \circ q \circ t \circ u_1	&= q_i \circ \pi_V^i \circ t \circ u_1 \\
							&= q_i \circ t_i \circ \pi_E^i \circ u_1 \\
							&= q_i \circ t_i \circ u_1^i \circ a_i \\
							&= q_i \circ t_i \circ u_2^i \circ a_i \\
							&= q_i \circ t_i \circ \pi_E^i \circ u_2 \\
							&= q_i \circ \pi_V^i \circ t \circ u_2 \\
							&= l_i \circ m \circ q \circ t \circ u_2
	\end{align*}
	By universal property of limits, we have that \- $m\circ q \circ t \circ u_1 = m \circ q \circ t \circ u_2$, and, since $m$ is mono, $q \circ t \circ u_1 = q \circ t \circ u_2$, hence the thesis.
\end{proof}

\begin{cor}\label{cor:I_creates_limits}
	$I$ creates limits.
\end{cor}

\begin{cor}\label{cor:reg_mono_EGG}
	$h: \eqgraph{G \to H}$ is a regular mono in $EGG$ if and only if it is a regular mono in $\EqGrph$.
\end{cor}

\begin{lemma}\label{lem:I_pres_pushouts_of_monos}
	$I$ preserves pushouts along regular monomorphisms.	
\end{lemma}

\begin{proof}
	Let $\eqgraph{G}_1 = (A_1, B_1, C_1, s_1, t_1, q_1)$, $\eqgraph{G}_2 = (A_2, B_2, C_2, s_2, t_2, q_2)$ and $\eqgraph{G}_3 = (A_3, B_3, C_3, s_3, t_3, q_3)$ be e-graphs,  $h: \eqgraph{G}_1 \to \eqgraph{G}_2$ and $m: \eqgraph{G}_1 \to \eqgraph{G}_3$ be two morphisms with $m$ mono. Let now $\eqgraph{P} = (A, B, C, s, t, q)$, together with $k: \eqgraph{G}_3 \to \eqgraph{P}$ and $n: \eqgraph{G}_2 \to \eqgraph{P}$, the pushout of $h$ and $m$.
	Let $(K_i, \pi_i^1, \pi_i^2)$ the kernel pair of $q_i\circ s_i$, for $i = 1, 2, 3$, and let $(U, u_1, u_2)$ be the kernel pair of $q \circ s$.
	Consider the following cube in $\Set$
	\[\begin{tikzcd}[row sep=25, column sep = 25]
	& {A_1} && {A_2} \\
	{A_3} && A \\
	& {C_1} && {C_2} \\
	{C_3} && C
	\arrow["{h_E}", from=1-2, to=1-4]
	\arrow["{m_E}"', from=1-2, to=2-1]
	\arrow["{q_1\circ s_1}"'{pos=0.7}, from=1-2, to=3-2]
	\arrow["{n_E}", from=1-4, to=2-3]
	\arrow["{q_2\circ s_2}", from=1-4, to=3-4]
	\arrow["{k_E}"{pos=0.7}, from=2-1, to=2-3, crossing over]
	\arrow["{q_3\circ s_3}"', from=2-1, to=4-1]
	\arrow["{h_C}"{pos=0.8}, from=3-2, to=3-4]
	\arrow["{m_C}"', from=3-2, to=4-1]
	\arrow["{n_C}", from=3-4, to=4-3]
	\arrow["{k_C}"', from=4-1, to=4-3]
	\arrow["{q\circ s}"'{pos=0.2}, from=2-3, to=4-3, crossing over]
	\end{tikzcd}\]

	Since $m$ is a regular mono then $m_C$ is mono by \Cref{cor:reg_mono_EGG} and \Cref{cor:regmono}. Then, by adhesivity of $\Set$, the bottom face of the cube above is a Van Kampen square and thus a pullback. Therefore by \Cref{lemma:pushouts_kernel_pairs}, the square below is a pushout.
	
	\[\begin{tikzcd}[row sep=27, column sep = 27]
		K_1 \ar[r, "{f_h}"] \ar[d, "{f_m}"swap] & K_2 \ar[d, "{f_n}"] \\
		K_3 \ar[r, "{f_k}"swap] & U
	\end{tikzcd}\]

	Hence, we end up with the following situation
	\[
		\begin{tikzcd}[row sep = 20, column sep = 20]
	& {K_1} && {K_2} \\
	{K_3} && U \\
	& {A_1} && {A_2} \\
	{A_3} && A \\
	& {C_1} && {C_2} \\
	{C_3} && C
	\arrow["{f_h}", from=1-2, to=1-4]
	\arrow["{f_m}"', from=1-2, to=2-1]
	\arrow["{\pi_1^1}"'{pos=0.7}, from=1-2, to=3-2]
	\arrow["{f_n}", from=1-4, to=2-3]
	\arrow["{\pi^1_2}", from=1-4, to=3-4]
	\arrow["{f_k}"{pos=0.8}, from=2-1, to=2-3, crossing over]
	\arrow["{\pi_3^1}"', from=2-1, to=4-1]
	\arrow["{u_1}"'{pos=0.3}, from=2-3, to=4-3, crossing over]
	\arrow["{h_E}"{pos=0.8}, from=3-2, to=3-4]
	\arrow["{m_E}"', from=3-2, to=4-1]
	\arrow["{q_1\circ s_1}"'{pos=0.7}, from=3-2, to=5-2]
	\arrow["{n_E}", from=3-4, to=4-3]
	\arrow["{q_2\circ s_2}", from=3-4, to=5-4]
	\arrow["{k_E}"{pos=0.7}, from=4-1, to=4-3, crossing over]
	\arrow["{q_3\circ s_3}"', from=4-1, to=6-1]
	\arrow["{h_C}"{pos=0.8}, from=5-2, to=5-4]
	\arrow["{q\circ s}"'{pos=0.2}, from=4-3, to=6-3, crossing over]
	\arrow["{m_C}"', from=5-2, to=6-1]
	\arrow["{n_C}", from=5-4, to=6-3]
	\arrow["{k_C}"', from=6-1, to=6-3]
	\end{tikzcd}
		\qquad
		\begin{tikzcd}[row sep = 20, column sep = 20]
	& {K_1} && {K_2} \\
	{K_3} && U \\
	& {A_1} && {A_2} \\
	{A_3} && A \\
	& {C_1} && {C_2} \\
	{C_3} && C
	\arrow["{f_h}", from=1-2, to=1-4]
	\arrow["{f_m}"', from=1-2, to=2-1]
	\arrow["{\pi_1^2}"'{pos=0.7}, from=1-2, to=3-2]
	\arrow["{f_n}", from=1-4, to=2-3]
	\arrow["{\pi^2_2}", from=1-4, to=3-4]
	\arrow["{f_k}"{pos=0.8}, from=2-1, to=2-3, crossing over]
	\arrow["{\pi_3^2}"', from=2-1, to=4-1]
	\arrow["{u_2}"'{pos=0.3}, from=2-3, to=4-3, crossing over]
	\arrow["{h_E}"{pos=0.8}, from=3-2, to=3-4]
	\arrow["{m_E}"', from=3-2, to=4-1]
	\arrow["{q_1\circ s_1}"'{pos=0.7}, from=3-2, to=5-2]
	\arrow["{n_E}", from=3-4, to=4-3]
	\arrow["{q_2\circ s_2}", from=3-4, to=5-4]
	\arrow["{k_E}"{pos=0.7}, from=4-1, to=4-3, crossing over]
	\arrow["{q_3\circ s_3}"', from=4-1, to=6-1]
	\arrow["{h_C}"{pos=0.8}, from=5-2, to=5-4]
	\arrow["{q\circ s}"'{pos=0.2}, from=4-3, to=6-3, crossing over]
	\arrow["{m_C}"', from=5-2, to=6-1]
	\arrow["{n_C}", from=5-4, to=6-3]
	\arrow["{k_C}"', from=6-1, to=6-3]
		\end{tikzcd}\]

	Computing, we have
	\[
		\begin{split}
			q \circ t \circ u_1 \circ f_n &= q \circ t \circ n_E \circ \pi_2^1 \\ &= n_C \circ q_2 \circ s_2 \circ \pi_2^1 \\&= n_C \circ q_2 \circ s_2 \circ \pi_2^2 \\&=q \circ t \circ u_2 \circ f_n 
			\end{split}
			\qquad
			\begin{split} q \circ t \circ u_1 \circ f_k &= q \circ t \circ k_E \circ \pi_3^1 \\ &= k_C \circ q_3 \circ s_3 \circ \pi_3^1 \\&= k_C \circ q_3 \circ s_3 \circ \pi_3^2 \\&=q \circ t \circ u_2 \circ f_k
			\end{split}
	\]
	By universal property of pushouts, there must exists a unque arrow $v: U \to C$ such that 
	\[\begin{split}v \circ f_n &= q \circ t \circ u_1 \circ f_n\\v\circ f_n  &= q \circ t \circ u_2 \circ f_n \end{split}\qquad\begin{split}v \circ f_k &= q \circ t \circ u_1 \circ f_k\\v \circ f_k &= q \circ t \circ u_2 \circ f_k\end{split}\]
		 hence, we deduce $q \circ t \circ u_1 = q \circ t \circ u_2$, and the thesis follows.
\end{proof}


\begin{cor}
	$I$ creates pushouts along regular monos.
\end{cor}

By direct application of \Cref{th:crit_for_adh}, we can conclude what follows.

\begin{cor}
	$\EGG$ is $\Reg(\EGG)-adhesive$.
\end{cor}

% APPUNTI
%Chiamiamo $(A_i, B_i, Q_i, s_i, t_i, q_i)$ le componenti del diagramma. Chiamiamo $(U_i, u_{1, i} , u_{2,i})$ il kernel pair di $q_i\circ s_i$.  Sia $(A, B, Q, s,t,q )$ il limite in $\EqGrph$ e $(U, u_1, u_2)$ il kernel pair di $q\circ s$. Vogliamo far vedere che $q\circ t \circ u_1= q\circ t\circ u_2$. Sia anche $(L, l_i)$ il limite del diagramma dato dai $Q_i$. In particolare abbiamo $m\colon Q\to L$ mono.

%Osserviamo che 
%\begin{align*}
%q_i\circ s_i\circ \pi^i_E\circ u_1 &= q_i\circ \pi^i_V\circ s\circ u_1\\&= \pi^i\circ q\circ s\circ u_1\\&=\pi^i\circ q\circ s\circ u_2=\dots (percorso al contrario)
%\end{align*}

%Dunque per ogni $i$ esiste una freccia $a_i\colon U\to U_{i}$ tale che il diagramma sotto commuta
%\[
%\xymatrix{U \ar@{.>}[dr]^{a_i}\ar[r]^{u_1} \ar[d]_{u_2}& A \ar[drr]^{\pi^i_E}\\A \ar[ddr]_{\pi^i_E}&U_i \ar[dd]_{u_{2,i}} \ar[rr]^{u_{1,i}}& & A_i \ar[d]^{s_i}\\
%& & &B_i \ar[d]^{q_i}\\&A_i \ar[r]_{s_i} & B_i \ar[r]_{s_i} & Q_i}
%\]


%\begin{align*}
%	l_i\circ m\circ q\circ t \circ u_1 &= q_i \circ \pi^{i}_V\circ t\circ u_1 \\&=q_i\circ t_i\circ \pi^i_E\circ u_1\\&=q_i\circ t_i\circ u_{1,i}\circ a_i\\&=q_i\circ t_i\circ u_{2,i}\circ a_i\\=\dots (uguaglianze al contrario)
%\end{align*}

%Dunque le frecce $m\circ q\circ t \circ u_1, m\circ q\circ t \circ u_2\colon U\rightrightarrows L$ sono uguali. Ma $m$ è mono per costruzione e allora $q\circ t \circ u_1=q\circ t \circ u_2$.

%FINE APPUNTI 

