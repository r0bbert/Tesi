\section{Graphs with Equivalences}\label{sect:eq_graphs}

A (directed graph) $\graph{G}$ is a structure consisting of a set of edge, a set of nodes and two functions, one assigning a \emph{source} node and one assigning a \emph{target} node to an edge. Formally, $\graph{G}$ is a quadruple $(V_\graph{G}, E_\graph{G}, s_\graph{G}, t_\graph{G})$, where $V_\graph{G}$ is the set of nodes, $E_\graph{G}$ is the set of edges, and $s_\graph{G}, t_\graph{G}: E_\graph{G} \rightarrow V_\graph{G}$ are the source and the target functions.

A \emph{graph homomorphism} $h: \graph{G \rightarrow H}$ is then a pair of functions $h = (h_V: V_\graph{G} \rightarrow V_\graph{H}, h_E: E_\graph{G} \rightarrow E_\graph{H})$ such that
    \[
        h_V \circ s_\graph{G} = s_\graph{H} \circ h_E
    \]
    and
    \[
        h_V \circ t_\graph{G} = t_\graph{H} \circ h_E
    \]
that is, a structure preserving map.

We can then generalize such notion to something more abstract, considering a graph to be nothing more than a presheaf from the category $(E \rightrightarrows V)$ to the category of sets.
Having two of such presheaves, a natural transformation from one to another encapsulates the behavior of a graph morphism due to naturality. We can now define the category of graphs.

\begin{definition}[Category of Graphs]\label{def:cat_of_graph}
    We denote as $\Graph$ the category $$[E \mathrel{\mathop{\rightrightarrows}^{s}_{t}} V , \Set]$$
\end{definition}

Since $\Graph$ is a category of presheaves, \Cref{lemma:limits_of_presheaves} guarantees the existence of limits and colimits, and gives us an easy way to compute them.

\begin{cor}\label{cor:graph_has_co_limits}
    $\Graph$ has all limits and colimits.
\end{cor}

A graph with equivalence is a 6-tuple $\eqgraph{G} =  (E, V, C, s, t, q)$, where $E$ and $V$ are, respectively, the edges and the vertices sets, and $C$ is the set of the equivalence classes among vertices, $s,t : E \to V$ are the source and target functions and $q: V \to C$ is the \emph{quotient} function, that is, the map from a vertex to its equivalence class. For this definition to make sense, $q$ needs to be surjective. A morphisms $h$ from a graph with equivalence $\eqgraph{G} =  (E, V, C, s, t, q)$ to another $\eqgraph{H} = (E', V', C', s', t',  q')$ is a triple $h = (h_E, h_V, h_C)$ of functions $h_V : V \to V'$, $h_E : E \to E'$ and $h_C : C \to C'$ such that
\[
    h_E \circ s = s' \circ h_V \qquad h_E \circ t = t' \circ h_V'\qquad h_C \circ q = q' \circ h_V
\]

\begin{remark}
     A graph with equivalence can be viewed as a graph endowed with an equivalence relation over its set of vertices, $(\graph G, \sim_\graph{G})$. An homomorphism between two graphs with equivalences $h :\eqgraph{G} = (\graph{G}, \sim_\graph{G})\rightarrow \eqgraph{H} = (\graph{H}, \sim_\graph{H})$ is a graph homomorphism $h = (h_V, h_E):\graph{G} \rightarrow \graph{H}$ such that if $v_1 \sim_\graph{G} v_2$ then $h_V(v_1) \sim_\graph{H} h_V(v_2)$. In $\Set$, it is possible to formalize an equivalence relation $\sim$ over $X$ as a surjective function sending each element $x$ on its equivalence class $[x]_{\sim}$, and this justify our formalization via surjective functions (i.e., epimorphisms).
\end{remark}

As we have done for graphs, we can think to a graph with equivalence as a presheaf, this time from a category $E \rightrightarrows V \rightarrow C$, where the image of $C$ along the presheaf is the set of the equivalence classes, requiring that the morphism $V\rightarrow C$ is an epimorphism (that is, a surjective function).

\begin{definition}[Category of Graphs with Equivalences]\label{def:eq_grphs}
    The category $\EqGrph$ is the subcategory of $$[E \mathrel{\mathop{\rightrightarrows}^{s}_{t}} V \xrightarrow{q} C, \Set]$$ such that, for each $\eqgraph{G} \in \Ob(\EqGrph)$, $\eqgraph{G}(q)$ is an epimorphism. 
\end{definition}

\begin{obs}\label{obs:eq_grph_morph_det_by_first_two_comp}
    Morphisms of graphs with equivalences are uniquely determined by the first two components. That is, if $h_1 = (h_E, h_V, \phi)$ and $h_2 = (h_E, h_V, \psi)$, then $\phi = \psi$. Indeed, consider two arrows $h_1, h_2 : \eqgraph{G \to H}$, where $\eqgraph{G} = (E_G, V_G, C_G, s_G, t_G, q_G)$ and $\eqgraph{H} = (E_H, V_H, C_H, s_H, t_H, q_H)$. Then, we have the following situation
    \[
        \begin{tikzcd}
            V_G \arrow[r, "{h_V}"] \arrow[d, "{q_G}"swap] & V_H \arrow[d, "{q_H}"] & V_G \arrow[l, "{h_V}"swap] \arrow[d, "{q_G}"] \\
            C_G \arrow[r, "{\phi}"swap] & C_H & \arrow[l, "{\psi}"] C_G
        \end{tikzcd}
     \]
     Then, we have:
     \begin{align*}
         \psi \circ q_G &= q_H \circ h_V \\
                        &= \phi \circ q_G
     \end{align*}
     From the fact that $q_G$ is epi, we can conclude $\phi = \psi$.
\end{obs}

A graph with equivalence is then a graph with an extra structure, the quotient map. Hence, it is possible to get the underlying graph by forgetting it. Such action is described by the \emph{forgetful functor} $U : \EqGrph \to \Graph$, that maps each graph with equivalence $\eqgraph{G} =  (E, V, C, s, t, q)$ onto $U(\eqgraph{G}) = (E, V, s, t)$, and each morphisms $h = (h_E, h_V, h_C)$ onto $U(h) = (h_E, h_V)$. $U$ is effectively a functor, since, on identities, $U((id_E, id_V, id_C)) = (id_E, id_V)$, and on compositions 
\[\begin{split}
	U(h \circ k) &= U((h_E \circ k_E, h_V \circ k_V, h_C \circ k_C))\\ &= (h_E \circ k_E, h_V \circ k_V) \\&= (h_E, h_V) \circ (k_E \circ k_V) \\&= U(h) \circ U(k)
\end{split}\]

\begin{prop}\label{prop:U_is_faithf}
    The forgetful functor $U: \EqGrph \to \Graph$ is faithful.
\end{prop}

\begin{proof}
    Let $\eqgraph{G} = (E_G, V_G, C_G, s_G, t_G, q_G)$ and $\eqgraph{H} = (E_H, V_H, C_H, s_H, t_H, q_H)$  be two graphs with equivalences, and let $h, k : \eqgraph{G \to H}$.
    If $U(h) = U(k)$ (i.e., the first two component of $h$ and $k$ are the same), from \Cref{obs:eq_grph_morph_det_by_first_two_comp}, we can conclude that $h = k$. Then, the restriction $U_{\eqgraph{G, H}} : \EqGrph(\eqgraph{G, H}) \to \Graph(U(\eqgraph{G}), U(\eqgraph{H}))$ is injective, therefore $U$ is faithful.
\end{proof}

Another functor that will be useful later is $V: \EqGrph \to \Set$, sending $(E_G, V_G, C_G, s_G, t_G, q_G)$ to $C_G$ and $h = (h_E, h_V, h_C)$ to $h_C$.

\begin{prop}\label{prop:eqgrph_complete}
    $\EqGrph$ has all limits, colimits and $U$ preserves limits and colimits.
\end{prop}

\begin{proof}
    Let $D : \cat I \to \EqGrph$ be a diagram. In the following, we will denote the graph with equivalence $D(i)$ as $(E_i, V_i, C_i, s_i, t_i, q_i)$.
    Let now be the graph $(A, B, s, t)$ the limit of $U \circ D$, with projections $(\pi_E^i, \pi_V^i):(A, B, s, t) \to (E_i, V_i, s_i, t_i)$. Notice now that $(B, (q_i\circ \pi_V^i)_{i \in \cat I})$ is a cone for $V \circ D$. To see this, let $\alpha : i \to j$ be an arrow of $\cat I$, $D(\alpha) = (h_E, h_V, h_C)$, $U \circ D (\alpha) = (h_E, h_V)$. From the definition of cone, we have that $U \circ D (\alpha) \circ (\pi_E^i, \pi_V^i) = (\pi_E^j, \pi_V^j)$, hence $h_V \circ \pi_V^i = \pi_V^j$. 
    Consider now the following diagram in $\Set$
    \[
        \begin{tikzcd}[row sep=25 pt, column sep = 25 pt]
            & B \arrow[dl, "{\pi_V^i}"swap] \arrow[dr, "{\pi_V^j}"] & \\
            V_i \arrow[rr, "{h_V}"] \arrow[d, "{q_i}" swap] & & V_j \arrow[d, "{q_j}"] \\
            C_i \arrow[rr, "{h_C}" swap] & & C_j 
        \end{tikzcd}
    \]
    So we have $q_j \circ h_V \circ \pi_V^i = q_j \circ \pi_V^j$, by definition of graph with equivalence, $h_C \circ q_i \circ \pi_V^i = q_j$, and, by definition of $V$, $V \circ D (\alpha) \circ q_i \circ \pi_V^i = q_j \circ \pi_V^j$.
    Suppose now $(L, (l_i)_{i \in \cat I})$ be a limit for $V \circ D$, so that we have an arrow $l: B \to L$. This arrow is not epi in general, so let $Q$ be its image, $q: Q \to B$ be the resulting epi and $m: Q \to L$ the corresponding mono, as the diagram below shows. By definition, the external rectangle commutes, so, for each $i$ object of $\cat I$, \Cref{prop:prop_epi_mono_Set} yields the dotted arrow $\pi_C^i$.
    \[
        \begin{tikzcd}
            B \arrow[r, "{\pi_V^i}"] \arrow[d, "q" swap] & B_ i \arrow[r, "{q_i}"] & Q_i \arrow[d, "{id_{Q_i}}"] \\
            Q \arrow[urr, dashed, "\pi_C^i"] \arrow[r, "m" swap] &L \arrow[r, "{l_i}"swap] & Q_i
        \end{tikzcd}
    \]
    We have to show that in this way we get a cone over the diagram $D$. Let $\alpha : i\to j$ be an arrow of $\cat{I}$, then we have:
    \begin{align*}
    U(D(\alpha)\circ (\pi_E^i, \pi_V^i, \pi_C^i))  &=  U(D(\alpha))\circ(\pi_E^i, \pi_V^i)\\
                                                   &=  (\pi_E^j, \pi_V^j)\\
                                                   &=  U(D(\alpha)\circ (\pi_E^j, \pi_V^j, \pi_C^j))
    \end{align*}
    And faithfulness of $U$ yields the thesis.

	To see that this cone is terminal, let $((E, F, G, a, b, c), (\tau_i)_{i \in \cat I})$, where $\tau_i = (\tau_E^i, \tau_V^i, \tau_C^i)$, be another cone. By construction, we have an arrow $(\tau_E, \tau_V):(E, F, a, b) \to (A, B, s, t)$ such that
    \[
        \begin{tikzcd}[row sep = 26, column sep = 26]
            & E \arrow[dl, dashed, "{\tau_E}" swap] \arrow[dr, "{\tau_E^i}"] & \\
            A \arrow[rr, "{\pi_E^i}" swap] & & A_i 
        \end{tikzcd}
        %
        \qquad
        %
        \begin{tikzcd}[row sep = 26, column sep = 26]
            & F \arrow[dl, dashed, "{\tau_V}" swap] \arrow[dr, "{\tau_V^i}"] & \\
            B \arrow[rr, "{\pi_V^i}" swap] & & B_i 
        \end{tikzcd}
    \]

    For the same reason as before, $(G, (\tau_C^i)_{i\in \cat I})$ is a cone over $V \circ D$, thus there exists an arrow $\tau : G \to L$ such that $l_i \circ \tau = \tau_C^i$. At this point, we get
    \begin{align*}
        l_i\circ \tau \circ c 
                        &= \tau_C^i\circ c              && \\
                        &=q_i\circ \tau_V^i             && \textit{$\tau_i$ is a morphism in $\EqGrph$}\\
                        &=q_i\circ \pi_V^i\circ \tau_V  && \textit{Diagram above} \\
                        &=l_i\circ l\circ \tau_V        && \textit{$(B, (q_i\circ \pi_V^i)_{i \in \cat I})$ cone} 
    \end{align*} 

	Therefore, the outer part of the rectangle below commutes, and by \Cref{prop:prop_epi_mono_Set} there exists a unique $\tau_C: G \to Q$
    \[
        \begin{tikzcd}
            F \arrow[r, "{\tau_V}"] \arrow[d, "c"swap] & B \arrow[r, "q"] & Q \arrow[d, "m"] \\
            G \arrow[urr, dashed, "{\tau_C}"] \arrow[rr, "{\tau}"swap] & & L
        \end{tikzcd}
    \]
     Faithfulness of $U$ and \Cref{obs:eq_grph_morph_det_by_first_two_comp} guarantees that $(\tau_E, \tau_V, \tau_C)$ is the unique arrow such that $(\pi_E^i, \pi_V^i, \pi_C^i) \circ (\tau_E, \tau_V, \tau_C) = (\tau_E^i, \tau_V^i, \tau_C^i)$.

     For colimits, let $(A, B, s, t)$ be the colimit of $U \circ D$, together with morphisms $(\kappa_i)_{i \in \cat I} = (\kappa_E^i, \kappa_V^i)_{i \in \cat I}$, and suppose $((c_i)_{i \in \cat I}, C)$ be the colimit of $V \circ D$. Then, we have the following situation
     \[
	     \begin{tikzcd}[row sep = 25, column sep=25]
			& B & \\
			B_i \ar[rr, "{h_V}"] \ar[ur, "{\kappa_V^i}"] \ar[d, "{q_i}"swap] & & B_j \ar[ul, "{\kappa_V^j}"swap] \ar[d, "q_j"] \\
			C_i \ar[rr, "{h_C}" swap] \ar[dr, "{c_i}"swap] & & C_j \ar[dl, "{c_j}"] \\
			& C &
		\end{tikzcd}
     \]
     Then, $((c_i \circ q_i)_{i \in \cat I}, C)$ is a cocone for all $B_i$, and, since $((\kappa_i)_{i \in \cat I}, B)$ is the colimit of all $B_i$ (\Cref{lemma:limits_of_presheaves}), there exists a unique arrow $q: B \to C$ such that, for each $i$, $q \circ \kappa_V^i = c_i \circ q_i$. Such $q$ is epi, by application of \Cref{lemma:nat_trans_reg_epi_canonical_arrow_reg_epi}, and thus $(A, B, C, s, t, q)$, together with arrows $(\kappa_E^i, \kappa_V^i, c_i)_{i \in \cat I}$ is the colimit of $D$.

\end{proof}

\begin{cor}\label{cor:mono_in_EqGrph}
    Let $\eqgraph{G} = (E_G, V_G, C_G, s_G, t_G, q_G)$ and $\eqgraph{H} = (E_H, V_H, C_H, s_H, t_H, q_H)$ be two graphs with equivalences. Then, an arrow $h = (h_E, h_V, h_C): \eqgraph{G} \to \eqgraph{H}$ in $\EqGrph$ is mono if and only if $h_E$ and $h_V$ are mono in $\Set$.
\end{cor}

\begin{proof}
    The ``if'' part is given by the fact that $U$ is faithful, and hence reflects monomorphisms. Since a morphism in a category of presheaves is mono if and only if it is injective on each component, we have that, if  $U(h)$ is mono, that is, $h_E$ and $h_V$ are injective in $\Set$, then $h$ is mono.
	For the ``only if'' part, suppose $f = (f_E, f_V, f_C)$, $g=(g_E, g_V, g_C)$, $f, g : \eqgraph{H \to K}$, where $\eqgraph{K} = (E_K, V_K, C_K, s_K, t_K, q_K)$, be such that $h \circ f = h \circ g$. Then, we have
    \begin{align*}
        h \circ f   &= (h_E \circ f_E, h_V \circ f_V, h_C \circ f_C) \\
                    &= (h_E \circ f_E, h_V \circ f_V, h_V \circ f_V \circ q_K) \\
                    &= (h_E \circ g_E, h_V \circ g_V, h_V \circ g_V \circ q_K)
    \end{align*}    
    Since $q_K$ is epi, we have, on the third component, that $h_V \circ f_V \circ q_K = h_V \circ g_V \circ q_K$ implies $f_C = g_C$, and hence $f = g$    
\end{proof}

\begin{cor}\label{cor:regmono}Let $\eqgraph{G} = (E_G, V_G, C_G, s_G, t_G, q_G)$ and $\eqgraph{H} = (E_H, V_H, C_H, s_H, t_H, q_H)$ be two graphs with equivalences, and let $ h = (h_E, h_V, h_C): \eqgraph{G}\to \eqgraph{H} $ be a morphism of $\EqGrph$, then the following are equivalent:
	\begin{enumerate}
		\item $h$ is a regular mono;
		\item $h_E$, $h_V$, $h_C$ are all monos;
%		\item $h_E$ and $h_V$ are mono and for every $v, v'\in V_H$, $q_H(h_V(v))=q_H(h_V(v'))$ if and only if $q_G(v)=q_G(v')$.
            \item $h_E$ and $h_V$ are mono and $(K, \pi_1, \pi_2)$ is the kernel pair of $q_H \circ h_V$ if and only if $(K, \pi_1, \pi_2)$ is the kernel pair of $q_G$.
	\end{enumerate}
\end{cor}
\begin{proof}
	$1\Rightarrow 2.$ If $h$ is mono, from \Cref{cor:mono_in_EqGrph} we have that $h_E$ and $h_V$ are monos. To derive $h_C$ mono, suppose $f, g :  \eqgraph{H \to K}$, where $\eqgraph{K} = (E_K, V_K, C_K, s_K, t_K, q_K)$to be the arrows equalized by $h$. Then we have
    \begin{align*}
        f_C \circ h_C \circ q_G  &=  f_C \circ q_H \circ h_V \\
                                            &=  q_K \circ f_V \circ h_V \\
                                            &=  q_K \circ g_V \circ h_V \\
                                            &=  g_C \circ h_C \circ q_G
    \end{align*}
    since $q_G$ is epi, we have that $f_C \circ h_C = g_C \circ h_C$, hence $h_C$ is an equalizer for $f_C$ and $g_C$, thus a monomorphism.
	
	\smallskip \noindent
	% $2\Rightarrow 3.$ The leftward side of the statement is satisfied by the definition of morphism of graphs with equivalences. For the remaining part, we have
 %    \begin{align*}
 %        (\eqgraph{H}(q) \circ h_V) (v)  &=  (\eqgraph{H}(q) \circ h_V)(v') \\
 %        (h_C \circ \eqgraph{G}(q))(v)   &=  (h_C \circ \eqgraph{G}(q))(v')  
 %    \end{align*}
 %        since $h_C$ is mono, we can conclude $\eqgraph{G}(q)(v)= \eqgraph{G}(q)(v')$.
        \smallskip \noindent
	$2\Rightarrow 3.$ We note that, by \Cref{cor:kermono}, $(K, \pi_1, \pi_2)$ is the kernel pair of $q_G$ if and only if it is the kernel pair also of $h_C \circ q_G$, since $h_C$ is mono by hypothesis. The thesis follows from $h_C \circ q_G = q_H \circ h_V$, and from the hypothesis of $h_E$ mono.
	
	\smallskip \noindent 
	$3\Rightarrow 1.$\todo{Esercizio} \color{green}{idea: force the comm. of the diagram on the last two components to obtain the two arrows that are equalized, and show that the condition in 3 is sufficient to conclude reg. mono} \color{black}
\end{proof}

\begin{remark}
    It is possible to restate the third point of the \Cref{cor:regmono}, by \Cref{ex:kernel_pairs_in_Set}, as 
    \begin{displayquote}
    \textit{$h_E$ and $h_V$ are mono and, for every $v, v'\in V_H$, $q_H(h_V(v))=q_H(h_V(v'))$ if and only if $q_G(v)=q_G(v')$}
    \end{displayquote}
    That is, a regular monomorphism in $\EqGrph$ is a morphism that reflects equivalences besides preserving them.
\end{remark}

Let us turn to another functor $\EqGrph\to \Graph$.

\begin{definition}
The \emph{quotient functor} $Q:\EqGrph\to \Graph $ is defined as the one sending $(E_G, V_G, C_G, s_G, t_G, q_G)$ to $(E_G, C_G, q_G\circ s_G, q_G\circ t_G)$ and an arrow $(h_E, h_V, h_C) \colon (E_G, V_G, C_G, s_G, t_G, q_G)\to (E_H, V_H, C_H, s_H, t_H, q_H)$ to $(h_E, h_C)$.
\end{definition}

\begin{remark}
    The action of the functor on a morphism of graphs with equivalences gives a morphism of graphs, in fact $q_H \circ s_H \circ h_E = q_H \circ h_V \circ s_G = h_C \circ q_G \circ s_G$. The same is valid for $t_H$ and $t_G$. 
\end{remark}

\begin{lemma}\label{lemma:quot_funct_left_adj}
    $Q$ is a left adjoint.
\end{lemma}

\begin{proof}
    Let $R((A, B, s, t))$ be $(A, B, B, s, t, id_B)$, so that $Q(R((A, B, s, t))) = (A, B, s, t)$. Now, suppose that $h = (h_E, h_V): Q((E, V, C, s', t', q)) \to (A, B, s, t)$  is an arrow in $\Graph$, and consider the triple $(h_E, h_V, h_V \circ q)$. Since $h$ is a morphism of $\Graph$, 
    \[h_V\circ q\circ s'= s\circ h_E \qquad  h_V\circ q\circ t' = t\circ h_E\]
    Then we have the following squares:
    \[
        \begin{tikzcd}
            E \arrow[r, "{h_E}"] \arrow[d, "{s_G}"swap] & A \arrow[d, "s"] \\
            V \arrow[r, "{h_V \circ q}"swap] & B
        \end{tikzcd}
        \qquad
        \begin{tikzcd}
            E \arrow[r, "{h_E}"] \arrow[d, "{t_G}"swap] & A \arrow[d, "t"] \\
            V \arrow[r, "{h_V \circ q}"swap] & B
        \end{tikzcd}
        \qquad
        \begin{tikzcd}
            V \arrow[r, "{h_V\circ q}"] \arrow[d, "q" swap] & B \arrow[d, "{id_B}"] \\
            C \arrow[r, "{h_V}"swap] & B
        \end{tikzcd}
    \]

    We have therefore found a morphism $(E, V, C, s', t', q) \to R((A, B, s, t))$ whose image through $Q$ fits in the diagram below.
    \[
        \begin{tikzcd}
            (A, B, s, t) \arrow[r, "{id_A, id_B}"] & (A, B, s, t)\\
            (E, C, q\circ s', q\circ t') \arrow[u, "{Q((h_E, h_V \circ q, h_V))}"] \arrow[ur, "{(h_E, h_V)}" swap] 
        \end{tikzcd}
    \]
    Such arrow is unique. Suppose $f = (f_E, f_V, f_C)$ to be another arrow wit such property. Then, it must be $(id_A, id_B) \circ Q(f) = (f_E, f_C) = (h_E, h_C)$. Finally, $f_C = f_V \circ q = h_V \circ q$. 
\end{proof}

\begin{prop}\label{prop:quot_creat_colims}
    $Q$ creates colimits.
\end{prop}


\begin{proof}
    Preserve from \Cref{th:adjoints_preserves_lim}. Remain to see Reflect.
	Let $D: \cat I \to \EqGrph$ be a diagram, and let $((c_i)_{i\in \cat I}, \eqgraph{C})$ be the colimit of $Q \circ D$, where $\eqgraph{C} = (A, C, q\circ s, q\circ t)$, and $D(i)$ is $(A_i, B_i, C_i, s_i, t_i, q_i)$.
	Let now $T: \EqGrph \to \Set$ be the functor mapping each graph with equivalence onto its second component, $T((X, Y, Z, x, y, z)) = Y$, and each morphims onto its second component.
	Let then $((b_i)_{i\in \cat I}, B)$ be the colimit of $T \circ D$.
	Consider the following situation.
	\[\begin{tikzcd}[row sep = 24, column sep = 24]
		& B & \\
		B_i \ar[ur, "{b_i}"] \ar[rr, "h_V"] \ar[d, "{q_i}"] & & B_j \ar[ul, "{b_j}"swap] \ar[d, "{q_j}" swap] \\
		C_i \ar[dr, "{c_C^i}" swap] \ar[rr, "{h_C}"swap] & & C_j \ar[dl, "{c_C^j}"] \\
		& C &
	\end{tikzcd}\]

	Now, since $((c^i_C \circ q_i)_{i \in \cat I}, C)$ is a cocone for $T \circ D$, there exists a unique $q: B \to C$, which is epi by \Cref{lemma:nat_trans_reg_epi_canonical_arrow_reg_epi}.
	Consider now the functor $W: \EqGrph \to \Set$ mapping each $(X, Y, Z, x, y, z)$ onto $X$, and each morphism on its first component. By \Cref{prop:eqgrph_complete} and \Cref{lemma:limits_of_presheaves}, we have that $((c_E^i)_{i \in \cat I}, A)$ is the colimit of $W \circ D$.
	Notice that $((b_i \circ s_i)_{i \in \cat{I}}, B)$ and $((b_i \circ t_i)_{i \in \cat I}, B)$ are cocones for $W \circ D$, so let $s$ and $t$ be, respectively, the mediating arrow for the first one and the mediating arrow for the second one. It remains now to show that $(A, B, C, s, t, q)$, together with $(c_E^i, b_i, c_C^i)_{i \in \cat I}$, is a colimit for $D$, but this follows by the proof of \Cref{prop:eqgrph_complete}.
\end{proof}

\begin{example}
	$Q$ does not preserve limits. Indeed, let $\eqgraph{G}_1 = (E_1, A, A, s_1, t_1, id_A)$, $\eqgraph{G}_2 = (E_2, B, B, s_2, t_2, id_B)$ and $\eqgraph{G}_3 = (E_3, A + B, \terminal, s_3, t_3, !_{A + B})$, and let $h = (h_E, \iota_A, !_A): \eqgraph{G}_1 \to \eqgraph{G}_3$, $k = (k_E, \iota_B, !_B): \eqgraph{G}_2 \to \eqgraph{G}_3$, where $(\iota_A, \iota_B, A + B)$ is the coproduct of $A$ and $B$, $\terminal$ is the intial object (in $\Set$, the singleton set as shown in \Cref{ex:set_init_term}), and $!_X$ the unique arrow $X \to \terminal$.
	The following two diagrams show the pullback of $h$ and $k$ and the pullback of $Q(h)$ and $Q(k)$, on the second component (the vertices of the graphs)
	\[\begin{tikzcd}[row sep = 20, column sep = 20]
		\initial \ar[r, "{p_1}"] \ar[d, "{p_2}" swap] & A \ar[d, "{\iota_A}"] \\
		B \ar[r, "{\iota_B}"swap] & A+B
	\end{tikzcd}\
	\qquad
	\begin{tikzcd}[row sep = 20, column sep = 20]
		{A \times B} \ar[r, "{\pi_A}"] \ar[d, "{\pi_B}"swap] & A \ar[d, "{!_A}"] \\
		B \ar[r, "{!_B}"swap] & \terminal
	\end{tikzcd}\]

	But the arrow $\initial \to A\times B$ is not epi in general (this is easy to see taking $\Set$ as example), hence such pullback is not preserved by $Q$.
\end{example}


\subsection{Adhesivity of $\EqGrph$}

\begin{lemma}\label{lemma:stab}
	In $\EqGrph$, pushouts along regular monos are stable.
\end{lemma}

\begin{proof}
	Let $\graph{G}_i = (A_i, B_i, C_i, s_i, t_i, q_i)$, $\graph{G}'=(A_i', B_i', C_i', s_i', t_i', q_i')$, for $i = 1, 2, 3, 4$, and, in the diagram above, suppose all the vertical faces are pullbacks, the bottom face is a pushout and $h$ is regular mono.
	\[\begin{tikzcd}[row sep=23, column sep =23]
	& {\mathcal{G}_1'} && {\mathcal{G}_2'} \\
	{\mathcal{G}_3'} && {\mathcal{G}_4'} \\
	& {\mathcal{G}_1} && {\mathcal{G}_2} \\
	{\mathcal{G}_3} && {\mathcal{G}_4}
	\arrow["{h'}", from=1-2, to=1-4]
	\arrow["{k'}"', from=1-2, to=2-1]
	\arrow["b"'{pos=0.7}, from=1-2, to=3-2]
	\arrow["{t'}", from=1-4, to=2-3]
	\arrow["c", from=1-4, to=3-4]
	\arrow["{p'}"{pos=0.8}, from=2-1, to=2-3, crossing over]
	\arrow["a"', from=2-1, to=4-1]
	\arrow["h"{pos=0.7}, from=3-2, to=3-4]
	\arrow["d"'{pos=0.7}, from=2-3, to=4-3, crossing over]
	\arrow["k"', from=3-2, to=4-1]
	\arrow["t", from=3-4, to=4-3]
	\arrow["p"', from=4-1, to=4-3]
	\end{tikzcd}\]
	By \Cref{prop:eqgrph_complete} and \Cref{cor:mono_in_EqGrph}, the following two cubes have $\mathcal{M}$-pushouts as bottom faces and pullbacks as vertical faces, thus their top faces are $\mathcal{M}$-pushouts.
	\[\begin{tikzcd}[row sep=20, column sep = 20]
	& {A_1'} && {A_2'} \\
	{A_3'} && {A_4'} \\
	& {A_1} && {A_2} \\
	{A_3} && {A_4}
	\arrow["{h'_V}", from=1-2, to=1-4]
	\arrow["{k'_V}"', from=1-2, to=2-1]
	\arrow["{b_V}"'{pos=0.7}, from=1-2, to=3-2]
	\arrow["{t'_V}", from=1-4, to=2-3]
	\arrow["{c_V}", from=1-4, to=3-4]
	\arrow["{p'_V}"{pos=0.8}, from=2-1, to=2-3, crossing over]
	\arrow["{a_V}"', from=2-1, to=4-1]
	\arrow["{h_V}"{pos=0.7}, from=3-2, to=3-4]
	\arrow["{d_V}"'{pos=0.7}, from=2-3, to=4-3, crossing over]
	\arrow["{k_V}"', from=3-2, to=4-1]
	\arrow["{t_V}", from=3-4, to=4-3]
	\arrow["{p_V}"', from=4-1, to=4-3]
	\end{tikzcd}\qquad
	\begin{tikzcd}[row sep=20, column sep = 20]
	& {B_1'} && {B_2'} \\
	{B_3'} && {B_4'} \\
	& {B_1} && {B_2} \\
	{B_3} && {B_4}
	\arrow["{h'_E}", from=1-2, to=1-4]
	\arrow["{k'_E}"', from=1-2, to=2-1]
	\arrow["{b_E}"'{pos=0.7}, from=1-2, to=3-2]
	\arrow["{t'_E}", from=1-4, to=2-3]
	\arrow["{c_E}"', from=1-4, to=3-4]
	\arrow["{p'_E}"{pos=0.8}, from=2-1, to=2-3, crossing over]
	\arrow["{a_E}"', from=2-1, to=4-1]
	\arrow["{h_E}"{pos=0.7}, from=3-2, to=3-4]
	\arrow["{d_E}"'{pos=0.7}, from=2-3, to=4-3, crossing over]
	\arrow["{k_E}"', from=3-2, to=4-1]
	\arrow["{t_E}", from=3-4, to=4-3]
	\arrow["{p_E}"', from=4-1, to=4-3]
	\end{tikzcd}
	\]
	Now, using \Cref{cor:cube}, we can consider a third cube, which, by \Cref{prop:quot_creat_colims}, has a $\mathcal{M}$-pushout as bottom face and pullbacks as vertical faces, so that its top face is a $\mathcal M$-pushout too.
	\[\begin{tikzcd}[row sep=20, column sep = 20]
	& {T} && {U} \\
	{Y} && {C_4'} \\
	& {C_1} && {C_2} \\
	{C_3} && {C_4}
	\arrow["{x_1}", from=1-2, to=1-4]
	\arrow["{w}"', from=1-2, to=2-1]
	\arrow["{x_2}"'{pos=0.7}, from=1-2, to=3-2]
	\arrow["{u_1}", from=1-4, to=2-3]
	\arrow["{u_2}", from=1-4, to=3-4]
	\arrow["{y_1}"{pos=0.8}, from=2-1, to=2-3, crossing over]
	\arrow["{y_2}"', from=2-1, to=4-1]
	\arrow["{h_C}"{pos=0.7}, from=3-2, to=3-4]
	\arrow["{d_C}"'{pos=0.7}, from=2-3, to=4-3, crossing over]
	\arrow["{k_C}"', from=3-2, to=4-1]
	\arrow["{t_C}", from=3-4, to=4-3]
	\arrow["{p_C}"', from=4-1, to=4-3]
	\end{tikzcd}\]

	Moreover, by the proof of \Cref{prop:eqgrph_complete}, we know that there are monos $m_2:C_2'\to U$ and $m_3:C_3'\to Y$ fitting in the diagrams
	\[\begin{tikzcd}[row sep=18, column sep = 18]
	{B'_3} && {B'_4} \\
	\\
	{C'_3} && Y && {C'_4} \\
	\\
	{B_3} && {C_3} && {C_4}
	\arrow["{p'_V}", from=1-1, to=1-3]
	\arrow["{q'_3}", from=1-1, to=3-1]
	\arrow["{q'_4}", from=1-3, to=3-5, bend left = 20]
	\arrow["{m_3}", from=3-1, to=3-3]
	\arrow["{y_1}", from=3-3, to=3-5]
	\arrow["{y_2}"', from=3-3, to=5-3]
	\arrow["{d_C}", from=3-5, to=5-5]
	\arrow["{q_3}"', from=5-1, to=5-3]
	\arrow["{p_C}"', from=5-3, to=5-5]
        \arrow["{c_V}"', from=1-1, to=5-1, bend right = 20, shift right=1]
	\end{tikzcd}\qquad\begin{tikzcd}[row sep=18, column sep = 18]
	{B'_2} && {B'_4} \\
	\\
	{C'_2} && U && {C'_4} \\
	\\
	{B_2} && {C_2} && {C_4}
	\arrow["{t'_V}", from=1-1, to=1-3]
	\arrow["{q'_2}", from=1-1, to=3-1]
	\arrow["{q'_4}", from=1-3, to=3-5, bend left = 20]
	\arrow["{m_2}", from=3-1, to=3-3]
	\arrow["{u_1}", from=3-3, to=3-5]
	\arrow["{u_2}"', from=3-3, to=5-3]
	\arrow["{d_C}", from=3-5, to=5-5]
	\arrow["{q_2}"', from=5-1, to=5-3]
	\arrow["{t_C}"', from=5-3, to=5-5]
        \arrow["{b_V}"', from=1-1, to=5-1, bend right = 20, shift right = 1]
	\end{tikzcd}\]

	For $C'_1$, we can make a similar argument, let $S$ be the pullback of $m_2$ along $x_1$, using \Cref{lemma:pullback_lemma} and, again, the proof of \Cref{prop:eqgrph_complete} we know that $q'_1$ arise as the factorization of the arrow $B'_1\to S$ induced by $q'_2\circ h'_2$ and $a_2$ so that we have a diagram.
	\[\begin{tikzcd}[row sep = 16, column sep =16]
	{B'_1} && {B'_2} \\
	\\
	{C'_1} && S && {C'_2} \\
	\\
	{B_1} && T && U \\
	\\
	&& {C_1} && {C_2}
	\arrow["{h'_V}", from=1-1, to=1-3]
	\arrow["{q'_1}", from=1-1, to=3-1]
	\arrow["{q'_2}", from=1-3, to=3-5, bend left = 20]
	\arrow["{m_1}", from=3-1, to=3-3]
	\arrow["{s_1}", from=3-3, to=3-5]
	\arrow["{s_2}"', from=3-3, to=5-3]
	\arrow["{m_2}", from=3-5, to=5-5]
	\arrow["{q_1}"', from=5-1, to=7-3, bend right = 20, shift right = 1]
	\arrow["{x_1}", from=5-3, to=5-5]
	\arrow["{x_2}"', from=5-3, to=7-3]
	\arrow["{u_2}", from=5-5, to=7-5]
	\arrow["{h_C}"', from=7-3, to=7-5]
        \arrow["{a_V}"', from=1-1, to=5-1, bend right = 20, shift right = 1]
	\end{tikzcd}\]
	
	Moreover, we have that
	\[
		\begin{split}
			s_1 \circ m_1 \circ q_1 &= q'_2 \circ h'_V \\ &= h'_C \circ q'_1
		\end{split}\qquad\begin{split}
			q'_4 \circ t'_V &= t'_C \circ q'_2\\ &= u_1\circ m_2 \circ q'_2
		\end{split}
	\]\[
		\begin{split}
			y_1 \circ m_3 \circ q'_3 &= p'_V \circ q'_4\\ &= p'_C \circ q'_3
		\end{split}%\qquad\begin{split}TODO:\ w \circ s_2 \circ m_1 = m_3 \circ k_C'\end{split}
	\]
	Hence, $s_1\circ m_1 = h'_C$, $w \circ s_2 \circ m_1 = m_3 \circ k_C'$, $t'_C = u_1 \circ m_2$ and $p'_C = y_1 \circ m_3$.
	Let now $z: \graph{G}'_2 \to \graph H = (E, V, Q, s, t, q)$, $w: \graph G_3^1 \to \graph H$ be two morphisms such that $z \circ h' = w \circ k'$, and let $\phi:B'_4 \to V$ be the arrow induced by $z_V$ and $w_V$. We want to construct the arrow $v: C'_4 \to Q$ in the diagram below.
	\[\begin{tikzcd}[row sep = 20, column sep = 20]
	& {B_1'} && {B_2'} \\
	{B_3'} && {B_4'} && V \\
	& T && U \\
	Y && {C_4'} && Q \\
	& {C_1} && {C_2} \\
	{C_3} && {C_4}
	\arrow["{h_V'}", from=1-2, to=1-4]
	\arrow["{k_V'}"', from=1-2, to=2-1]
	\arrow["{s_2\circ m_1 \circ q_1'}"{description, pos=0.7}, from=1-2, to=3-2]
	\arrow["{t_V'}", from=1-4, to=2-3]
	\arrow["{z_V}", from=1-4, to=2-5]
	\arrow["{m_2\circ q'_2}"{pos=0.7}, from=1-4, to=3-4]
	\arrow["{m_3 \circ q'_3}"', from=2-1, to=4-1]
	\arrow["q", from=2-5, to=4-5]
	\arrow["{x_1}"{pos=0.8}, from=3-2, to=3-4]
	\arrow["w"', from=3-2, to=4-1]
	\arrow["{x_2}"'{pos=0.7}, from=3-2, to=5-2]
	\arrow["{u_1}", from=3-4, to=4-3]
	\arrow["{u_2}"{pos=0.7}, from=3-4, to=5-4]
	\arrow["{y_2}"', from=4-1, to=6-1]
	\arrow["{h_C}"{pos=0.8}, from=5-2, to=5-4]
	\arrow["{k_C}", from=5-2, to=6-1]
	\arrow["{t_C}", from=5-4, to=6-3]
	\arrow["{p_C}"', from=6-1, to=6-3]
	\arrow["{p_V'}"{pos=0.8}, from=2-1, to=2-3, crossing over]
	\arrow["\phi"{pos=0.8}, from=2-3, to=2-5, crossing over]
	\arrow["{q'_4}"'{pos=0.7}, from=2-3, to=4-3, crossing over]
	\arrow["{y_1}"{pos=0.7}, from=4-1, to=4-3, crossing over]
	\arrow["{d_C}"'{pos=0.7}, from=4-3, to=6-3, crossing over]
	\arrow["v"{pos=0.8}, dashed, from=4-3, to=4-5, crossing over]
	\end{tikzcd}\]
	
	By \Cref{prop:reg_epi_coeq}, $d_C$ is the coequalizer of its kernel pair. On the other hand, by \Cref{lemma:pushouts_kernel_pairs}, we know that the top face of the cube below is a pushout.
	\[\begin{tikzcd}[row sep=20, column sep = 20]
	& {K_{s_2\circ m_1 \circ q_1'}} && {K_{m_2\circ q_2'}} \\
	{K_{m_3 \circ q_3'}} && {K_{q_4'}} \\
	& {B'_1} && {B'_2} \\
	{B'_3} && {B'_4}
	\arrow["{k_{h'_V}}", from=1-2, to=1-4]
	\arrow["{k_{k_V'}}"', from=1-2, to=2-1]
	\arrow["{\pi_{s_2\circ m_1 \circ q_1'}^1}"'{pos=0.7}, from=1-2, to=3-2]
	\arrow["{k_{t_V'}}", from=1-4, to=2-3]
	\arrow["{\pi_{m_2\circ q_2'}^1}", from=1-4, to=3-4]
	\arrow["{k_{p_V'}}"{pos=0.8}, from=2-1, to=2-3, crossing over]
	\arrow["{\pi_{m_3 \circ q_3'}^1}"', from=2-1, to=4-1]
	\arrow["{h_V'}"{pos=0.7}, from=3-2, to=3-4]
	\arrow["{\pi_{q_4'}^1}"'{pos=0.7}, from=2-3, to=4-3, crossing over]
	\arrow["{k_V'}"', from=3-2, to=4-1]
	\arrow["{t_V'}", from=3-4, to=4-3]
	\arrow["{p_V'}"', from=4-1, to=4-3]
	\end{tikzcd}\]
	And, since $m_3$ and $m_2$ are monos,
	\[\begin{split}
		q'_3 \circ \pi_{m_3 \circ q'_3}^1 = q'_3 \circ \pi_{m_3\circ q_3'}^2
	\end{split}\qquad\begin{split}
		q_2' \circ \pi_{m_2 \circ q_2'}^1 = q'_2 \circ \pi_{m_2 \circ q_2'}^2
	\end{split}\]
	
	Computing, we obtain
	\[\begin{split}
		q \circ \phi \circ \pi_{q_4'}^1 \circ k_{p_V'} &= q \circ \phi \circ p_V' \circ \pi_{m_3 \circ q_3'}^1 \\ &= q \circ w_V \circ \pi_{m_3 \circ q_3'}^1 \\ &= w_C \circ q_3' \circ \pi_{m_3 \circ q_3'}^1 \\ &= w_C \circ q_3' \circ \pi_{m_3 \circ q_3}^2 \\ &= q \circ w_V \circ \pi_{m_3 \circ q_3'} \\ &= q \circ \phi \circ p_V' \circ \pi_{m_3 \circ q_3'}^2 \\ &= q \circ \phi \circ \pi_{q'_4}^2 \circ k_{p'_V}
	\end{split}\qquad\begin{split}
		q \circ \phi \circ \pi_{q'_4}^1 \circ k_{t_V'} &= q \circ \phi \circ t_V' \circ \pi_{m_2 \circ q_2'}^1 \\ &= q \circ z_V \circ \pi_{m_2 \circ q_2'}^1 \\ &= z_C \circ q_2' \circ \pi_{m_2 \circ q_2'}^1 \\ &= z_C \circ q_2' \circ \pi_{m_2 \circ q_2}^2 \\ &= q \circ z_V \circ \pi_{m_2 \circ q_2'} \\ &= q \circ \phi \circ t_V' \circ \pi_{m_2 \circ q_2'}^2 \\ &= q \circ \phi \circ \pi_{q'_4}^2 \circ k_{t'_V}
	\end{split}\]

	Since the previous cube has a pushout as top face, by universal property, we have
	\[
		q \circ \phi \circ \pi_{q'_4}^1 = q \circ \phi \circ \pi_{q'_4}^2
	\]
	hence, $v$ is the mediating arrow.
	\[
		v \circ q'_4 \circ \pi_{q'_4}^1 = v \circ q'_4 \circ \pi_{q'_4}^2
	\]
\end{proof}


\begin{lemma}\label{lemma:van_kampen}
	In $\EqGrph$, pushouts along regular monos are $\Reg(\EqGrph)$-Van Kampen.
\end{lemma}

\begin{proof}
	In lieu of \Cref{lemma:stab}, it is enough to proof that, given a cube as the one below, with pullbacks as back faces, pushouts as bottom and top faces and such that $h$ is a regular mono, the front faces are pullbacks too, where $\graph{G}_i = (A_i, B_i, C_i, s_i, t_i, q_i)$, $\graph{G}'=(A_i', B_i', C_i', s_i', t_i', q_i')$, for $i = 1, 2, 3, 4$.
        \[\begin{tikzcd}[row sep=23, column sep =23]
        & {\mathcal{G}_1'} && {\mathcal{G}_2'} \\
        {\mathcal{G}_3'} && {\mathcal{G}_4'} \\
        & {\mathcal{G}_1} && {\mathcal{G}_2} \\
        {\mathcal{G}_3} && {\mathcal{G}_4}
        \arrow["{h'}", from=1-2, to=1-4]
        \arrow["{k'}"', from=1-2, to=2-1]
        \arrow["b"'{pos=0.7}, from=1-2, to=3-2]
        \arrow["{t'}", from=1-4, to=2-3]
        \arrow["c", from=1-4, to=3-4]
        \arrow["{p'}"{pos=0.8}, from=2-1, to=2-3, crossing over]
        \arrow["a"', from=2-1, to=4-1]
        \arrow["h"{pos=0.7}, from=3-2, to=3-4]
        \arrow["d"'{pos=0.7}, from=2-3, to=4-3, crossing over]
        \arrow["k"', from=3-2, to=4-1]
        \arrow["t", from=3-4, to=4-3]
        \arrow["p"', from=4-1, to=4-3]
        \end{tikzcd}\]
        By \Cref{prop:eqgrph_complete} and \Cref{cor:mono_in_EqGrph}, the following two cubes have $\mathcal{M}$-pushouts as bottom faces and pullbacks as back faces, thus their front faces are pullbacks too.
        \[\begin{tikzcd}[row sep=20, column sep = 20]
        & {A_1'} && {A_2'} \\
        {A_3'} && {A_4'} \\
        & {A_1} && {A_2} \\
        {A_3} && {A_4}
        \arrow["{h'_V}", from=1-2, to=1-4]
        \arrow["{k'_V}"', from=1-2, to=2-1]
        \arrow["{b_V}"'{pos=0.7}, from=1-2, to=3-2]
        \arrow["{t'_V}", from=1-4, to=2-3]
        \arrow["{c_V}", from=1-4, to=3-4]
        \arrow["{p'_V}"{pos=0.8}, from=2-1, to=2-3, crossing over]
        \arrow["{a_V}"', from=2-1, to=4-1]
        \arrow["{h_V}"{pos=0.7}, from=3-2, to=3-4]
        \arrow["{d_V}"'{pos=0.7}, from=2-3, to=4-3, crossing over]
        \arrow["{k_V}"', from=3-2, to=4-1]
        \arrow["{t_V}", from=3-4, to=4-3]
        \arrow["{p_V}"', from=4-1, to=4-3]
        \end{tikzcd}\qquad
        \begin{tikzcd}[row sep=20, column sep = 20]
        & {B_1'} && {B_2'} \\
        {B_3'} && {B_4'} \\
        & {B_1} && {B_2} \\
        {B_3} && {B_4}
        \arrow["{h'_E}", from=1-2, to=1-4]
        \arrow["{k'_E}"', from=1-2, to=2-1]
        \arrow["{b_E}"'{pos=0.7}, from=1-2, to=3-2]
        \arrow["{t'_E}", from=1-4, to=2-3]
        \arrow["{c_E}"', from=1-4, to=3-4]
        \arrow["{p'_E}"{pos=0.8}, from=2-1, to=2-3, crossing over]
        \arrow["{a_E}"', from=2-1, to=4-1]
        \arrow["{h_E}"{pos=0.7}, from=3-2, to=3-4]
        \arrow["{d_E}"'{pos=0.7}, from=2-3, to=4-3, crossing over]
        \arrow["{k_E}"', from=3-2, to=4-1]
        \arrow["{t_E}", from=3-4, to=4-3]
        \arrow["{p_E}"', from=4-1, to=4-3]
        \end{tikzcd}
        \]


	On the other hand we can consider the diagrams below, in which the inner squares are pullbacks. Since the outer diagrams commute, by definition of porphism of $\EqGrph$, then we have the existence of $m_2\colon C'_2\to U$, $m_3\colon C'_3\to Y $, $a_3\colon B'_3\to Y$ and $a_2\colon B'_2\to Y$.

	\[\begin{tikzcd}[row sep = 20, column sep = 20]
	{C'_3} &&& {C_2'} \\
	& Y & {C'_4} & {} & U & {C_4'} \\
	& {C_3} & {C_4} && {C_2} & {C_4}
	\arrow["{m_3}", dashed, from=1-1, to=2-2]
	\arrow["{p'_C}", from=1-1, to=2-3, bend left = 20]
	\arrow["{c_C}"', from=1-1, to=3-2, bend right = 20]
	\arrow["{m_2}", dashed, from=1-4, to=2-5]
	\arrow["{t'_C}", from=1-4, to=2-6, bend left = 20]
	\arrow["{b_C}"', from=1-4, to=3-5, bend right = 20]
	\arrow["{y_1}", from=2-2, to=2-3]
	\arrow["{y_2}"', from=2-2, to=3-2]
	\arrow["{d_C}", from=2-3, to=3-3]
	\arrow["{u_1}", from=2-5, to=2-6]
	\arrow["{u_2}"', from=2-5, to=3-5]
	\arrow["{d_C}", from=2-6, to=3-6]
	\arrow["{p_C}"', from=3-2, to=3-3]
	\arrow["{t_C}"', from=3-5, to=3-6]
	\end{tikzcd}\]
	\[\begin{tikzcd}[row sep = 20, column sep = 20]
	{B_3'} & {B'_4} && {B'_2} & {B'_4} \\
	{C'_3} & Y & {C'_4} & {C_2'} & U & {C_4'} \\
	& {C_3} & {C_4} && {C_2} & {C_4}
	\arrow["{p'_V}", from=1-1, to=1-2]
	\arrow["{q'_3}"', from=1-1, to=2-1]
	\arrow["f", dashed, from=1-1, to=2-2]
	\arrow["{q'_4}", from=1-2, to=2-3]
	\arrow["{t'_V}", from=1-4, to=1-5]
	\arrow["{q'_2}"', from=1-4, to=2-4]
	\arrow["g", dashed, from=1-4, to=2-5]
	\arrow["{q'_4}", from=1-5, to=2-6]
	\arrow["{c_C}"', from=2-1, to=3-2]
	\arrow["{y_1}", from=2-2, to=2-3]
	\arrow["{y_2}"', from=2-2, to=3-2]
	\arrow["{d_C}", from=2-3, to=3-3]
	\arrow["{b_C}"', from=2-4, to=3-5]
	\arrow["{u_1}", from=2-5, to=2-6]
	\arrow["{u_2}"', from=2-5, to=3-5]
	\arrow["{d_C}", from=2-6, to=3-6]
	\arrow["{p_C}"', from=3-2, to=3-3]
	\arrow["{t_C}"', from=3-5, to=3-6]
\end{tikzcd}\]

	Now, notice that $m_3$ and $m_2$ are monos because $c_C$ and $b_C$ are monos (\Cref{cor:regmono}). By the proof of \Cref{prop:eqgrph_complete}, to conclude it is enough to show that
	\[\begin{split}
		m_3 \circ q'_3 = f
	\end{split}
	\qquad
	\begin{split}
		m_2\circ q'_2 = g
	\end{split}\]
	Indeed, if the previous equations hold, then $C'_3$ and $C'_2$ are epi-mono factorizations of $f$ and $g$, and the thesis follows from \Cref{cor:unique} and the proof of \Cref{prop:eqgrph_complete}.
	
	Now, if we compute, we obtain:
	\[\begin{split}
		y_1 \circ f &= q'_4 \circ p'_V \\ &= p'_C \circ q'_3 \\ &= y_1 \circ m_3 \circ q'_3
	\end{split}
	\qquad
	\begin{split}
		u_1 \circ g &= q'_4 \circ t'_V \\ &= t'_C \circ q'_3 \\ &= u_1 \circ m_2 \circ q'_2
	\end{split}\]
	\[\begin{split}
		y_2 \circ f &= d_C \circ q'_3 \\ &= y_2 \circ m_3 \circ q'_3
	\end{split}
	\qquad
	\begin{split}
		u_2 \circ g &= d_V \circ q'_2 \\ &= u_2 \circ m_2 \circ q'_2
	\end{split}\]
	Concluding the proof.
\end{proof}

From \Cref{prop:eqgrph_complete} and \Cref{lemma:stab,lemma:van_kampen}, by \Cref{th:crit_for_adh}, we can deduce at once the following.

\begin{cor}\label{cor:eqgrph_reg_adh}
	$\EqGrph$ is $\Reg(\EqGrph)$-adhesive.
\end{cor}
