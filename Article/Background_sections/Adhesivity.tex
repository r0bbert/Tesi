\section{Adhesivity}\label{sect:adh}

The next section is about adhesivity.
An adhesive category is intuitively a category in which pushouts of (some) monomorphisms exist and they behave more or less as they do among sets. 

\begin{definition}(Van Kampen property)
    Let $\mathcal A$ be a subclass of $\Hom(\cat C)$, and consider the diagram below:
    % \[
    % \begin{tikzcd}
    %     A \arrow[r, "f"] \arrow[d, "m" swap] & B \arrow[d, "n"] \\
    %     C \arrow[r, "g" swap]                & D
    % \end{tikzcd}
    % \]
    \[
    \begin{tikzcd}[row sep = scriptsize]
        A' \arrow[ddd, "m'"swap] \arrow[rrr, "f'"] \arrow[dr, "a"swap] & & & B \arrow[ddd, "n'"] \arrow[dl, "b"] \\
        & A \arrow[r, "f"] \arrow[d, "m" swap] & B \arrow[d, "n"]   & \\
        & C \arrow[r, "g" swap]                & D                  & \\
        C' \arrow[rrr, "g'"swap] \arrow[ur, "c"] & & & D' \arrow[ul, "d" swap] 
    \end{tikzcd}
    \]
    \[\begin{tikzcd}[row sep=25, column sep=25]
	& {A'} && {B'} \\
	{C'} && {D'} \\
	& A && B \\
	C && D
	\arrow["{f'}", from=1-2, to=1-4]
	\arrow["{m'}"', from=1-2, to=2-1]
	\arrow["a"'{pos=0.7}, from=1-2, to=3-2]
	\arrow["{n'}"', from=1-4, to=2-3]
	\arrow["b", from=1-4, to=3-4]
	\arrow["{g'}"{pos=0.7}, from=2-1, to=2-3, crossing over]
	\arrow["c"', from=2-1, to=4-1]
	\arrow["f"'{pos=0.3}, from=3-2, to=3-4]
	\arrow["d"{pos=0.3}, from=2-3, to=4-3, crossing over]
	\arrow["m", from=3-2, to=4-1]
	\arrow["n", from=3-4, to=4-3]
	\arrow["g"', from=4-1, to=4-3]
    \end{tikzcd}\]
    we say that the inner square is an \emph{$\mathcal A$-Van Kampen} square if:
    \begin{itemize}
        \item it is a pushout;
        \item $a, b, c, d \in \mathcal{A}$;
        \item whenever the top and the left squares are pullbacks then the outer square is a pushout if and only of the right and the bottom squares are pullbacks.
    \end{itemize}
\end{definition}

We are now ready to give the notion of $\mathcal M$-adhesivity. % CITE: https://web3.arxiv.org/pdf/2407.06181

\begin{definition}[$\mathcal{M}$-adhesivity]\label{def:adh}
    Let $\cat C$ be a category and $\mathcal M \subseteq \Mono(\cat C)$ containing all isomorphisms, closed under composition and stable under pullbacks and pushouts (\Cref{def:stab_under_pb_po}).
    Then $\cat C$ is \emph{$\mathcal M$-adhesive} if
    \begin{enumerate}
        \item every cospan $C \xrightarrow[]{g} D \xleftarrow[]{m} B$ with $m \in \mathcal M$ can be completed to a pullback (such pullbacks are called $\mathcal M$-pullbacks);
        \item every span $C \xleftarrow{m} A \xrightarrow{f} B$ with $ m \in \mathcal M$ can be completed to a pushout (such pushouts are called $\mathcal M$-pushouts);
        \item pushouts along $\mathcal M$-arrows are $\mathcal M$-Van Kampen squares.
    \end{enumerate}
    We also say that $\cat C$ is \emph{adhesive} when it is $\Mono(\cat C)$-adhesive, and \emph{quasiadhesive} when it is $\Reg(\cat C)$-adhesive. \todo{NO, questo è sbagliato: bisogna introdurre la nozione di adesività forte (cioè VK vale per tutti i cubi)}
\end{definition}

\begin{obs}
    $\Set$ is adhesive.
    % It is easy to check the first two points of the \Cref{def:adh} while for the Van Kampen property of pushouts along monos [...]
\end{obs}

Here it follows an interesting property of adhesive categories~\cite{lack2011embeddingtheoremadhesivecategories}.

\begin{prop}\label{prop:monos_are_preserved_by_pullbacks_in_adh_cats}
    In any adhesive category, the pushout of a monomorphism along any morphism is a monomorphism, and the resulting square is also a pullback. \todo{almeno la seconda parte devi provarla o citarla per le M-adesive}
\end{prop}

\begin{proof}
	Let the following square be a pushout, with $m$ mono.
	\[\begin{tikzcd}[row sep = 26, column sep = 26]
		A \ar[r, "f"] \ar[d, "m"swap] & B \ar[d, "n"] \\
		C \ar[r, "g"swap] & D
	\end{tikzcd}\]
	Consider now the following cube.
	\[\begin{tikzcd}[row sep = 20, column sep = 20]
	& A && B \\
	A && B \\
	& A && B \\
	C && D \\
	& {}
	\arrow["f", from=1-2, to=1-4]
	\arrow["{id_A}"', from=1-2, to=2-1]
	\arrow["{id_A}"'{pos=0.7}, from=1-2, to=3-2]
	\arrow["{id_B}"', from=1-4, to=2-3]
	\arrow["{id_B}", from=1-4, to=3-4]
	\arrow["m"', from=2-1, to=4-1]
	\arrow["f"{pos=0.7}, from=3-2, to=3-4]
	\arrow["m"', from=3-2, to=4-1]
	\arrow["n", from=3-4, to=4-3]
	\arrow["g"', from=4-1, to=4-3]
	\arrow["f"{pos=0.7}, from=2-1, to=2-3, crossing over]
	\arrow["n"'{pos=0.7}, from=2-3, to=4-3, crossing over]
	\end{tikzcd}\]

	We have then that the top face is a pullback, and by \Cref{prop:kermono}, the left face is a pullback. Since the bottom face is a pushout by hypothesis, we have that the front face and the right one are pullbacks (by adhesivity). Hence, we can conclude that the starting square (which is the front and the bottom face of the cube) is a pushout, and by \Cref{prop:kermono}, since the left face is a pullback, $n$ is mono.
\end{proof}

Verifying $\mathcal M$-adhesivity using the definition above may turn out to be very complex, so we can make use of the following result~\cite{castelnovo2022newcriterionmathcalmmathcalnadhesivity}. 

\begin{theorem}\label{th:crit_for_adh}
    Let $\cat C$ be a category, $\mathcal M \subseteq \Mono(\cat C)$ containing all isomorphisms, closed under composition and stable under pullbacks and pushouts. Let now $F: \cat{C \rightarrow D}$ be a functor with $\cat D$ $\mathcal{N}$-adhesive for some $\mathcal{N} \subseteq \Mono(\cat D)$.
    If $F$ is such that $F(\mathcal{M}) \subseteq \mathcal N$ and creates pullbacks and $\mathcal{M}$-pushout, then $\cat C$ is $\mathcal M$-adhesive.
\end{theorem}

The idea behind this theorem is to simplify calculations to show that a certain category is adhesive for some subclass of monomorphisms, considering a functor from the category of which we want to prove adhesivity to a category we know it is adhesive, requiring that such functor has some properties.

\begin{proof}
    In order to prove $\mathcal M$-adhesivity of $\cat C$, we have to verify the condition in \Cref{def:adh}.
    \begin{itemize}
        \item Let $C \xrightarrow[]{g} D \xleftarrow[]{m} B$ with $m \in \mathcal M$ be a cospan in $\cat C$. Applying $F$, we obtain $F(C) \xrightarrow[]{F(g)} F(D) \xleftarrow[]{F(m)} B$, with $F(m) \in \mathcal{N}$ by hypothesis. Then, there exists a pullback $(P_F, p_{F(B)}, p_{(F(D))})$ in $\cat D$, which is an $\mathcal N$-pullback (\Cref{def:pullback_pushout}). Since $F$ creates pullbacks, hence lifts them (\Cref{obs:funct_creat_lim_then_lift}), there exist a pullback $(P, p_B, p_D)$ in $\cat C$.
        \item Let $C \xleftarrow{m} A \xrightarrow{f} B$ with $ m \in \mathcal M$ be a cospan in $\cat C$. Analogously to the previous point, applying the functor $F$ we obtain $F(C) \xleftarrow{F(m)} F(A) \xrightarrow{F(f)} F(B)$ with $ F(m) \in \mathcal N$, and there exists a $\mathcal N$-pushout $(q_{F(C)}, q_{F(B)}, F(Q))$ in $\cat D$. Since $F$ reflects pushouts, $(q_C, q_B, Q)$ is a $\mathcal{M}$-pushout in $\cat C$.
        \item the Van Kampen property of $\mathcal M$-pullbacks follows from the closure under pullbacks and pushouts of $\mathcal M$ and from the fact that $F$ reflects pullbacks.
    \end{itemize}
    
\end{proof}

\begin{cor}\label{cor:adhesivity_functor_categories}
    Let $\cat A$ be a $\mathcal M$-adhesive category for some $\mathcal M \subseteq \Hom(\cat A)$. Then, if $\cat C$ is a category whith all the pullbacks, the functor category $[\cat C, \cat A]$ is $\mathcal M^{\cat C}$-adhesive, where
    \[
        \mathcal{M}^\cat C = \{\eta \in \Hom([\cat C, \cat A]) \mid \eta_C \in \mathcal M \textit{ for each object $C$ of $\cat C$ }\}
    \]
\end{cor}

Since $\Set$ is adhesive, we can conclude what follows.

\begin{cor}\label{cor:presh_adhesive}
    Every category of presheaves is adhesive.
\end{cor}

\color{blue}\todo{Ti inserisco una versione un po' più generale del lemma sotto}


\begin{lemma}\label{lemma:pushouts_kernel_pairs}
	Let $\cat C$ be an \todo{Qua serve adesività forte} $\mathcal{M}$-adhesive category with all pullbacks, and suppose that in the cube below the top face is an $\mathcal{M}$-pushout.
	\[\begin{tikzcd}[row sep=25, column sep=25]
		& {A'} && {B'} \\
		{C'} && {D'} \\
		& A && B \\
		C && D
		\arrow["{f'}", from=1-2, to=1-4]
		\arrow["{m'}"', from=1-2, to=2-1]
		\arrow["a"'{pos=0.7}, from=1-2, to=3-2]
		\arrow["{n'}"', from=1-4, to=2-3]
		\arrow["b", from=1-4, to=3-4]
		\arrow["{g'}"{pos=0.7}, from=2-1, to=2-3, crossing over]
		\arrow["c"', from=2-1, to=4-1]
		\arrow["f"'{pos=0.3}, from=3-2, to=3-4]
		\arrow["d"{pos=0.3}, from=2-3, to=4-3, crossing over]
		\arrow["m", from=3-2, to=4-1]
		\arrow["n", from=3-4, to=4-3]
		\arrow["g"', from=4-1, to=4-3]
	\end{tikzcd}\]
	
	Then, the square below is a pushout.
	\[
	\begin{tikzcd}%[row sep=25, column sep=25]
		{K_a} \ar[r, "{k_{f'}}"] \ar[d, "{k_{m'}}"'] & {K_b} \ar[d, "{k_{n'}}"] \\
		{K_c} \ar[r, "{k_{g'}}"'] & {K_d}
	\end{tikzcd}
	\]
\end{lemma}

\begin{proof}
	Since $f'$, being the pullback of $f$, is in $\mathcal{M}$ then by  \Cref{prop:monos_are_preserved_by_pullbacks_in_adh_cats} the bottom face of the cube is a pullback. Thus \Cref{lemma:kern_pairs_pres_pullbacks} entails that in the following cube the vertical faces are pullbacks \todo{Nota che le frecce verticali NON sono mono in generale: qua serve adesività forte}.
	\[\begin{tikzcd}[row sep = 20, column sep = 20]
		& {K_a} && {K_b} \\
		{K_c} && {K_d} \\
		& {A'} && {B'} \\
		{C'} && {D'}
		\arrow["{k_{f'}}", from=1-2, to=1-4]
		\arrow["{k_{m'}}"', from=1-2, to=2-1]
		\arrow["{\pi_a^1}"{pos=0.7}, from=1-2, to=3-2]
		\arrow["{k_{n'}}", from=1-4, to=2-3]
		\arrow["{\pi_b^1}", from=1-4, to=3-4]
		\arrow["{k_{g'}}"{pos=0.7}, from=2-1, to=2-3, crossing over]
		\arrow["{\pi_c^1}"', from=2-1, to=4-1]
		\arrow["{f'}"'{pos=0.3}, from=3-2, to=3-4]
		\arrow["{\pi_d^1}"{pos=0.7}, from=2-3, to=4-3, crossing over]
		\arrow["{m'}", from=3-2, to=4-1]
		\arrow["{n'}", from=3-4, to=4-3]
		\arrow["{g'}"', from=4-1, to=4-3]
	\end{tikzcd}\]
	
Now the thesis follows from strong $\mathcal{M}$-adhesivity.
\end{proof}


\color{black}

\begin{lemma}\label{lemma:pushouts_kernel_pairs}
    Let $\cat C$ be an \todo{Qua serve adesività forte} $\mathcal{M}$-adhesive category with all pullbacks, and suppose that the cube below is given, in which every face is a pullback and the bottom one is a $\mathcal{M}$-pushout.
        \[\begin{tikzcd}[row sep=25, column sep=25]
	& {A'} && {B'} \\
	{C'} && {D'} \\
	& A && B \\
	C && D
	\arrow["{f'}", from=1-2, to=1-4]
	\arrow["{m'}"', from=1-2, to=2-1]
	\arrow["a"'{pos=0.7}, from=1-2, to=3-2]
	\arrow["{n'}"', from=1-4, to=2-3]
	\arrow["b", from=1-4, to=3-4]
	\arrow["{g'}"{pos=0.7}, from=2-1, to=2-3, crossing over]
	\arrow["c"', from=2-1, to=4-1]
	\arrow["f"'{pos=0.3}, from=3-2, to=3-4]
	\arrow["d"{pos=0.3}, from=2-3, to=4-3, crossing over]
	\arrow["m", from=3-2, to=4-1]
	\arrow["n", from=3-4, to=4-3]
	\arrow["g"', from=4-1, to=4-3]
    \end{tikzcd}\]
    
    Then, the square below is a pushout.
    \[
        \begin{tikzcd}%[row sep=25, column sep=25]
            {K_a} \ar[r, "{k_{f'}}"] \ar[d, "{k_{m'}}"'] & {K_b} \ar[d, "{k_{n'}}"] \\
            {K_c} \ar[r, "{k_{g'}}"'] & {K_d}
        \end{tikzcd}
    \]
\end{lemma}

\begin{proof}
    By \Cref{lemma:kern_pairs_pres_pullbacks}, in the following cube the vertical faces are pullbacks.
    \[\begin{tikzcd}[row sep = 20, column sep = 20]
        & {K_a} && {K_b} \\
        {K_c} && {K_d} \\
        & {A'} && {B'} \\
        {C'} && {D'}
        \arrow["{k_{f'}}", from=1-2, to=1-4]
        \arrow["{k_{m'}}"', from=1-2, to=2-1]
        \arrow["{\pi_a^1}"{pos=0.7}, from=1-2, to=3-2]
        \arrow["{k_{n'}}", from=1-4, to=2-3]
        \arrow["{\pi_b^1}", from=1-4, to=3-4]
        \arrow["{k_{g'}}"{pos=0.7}, from=2-1, to=2-3, crossing over]
        \arrow["{\pi_c^1}"', from=2-1, to=4-1]
        \arrow["{f'}"'{pos=0.3}, from=3-2, to=3-4]
        \arrow["{\pi_d^1}"{pos=0.7}, from=2-3, to=4-3, crossing over]
        \arrow["{m'}", from=3-2, to=4-1]
        \arrow["{n'}", from=3-4, to=4-3]
        \arrow["{g'}"', from=4-1, to=4-3]
    \end{tikzcd}\]

	$f'$ is in $\mathcal{M}$ as it is the pullback of $\mathcal{M}$, thus the bottom face of the cube is a Van Kampen pushout and the thesis follows.

\end{proof}


