
\section{Basic Notions}\label{sect:basic_nots}

%The very fist concept to get familiar with is what is a category. We can see a category as a general construction that can provide any level of abstraction needed. In fact, the definition is not about what the category represents, but instead a more general scheme that \emph{may} represent something. The definition is about what a category is composed by, and how the objects in it interact between themselves.

This section is all about basic definitions and examples, to get familiar with the formalism of categories.

\subsection{Categories}\label{ssect:cats}

\begin{definition}[Category]\label{def:category}
    A \emph{category} $\cat{C}$ comprises:
    \begin{enumerate}
        \item A collection of \emph{objects} $\Ob(\cat{C})$;
        \item A collection of \emph{arrows} (or \emph{morphisms}) $\Hom(\cat{C})$, often called \emph{homset}.
    \end{enumerate}
    Two operators, $dom$ and $cod$, that map every morphism \-$f \in \Hom(\cat{C})$ to two objects, respectively, its \emph{domain} and its \emph{codomain}. In case $dom\ f = A$ and $cod\ f = B$, we will write $f: A \rightarrow B$. The collection of morphisms from an object $A$ to an object $B$ is denoted as $\cat{C}(A, B)$.
    An operator $\circ$ of \emph{composition} maps every couple of morphisms $f$, $g$ with $cod\ f = dom \ g$ (in this case $f$ and $g$ are said to be composable) to a morphism $g \circ f : dom\ f \rightarrow cod \ g$. The composition operator is associative, i.e., for each composable arrows $f$, $g$ and $h$, it holds that
    $$
        h \circ (g \circ f) = (h\circ g) \circ f
    $$
    For each object $A$, an \emph{identity} morphism $id_A : A \rightarrow A$ (or, when it is clear from the context, just denoted $A$)  such that, for each $f: A \rightarrow B$:
    \[
        id_B \circ f = f = f \circ id_A 
    \]
\end{definition}

The most important thing here is not the structure of the objects, but instead how this structure is preserved by the morphisms.

\begin{example}\label{ex:0_cat}
    A trivial example of category is the one with no objects, and hence no morphisms. Such category is denoted by $\textbf 0$ and is called \emph{empty category}.
\end{example}

\begin{example}\label{ex:1_cat}
    The category with just one object and just one arrow, the identity arrow on that object, is denoted $\textbf 1$. In particular, the only object of this category is $\bullet$, and the only arrow is $id_{\bullet}$.
\end{example}

Given an arrow $f A \rightarrow B$ in a category $\cat C$, we say that $f$ \emph{factors through} $g: C \rightarrow B$ if there exists an arrow $h: A \rightarrow C$ such that $f = h \circ g$.

\begin{definition}\label{def:dual_cat}[Dual Category]
    Given a category $\cat C$, there exist a category $\cat C^{op}$ such that:
    \begin{itemize}
        \item $Ob(\cat C^{op}) = Ob(\cat C)$;
        \item if $f: A \rightarrow B$ is a morphism in $\cat C$, then $f: B\rightarrow A$ is a morphism in $\cat C ^ {op}$.
    \end{itemize}
    Hence, given $f : A \rightarrow B$ and $g: B \rightarrow C$ arrows in $\cat C$, as $g \circ f: A \rightarrow C$ is an arrow in $\cat C$, then $f \circ g: C \rightarrow A$ is an arrow in $\cat C ^{op}$.
    Such category is called \emph{dual category} or \emph{opposite category}.
\end{definition}

Duality is a concept that we will encounter most of the time. Given a property $P$ valid for a category $\cat C$, we will refer to the same property in the opposite category $\cat C^{op}$ as the \emph{dual} of $P$, without explicitly constructing $\cat C^{op}$. There exist some properties that coincide exactly with their dual, and such properties are said to be \emph{self dual} properties.

%  \emph{dual category} of a category $\cat{C}$, denoted $\cat{C}^{op}$, in which the objects are the same of $\cat{C}$, and the arrows are the opposite of the arrows in $\cat{C}$, i.e., if $f: A \rightarrow B$ is an arrow of $\cat{C}$, then $f: B \rightarrow A$ is an arrow of $\cat{C}^{op}$.
% Each definition in category theory has a dual form. In general, if a statement $S$ is true in a category $\cat{C}$, then the opposite of the statement, $S^{op}$, obtained switching the words "domain" and "codomain" and replacing each composite $g \circ f$ into $f \circ g$, is still true in the category $\cat{C}^{op}$. Moreover, since every category is the opposite of its opposite, if a statement $S$ is true for every category, then $S^{op}$ is also true for every category ~\cite[pp8-9]{pierce91}. {\color{red} la discussione sul duale temo confonda}

To represent morphisms of a category $\cat{C}$ it is possible to use \emph{diagrams}, as the one below, in which the vertices are objects of $\cat{C}$, and the edges are morphisms of $\cat{C}$.
    \[
    \begin{tikzcd}
        X \arrow[d, "g'" swap] \arrow[r, "f'"] & Z \arrow[d, "g"] \\
        W \arrow[r, "f" swap] & Y        
    \end{tikzcd}
    \]
The diagram is said to commute whenever  $f \circ g' = g \circ f'$. Unique morphisms are represented with dashed arrows.
A more rigorous definition of what a diagram is will be given later (\Cref{def:diagram}).

\begin{example}
    It is easy to see that taking sets as objects and (total) functions as arrows, we obtain a category. In fact, given two functions $f: A \rightarrow B$ and $g: B \rightarrow C$, it is possible to compose them obtaining an arrow $g \circ f : A \rightarrow C$, and the composition is associative. For each set $A$ there exists an identity function $id_A: A \rightarrow A$ such that $id_A(a) = a$ for each $a\in A$.
    This category is denoted as $\Set$.
\end{example}

\begin{remark}\label{rem:small_cats}
    It is important to note that the \Cref{def:category} above does not specify what kind of collections
    %for a category $\cat{C}$, 
    $\Ob(\cat{C})$ and $\Hom(\cat{C})$ are.
    Taking $\Set$ as example, the collection $\Ob(\Set)$ cannot be a set itself, due to Russel's paradox. It would be more appropriate referring to a category $\cat{C}$ which $\Ob(\cat{C})$ and $\Hom(\cat{C})$ are both sets as a \emph{small category}, but it is assumed in this work, except where it is made explicit, for a category to be small.
    Another clarification must to be done, still considering $\Set$. Given two sets $A$ and $B$, it is possible to construct the set $B^A$ of all functions from $A$ to $B$. This is isomorphic to $\Set(A, B)$, for each pair of sets $A$ and $B$.
    A category $\cat C$ where, for each pair of objects $A$ and $B$, $\cat C (A, B)$ is a set is said to be \emph{locally small}.
\end{remark}

\subsection{Mono, Epi and Iso}\label{ssect:Mono_epi_iso}

Between the morphisms of a category, it is possible to distinguish some that have certain properties, as functions between sets can be surjective, injective or bijective.

\begin{definition}[Monomorphism]\label{def:mono}
    An arrow $f:B\rightarrow C$ in a category $\cat{C}$ is a \emph{monomorphism} if, for any pair of arrows of $\cat{C}$ $g:A \rightarrow B$, $h: A \rightarrow B$, the equality $f \circ g = f \circ h$ implies $g = h$. The class of monomorphisms of $\cat C$ is denoted $\Mono(\cat C)$.
\end{definition}

\begin{remark}\label{rem:fact_of_subobject_is_unique}
    For a morphism, from an algebraic point of view, being mono means being \emph{left cancelable}.
    %This fact can led us to define a particular kind of class of morphisms, which will reveal useful further.
    Let $A$ be an object in a category $\cat C$. Given two monomorphism $m: X \to A$ and $n: Y \to A$, then if  $h: X \to Y$ is a morphism such that $m = n \circ h$, then is the unique one: suppose $k$ is another morphism such that $m = n \circ k$. We can conclude $h = k$ observing that $n \circ h = n \circ k$ implies $h = k$ when $n$ is mono, which is by hypothesis.    
\end{remark}


\begin{definition}[Subobject]\label{def:subobj}
    Starting from this consideration, we can define a preorder on monomorphisms, placing $m \leq n$ if $m = n \circ h$ for some $h$. Such preorder induces an equivalence relation $\equiv$ on monomorphisms with codomain $A$, where $m \equiv n$ whenever $m \leq n$ and $n \leq m$, and the corresponding equivalence class is called \emph{subobject} of $A$.
\end{definition}


% \begin{definition}[Subobjects]
%     Let $C$ be an object in a category $\cat C$. Then, if $m: A \rightarrow C$ is mono\todo{sinceramente non trovo molto sensata la scelta di chiamare sottoggetto la coppia $(A, m)$. Lo standard, per quanto ne so, è usare sottoggetto per una classe di equivalenza di mono o, al massimo, per i mono}, $(A, m)$ is said to be a \emph{subobject} of $C$. Factorization of morphisms induces a preorder on subobjects of an object. $(A, m) \leq (B, n)$ whenever there exists a morphism $f : A\rightarrow B$ such that $m = n \circ f$.
%     \[
%         \begin{tikzcd}[row sep=9]
%             A \arrow[dr, "m"] \arrow[dd, dashed, "f"swap] & \\
%             & B \\
%             C \arrow[ur, "n"swap] &
%         \end{tikzcd}
%     \]
%     Two subobject $(A, m)$ and $(B, n)$ can are said to be \emph{equivalent subobjects}, written $(A, m) \approx (B, n)$ if $(A, m)\leq (B, n)$ and $(B, n) \leq (A, m)$. 
% \end{definition}

% An useful fact about subobjects is how factorization behaves. In particular $(A, m)$ and $(B, n)$ are subobjects of $C$. Then, if $(A, m)\leq (B, n)$, we have $m = n \circ h$ for some $h$. Suppose $k$ is another morphism such that $m = n \circ k$. We can conclude $h = k$ observing that $n \circ h = n \circ k$ implies $h = k$ when $n$ is mono, which is by hypothesis. This is to say what follows.

% \begin{prop}\label{prop:fact_of_subobjects_is_unique}
%     Let $(A, m)$ and $(B, n)$ be subobjects of $C$ in a category $\cat C$, with $(A, m)\leq (B, n)$. Then, the factorization of $m$ through $n$ is unique.
% \end{prop}



\begin{definition}[Epimorphism]\label{def:epi}
    An arrow $f: A\rightarrow B$ in a category $\cat{C}$ is an \emph{epimorphism} if, for any pair of arrows of $\cat{C}$ $g : B \rightarrow C$, $h: B \rightarrow C$, the equality $g \circ f = h \circ f$ implies $g = h$.
\end{definition}

\begin{definition}[Isomorphism]\label{def:iso}
    An arrow $f:A \rightarrow B$ is an \emph{isomorphism} if there is an arrow $f^{-1}:B \rightarrow A$, called the \emph{inverse} of $f$, such that $f^{-1}\circ f = id_{A}$ and $f \circ f^{-1} = id_{B}$. Two objects are said to be \emph{isomorphic} if there is an isomorphism between them.
\end{definition}

\begin{example}
    In $\Set$, monomorphisms are injective functions, epimorphisms are surjective functions and isomorphisms are bijections.
\end{example}

\begin{remark}
    Mono and epi are dual concepts. This fact is easily shown by considering how a monomorphism $m$ in a category $\cat C$ behaves in the dual category $\cat C^{op}$.
    In $\cat C$ we have that $m \circ f = m \circ g$ implies $f = g$. In $\cat C^{op}$, the we can state that $f \circ m = g \circ m$ implies $f = g$, obtaining the definition of epi. 
\end{remark}

\begin{prop}\label{prop:epi_mono_prop}
    The following statements hold for every pair of composable arrows $f$ and $g$ for any category $\cat C$:
    \begin{enumerate}
        \item if both $f$ and $g$ are mono, then $g \circ f$ is mono;
        \item if $g \circ f$ is mono, then $f$ is mono;
        \item if both $f$ and $g$ are epi, then $g \circ f$ is epi;
        \item if $g \circ f$ is epi, then $g$ is epi.
    \end{enumerate}
\end{prop}

\begin{obs}\label{obs:equiv_subobj_implies_isom_of_domains}
    Consider two monomorphisms $m: X \to A$ and $n: Y \to A$, and suppose $m \equiv n$ (in the sense of \Cref{def:subobj}). This corresponds to having a isomorphism between $X$ and $Y$. in fact, since $m \leq n$, there exists s unique $h: Y \to X$ such that $m = n \circ h$, and, since $n \leq m$, there exists a unique $k: X \to Y$ such that $n = m \circ k$. But then
    \[
        \begin{split}
            m &= n \circ h \\
              &= m \circ h \circ k
        \end{split}
        \qquad
        \begin{split}
            n &= m \circ k \\
              &= n \circ h \circ k
        \end{split}
    \]
    Since $m$ is mono, we obtain $id_X = h \circ k$, and, since $n$ is mono, $id_Y = k\circ h$, thus, an isomorphism. 
\end{obs}

The next proposition will be useful later.

\begin{prop}\label{prop:prop_epi_mono_Set}
    In $\Set$, for every commutative square as the one below, if $e: X \to Y$ is epi and $m: M \to Z$ is mono, then there exists a unique morphism $h: Y \to M$ making the whole diagram below commutative.
    \[
        \begin{tikzcd}
            X \ar[r, "f"] \ar[d, "e"swap] & M \ar[d, "m"] \\
            Y \ar[ur, dashed, "h"] \ar[r, "g"swap] & Z
        \end{tikzcd}
    \]
\end{prop}

%{\color{red}\begin{proof}
%    Before we start proving the statement, we have to note that, given a function $t: A \to B$, it is possible to decompose it as a composition of a injective function and a surjective function, considering the function $A \to t(A)$ sending each element onto its image along $t$, and then applying the inclusion $t(A) \to B$, and such functions are unique. Another way to factorize $t$ is via a composition of a surjective function and an injective function. Consider the equivalence relation $\sim$ defined on $A$, where $a \sim a'$ whenever $t(a) = t(a')$. This equivalence relation induces a map $A \to A/_\sim$, which is surjective. The function $A/_\sim \to B$, mapping each equivalence class onto its image along $t$ is then injective, and this factorization in unique too.

%    Let now be $f = f_i \circ f_s$ be the decomposition of $f$ with $f_s$ surjective (i.e., epi in $\Set$) and $f_i$ injective (i.e., mono in $\Set$), and $g = g_i \circ g_s$ be the decomposition of $g$ with $g_i$ injective and $g_s$ surjective, having the following situation.
%    \[
%        \begin{tikzcd}
%            X \ar[r, "{f_s}"] \ar[d, "e"] & f(X) \ar[r, "{f_i}"] & M \ar[d, "m"] \\
%            Y \ar[r, "{g_i}"swap] \ar[urr, "h"]& Y/_\sim \ar[r, "{g_s}"swap] & Z
%        \end{tikzcd}
%   \]
%    For the diagram above to commute, must be $f_i \circ f_s = h \circ e$ and $g_s \circ g_i = m \circ h$\todo{completare!!!}
%\end{proof}}
\begin{proof}
	Let $y \in Y$. Since $e$ is epi (i.e., surjective), there exists $x \in X$ such that $e(x) = y$. Then, we put $h(y) = f(x)$. $h$ is well defined, in fact, let $x' \in X$ be such that $e(x') = y$. In this case, we have $h(x) = f(x')$, and so
	\begin{align*}
		(m \circ f) (x')  &= (g \circ e) (x') \\
				&= g(y)		\\
				&= (g \circ e) (x)	\\
				&= (m \circ f) (x)
	\end{align*}
	since $m$ is mono by hypothesis, $f(x) = f(x')$.
\end{proof}
{\color{red}
\begin{cor}\label{cor:unique}
	% La fattorizzazione di un morfismo è unica a meno di isomorfismo \todo{Scrivilo bene (l'isomorfismo dovrà rispettare certe coerenze) e dimostralo}
	In $\Set$, let $f:A \to B$ be a morphism, and let $f=m \circ e$, $f=m'\circ e'$ be two factorizations of $f$, with $e: A \to X$ and $e': A \to Y$ epis and $m: X \to B$, $m': Y \to B$ monos.
	Then, there exists an isomorphism $\phi: X \to Y$.
\end{cor}

\begin{proof}
	Representing the hypothesis in diagrams, we have the following situation
	\[\begin{tikzcd}
		A \ar[r, "e"] \ar[d, "e'"swap] & X \ar[d, "m"] & A \ar[r, "e'"] \ar[d, "e"swap] & Y \ar[d, "m'"] \\
		Y \ar[r, "m'"swap] \ar[ur, dashed, "h"] & B    & X \ar[r, "m"swap] \ar[ur, dashed, "k"] & B 
	\end{tikzcd}\]
	where the dashed arrows are yielded by \Cref{prop:prop_epi_mono_Set}. Hence, we have that
	\[\begin{split}
		id_X \circ e &= h \circ e' \\ &= h \circ k \circ e
	\end{split}\qquad\begin{split}
		id_Y \circ e' &= k \circ e \\ &= k \circ h \circ e'
	\end{split}\]
	Hence, $h \circ k = id_X$, $k \circ h = id_Y$, obtaining the isomorphism.
\end{proof}
}
\subsection{Categories from other categories}\label{ssect:cats_from_cats}

Starting from a category, it is possible to construct other categories with some interesting properties, as the following examples show.

The first notion to introduce is the one of subcategory.

\begin{definition}[Subcategory]
    A category $\cat{D}$ is a \emph{subcategory} of a category $\cat{C}$ if:
    \begin{enumerate}
        \item each object of $\cat{D}$ is an object of $\cat{C}$;
        \item \label{inc} each morphism between two objects of $\cat{D}$ is a morphism of $\cat{C}$; and
        \item \label{comp} composites and identities of $\cat{D}$ are the same of $\cat{C}$.
    \end{enumerate}

    If the inclusion at~\ref{inc} is an equality (i.e. $\cat{D}(A, B) = \cat{C}(A, B)$ for each couple of objects $A$, $B$ of $\cat{D}$), then $\cat{D}$ is said to be a \emph{full subcategory} of $\cat{C}$.
    Another way to express that composites are the same (point~\ref{comp}) is to say that if $f, g \in \Hom(\cat D)$ are composable, then $g \circ f \in \Hom(\cat D)$, i.e., $\Hom(\cat D)$ is \emph{closed under composition}.
\end{definition}

An object of a category marks out a category itself. This is the case of slice (and coslice) categories.

\begin{definition}[Slice Category]\label{def:slice_cat}
    Given a category $\cat{C}$ and an object $X \in \Ob(\cat{C})$, the \emph{slice category} $\cat{C}/X$ is the category that has pairs $(A, f)$ as objects, where $A$ is an object of $C$ and $f: A \rightarrow X$ is an arrow in $\cat{C}$, and arrows $\phi: (A, f) \rightarrow (B, g)$ are given by a morphism $\phi: A \rightarrow B$ of $\cat{C}$ such that the following diagram commutes:
    \[
        \begin{tikzcd}
            A \arrow[r, "{\phi}"] \arrow[dr, "f" swap] & B \arrow[d, "g"] \\
            & X
        \end{tikzcd}
    \]
    -- i.e, $g \circ \phi = f$.
    Composition between two arrows in $\cat{C}/X$ $\phi: (A, f) \rightarrow (B, g)$ and $\psi: (B, g) \rightarrow (C, h)$ is the arrow $\psi \circ \phi : (A, f) \rightarrow (C, h)$ obtained in the obvious way:
    \[
        \begin{tikzcd}
            A \arrow[bend left=30]{rr}{\psi \circ \phi}  \arrow[r, "{\phi}"] \arrow[dr, "f" swap] & B \arrow[r, "{\psi}"] \arrow[d, "g"] & C \arrow[dl, "h"] \\
            & X & 
        \end{tikzcd}
    \]

    The dual definition of \emph{coslice category}, noted $X/\cat{C}$ (where $X \in \Ob(\cat{C})$), is obtained by taking as objects the morhisms of $\cat{C}$ with domain $X$ and as arrows the morphisms $\phi: (A, f) \rightarrow (B, g)$ such that $f:X\rightarrow A, g:B \rightarrow X \text{ of }\cat{C}$ and $g = \phi \circ f$. 
\end{definition}

Furthermore, it is possible to raise a new category from two old ones by taking their product, as the following definition shows.

\begin{definition}[Product category]
    Given two categories $\cat C, \cat D$, the \emph{product category} $\cat{C\times D}$ has as objects pairs of objects $(A, B)$, where $A \in \Ob(\cat C), B \in \Ob(\cat D)$, and as arrows pairs of morphisms $(f, g)$, where $f$ is an arrow in $\cat C$ and $g$ is an arrow in $\cat D$. Composition and identities are defined pairwise: $(f, g) \circ (h, k) = (f \circ h, g \circ k)$, and $id_{(A, B)} = (id_A, id_B)$.
\end{definition}

