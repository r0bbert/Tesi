\commentato{
\section{Equipping hypergraphical structures with equivalences}


\subsection{Hypergraphs with equivalences}

\subsection{Term graphs with equivalences}

\todo{Il materiale va distribuito nelle varie sottosezioni}

\begin{definition}
	A \emph{hypergraph with equivalence} $\mathcal{G} = (E_\mathcal{G}, V_{\mathcal{G}}, C_\mathcal{G}, s_\mathcal{G}, t_\mathcal{G}, q_\mathcal{G})$ is a 6-tuple such that $\mathcal{G} = (E_\mathcal{G}, V_{\mathcal{G}}, s_\mathcal{G}, t_\mathcal{G})$ is a hypergraph, $C_\mathcal{G}$ is an object and $q_{\mathcal{G}}: V_{\mathcal{G}}\to C_{\mathcal{G}}$ is a regular epimorphism called \emph{quotient map}. 
	
	A morphism $h:\mathcal{G\to H}$ is a triple $(h_E, h_V, h_C)$ such that the following diagrams commute.
	\[\xymatrix{
		{E_\mathcal{G}}\ar[r]^{s_\mathcal{G}}\ar[d]_{h_E} & {V_{\mathcal{G}}^\star}\ar[d]^{h_V^\star} & {E_\mathcal{G}}\ar[r]^{t_\mathcal{G}}\ar[d]_{h_E} & {V_{\mathcal{G}^\star}}\ar[d]^{h_V^\star} & {V_\mathcal{G}}\ar[r]^{q_\mathcal{G}}\ar[d]_{h_V} & {C_{\mathcal{G}}} \ar[d]^{h_C} \\
		{E_\mathcal{H}}\ar[r]_{s_\mathcal{H}} & {V_{\mathcal{H}}^\star}	& {E_\mathcal{H}}\ar[r]_{t_\mathcal{H}} & {V_{\mathcal{H}^\star}}& {V_\mathcal{H}}\ar[r]_{q_\mathcal{H}} & {C_\mathcal{H}}
	}\]
	The category of hypergraphs with equivalences and their morphisms is denoted $\EqHyp$.

\end{definition}

\begin{remark}\label{rem:eqhyp_morphs}
	Morphisms of hypergraphs with equivalences are uniquely determined by the first two components. That is, if $h_1 = (h_E, h_V, f)$ and $h_2 = (h_E, h_V, g)$ are two morphisms $\mathcal{G \to H}$, then we have
	\[\xymatrix{
			{V_\mathcal{G}} \ar[r]^{h_V}\ar[d]_{q_\mathcal{G}} & V_\mathcal{H} \ar[d]^{q_\mathcal{H}} & V_\mathcal{G}\ar[l]_{h_V}\ar[d]^{q_\mathcal{G}}\\
			C_{\mathcal{G}}\ar[r]_{f} & C_{\mathcal{H}} & C_{\mathcal{G}}\ar[l]^{g}
	}\]
	Hence $f \circ q_\mathcal{G} = q_\mathcal{H}\circ h_V  =g\circ q_\mathcal{G}$, and since $q_\mathcal{G}$ is epi, we obtain $f = g$.
\end{remark}

$\EqHyp$ has a forgetful functor $U_{\EqHyp}:\EqHyp \to \Set$, which sends each $\mathcal{G} = (E_\mathcal{G}, V_{\mathcal{G}}, C_\mathcal{G}, s_\mathcal{G}, t_\mathcal{G}, q_\mathcal{G})$ into $V_\mathcal{G}$, and each $h = (h_E, h_V, h_C)$ onto $h_V$. 

\begin{proposition}
	$U_\EqHyp$ has a left adjoint $\Delta_{\EqHyp}: \Set \to \EqHyp$.
\end{proposition}

\begin{proof}
	For each set $X$, define $\Delta_\EqHyp(X):= (\emptyset, X, \{\bullet\}, ?_X, ?_X, !_X)$. Consider now $h: \Delta_{\EqHyp}(X) \to \mathcal{H}$.
	\[\xymatrix@C=2.3cm{
			\Delta_{\EqHyp}(X) \ar@{.>}[d]_{\Delta_\EqHyp(f)} \ar[dr]^{h} & \\
			\Delta_\EqHyp(U_{\EqHyp}(\mathcal{H})) \ar[r]_{\epsilon_{\mathcal{H}}} & \mathcal{H}
	}\]
	Where $\Delta_\EqHyp(U_{\EqHyp}(\mathcal{H})) = (\emptyset, V_\mathcal{H}, \{\bullet\}, ?_{V_\mathcal{H}}, ?_{V_\mathcal{H}}, !_{V_\mathcal{H}})$ and $\epsilon_{\mathcal{H}} = (?_{E_\mathcal{H}}, \id{V_\mathcal{H}}, g)$.
	Note that, since $\Delta_{\EqHyp}(X)$ has the empty set as object of edges, $h_E = ?_{E_\mathcal{H}}$, then, the unique arrow that fits in the diagram is $\Delta_{\EqHyp}(f) = (?_{E_\mathcal{H}}, h_V, \id{\{\bullet\}})$.

\end{proof}

We now define another functor $T: \EqHyp \to \hyp$, which ``forgets'' the quotient part, mapping each hypergraph with equivalence $\mathcal{G} = (E_\mathcal{G}, V_{\mathcal{G}}, C_\mathcal{G}, s_\mathcal{G}, t_\mathcal{G}, q_\mathcal{G})$ onto $T(\mathcal{G})=(E_{\mathcal{G}}, V_{\mathcal{G}}, s_\mathcal{G}, t_{\mathcal{G}})$. Then, we have the following result.

\begin{proposition}
	$T$ has a left adjoint $L: \hyp \to \EqHyp$.
\end{proposition}

\begin{proof}
	Let $\mathcal{G}$ be a hypergraph, and define $L(\mathcal{G}) := (E_\mathcal{G}, V_{\mathcal{G}}, \{\bullet\}, s_\mathcal{G}, t_\mathcal{G}, !_{V_\mathcal{G}})$. Let now $h: L(\mathcal{G})\to \mathcal{H}$ be a morphism in $\EqHyp$, and consider the following situation.
	\[\xymatrix@C=2.3cm{
		L(\mathcal{G}) \ar@{.>}[d]_{L(f)} \ar[dr]^{h}\\ L(T(\mathcal{H})) \ar[r]_{\epsilon_{\mathcal{H}}} & \mathcal{H}
	}\]
	for $L(T(\mathcal{H}))=(E_\mathcal{H}, V_{\mathcal{H}}, \{\bullet\}, s_\mathcal{H}, t_\mathcal{H}, !_{V_\mathcal{H}})$. Then $\epsilon_\mathcal{H} = (\id{E_\mathcal{H}}, \id{V_\mathcal{H}}, h_C)$ (by \Cref{rem:eqhyp_morphs}, the last component is uniquely determined by the first two), and $L(f)$ must be $(h_E, h_V, \id{\{\bullet\}})$.
\end{proof}

\begin{remark}
	$T$ is faithful. Indeed, consider two morphisms $h = (h_E, h_V, h_C)$ and $k = (k_E, k_V, k_C)$, and suppose $T(h) = T(k)$, that is, $(h_E, h_V) = (k_E, k_V)$.
	By \Cref{rem:eqhyp_morphs}, we can conclude also $h_C = k_C$, and hence the faithfulness of $T$.
\end{remark}
}

\section{Hypergraphs and term graphs with equivalences}


\begin{definition}
	A \emph{hypergraph with equivalence} $\mathcal{G} = (E_\mathcal{G}, V_{\mathcal{G}}, C_\mathcal{G}, s_\mathcal{G}, t_\mathcal{G}, q_\mathcal{G})$ is a 6-tuple such that $\mathcal{G} = (E_\mathcal{G}, V_{\mathcal{G}}, s_\mathcal{G}, t_\mathcal{G})$ is a hypergraph, $C_\mathcal{G}$ is an object and $q_{\mathcal{G}}: V_{\mathcal{G}}\to C_{\mathcal{G}}$ is a regular epi called \emph{quotient map}. 
	%
	A morphism $h:\mathcal{G\to H}$ is a triple $(h_E, h_V, h_C)$ such that the following diagrams commute
	\[\xymatrix{
		{E_\mathcal{G}}\ar[r]^{s_\mathcal{G}}\ar[d]_{h_E} & {V_{\mathcal{G}}^\star}\ar[d]^{h_V^\star} & {E_\mathcal{G}}\ar[r]^{t_\mathcal{G}}\ar[d]_{h_E} & {V_{\mathcal{G}^\star}}\ar[d]^{h_V^\star} & {V_\mathcal{G}}\ar[r]^{q_\mathcal{G}}\ar[d]_{h_V} & {C_{\mathcal{G}}} \ar[d]^{h_C} \\
		{E_\mathcal{H}}\ar[r]_{s_\mathcal{H}} & {V_{\mathcal{H}}^\star}	& {E_\mathcal{H}}\ar[r]_{t_\mathcal{H}} & {V_{\mathcal{H}^\star}}& {V_\mathcal{H}}\ar[r]_{q_\mathcal{H}} & {C_\mathcal{H}}
	}\]
	The category of hypergraphs with equivalences and their morphisms is denoted $\EqHyp$.

\end{definition}

\begin{remark}\label{rem:eqhyp_morphs}
	Morphisms of hypergraphs with equivalences are uniquely determined by the first two components. That is, if $h_1 = (h_E, h_V, f)$ and $h_2 = (h_E, h_V, g)$ are two morphisms $\mathcal{G \to H}$, then we have
	\[\xymatrix{
			{V_\mathcal{G}} \ar[r]^{h_V}\ar[d]_{q_\mathcal{G}} & V_\mathcal{H} \ar[d]^{q_\mathcal{H}} & V_\mathcal{G}\ar[l]_{h_V}\ar[d]^{q_\mathcal{G}}\\
			C_{\mathcal{G}}\ar[r]_{f} & C_{\mathcal{H}} & C_{\mathcal{G}}\ar[l]^{g}
	}\]
	Hence $f \circ q_\mathcal{G} = q_\mathcal{H}\circ h_V =g\circ q_\mathcal{G}$, and since $q_\mathcal{G}$ is epi, we obtain $f = g$.
\end{remark}

$\EqHyp$ has a forgetful functor $U_{\EqHyp}:\EqHyp \to \Set$, which sends each $\mathcal{G} = (E_\mathcal{G}, V_{\mathcal{G}}, C_\mathcal{G}, s_\mathcal{G}, t_\mathcal{G}, q_\mathcal{G})$ into $V_\mathcal{G}$, and each $h = (h_E, h_V, h_C)$ onto $h_V$. 

\begin{proposition}
	$U_\EqHyp$ has a left adjoint $\Delta_{\EqHyp}: \Set \to \EqHyp$.
\end{proposition}

\begin{proof}
	For each set $X$, define $\Delta_\EqHyp(X):= (\emptyset, X, \{\bullet\}, ?_X, ?_X, !_X)$. Consider now $h: \Delta_{\EqHyp}(X) \to \mathcal{H}$.
	\[\xymatrix@C=2.3cm{
			\Delta_{\EqHyp}(X) \ar@{.>}[d]_{\Delta_\EqHyp(f)} \ar[dr]^{h} & \\
			\Delta_\EqHyp(U_{\EqHyp}(\mathcal{H})) \ar[r]_{\epsilon_{\mathcal{H}}} & \mathcal{H}
	}\]
	Where $\Delta_\EqHyp(U_{\EqHyp}(\mathcal{H})) = (\emptyset, V_\mathcal{H}, \{\bullet\}, ?_{V_\mathcal{H}}, ?_{V_\mathcal{H}}, !_{V_\mathcal{H}})$ and $\epsilon_{\mathcal{H}} = (?_{E_\mathcal{H}}, \id{V_\mathcal{H}}, g)$.
	Note that, since $\Delta_{\EqHyp}(X)$ has the empty set as object of edges, $h_E = ?_{E_\mathcal{H}}$, then, the unique arrow that fits in the diagram is $\Delta_{\EqHyp}(f) = (?_{E_\mathcal{H}}, h_V, \id{\{\bullet\}})$.
\end{proof}

We now define another functor $T: \EqHyp \to \hyp$, which ``forgets'' the quotient part, mapping each hypergraph with equivalence $\mathcal{G} = (E_\mathcal{G}, V_{\mathcal{G}}, C_\mathcal{G}, s_\mathcal{G}, t_\mathcal{G}, q_\mathcal{G})$ onto $T(\mathcal{G})=(E_{\mathcal{G}}, V_{\mathcal{G}}, s_\mathcal{G}, t_{\mathcal{G}})$. Then, we have the following result.

\begin{proposition}
	$T$ has a left adjoint $L: \hyp \to \EqHyp$.
\end{proposition}

\begin{proof}
	Let $\mathcal{G}$ be a hypergraph, and define $L(\mathcal{G}) := (E_\mathcal{G}, V_{\mathcal{G}}, \{\bullet\}, s_\mathcal{G}, t_\mathcal{G}, !_{V_\mathcal{G}})$. Let now $h: L(\mathcal{G})\to \mathcal{H}$ be a morphism in $\EqHyp$, and consider the following situation.
	\[\xymatrix@C=2.3cm{
		L(\mathcal{G}) \ar@{.>}[d]_{L(f)} \ar[dr]^{h}\\ L(T(\mathcal{H})) \ar[r]_{\epsilon_{\mathcal{H}}} & \mathcal{H}
	}\]
	for $L(T(\mathcal{H}))=(E_\mathcal{H}, V_{\mathcal{H}}, \{\bullet\}, s_\mathcal{H}, t_\mathcal{H}, !_{V_\mathcal{H}})$. Then, $\epsilon_\mathcal{H} = (\id{E_\mathcal{H}}, \id{V_\mathcal{H}}, h_C)$ (by \Cref{rem:eqhyp_morphs}, the last component is uniquely determined by the first two), and $L(f)$ must be $(h_E, h_V, \id{\{\bullet\}})$.
\end{proof}

\begin{remark}
	$T$ is faithful. Indeed, consider two morphisms $h = (h_E, h_V, h_C)$ and $k = (k_E, k_V, k_C)$, and suppose $T(h) = T(k)$, that is, $(h_E, h_V) = (k_E, k_V)$.
	By \Cref{rem:eqhyp_morphs}, we can conclude also $h_C = k_C$, and hence the faithfulness of $T$.
\end{remark}

Let now $K: \EqHyp \to \Set$ be the functor which sends each hypergraph with equivalence $\mathcal{G} = (E, V, C, s, t, q)$ onto $K(\mathcal{G}) = C$, and each morphism $(h_E, h_V, h_C)$ to $h_C$.

\begin{proposition}\label{prop:eqhyp_complete}
	\EqHyp \, is complete and cocomplete, and $T$ preserves limits and colimits.
\end{proposition}

\begin{proof}
	Let $D: \cat{I}\to \EqHyp$ be a diagram, and, for each $i \in \cat{I}$, $D(i) = (E_i, V_i, C_i, s_i, t_i, q_i)$.
	Suppose now $(E, V, s, t)$, together with morphisms $(\pi_i^E, \pi_i^V)$, be the limit of $T \circ D$.
	Then, $V$, together with $(q_i\circ \pi_V^i)_{i\in \cat I}$, is a cone for $K \circ D$. Indeed, let $\alpha: i \to j$ be an arrow of $\cat I$, $D(\alpha) = (h_E,h_V, h_C)$.
	By definition of $T$, $(T \circ D)(\alpha) = (h_E, h_V)$, hence we have
	\[\xymatrix{
		& V \ar[dl]_{\pi_V^i} \ar[dr]^{\pi_V^j} & \\V_i\ar[rr]^{h_V}\ar[d]_{q_i}&&V_j\ar[d]^{q_j}\\C_i\ar[rr]_{h_C}&&C_j
	}\]
	Suppose now that $L$, with morphisms $(l_i)_{i\in \cat I}$ be the limit of $K\circ D$. Hence, we have an arrow $l:V \to L$, which is not epi in general.
	Let then $l = m \circ q$ be the epi-mono factorization of it. Consider the following situation, where the outer rectangle commutes by definition, and the dotted arrow is yielded by ({\color{red}cite left lifting prop}).
	\[\xymatrix{
			V\ar[r]^{\pi_V^i}\ar[d]_{q}&V_i\ar[r]^{q_i}&C_i\ar[d]^{\id{C_i}}\\C\ar[r]_{m}\ar@{.>}[urr]^{\pi_C^i}&L\ar[r]_{l_i}&{C_i}
	}\]
	Thus, $(E, V, C, s, t, q)$, together with $(\pi_E^i, \pi_V^i, \pi_C^i)$ is a cone over $D$. {\color{red} remain to show that this cone is terminal}
	
	Suppose now $(E', V', s', t')$, together with $(\kappa_E^i, \kappa_V^i)_{i \in \cat I}$, be the colimit of $T \circ D$, and $C'$, with $(c_i)_{i \in \cat I}$ be the colimit of $K \circ D$.
	Then, we have the folliwing situation.
	\[\xymatrix{
								  &  {V'} &                             \\
		V_i \ar[ur]^{\kappa_i} \ar[rr]^{h_V} \ar[d]_{q_i} && V_j \ar[ul]_{\kappa_j} \ar[d]^{q_j}\\
		C_i \ar[dr]_{c_i}      \ar[rr]_{h_C}              &&C_j  \ar[dl]^{c_j}                  \\
								  &  {C'} &
	}\]
	Then, $C'$ with morphisms $(c_i \circ q_i)_{i\in \cat I}$ is a conone for $U \circ D$. Then, there exists a unique morphism $q': V' \to C'$ such that $q'\circ \kappa_V^i = c_i\circ q_i$.
	Such morphism is epi ({\color{red}cite Lemma 1.3.45 of the thesis}), and thus $(E', V', C', s', t', q')$, together with $(\kappa_E^i, \kappa_V^i, c_i)_{i\in \cat I}$ is the colimit of $D$.
	
\end{proof}

\begin{corollary}\label{cor:mono1}
	An arrow $h = (h_E, h_V, h_C): \mathcal{G \to H}$ is mono if and only if $T(h)$ is mono.
\end{corollary}

\begin{proof}
	The ``if'' part is given by the faithfulness of $T$.
	The ``only if'' part is given by \Cref{rem:eqhyp_morphs}.
\end{proof}

\begin{corollary}\label{cor:mono2}
	If $h = (h_E, h_V, h_C): \mathcal{G\to H}$ is a regular mono in $\EqHyp$, then $h_E, h_V$ and $h_C$ are all monos.
\end{corollary}

\begin{proof}
	If $h$ is mono, from \Cref{cor:mono1} we have that $h_E$ and $h_V$ are monos. Suppose now $f, g: \mathcal{H\rightrightarrows K}$ be the arrows equalized by $h$. Then, we have:
	\begin{align*} f_C \circ h_C \circ q_\mathcal{G} &= f_C\circ q_\mathcal{H} \circ h_V\\&=q_\mathcal{K}\circ g_V \circ h_V\\&=q_\mathcal{K}\circ f_V \circ h_V\\&=g_C \circ h_C \circ q_\mathcal{G} \end{align*}
	Since $q_\mathcal{G}$ is epi, we have $f_C \circ h_C = g_C \circ h_C$, hence $h_C$ is an equalizer for $f_C$ and $g_C$, and thus a mono.
\end{proof}

\begin{proposition}
	Let $h = (h_E, h_V, h_C): \mathcal{G \to H}$ be a regular mono in $\EqHyp$.
	Then, $h_E$ and $h_V$ are monos and $(K, \pi_1, \pi_2)$ is the kernel pair of $q_\mathcal{H}\circ h_V$ if and only if $(K, \pi_1, \pi_2)$ is the kernel pair of $q_\mathcal{G}$.
\end{proposition}

\begin{proof}
	By \Cref{cor:mono2}, we have that $h_E, h_V$ and $h_C$ are all monos.
	Hence, by \Cref{cor:kermono}, $(K, \pi_1, \pi_2)$ is the kernel pair of $q_\mathcal{G}$ if and only if it is the kernel pair also of $h_C \circ q_\mathcal{G}$, since $h_C$ is mono by hypothesis.
	The thesis follows from $h_C \circ q_\mathcal{G} = q_H \circ h_V$, and from the hypothesis of $h_E$ mono.
\end{proof}


\begin{remark}
    It is possible to restate the last proposition, by \Cref{ex:kernel_pairs_in_Set}, as 
    \begin{displayquote}
    \textit{$h_E$ and $h_V$ are mono and, for every $v, v'\in V_H$, $q_H(h_V(v))=q_H(h_V(v'))$ if and only if $q_G(v)=q_G(v')$}
    \end{displayquote}
    That is, a regular mono in $\EqHyp$ is a morphism that both reflects and preserves equivalences.
    % besides preserving them.
\end{remark}

Let us turn to another functor $\EqHyp \to \hyp$.

\begin{definition}
The \emph{quotient functor} $Q:\EqHyp\to \hyp $ is defined as
the one sending $(E, V, C, s, t, q)$ to $(E, C, q^{\star}\circ s, q^{\star}\circ t)$ and an arrow $(h_E, h_V, h_C)$ to $(h_E, h_C)$.
\end{definition}

\begin{remark}
	The action of the functor on a morphism of hypergraphs with equivalences gives a morphism of hypergraphs,
	in fact $q^{\star}_\mathcal{H} \circ s_\mathcal{H} \circ h_E = q^{\star}_\mathcal{H} \circ h_V^\star \circ s_\mathcal{G} = h_C^\star \circ q^{\star}_\mathcal{G} \circ s_\mathcal{G}$.
	The same is valid for $t_\mathcal{H}$ and $t_\mathcal{G}$. 
\end{remark}

\begin{lemma}\label{lemma:quot_funct_left_adj}
    $Q$ is a left adjoint.
\end{lemma}

\begin{proof}
	Let $R((A, B, s, t))$ be $(A, B, B, s, t, \id{B})$, so that $Q(R((A, B, s, t))) = (A, B, s, t)$. Now, suppose that $h = (h_E, h_V): Q((E, V, C, s', t', q)) \to (A, B, s, t)$  is an arrow in $\hyp$, and consider the triple $(h_E, h_V, h_V \circ q)$. Since $h$ is a morphism of $\hyp$, we have $h_V^{\star}\circ q^{\star}\circ s'= s\circ h_E$ and $h_V^{\star}\circ q^{\star}\circ t' = t\circ h_E$.
    Then we have the following squares
    \[\xymatrix{
		    E \ar[r]^{h_E}\ar[d]_{s'} & A \ar[d]^{s} & E \ar[r]^{h_E}\ar[d]_{t'} & A \ar[d]^{t} & V \ar[r]^{h_V \circ q} \ar[d]_{q} & B \ar[d]^{\id{B}} \\
		    V^{\star} \ar[r]_{h_V^\star\circ q^\star} & B^{\star} & V^{\star} \ar[r]_{h_V^\star\circ q^\star} & B^{\star} & C \ar[r]_{h_V} & B
    }\]

    \iffalse\[
        \begin{tikzcd}
            E \arrow[r, "{h_E}"] \arrow[d, "{s_G}"swap] & A \arrow[d, "s"] \\
            V \arrow[r, "{h_V \circ q}"swap] & B
        \end{tikzcd}
        \qquad
        \begin{tikzcd}
            E \arrow[r, "{h_E}"] \arrow[d, "{t_G}"swap] & A \arrow[d, "t"] \\
            V \arrow[r, "{h_V \circ q}"swap] & B
        \end{tikzcd}
        \qquad
        \begin{tikzcd}
            V \arrow[r, "{h_V\circ q}"] \arrow[d, "q" swap] & B \arrow[d, "{id_B}"] \\
            C \arrow[r, "{h_V}"swap] & B
        \end{tikzcd}
    \]\fi
    We have therefore found a morphism $(E, V, C, s', t', q) \to R((A, B, s, t))$ whose image through $Q$ fits in the diagram below
    \[\xymatrix{
		    (A, B, s, t) \ar[r]^{(\id{A}, \id{B})} & (A, B, s, t)\\
		    (E, C, q^\star\circ s', q^\star\circ t') \ar[u]^{Q(h_E, h_V \circ q, h_V)} \ar[ur]_{(h_E, h_V)}
    }\]   
    Such arrow is unique. Suppose $f = (f_E, f_V, f_C)$ to be another arrow with such property.
    Then, it must be $(\id A, \id B) \circ Q(f) = (f_E, f_C) = (h_E, h_C)$. Finally, $f_C = f_V \circ q = h_V \circ q$. 
\end{proof}

\begin{proposition}\label{prop:quot_creat_colims}
    $Q$ creates colimits.
\end{proposition}


\begin{proof}
    Since $Q$ is a left adjoint, it preserves colimits.
	Let $D: \cat I \to \EqHyp$ be a diagram, and let $\mathcal{C}$, together with  $(c_i)_{i\in \cat I}$ be the colimit of $Q \circ D$, where $\mathcal{C} = (A, C, q\circ s, q\circ t)$, and $D(i)$ is $(A_i, B_i, C_i, s_i, t_i, q_i)$.
	\iffalse
	Let now $T: \EqHyp \to \Set$ be the functor mapping each graph with equivalence onto its second component, $T((X, Y, Z, x, y, z)) = Y$, and each morphims onto its second component.\fi
	Let $((\kappa_i)_{i\in \cat I}, V)$ be the colimit of $U_\EqHyp \circ D$.
	Consider the following situation
	\[\xymatrix{
								  &  {V} &                             \\
		V_i \ar[ur]^{\kappa_i} \ar[rr]^{h_V} \ar[d]_{q_i} && V_j \ar[ul]_{\kappa_j} \ar[d]^{q_j}\\
		C_i \ar[dr]_{c_i}      \ar[rr]_{h_C}              &&C_j  \ar[dl]^{c_j}                  \\
								  &  {C} &
	}\]
	Now, since $((c^i_C \circ q_i)_{i \in \cat I}, C)$ is a cocone for $U_\EqHyp \circ D$, there exists a unique $q: V \to C$, which is epi by \Cref{lemma:nat_trans_reg_epi_canonical_arrow_reg_epi}.
	Consider now the functor $W: \EqHyp \to \Set$ mapping each $(X, Y, Z, x, y, z)$ onto $X$, and each morphism on its first component.
	By \Cref{prop:eqhyp_complete} and \Cref{lemma:limits_of_presheaves}, we have that $((c_E^i)_{i \in \cat I}, E)$ is the colimit of $W \circ D$.
	Notice that $((\kappa_i \circ s_i)_{i \in \cat{I}}, B)$ and $((\kappa_i \circ t_i)_{i \in \cat I}, B)$ are cocones for $W \circ D$, so let $s$ and $t$ be, respectively, the mediating arrow for the first one and the mediating arrow for the second one. It remains now to show that $(E, V, C, s, t, q)$, together with $(c_E^i, \kappa_i, c_C^i)_{i \in \cat I}$, is a colimit for $D$, but this follows by the proof of \Cref{prop:eqgrph_complete}.
\end{proof}

\begin{example}
	$Q$ does not preserve limits. Indeed, let $\mathcal{G}_1 = (E_1, A, A, s_1, t_1, id_A)$, $\mathcal{G}_2 = (E_2, B, B, s_2, t_2, id_B)$ and $\mathcal{G}_3 = (E_3, A + B, \terminal, s_3, t_3, !_{A + B})$, and let $h = (h_E, \iota_A, !_A): \mathcal{G}_1 \to \mathcal{G}_3$, $k = (k_E, \iota_B, !_B): \mathcal{G}_2 \to \mathcal{G}_3$, where $(\iota_A, \iota_B, A + B)$ is the coproduct of $A$ and $B$, $\terminal$ is the intial object (in $\Set$, the singleton set as shown in \Cref{ex:set_init_term}), and $!_X$ the unique arrow $X \to \terminal$.
	The following two diagrams show the pullback of $h$ and $k$ and the pullback of $Q(h)$ and $Q(k)$, on the second component (the vertices of the graphs)
	\[\xymatrix{
			\initial \ar[r]^{p_1} \ar[d]_{p_2}& A \ar[d]^{\iota_A} \\
			B \ar[r]_{\iota_B} & A+B
	}
	\qquad
	\xymatrix{
		{A \times B} \ar[r]^{\pi_A} \ar[d]_{\pi_B} & A \ar[d]^{!_A} \\
		B \ar[r]_{!_B} & \terminal
	}\]

	But the arrow $\initial \to A \times B$ is not epi in general (this is easy to see taking $\Set$ as example), hence such pullback is not preserved by $Q$.
\end{example}

\begin{lemma}\label{lemma:stab}
	In $\EqHyp$, pushouts along regular monos are stable.
\end{lemma}

\begin{proof}
	Let $\mathcal{G}_i = (A_i, B_i, C_i, s_i, t_i, q_i)$, $\mathcal{G}'_i=(A'_i, B'_i, C'_i, s'_i, t'_i, q'_i)$, for $i \in \{1, 2, 3, 4\}$, be hypergraphs with equivalence, 
	and, in the diagram below, suppose all the vertical faces are pullbacks, the bottom face is a pushout and $h$ is regular mono.
                \[
		\xymatrix@C=10pt@R=6pt{&\mathcal{G}_1'\ar[dd]|\hole_(.65){a}\ar[rr]^{h'} \ar[dl]_{k'} && \mathcal{G}_2' \ar[dd]^{b} \ar[dl]_{t'} \\ \mathcal{G}_3'  \ar[dd]_{c}\ar[rr]^(.7){p'} & & \mathcal{G}_4' \ar[dd]_(.3){d}\\&\mathcal{G}_1\ar[rr]|\hole^(.65){h} \ar[dl]^{k} && \mathcal{G}_2 \ar[dl]^{t} \\\mathcal{G}_3 \ar[rr]_{p} & & \mathcal{G}_4 }
	\]
	By \Cref{prop:eqhyp_complete} and \Cref{cor:mono1}, the following cubes in $\Set$ have pushouts as bottom faces and pullbacks as vertical faces, hence their top faces are pushouts.
	\[
        \xymatrix@C=10pt@R=6pt{&A_1'\ar[dd]|\hole_(.65){a_E}\ar[rr]^{h_E'} \ar[dl]_{k_E'} && A_2' \ar[dd]^{b_E} \ar[dl]_{t_E'} \\ A_3'  \ar[dd]_{c_E}\ar[rr]^(.7){p_E'} & & A_4' \ar[dd]_(.3){d_E}\\&A_1\ar[rr]|\hole^(.65){h_E} \ar[dl]^{k_E} && A_2 \ar[dl]^{t_E} \\A_3 \ar[rr]_{p_E} & & A_4 }
	\qquad
        \xymatrix@C=10pt@R=6pt{&B_1'\ar[dd]|\hole_(.65){a_V}\ar[rr]^{h_V'} \ar[dl]_{k_V'} && B_2' \ar[dd]^{b_V} \ar[dl]_{t_V'} \\ B_3'  \ar[dd]_{c_V}\ar[rr]^(.7){p_V'} & & B_4' \ar[dd]_(.3){d_V}\\&B_1\ar[rr]|\hole^(.65){h_V} \ar[dl]^{k_V} && B_2 \ar[dl]^{t_V} \\B_3 \ar[rr]_{p_V} & & B_4 }
	\]
	Consider now the following pullbacks.
	\[\xymatrix{
			Y \ar[d]_{y_2}\ar[r]^{y_1}&{C_4'}\ar[d]^{d_C}&U\ar[d]_{u_2}\ar[r]^{u_1}&{C_4'}\ar[d]^{d_C}&T\ar[r]^{x_1}\ar[d]_{x_2}&U\ar[d]^{u_2}\\
			C_3\ar[r]_{p_C}&C_4&C_2\ar[r]_{t_C}&C_4&C_1\ar[r]_{h_C}&C_2
	}\]
	Thus, [???]\todo{inserire in sezione 1 la proposizione} yields the following situation, in which the bottom face is a pushout, and the vertical faces are pullbacks, hence the top face is a pushout too.
        \[\xymatrix@C=10pt@R=6pt{&T\ar[dd]|\hole_(.65){x_2}\ar[rr]^{x_1} \ar@{.>}[dl]_{w} && U \ar[dd]^{u_2} \ar[dl]_{u_1} \\ Y  \ar[dd]_{y_2}\ar[rr]^(.7){y_1} & & C_4' \ar[dd]_(.3){d_C}\\&C_1\ar[rr]|\hole^(.65){h_C} \ar[dl]^{k_C} && C_2 \ar[dl]^{t_C} \\C_3 \ar[rr]_{p_C} & & C_4 }\]
	By the proof of \Cref{prop:eqhyp_complete}, we have that $m_2 \circ q'_3 : B'_3 \to Y$ and $m_3 \circ q'_2 : B_2' \to U$ are two epi-mono factorizations, with $m_2$ and $m_3$ monos.
	At the same way, let the following square to be a pullback.
	\[\xymatrix{S\ar[r]^{s_1}\ar[d]_{s_2}&C'_2\ar[d]^{m_3}\\T\ar[r]_{x_1}&U}\]
	Hence, in the following diagram, the outer rectangle is a pullback.
	\[\xymatrix{
			S\ar[r]^{s_1}\ar[d]_{s_2}&C'_2\ar[d]^{m_3}\\T\ar[r]_{x_1}\ar[d]_{x_2}&U\ar[d]^{u_2}\\C_1\ar[r]_{h_C}&C_2
	}\]
	By the same argument as before, there exists a mono $m_1$ such that $m_1\circ q_1' : B_1' \to S$.
	% Uguaglianze da capire se servono davvero

	We have to show that the top face of the cube ate the beginning of the proof is a pushout.
	Suppose then that $z: \mathcal{G}_2' \to \mathcal{H}$ and $w: \mathcal{G}_3' \to \mathcal{H}$, with $\mathcal{H} = (E, V, C, s, t, q)$,
	are two morphisms such that $z \circ h' = w \circ h'$, and let $v_V: B'_4 \to V$ the arrow induced by $z_V$ and $w_V$.
	We want to construct the dotted arrow $v_C$ which fits in the diagram below.
	\[
	\xymatrix@C=10pt@R=6pt{
		&B_1'\ar[dd]|\hole_(.65){q_1'}\ar[rr]^{h_V'} \ar[dl]_{k_V'} && B_2' \ar[dd]|\hole_(.65){q_2'} \ar[dl]_{t_V'}\ar[dr]^{z_V}\\
		B_3'  \ar[dd]_{q_3'}\ar[rr]^(.7){p_V'} & & B_4' \ar[dd]_(.3){q_4'} \ar[rr]^(.7){v_V}&&  V\ar[dd]^{q}\\
						       &C_1'\ar[rr]|\hole^(.65){h_C'} \ar[dl]_{k_C'} && C_2' \ar[dl]^{t_C'}\ar[dr]^{z_C}\\
		C'_3 \ar[rr]_{p_C'} & & C'_4\ar@{.>}[rr]_{v_C} && C
	}\]
	By \Cref{lem:mpo}, we know that the top face of the cube below is a pushout.
\[\xymatrix@C=15pt@R=15pt{&K_{s_2\circ m_1\circ q'_1}\ar[dd]|\hole_(.65){\pi_{s_2\circ m_1\circ q'_1}^{1}}\ar[rr]^{k_{h'_2}} \ar[dl]_{k_{k'_2}} && K_{m_2\circ q'_2} \ar[dd]^{\pi_{m_2\circ q'_2}^{1}} \ar[dl]_{k_{t'_2}} \\ K_{m_3\circ q'_3}  \ar[dd]_{\pi_{m_3\circ q'_3}^{1}}\ar[rr]^(.7){k_{p'_2}} & & K_{q'_4} \ar[dd]_(.3){\pi_{q'_4}^{1}}\\&B'_1\ar[rr]|\hole^(.65){h'_2} \ar[dl]^{k'_2} && B'_2 \ar[dl]^{t'_2} \\B'_3 \ar[rr]_{p'_2} & & C_4}\]
	And, since $m_3$ and $m_2$ are monos,
	\[\begin{split}
		q'_3 \circ \pi_{m_3 \circ q'_3}^1 = q'_3 \circ \pi_{m_3\circ q_3'}^2
	\end{split}\qquad\begin{split}
		q_2' \circ \pi_{m_2 \circ q_2'}^1 = q'_2 \circ \pi_{m_2 \circ q_2'}^2
	\end{split}\]
	Computing, we obtain
	\[\begin{split}
		q \circ v_V \circ \pi_{q_4'}^1 \circ k_{p_V'} &= q \circ v_V \circ p_V' \circ \pi_{m_3 \circ q_3'}^1 \\
							      &= q \circ w_V \circ \pi_{m_3 \circ q_3'}^1 \\
							      &= w_C \circ q_3' \circ \pi_{m_3 \circ q_3'}^1 \\
							      &= w_C \circ q_3' \circ \pi_{m_3 \circ q_3}^2 \\
							      &= q \circ w_V \circ \pi_{m_3 \circ q_3'}^2 \\
							      &= q \circ v_V \circ p_V' \circ \pi_{m_3 \circ q_3'}^2 \\
							      &= q \circ v_V \circ \pi_{q'_4}^2 \circ k_{p'_V}
	\end{split}\qquad\begin{split}
		q \circ v_V \circ \pi_{q'_4}^1 \circ k_{t_V'} &= q \circ v_V \circ t_V' \circ \pi_{m_2 \circ q_2'}^1 \\
							      &= q \circ z_V \circ \pi_{m_2 \circ q_2'}^1 \\
							      &= z_C \circ q_2' \circ \pi_{m_2 \circ q_2'}^1 \\
							      &= z_C \circ q_2' \circ \pi_{m_2 \circ q_2}^2 \\
							      &= q \circ z_V \circ \pi_{m_2 \circ q_2'}^2 \\
							      &= q \circ v_V \circ t_V' \circ \pi_{m_2 \circ q_2'}^2 \\
							      &= q \circ v_V \circ \pi_{q'_4}^2 \circ k_{t'_V}
	\end{split}\]
	Since the previous cube has a pushout as top face, by universal property, we have
	\[
		q \circ v_V \circ \pi_{q'_4}^1 = q \circ v_V \circ \pi_{q'_4}^2
	\]
	hence, $v_C$ is the mediating arrow.
	\[
		v_C \circ q'_4 \circ \pi_{q'_4}^1 = v_C \circ q'_4 \circ \pi_{q'_4}^2
	\]
\end{proof}

\begin{lemma}\label{lemma:van_kampen}
	In $\EqHyp$, pushouts along regular monos are $\reg(\EqHyp)$-Van Kampen.
\end{lemma}

\begin{proof}
	In lieu of \Cref{lemma:stab}, it is enough to proof that, given a cube as the one below, with pullbacks as back faces, pushouts as bottom and top faces and such that $h$ is a regular mono,
	the front faces are pullbacks too, where $\mathcal{G}_i = (A_i, B_i, C_i, s_i, t_i, q_i)$, $\mathcal{G}'=(A_i', B_i', C_i', s_i', t_i', q_i')$, for $i = 1, 2, 3, 4$.
  	\[
	\xymatrix@C=10pt@R=6pt{&\mathcal{G}_1'\ar[dd]|\hole_(.65){a}\ar[rr]^{h'} \ar[dl]_{k'} && \mathcal{G}_2' \ar[dd]^{b} \ar[dl]_{t'} \\ \mathcal{G}_3'  \ar[dd]_{c}\ar[rr]^(.7){p'} & & \mathcal{G}_4' \ar[dd]_(.3){d}\\&\mathcal{G}_1\ar[rr]|\hole^(.65){h} \ar[dl]^{k} && \mathcal{G}_2 \ar[dl]^{t} \\\mathcal{G}_3 \ar[rr]_{p} & & \mathcal{G}_4 }
	\]
	By \Cref{prop:eqhyp_complete} and \Cref{cor:mono_in_EqGrph}, the following two cubes have $\mathcal{M}$-pushouts as bottom faces and pullbacks as back faces,
	thus their front faces are pullbacks too.
	\[
        \xymatrix@C=10pt@R=6pt{&A_1'\ar[dd]|\hole_(.65){a_E}\ar[rr]^{h_E'} \ar[dl]_{k_E'} && A_2' \ar[dd]^{b_E} \ar[dl]_{t_E'} \\ A_3'  \ar[dd]_{c_E}\ar[rr]^(.7){p_E'} & & A_4' \ar[dd]_(.3){d_E}\\&A_1\ar[rr]|\hole^(.65){h_E} \ar[dl]^{k_E} && A_2 \ar[dl]^{t_E} \\A_3 \ar[rr]_{p_E} & & A_4 }
	\qquad
        \xymatrix@C=10pt@R=6pt{&B_1'\ar[dd]|\hole_(.65){a_V}\ar[rr]^{h_V'} \ar[dl]_{k_V'} && B_2' \ar[dd]^{b_V} \ar[dl]_{t_V'} \\ B_3'  \ar[dd]_{c_V}\ar[rr]^(.7){p_V'} & & B_4' \ar[dd]_(.3){d_V}\\&B_1\ar[rr]|\hole^(.65){h_V} \ar[dl]^{k_V} && B_2 \ar[dl]^{t_V} \\B_3 \ar[rr]_{p_V} & & B_4 }
	\]
	On the other hand we can consider the diagrams below, in which the inner squares are pullbacks.
	Since the outer diagrams commute, by definition of morphism of $\EqHyp$, then we have the existence of $m_2\colon C'_2\to U$, $m_3\colon C'_3\to Y $, $a_3\colon B'_3\to Y$ and $a_2\colon B'_2\to Y$.
	\[\xymatrix{C'_3 \ar@{.>}[dr]^{m_3} \ar@/^.3cm/[drr]^{p'_3} \ar@/_.3cm/[ddr]_{d_3} &&& C'_2 \ar@{.>}[dr]^{m_2} \ar@/^.3cm/[drr]^{t'_3} \ar@/_.3cm/[ddr]_{d_2}\\&Y \ar[r]_{y_1} \ar[d]_{y_2}& C'_4\ar[d]^{d_3}&& U \ar[d]_{u_2}\ar[r]_{u_1}& C'_4 \ar[d]^{d_3}\\&C_3 \ar[r]_{p_3}& C_4 &&C_2 \ar[r]_{t_3}& C_4 }\]

		\[\xymatrix{B'_3 \ar@{.>}[dr]^{a_3} \ar[r]^{p'_2} \ar[d]_{q'_3} &B'_4\ar[dr]^{q'_4}&& B'_2\ar@{.>}[dr]^{a_2} \ar[r]^{t'_2} \ar[d]_{q'_2} &B'_4\ar[dr]^{q'_4}\\C'_3 \ar[dr]_{d_3} &Y \ar[r]_{y_1} \ar[d]_{y_2}& C'_4\ar[d]^{d_3}&C_2' \ar[dr]_{d_2}& U \ar[d]_{u_2}\ar[r]_{u_1}& C'_4 \ar[d]^{d_3}\\&C_3 \ar[r]_{p_3}& C_4 &&C_2 \ar[r]_{t_3}& C_4 }\]

	Now, notice that  $m_3$ and $m_2$ are monos because $d_3$ and $d_2$ are regular monos. By the proof of \Cref{prop:eqhyp_complete}, to conclude it is enough to show that
	\[m_3\circ q'_3 = a_3 \qquad m_2\circ q'_2=a_2\]

	Indeed, if the previous equations hold, then $C'_3$ and $C'_2$ are epi-mono factorizations of $a_3$ and $a_2$ and the thesis follows from \Cref{cor:unique} and the proof of \Cref{prop:eqhyp_complete}.

	No if we compute we have:
	\[\begin{split}
		y_1\circ a_3&= q'_4\circ p'_2\\&=p'_3 \circ q'_3\\&=y_1\circ m_3\circ q'_3  
	\end{split}\qquad \begin{split}
		u_1\circ a_2&= q'_4\circ t'_2\\&=t'_3 \circ q'_3\\&=u_1\circ m_2\circ q'_2  
	\end{split}\]
	\[\begin{split}
	y_2\circ a_3&= d_3\circ q_3'\\&=y_2\circ m_3\circ q'_3
	\end{split}\qquad \begin{split}
	u_2\circ a_2&= d_2\circ q_2'\\&=u_2\circ m_2\circ q'_2
	\end{split}\]
	And we have done.
\end{proof} 
