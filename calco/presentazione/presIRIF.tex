\documentclass[aspectratio=169]{beamer}
%\usepackage{beamer}
\usepackage{amsthm,amsmath,amssymb,mathrsfs, dsfont}
\usepackage[english]{babel}
\usepackage[T1]{fontenc}
\usepackage[utf8]{inputenc}
\usepackage{amsmath}
\usepackage{ragged2e}
\usepackage{xcolor}
\colorlet{shaded}{gray!40} 

\usepackage{graphicx}
\usepackage{tikz}
\usetikzlibrary{shapes.geometric}
\usetikzlibrary{arrows.meta,arrows}
\usepackage[draft]{tikzit}
\input{hypergraph.tikzdefs}
\input{hypergraph.tikzstyles}
\usetikzlibrary{decorations.markings}
\usepackage[all, cmtip]{xy}
\usepackage{svg}

%%% Bibliography
\usepackage[style=numeric,backend=bibtex]{biblatex}
\addbibresource{biblio.bib}
\renewcommand*{\bibfont}{\footnotesize}

% default color is dark

\def\colortheme{dark} % preset themes available are lighten, light, dark
\def\alertcolor{dark} % preset themes available are lighten, light, dark
\def\upperbar{\true} % i want upper index/navigation bar, \true or \false
\def\bottomsectionbar{\true} % i want bottom bar with Title and Frame/Slide number, \true or \false
\def\bottomtitlebar{\false} % i want bottom bar with Section and Institute, \true or \false

% % %
\if\upperbar\false
    \setbeamertemplate{headline}{}
\fi
% % %


\newcommand{\catname}[1]{\textbf{\textup{#1}}}
\newcommand{\tg}[0]{\catname{TG}_{\Sigma}}
\newcommand{\eg}{\catname{EGG}}
\newcommand{\egg}{\mathsf{e}\text{-}\catname{EqHyp}}
\newcommand{\hyp}{\catname{Hyp}}
\newcommand{\hyps}{\catname{Hyp}_{\Sigma}}
\newcommand{\EqHyp}{\catname{EqHyp}} %equivalence hypergraphs
\newcommand{\EqHyps}{\catname{EqHyp}_{\Sigma}}
\newcommand{\EqTG}{\catname{EqTG}}
\newcommand{\EqTGs}{\catname{EqTG}_{\Sigma}}
\newcommand{\eto}{\twoheadrightarrow}
\newcommand{\mto}{\rightarrowtail}


\newcommand{\pbc}{\mathsf{Pb}}
\newcommand{\pbe}{\mathsf{ePb}}

\title{EGGs are adhesive!}
%\subtitle{Subtitle}
\author{R. Biondo, D. Castelnovo and \textbf{F. Gadducci}}
\institute
{
  University of Pisa
}
\date{IRIF, \today}

\def\X{\textbf {\textup{X}}}
\def\Y{\textbf {\textup{Y}}}
\def\D{\textbf {\textup{D}}}

\newenvironment{noheadlineframe}
{%
	\setbeamertemplate{headline}{}%
	\begin{frame}
	}
	{%
	\end{frame}
}

\newcommand{\catname}[1]{\textbf{\textup{#1}}}
\newcommand{\lab}{\catname{LHyp}}
\newcommand{\lmo}{\catname{LMH}}
\newcommand{\hyp}{\catname{Hyp}}
\newcommand{\phyp}{\catname{pHyp}}
\newcommand{\pshyp}{\catname{pSep}}
\newcommand{\shyp}{\catname{Sep}}
\newcommand{\hyps}{\catname{Hyp}_{\Sigma}}
\newcommand{\EqHyp}{\catname{Hyp}_{\mathsf{eq}}} %equivalence hypergraphs
\newcommand{\EqsHyp}{\catname{Sep}_{\mathsf{eq}}} 
\newcommand{\EqpHyp}{\catname{pHyp}_{\mathsf{eq}}}
\newcommand{\EqpsHyp}{\catname{pSep}_{\mathsf{eq}}}
\newcommand{\Eqlmo}{\catname{LMH}_{\mathsf{eq}}}
\newcommand{\EqHyps}{\catname{Hyp}_{\mathsf{eq}, \mathcal{G}}}
\newcommand{\Eqhyps}{\catname{Hyp}_{\mathsf{eq}, \Sigma}}
\newcommand{\EqTG}{\catname{EqTG}}
\newcommand{\EqTGs}{\catname{EqTG}_{\Sigma}}
\newcommand{\GEqTGs}{\catname{GEqTG}_{\Sigma}}
\newcommand{\GEqATGs}{\catname{GEqATG}_{\Sigma}}
\newcommand{\gr}{\textbf{\textup{Graph}}}
\newcommand{\dgr}{\catname{SGraph}}
\newcommand{\dg}{\catname{DAG}}
\newcommand{\rt}{\mathsf{dcl_s}}
\newcommand{\rta}{\mathsf{dcl}}
\newcommand{\rtd}{\mathsf{dcl_{d}}}
\newcommand{\slice}[2]{(\catname{#1}\downarrow{#2})}
\newcommand{\tg}[0]{\catname{TG}_{\Sigma}}
\newcommand{\teg}[0]{\catname{TeGr}_{\Sigma}}
\newcommand{\sv}[0]{\mathsf{Sieves}}
\newcommand{\mono}[1]{\mathsf{Mon}(\catname{#1})}
\newcommand{\mo}[1]{{#1}_\mathsf{Mon}}
\newcommand{\pro}{\mathsf{prod}}
\newcommand{\spro}{\mathsf{ps}}
\newcommand{\prol}{\mathsf{lprod}}
\newcommand{\pred}[1]{{\downarrow}#1}
\newcommand{\colim}[0]{\mathrm{colim}}
\newcommand{\cod}{\mathsf{cod}}
\renewcommand{\sp}{\mathsf{sp}}
\newcommand{\pr}{\mathsf{pr}}
\newcommand{\is}[1]{{#1}_{is}}
\renewcommand{\sup}{\mathsf{sup}}
\newcommand{\cow}[1]{\mathsf{cwd}({#1})}
\renewcommand{\inf}{\mathsf{inf}}
\newcommand{\dom}{\mathsf{dom}}
\newcommand{\dwnarrow}{\downarrow \hspace{-2pt}}
\newcommand{\Dwnarrow}{\Downarrow \hspace{-2pt}}
\newcommand{\egg}{\mathsf{e}\text{-}\catname{Hyp}_\eq}
\newcommand{\eggl}{\mathsf{e}\text{-}\catname{Hyp}_{\eq, \mathcal{G}}}
\newcommand{\eggll}{\mathsf{e}\text{-}\catname{Hyp}_{\eq, \Sigma}}
\newcommand{\elm}{\mathsf{e}\text{-}\catname{LMO}_\eq}
\newcommand{\eps}{\mathsf{e}\text{-}\catname{pSep}_\eq}
\newcommand{\lelm}{\mathsf{e}\text{-}\catname{LMO}_{\eq, \mathcal{G}}}
\newcommand{\leps}{\mathsf{e}\text{-}\catname{pSep}_{\eq, \mathcal{G}}}
\newcommand{\lelmBis}{\mathsf{e}\text{-}\catname{LMO}_{\eq, \Sigma}}
\newcommand{\lepsBis}{\mathsf{e}\text{-}\catname{pSep}_{\eq, \Sigma}}

\begin{document}

\firstpage % optional. First page

%\indexoverview % optional. Make a slide with index 

\section{Intro: EGGs}\justifying

%\sectionoverview % optional. Make a slide with section structure index 

\begin{frame}{What are EGGs?} % normal itemize, one frame
    
    \emph{E-graphs (EGGS)} are a graphical formalism  for program optimisation  via an efficient implementation of equality saturation. 
    
    \pause \medskip 
    
     They can be  defined as (acyclic) term graphs with an additional notion of  equivalence on nodes that is closed under the operators of the signature.
     
     \pause \medskip The idea is that nodes representing the same term are in the same equivalence class. \pause 
     
     \begin{center}
     	\hspace{.25cm}
     		\begin{tikzpicture}[baseline=(nil.center)]\begin{pgfonlayer}{nodelayer}
     				\node[style=node](v1) at (-0.9, 0.3){};
     				\node[style=node](v2) at (-0.9, -0.3){};
     				\node[style=medium box] at (0, 0){$/$};
     				\node[style=none](divfst) at (-0.3, 0.3){};
     				\node[style=none](divsnd) at (-0.3, -0.3){};
     				\node[style=none](divout) at (0.3, 0){};
     				\node[style=node] (w) at (0.9, 0){};
     				\node[style=small box] at (0, -1){$1$};
     				\node[style=none](one) at (0.3, -1){};
     				\node[style=node](z) at (0.9, -1){};
     				\node[style=none](nil) at (0, -0.5){};
     			\end{pgfonlayer}        
     			\begin{pgfonlayer}{edgelayer}
     				\draw(v1) to (divfst.center);
     				\draw(v2) to (divsnd.center);
     				\draw(divout.center) to (w);
     				\draw(one.center) to (z);
     			\end{pgfonlayer}
     			\begin{pgfonlayer}{eqlayer}
     				\draw[dashed, rounded corners](-1.2, 0.6) rectangle (-0.6, -0.6);
     				\draw[dashed, rounded corners](0.6, 0.3) rectangle (1.2, -1.3);
     			\end{pgfonlayer}\begin{pgfonlayer}{background}
     				\draw[color=white] (-1.5, 0.2) rectangle (1.5, -0.2);
     		\end{pgfonlayer}\end{tikzpicture}
     \end{center}
     \begin{center}
       The equation $x / x = 1$, graphically.
     \end{center}     
\end{frame}

\begin{frame}{Why EGGs?}
     
     Equations are interpreted as (possibly bidirectionl) rules.
     
     \pause \medskip However, rewriting does not imply removing the left-hand side of the rule.
     
     \pause \medskip Instead, a rewriting step adds nodes, edges and equivalence classes to a source graph.
    
         \begin{center}
	\hspace{.25cm}
	\xymatrix{        
		\begin{tikzpicture}[baseline=(w.base)]\begin{pgfonlayer}{nodelayer}
				\node[style=node](v1) at (-0.9, 0.3){};
				\node[style=node](v2) at (-0.9, -0.3){};
				\node[style=medium box] at (0, 0){$/$};
				\node[style=none](divfst) at (-0.3, 0.3){};
				\node[style=none](divsnd) at (-0.3, -0.3){};
				\node[style=none](divout) at (0.3, 0){};
				\node[style=node] (w) at (0.9, 0){};
			\end{pgfonlayer}        
			\begin{pgfonlayer}{edgelayer}
				\draw(v1) to (divfst.center);
				\draw(v2) to (divsnd.center);
				\draw(divout.center) to (w);
			\end{pgfonlayer}
			\begin{pgfonlayer}{eqlayer}
				\draw[dashed, rounded corners](-1.2, 0.6) rectangle (-0.6, -0.6);
					\draw[dashed, rounded corners] (.6, -0.3) rectangle (1.2, 0.3);
			\end{pgfonlayer}\begin{pgfonlayer}{background}
				\draw[color=white] (-1.5, 0.2) rectangle (1.5, -0.2);
			\end{pgfonlayer}
		\end{tikzpicture}
		\ar@{=>}[r] &
		\begin{tikzpicture}[baseline=(nil.center)]\begin{pgfonlayer}{nodelayer}
				\node[style=node](v1) at (-0.9, 0.3){};
				\node[style=node](v2) at (-0.9, -0.3){};
				\node[style=medium box] at (0, 0){$/$};
				\node[style=none](divfst) at (-0.3, 0.3){};
				\node[style=none](divsnd) at (-0.3, -0.3){};
				\node[style=none](divout) at (0.3, 0){};
				\node[style=node] (w) at (0.9, 0){};
				\node[style=small box] at (0, -1){$1$};
				\node[style=none](one) at (0.3, -1){};
				\node[style=node](z) at (0.9, -1){};
				\node[style=none](nil) at (0, -0.5){};
			\end{pgfonlayer}        
			\begin{pgfonlayer}{edgelayer}
				\draw(v1) to (divfst.center);
				\draw(v2) to (divsnd.center);
				\draw(divout.center) to (w);
				\draw(one.center) to (z);
			\end{pgfonlayer}
			\begin{pgfonlayer}{eqlayer}
				\draw[dashed, rounded corners](-1.2, 0.6) rectangle (-0.6, -0.6);
				\draw[dashed, rounded corners](0.6, 0.3) rectangle (1.2, -1.3);
			\end{pgfonlayer}\begin{pgfonlayer}{background}
				\draw[color=white] (-1.5, 0.2) rectangle (1.5, -0.2);
		\end{pgfonlayer}
\end{tikzpicture}
	}
\end{center}
\begin{center}
The rule $x / x \Rightarrow 1$, graphically.
\end{center}
      \pause \medskip At the end, an ``optimal'' term is extracted.
\end{frame}

\begin{frame}{What we want to do} 
    \begin{itemize}
        \item Formalise the definition of EGGs in a categorical setting and study their properties. \pause \medskip 
        \item Develop the theory of EGGs rewriting using the Double Pushout (DPO) paradigm.
    \end{itemize}
\end{frame}

\section{DPO rewriting}

\begin{frame}{DPO rewriting systems} 
	\begin{overprint}

		\onslide<1|handout:1>
		
		A \emph{DPO-rewriting rule} is a pair of arrows $l:K\to L$ and $r:K\to R$ with the same domain. To rewrite an object $G$ according to a rule we proceed in three steps.
		
		\onslide<2| handout:0>
		
	A \emph{DPO-rewriting rule} is a pair  of arrows $l:K\to L$ and $r:K\to R$ with the same domain. To rewrite an object $G$ according to a rule we proceed in three steps.
		
		\medskip
		First: find a \emph{match} $m:L\to G$.
		\begin{center}
			\begin{tikzpicture}
				\node(L)at(0,0){$L$};
				\node(K)at(2,0){$K$};
				\node(R)at(4,0){$R$};
				\node(G)at(0,-2){$G$};
				\draw[->](K)--(L)node[pos=0.5, above]{$l$};
				\draw[->](K)--(R)node[pos=0.5, above]{$r$};
				\draw[->, dotted](L)--(G)node[pos=0.5, left]{$m$};
			\end{tikzpicture}
		\end{center}
		
		
		\onslide<3| handout:0>
		
		A \emph{DPO-rewriting rule} is a pair  of arrows $l:K\to L$ and $r:K\to R$ with the same domain. To rewrite an object $G$ according to a rule we proceed in three steps.		
		
		\medskip
		Second: ``remove'' the image of $m$  building a pushout square. \vspace{-0.075cm}
		\begin{center}
			\begin{tikzpicture}
				\node(L)at(0,0){$L$};
				\node(K)at(2,0){$K$};
				\node(R)at(4,0){$R$};
				\node(G)at(0,-2){$G$};
				\node(C)at(2,-2){$C$};
				\draw[->](K)--(L)node[pos=0.5, above]{$l$};
				\draw[->](K)--(R)node[pos=0.5, above]{$r$};
				\draw[->](L)--(G)node[pos=0.5, left]{$m$};
				\draw[->, dotted](K)--(C)node[pos=0.5, right]{$p$};
				\draw[->, dotted](C)--(G)node[pos=0.5, below]{$l'$};
			\end{tikzpicture}
		\end{center}		
		\onslide<4| handout:0>
		
	A \emph{DPO-rewriting rule} is a pair of arrows $l:K\to L$ and $r:K\to R$ with the same domain. To rewrite an object $G$ according to a rule we proceed in three steps.
		
		\medskip
		Third: glue $R$ into the ``hole'',  taking a pushout.\vspace{-0.075cm}
		\begin{center}\hspace{0.08cm}
			\begin{tikzpicture}
				\node(L)at(0,0){$L$};
				\node(K)at(2,0){$K$};
				\node(R)at(4,0){$R$};
				\node(G)at(0,-2){$G$};
				\node(C)at(2,-2){$C$};
				\node(H)at(4,-2){$H$};
				\draw[->](K)--(L)node[pos=0.5, above]{$l$};
				\draw[->](K)--(R)node[pos=0.5, above]{$r$};
				\draw[->](L)--(G)node[pos=0.5, left]{$m$};
				\draw[->](K)--(C)node[pos=0.5, right]{$p$};
				\draw[->](C)--(G)node[pos=0.5, below]{$l'$};
				\draw[->, dotted ](C)--(H)node[pos=0.5, below]{$r'$};
				\draw[->, dotted ](R)--(H)node[pos=0.5, right]{$q$};
			\end{tikzpicture}
		\end{center}		
	\end{overprint}  	
\end{frame}




\begin{frame}{$\mathcal{M}$-adhesive categories}
	A categorical framework for DPO rewriting are \emph{$\mathcal{M}$-adhesive categories}.
	
		\medskip \pause
		Let $\mathcal{M}$ be a class of monomorphisms in a category $\X$, which contains all isomorphisms and is stable under pullbacks and composition. 
		
		\medskip \pause
		$\X$ is \emph{$\mathcal{M}$-adhesive} if, given a cube below with the hooked arrows in $\mathcal{M}$, the bottom face a pushout and the back faces pullbacks, we have
		
		\parbox{5cm}{\centering \begin{tikzpicture}
				\node(C)at(-1,0.4){$C$};
				\node(A)at(1,0.8){$A$};
				\node(B)at(2,0.4){$B$};
				\node(D)at(0,0){$D$};
				\node(A')at(1,2){$A'$};
				\node(B')at(2,1.6){$B'$};
				\node(C')at(-1,1.6){$C'$};
				\node(D')at(0,1.2){$D'$};
				\draw[<-](B')--(A');
				\draw[>->](B')--(B);
				\draw[>->](C')--(C); 
				\draw[>->](D')--(D);
				\draw[->](A')--(C'); 
				\draw[->](B')--(D');
				\draw[->](C')--(D');
				\draw[->](A)--(B);
				\draw[->](C)--(D);
				\draw[->](B)--(D);
				\draw[>-](A')--(1,1.45);
				\draw[->](1,1.35)--(A);
				\draw[>-](A)--(0.05, 0.61);
				\draw[<-](C)--(-0.05,0.59);
		\end{tikzpicture}} \qquad 	\parbox{4cm}{\centering
			top face is a pushout \\ $\iff$ \\ front faces are pullbacks}
			

\end{frame}

\begin{frame}

If a category is $\mathcal{M}$-adhesive, many properties for DPO rewriting can be proved once and for all: Church-Rosser, termination, concurrent semantics\ldots

\pause \medskip
Also, $\mathcal{M}$-adhesivity is preserved by common categorical constructions, such as 
\pause
\begin{itemize}
	\item products\pause 
	\item functor categories \pause
	\item slices and coslices
\end{itemize}

\pause  \medskip
Moreover, a full subcategory closed under the relevant pullbacks and pushouts inherits the adhesivity properties of the larger one.

\pause \medskip
Yet, proving that a category is $\mathcal{M}$-adhesive can be ``tricky'' \ldots
\end{frame}


\section{Hypergraphical structures }


\begin{frame}{Hypergraphs}

A \emph{hypergraph} is a 4-uple $(E, V, s, t)$ made by two sets $E$ and $V$ (\emph{edges} and \emph{nodes}) and arrows $s, t\colon E \rightrightarrows V^\star$. \pause A \emph{hypergraph morphism} $(E, V, s, t)\to (F, W, s,' t')$  is a pair of arrow $h\colon E\to E'$ and $k\colon V\to V'$ fitting in the digram below.
	
	\[\xymatrix{ E \ar[r]^{h}\ar[d]_{s}& F \ar[d]^{s'} & E \ar[r]^{k} \ar[d]_{t}& F\ar[d]^{t'}\\ V^\star\ar[r]_{k^\star}& W^\star & V^\star \ar[r]_{k^\star}&W^\star}\]
		
	\pause 
\begin{theorem}
	The category $\hyp$ of hypergraphs is adhesive.
\end{theorem}	

\end{frame}

\begin{frame}{Labelling  hypergraphs}
	
	\parbox{.5\linewidth}{Given an algebraic signature $\Sigma$ one can build an hypergraph $\mathcal{G}_\Sigma$ with a single node and edges corresponding with operations of $\Sigma$.} \hfill
		\begin{minipage}[r]{.4\linewidth}
			\begin{tikzpicture}
				\begin{pgfonlayer}{nodelayer}
					\node[style=node] (v) at (0, 0){};
					\node[style=none] (vlab) at (0, -0.4){$v$};
					
					\node[style=small box] (n) at (1.8, 0) {};
					\node[style=none] (nlab) at (2.4, 0.0) {$1$};
					\node[style=none] (nattach) at (1.6, 0) {};
					
					
					\node[style=medium box] (times) at (-1, 1.8) {};
					\node[style=none] (timeslab) at (-1, 2.6) {$\cdot$};
					\node[style=none] (timesin1) at (-1.2, 1.25) {};
					\node[style=none] (timesin2) at (-0.8, 1.25) {};
					\node[style=none] (timesout) at (-0.8, 1.8){};
					
					\node[style=medium box] (div) at (1, 1.8) {};
					\node[style=none] (divlab) at (1, 2.6) {$(-)^{-1}$};
					\node[style=none] (divin1) at (0.8, 1.25) {};
					\node[style=none] (divin2) at (1.2, 1.25) {};
					\node[style=none] (divout) at (0.8, 1.8){};
				\end{pgfonlayer}
				
				
				\begin{pgfonlayer}{edgelayer}
					\draw[style=diredge, in=270, out= 160] (v) to (timesin2.center); 
					\draw[style=diredge,out=200, in= 270] (v) to (timesin1.center);
					\draw[style=diredge,in=0, out=0] (nattach.center) to (v);
					
					\draw[style=diredge,in=270, out=20] (v) to (divin2.center); 
					\draw[style=diredge,out=60, in= 270] (v) to (divin1.center);
					
					\draw[style=diredge, in=120, out=0] (timesout.center) to (v);
					\draw[style=diredge, in=90, out=180] (divout.center) to (v);
				\end{pgfonlayer}
				
			\end{tikzpicture}
		\end{minipage}
	\pause 
	
	\medskip 
	The category $\hyp_{\Sigma}$ of \emph{labelled hypergraphs} is the slice $\hyp/\mathcal{G}_\Sigma$.
	
	\pause 
	\begin{theorem}
		 \hspace{1pt}$\hyp_\Sigma$ is adhesive.
	\end{theorem}	
\end{frame}

\begin{frame}{Why labelling}

\begin{minipage}[l]{.3\linewidth}
DPO rewriting relates isomorphic classes of objects.
\end{minipage}
\qquad 
\begin{minipage}[l]{.7\linewidth}
%\begin{center}
        \begin{tikzpicture}
                \begin{pgfonlayer}{nodelayer}
                        \node[style=none] (lab1) at (-1.8, 0.8);
                        \node[style=small box] (h1) at (-1.8, 0.3) {};
                        \node[style=small box] (2) at (-1.8, -0.5) {};
                        \node[style=none] (lab2) at (-1.8, -1.0);
                        \node[style=none] (h1b) at (-1.8, 0.3) {};
                        \node[style=none] (th2) at (0, 0.3) {};
                        \node[style=none] (bh2) at (0, -0.3) {};
                        \node[style=none] (h2b) at (0, 0) {};
                        \node[style=none] (th3) at (1.8, 0){};
                        \node[style=none] (bh3) at (1.8, -0.6){};
                        \node[style=none] (h3c) at (1.8, -0.3){};
                        \node[style=none] (2c) at (-1.8, -0.5){};
                        \node[style=none] (lab3) at (0, 0.8);
                        \node[style=medium box] (h2) at (0, 0) {};
                        \node[style=none] (lab4) at (1.8, 0.5);
                        \node[style=medium box] (h3) at (1.8, -0.3) {};
                        \node[style=none](labv2) at (-0.9, -0.9);
                        \node[style=node] (v2) at (-0.9, -0.5) {};
                        \node[style=none] (labv1) at (-0.9, 0.7);
                        \node[style=node] (v1) at (-0.9, 0.3) {};
                        \node[style=none] (labv3) at (0.9, 0.4);
                        \node[style=node] (v3) at (0.9, 0) {};
                        \node[style=none] (labv4) at (2.7, 0.1);
                        \node[style=node] (v4) at (2.7, -0.3) {};
                \end{pgfonlayer}
                \begin{pgfonlayer}{edgelayer}
                        \draw (h1b.center) to (th2.center);
                        \draw (2c.center) to (v2);
                        \draw[in=180, out=30] (v2) to (bh2.center);
                        \draw (h2b.center) to (th3.center);
                        \draw (v4) to (h3c.center);
                        %\draw (h3c) to (v4);
                        \draw[in=180, out=-30](v2) to (bh3);
                \end{pgfonlayer}
        \end{tikzpicture}
%\end{center}
\end{minipage}
\pause
\begin{minipage}[l]{.3\linewidth}
In a slice category, morphisms preserve ``labels''.
\end{minipage}\qquad 
\begin{minipage}[l]{.2\linewidth}
%\begin{center}
	\begin{tikzpicture}
		\begin{pgfonlayer}{nodelayer}
			\node[style=node] (v) at (0, 0){};
			\node[style=none] (vlab) at (0, -0.4){$v$};
			
			\node[style=small box] (n) at (-1.8, 0.4) {};
			\node[style=none] (nlab) at (-1.8, 0.9) {$n$};
			\node[style=none] (nattach) at (-1.6, 0.4) {};
			
			\node[style=small box] (const) at (-1.8, -0.4) {};
			\node[style=none] (constlab) at (-1.8, -0.9){$a$};
			\node[style=none] (constattach) at (-1.6, -0.4){};
			
			\node[style=medium box] (times) at (0, 1.9) {};
			\node[style=none] (timeslab) at (0, 2.7) {$*$};
			\node[style=none] (timesin1) at (-0.3, 2.2) {};
			\node[style=none] (timesin2) at (-0.3, 1.6) {};
			\node[style=none] (timesout) at (0.2, 1.9){};
			
			\node[style=medium box] (div) at (0, -1.9) {};
			\node[style=none] (divlab) at (0, -2.7) {$/$};
			\node[style=none] (divin1) at (-0.3, -2.2) {};
			\node[style=none] (divin2) at (-0.3, -1.6) {};
			\node[style=none] (divout) at (0.2, -1.9){};
		\end{pgfonlayer}		
		\begin{pgfonlayer}{edgelayer}
%			\draw[style=diredge, in=180, out= 112.5] (v) to (timesin2.center); 
			\draw[in=180, out= 112.5] (v) to (timesin2.center); 
			\draw[out=135, in= 180] (v) to (timesin1.center);
			\draw[in=165, out=0] (nattach.center) to (v);
			
			\draw[out=0, in=-165] (constattach.center) to (v);
			\draw[in=180, out= -112.5] (v) to (divin2.center); 
			\draw[out=-135, in= 180] (v) to (divin1.center);
			
			\draw[in=30, out=0] (timesout.center) to (v);
			\draw[in=-30, out=0] (divout.center) to (v);
		\end{pgfonlayer}
	\end{tikzpicture}
	\end{minipage} \qquad \pause
	\begin{minipage}{.5\linewidth}
		\begin{tikzpicture}
			\begin{pgfonlayer}{nodelayer}
				\node[style=none] (lab1) at (-1.8, 0.8) {};
				\node[style=small box] (h1) at (-1.8, 0.3) {$a$};
				\node[style=small box] (2) at (-1.8, -0.5) {$2$};
				\node[style=none] (lab2) at (-1.8, -1.0) {};
				\node[style=none] (h1b) at (-1.8, 0.3) {};
				\node[style=none] (th2) at (0, 0.3) {};
				\node[style=none] (bh2) at (0, -0.3) {};
				\node[style=none] (h2b) at (0, 0) {};
				\node[style=none] (th3) at (1.8, 0){};
				\node[style=none] (bh3) at (1.8, -0.6){};
				\node[style=none] (h3c) at (1.8, -0.3){};
				\node[style=none] (2c) at (-1.8, -0.5){};
				\node[style=none] (lab3) at (0, 0.8) {};
				\node[style=medium box] (h2) at (0, 0) {$*$};
				\node[style=none] (lab4) at (1.8, 0.5) {};
				\node[style=medium box] (h3) at (1.8, -0.3) {$/$};
				\node[style=none](labv2) at (-0.9, -0.9) {};
				\node[style=node] (v2) at (-0.9, -0.5) {};
				\node[style=none] (labv1) at (-0.9, 0.7) {};
				\node[style=node] (v1) at (-0.9, 0.3) {};
				\node[style=none] (labv3) at (0.9, 0.4) {};
				\node[style=node] (v3) at (0.9, 0) {};
				\node[style=none] (labv4) at (2.7, 0.1) {};
				\node[style=node] (v4) at (2.7, -0.3) {};
			\end{pgfonlayer}
			\begin{pgfonlayer}{edgelayer}
				\draw (h1b.center) to (th2.center);
				\draw (2c.center) to (v2);
				\draw[in=180, out=30] (v2) to (bh2.center);
				\draw (h2b.center) to (th3.center);
				\draw (v4) to (h3c.center);
				%\draw (h3c) to (v4);
				\draw[in=180, out=-30](v2) to (bh3);
			\end{pgfonlayer}
        	\end{tikzpicture}
%\end{center}
	\end{minipage}

\end{frame}

\section{Adding equivalences}

\begin{frame}{Hypergraphs with equivalence}

A \emph{(labelled) hypergraph with equivalence} is a (labelled) hypergraph equipped with a surjection $q\colon V\eto Q$ from the set of nodes.

\pause \medskip
We get categories $\EqHyp$ and $\EqHyps$ asking that the nodes part of a morphism induces a morphism between the quotients. 
\[\xymatrix{V \ar@{>>}[d]_{q} \ar[r]^{k}& W \ar@{>>}[d]^p\\ Q \ar@{.>}[r]_{u} & P }\]

\pause 

\begin{block}{Remark}\justifying
$\EqHyps$ is actually the slice over $\mathcal{G}_\Sigma$ endowed with the quotient from $1$.
\end{block}
\end{frame}


\begin{frame}{A key example}

\begin{center}
\begin{tikzpicture}[baseline=(w)]
        \begin{pgfonlayer}{nodelayer}
                \node[style=small box](a1) at (-1.8, -0.5){$a$};
                \node[style=none] (a1att) at (-1.5, -0.5){};
                \node[style=node](v1) at (-0.9, -0.5){};
                \node[style=small box](a2) at (-1.8, 0.5){$a$};
                \node[style=none] (a2att) at (-1.5, 0.5){};
                \node[style=node](v2) at (-0.9, 0.5){};
                \node[style=medium box](div) at (0, 0){$/$};
                \node[style=none](divfst) at (-0.3, -0.3){};
                \node[style=none] (divsnd) at (-0.3, 0.3){};
                \node[style=none] (divout) at (0.3, 0){};
                \node[style=node] (w) at (0.9, 0){};
                
                 \node [below=.5cm, align=flush center,text width=2cm] at (v1)
                {
                	Figure $(1)$.
                };
        \end{pgfonlayer}
        \begin{pgfonlayer}{edgelayer}
                \draw(a1att.center) to (v1);
                \draw(a2att.center) to (v2);
                \draw[out=0, in=-120] (v1) to (divfst.center);
                \draw[out=0, in=120](v2) to (divsnd.center);
                \draw(divout.center) to (w);
        \end{pgfonlayer}
        \begin{pgfonlayer}{eqlayer}
        	\draw[dashed, rounded corners] (-1.2, -0.8) rectangle (-.6, -0.2);
        	\draw[dashed, rounded corners] (-1.2, 0.2) rectangle (-.6, 0.8);
        	\draw[dashed, rounded corners] (.6, -0.3) rectangle (1.2, 0.3);
        \end{pgfonlayer}
\end{tikzpicture}
\qquad \qquad 
\begin{tikzpicture}[baseline=(w)]
	\begin{pgfonlayer}{nodelayer}
		\node[style=small box](a1) at (-2.3, -0.5){$a$};
		\node[style=none] (a1att) at (-2.3, -0.5){};
		\node[style=node](v1) at (-1.4, -0.5){};
		\node[style=node](w1) at (-0.9, -0.5){};
		\node[style=small box](a2) at (-2.3, 0.5){$a$};
		\node[style=none] (a2att) at (-2.3, 0.5){};
		\node[style=node](v2) at (-1.4, 0.5){};
		\node[style=node](w2) at (-0.9, 0.5){};
		\node[style=medium box](div) at (0, 0){$/$};
		\node[style=none](divfst) at (-0.3, -0.3){};
		\node[style=none] (divsnd) at (-0.3, 0.3){};
		\node[style=none] (divout) at (0.3, 0){};
		\node[style=node] (w) at (0.9, 0){};
		
		      \node [below=.5cm, align=flush center,text width=2cm] at (-1.15,-0.5)
		{
			Figure $(2)$.
		};
	\end{pgfonlayer}
	\begin{pgfonlayer}{edgelayer}
		\draw(a1att.center) to (v1);
		\draw(a2att.center) to (v2);
		\draw[out=0, in=-120] (w1) to (divfst.center);
		\draw[out=0, in=120](w2) to (divsnd.center);
		\draw(divout.center) to (w);
	\end{pgfonlayer}
	\begin{pgfonlayer}{eqlayer}
		\draw[dashed, rounded corners] (-1.7, 0.2) rectangle (-0.6, 0.8);
		\draw[dashed, rounded corners] (-1.7, -0.8) rectangle (-0.6, -0.2);
			\draw[dashed, rounded corners] (.6, -0.3) rectangle (1.2, 0.3);
	\end{pgfonlayer}
\end{tikzpicture}
\end{center}
\begin{center}
\begin{tikzpicture}[baseline=(w)]
	\begin{pgfonlayer}{nodelayer}
		\node[style=small box](a) at (-1.8, 0){$a$};
		\node[style=none] (aatt) at (-1.5, 0){};
		\node[style=node](v) at (-0.9, 0){};
		\node[style=medium box](div) at (0, 0){$/$};
		\node[style=none](divfst) at (-0.3, -0.3){};
		\node[style=none] (divsnd) at (-0.3, 0.3){};
		\node[style=none] (divout) at (0.3, 0){};
		\node[style=node] (w) at (0.9, 0){};
		      \node [below=.5cm, align=flush center,text width=2cm] at (-0.9,-0.5)
		{
			Figure $(3)$.
		};
	\end{pgfonlayer}
	\begin{pgfonlayer}{edgelayer}
		\draw(aatt.center) to (v);
		\draw[out=-60, in=-180] (v) to (divfst.center);
		\draw[out=60, in=-180](v) to (divsnd.center);
		\draw(divout.center) to (w);
	\end{pgfonlayer}
	\begin{pgfonlayer}{eqlayer}
			\draw[dashed, rounded corners] (.6, -0.3) rectangle (1.2, 0.3);
				\draw[dashed, rounded corners] (-1.2, -0.3) rectangle (-.6, 0.3);
	\end{pgfonlayer}
\end{tikzpicture}
\qquad \qquad 
\begin{tikzpicture}[baseline=(w)]
	\begin{pgfonlayer}{nodelayer}
		\node[style=node](w1) at (-0.9, -0.5){};
		\node[style=small box](a2) at (-2.3, 0){$a$};
		\node[style=none] (a2att) at (-2.3, 0){};
		\node[style=node](v) at (-1.4, 0){};
		\node[style=node](w2) at (-0.9, 0.5){};
		\node[style=medium box](div) at (0, 0){$/$};
		\node[style=none](divfst) at (-0.3, -0.3){};
		\node[style=none] (divsnd) at (-0.3, 0.3){};
		\node[style=none] (divout) at (0.3, 0){};
		\node[style=node] (w) at (0.9, 0){};
		
		      \node [below=.5cm, align=flush center,text width=2cm] at (-1.15, -0.5)
		{
			Figure $(4)$.
		};
	\end{pgfonlayer}
	\begin{pgfonlayer}{edgelayer}
		\draw(a2att.center) to (v);
		\draw[out=0, in=-120] (w1) to (divfst.center);
		\draw[out=0, in=120](w2) to (divsnd.center);
		\draw(divout.center) to (w);
	\end{pgfonlayer}
	\begin{pgfonlayer}{eqlayer}
		\draw[dashed, rounded corners] (-1.7, -0.8) rectangle (-0.6, 0.8);
			\draw[dashed, rounded corners] (.6, -0.3) rectangle (1.2, 0.3);
	\end{pgfonlayer}
\end{tikzpicture}
\end{center}

\end{frame}

\begin{frame}{Limits and colimits in $\EqHyp$}
	Colimits are computed componentwise. If $(E_d, V_d, Q_d, s_d, t_d, q_d )$ is a diagram of hypergraphs with equivalence, then a colimit is given by $(E, V, Q, s, t, q)$, where $E$, $V$, $Q$ are the colimits of $\{E_d\}_{d\in \D}$, $\{V_d\}_{d\in \D}$, $\{Q_d\}_{d\in \D}$ and $s, t, q$ are the natural ones.
	
	\pause  \medskip 
	\begin{minipage}[l]{0.5\linewidth}
		Limits are more difficult. In the example on the right, the pullback of the underlying hypergraphs gives us the empty one. The pullback of quotients is the singleton.
	\end{minipage} \qquad \quad 
\begin{minipage}[r]{0.6\linewidth}
	\xymatrix{
		 &
		\begin{tikzpicture}[baseline=(v.base)]\begin{pgfonlayer}{nodelayer}
				\node[style=node](v) at (0, 0){};
				\node[style=none] at(0, 0.5){$c$};
			\end{pgfonlayer}
			\begin{pgfonlayer}{eqlayer}\draw[dashed, rounded corners] (-0.3, -0.3) rectangle (0.3, 0.3);\end{pgfonlayer}
		\end{tikzpicture}  \ar@{>->}@(d,)[d]!<0ex,0ex> \\ 
		\begin{tikzpicture}[baseline=(v.base)]\begin{pgfonlayer}{nodelayer}
				\node[style=node](v)at(0, 0){};
				\node[style=none]at(0, 0.5){$b$};
			\end{pgfonlayer}
			\begin{pgfonlayer}{eqlayer}\draw[dashed, rounded corners] (-0.3, -0.3) rectangle (0.3, 0.3);\end{pgfonlayer}
		\end{tikzpicture}
		\ar@{>->}@(r,)@<-.5ex>[r] & 
		\begin{tikzpicture}[baseline=(v.base)]\begin{pgfonlayer}{nodelayer}
				\node[style=node](v)at(-0.5, 0){};
				\node[style=none]at(-0.5, 0.5){$b$};
				\node[style=node]at(0.5, 0){};
				\node[style=none]at(0.5, 0.5){$c$};
			\end{pgfonlayer}\begin{pgfonlayer}{eqlayer}
				\draw[dashed, rounded corners](-0.8, 0.3) rectangle (0.8, -0.3);
		\end{pgfonlayer}\end{tikzpicture}
		 }
	\end{minipage}
\end{frame}

\begin{frame}{Limits and colimits in $\EqHyp$}

Luckily in $\catname{Set}$ we can factor any arrow as a surjection followed by an injection. 

\pause \medskip
General recipe for limits \pause 

\begin{enumerate}
	\item compute the limit $(E, V, s,t)$ of the underlying hypergraphs; \pause 
	\item compute the limit $L$ of the family of quotients; \pause 
	\item factor the natural arrow $V\to L$ as $V\eto Q \mto L$; \pause 
	\item $(E, V, Q, s,t, q )$ is a colimit for the original diagram, where $q\colon V\eto Q$.
\end{enumerate}	
	
\end{frame}


\begin{frame}{Hypergraphs with equivalence - adhesivity}
	
Pushouts along regular monos in $\EqHyp$ are not stable. 

\begin{center}
	\xymatrix@C=10pt@R=6pt{
		&\emptyset \ar@{>->}[dd]|\hole \ar@{>->}[rr] \ar@{>->}@(dl,)[dl] &&
		\begin{tikzpicture}[baseline=(v.base)]\begin{pgfonlayer}{nodelayer}
				\node[style=node](v) at (0, 0){};
				\node[style=none] at(0, 0.5){$c$};
			\end{pgfonlayer}
			\begin{pgfonlayer}{eqlayer}\draw[dashed, rounded corners] (-0.3, -0.3) rectangle (0.3, 0.3);\end{pgfonlayer}
		\end{tikzpicture} \ar@{>->}[dd] \ar@{>->}@(dl,)[dl]!<6ex,0ex> \\
		\begin{tikzpicture}[baseline=(v.base)]\begin{pgfonlayer}{nodelayer}
				\node[style=node](v)at(0, 0){};
				\node[style=none]at(0, 0.5){$b$};
			\end{pgfonlayer}
			\begin{pgfonlayer}{eqlayer}\draw[dashed, rounded corners] (-0.3, -0.3) rectangle (0.3, 0.3);\end{pgfonlayer}
		\end{tikzpicture}
		\ar@{>->}[dd]\ar@{>->}[rr] & &
		\begin{tikzpicture}[baseline=(v.base)]\begin{pgfonlayer}{nodelayer}
				\node[style=node](v)at(-0.5, 0){};
				\node[style=none]at(-0.5, 0.5){$b$};
				\node[style=node]at(0.5, 0){};
				\node[style=none]at(0.5, 0.5){$c$};
			\end{pgfonlayer}\begin{pgfonlayer}{eqlayer}
				\draw[dashed, rounded corners](-0.8, 0.3) rectangle (0.8, -0.3);
		\end{pgfonlayer}\end{tikzpicture}
		\ar@<2ex>@{>->}[dd]\\&
		\begin{tikzpicture}[baseline=(v.base)]\begin{pgfonlayer}{nodelayer}
				\node[style=node](v)at(0, 0){};
				\node[style=none]at(0, 0.5){$a$};
			\end{pgfonlayer}
			\begin{pgfonlayer}{eqlayer}\draw[dashed, rounded corners] (-0.3, -0.3) rectangle (0.3, 0.3);\end{pgfonlayer}
		\end{tikzpicture}
		\ar@{>->}[rr]|\hole \ar@{>->}@(dl,)[dl] && 
		\begin{tikzpicture}[baseline=(v.base)]\begin{pgfonlayer}{nodelayer}
				\node[style=node](v)at(-0.5, 0){};
				\node[style=none]at(-0.5, 0.5){$a$};
				\node[style=node]at(0.5, 0){};
				\node[style=none]at(0.5, 0.5){$c$};
			\end{pgfonlayer}\begin{pgfonlayer}{eqlayer}
				\draw[dashed, rounded corners](-0.8, 0.3) rectangle (0.8, -0.3);
		\end{pgfonlayer}\end{tikzpicture}
		\ar@{>->}@(,r)[dl] \\
		\begin{tikzpicture}[baseline=(v.base)]\begin{pgfonlayer}{nodelayer}
				\node[style=node](v)at(-0.5, 0){};
				\node[style=none]at(-0.5, 0.5){$a$};
				\node[style=node]at(0.5, 0){};
				\node[style=none]at(0.5, 0.5){$b$};
			\end{pgfonlayer}\begin{pgfonlayer}{eqlayer}
				\draw[dashed, rounded corners](-0.8, 0.3) rectangle (0.8, -0.3);
		\end{pgfonlayer}\end{tikzpicture}
		\ar@{>->}[rr] & & 
		\begin{tikzpicture}[baseline=(v.base)]\begin{pgfonlayer}{nodelayer}
				\node[style=node](v)at(0, 0){};
				\node[style=none]at(0, 0.5){$a$};
				\node[style=node]at(1, 0){};
				\node[style=none]at(1, 0.5){$b$};
				\node[style=node]at(2, 0){};
				\node[style=none]at(2, 0.5){$c$};
			\end{pgfonlayer}\begin{pgfonlayer}{eqlayer}
				\draw[dashed, rounded corners](-.3, 0.3) rectangle (2.3, -0.3);
	\end{pgfonlayer}\end{tikzpicture} }
\end{center}
\end{frame}



\begin{frame}{Hypergraphs with equivalence - adhesivity}
	\begin{minipage}[l]{.6\linewidth}
		We must find another class to test for adhesivity. Let $\pbc$ be the class of monos of $\EqHyp$ such that the square aside is a pullback.
	\end{minipage}\qquad 
	\begin{minipage}{.3\linewidth}
	\xymatrix{V \ar@{>>}[d]_{q} \ar@{>->}[r]^{k}& W \ar@{>>}[d]^p\\ Q \ar@{>->}[r]_{u} & P }
	\end{minipage}
	\pause 
	
	\begin{theorem}
		$\EqHyp$ is $\pbc$-adhesive.
	\end{theorem}

\pause \medskip
	In the same way can define a labelled analog $\pbc_\Sigma$ of $\pbc$.
	\pause 
	\begin{theorem}
		$\EqHyp_\Sigma$ is $\pbc_\Sigma$-adhesive.
	\end{theorem} 
	
\end{frame}

\section{Congruences}

\begin{frame}{Congruences}

\begin{minipage}[l]{.6\linewidth}
General equivalence relations are not enough. We want \emph{congruences}: 
if a relation identifies the sources of two hyperedges, it must identify their targets too. 
\end{minipage}\qquad 
\begin{minipage}[r]{.3\linewidth}
	\begin{tikzpicture}
        \begin{pgfonlayer}{nodelayer}
                \node[style=small box](a1) at (-1.8, -0.5){$a$};
                \node[style=none] (a1att) at (-1.5, -0.5){};
                \node[style=node](v1) at (-0.9, -0.5){};
                \node[style=small box](a2) at (-1.8, 0.5){$a$};
                \node[style=none] (a2att) at (-1.5, 0.5){};
                \node[style=node](v2) at (-0.9, 0.5){};
                \node[style=medium box](div) at (0, 0){$/$};
                \node[style=none](divfst) at (-0.3, -0.3){};
                \node[style=none] (divsnd) at (-0.3, 0.3){};
                \node[style=none] (divout) at (0.3, 0){};
                \node[style=node] (w) at (0.9, 0){};
        \end{pgfonlayer}
        \begin{pgfonlayer}{edgelayer}
                \draw(a1att.center) to (v1);
                \draw(a2att.center) to (v2);
                \draw[out=0, in=-120] (v1) to (divfst.center);
                \draw[out=0, in=120](v2) to (divsnd.center);
                \draw(divout.center) to (w);
        \end{pgfonlayer}
        \begin{pgfonlayer}{eqlayer}
        	\draw[dashed, rounded corners] (-1.2, -0.8) rectangle (-.6, -0.2);
        	\draw[dashed, rounded corners] (-1.2, 0.2) rectangle (-.6, 0.8);
        	\draw[dashed, rounded corners] (.6, -0.3) rectangle (1.2, 0.3);
        \end{pgfonlayer}
\end{tikzpicture}
\end{minipage}
\pause 

\begin{definition}\justifying 
	Let $\mathcal{G} = (E, V, Q, s, t, q)$ be a hypergraph with equivalence and $(S, \pi_1, \pi_2)$ a kernel pair for $q^\star \circ s$.
We will say that $\mathcal{G}$ is an \emph{e-hypergraph} if $q^\star \circ t \circ \pi_1 = q^\star \circ t \circ \pi_2$.
\end{definition}	
	
	\pause 
	Restricting to e-hypergraphs we get \pause 
\begin{itemize}\justifying
	\item $\egg$: the category of e-hypergraphs is a full subcategory of $\EqHyp$; \pause 
	\item $\egg_\Sigma$: the subcategory of $\EqHyp_\Sigma$ whose underlying hypergraph is in $\egg$, it is actually a slice of $\egg$.
\end{itemize}	
	
\end{frame}

\begin{frame}{The key example}

\begin{center}
\begin{tikzpicture}[baseline=(w)]
        \begin{pgfonlayer}{nodelayer}
                \node[style=small box](a1) at (-1.8, -0.5){$a$};
                \node[style=none] (a1att) at (-1.5, -0.5){};
                \node[style=node](v1) at (-0.9, -0.5){};
                \node[style=small box](a2) at (-1.8, 0.5){$a$};
                \node[style=none] (a2att) at (-1.5, 0.5){};
                \node[style=node](v2) at (-0.9, 0.5){};
                \node[style=medium box](div) at (0, 0){$/$};
                \node[style=none](divfst) at (-0.3, -0.3){};
                \node[style=none] (divsnd) at (-0.3, 0.3){};
                \node[style=none] (divout) at (0.3, 0){};
                \node[style=node] (w) at (0.9, 0){};
                
                 \node [below=.5cm, align=flush center,text width=2cm] at (v1)
                {
                	Figure $(1)$.
                };
        \end{pgfonlayer}
        \begin{pgfonlayer}{edgelayer}
                \draw(a1att.center) to (v1);
                \draw(a2att.center) to (v2);
                \draw[out=0, in=-120] (v1) to (divfst.center);
                \draw[out=0, in=120](v2) to (divsnd.center);
                \draw(divout.center) to (w);
        \end{pgfonlayer}
        \begin{pgfonlayer}{eqlayer}
        	\draw[dashed, rounded corners] (-1.2, -0.8) rectangle (-.6, -0.2);
        	\draw[dashed, rounded corners] (-1.2, 0.2) rectangle (-.6, 0.8);
        	\draw[dashed, rounded corners] (.6, -0.3) rectangle (1.2, 0.3);
        \end{pgfonlayer}
\end{tikzpicture}
\qquad \qquad 
\begin{tikzpicture}[baseline=(w)]
	\begin{pgfonlayer}{nodelayer}
		\node[style=small box](a1) at (-2.3, -0.5){$a$};
		\node[style=none] (a1att) at (-2.3, -0.5){};
		\node[style=node](v1) at (-1.4, -0.5){};
		\node[style=node](w1) at (-0.9, -0.5){};
		\node[style=small box](a2) at (-2.3, 0.5){$a$};
		\node[style=none] (a2att) at (-2.3, 0.5){};
		\node[style=node](v2) at (-1.4, 0.5){};
		\node[style=node](w2) at (-0.9, 0.5){};
		\node[style=medium box](div) at (0, 0){$/$};
		\node[style=none](divfst) at (-0.3, -0.3){};
		\node[style=none] (divsnd) at (-0.3, 0.3){};
		\node[style=none] (divout) at (0.3, 0){};
		\node[style=node] (w) at (0.9, 0){};
		
		      \node [below=.5cm, align=flush center,text width=2cm] at (-1.15,-0.5)
		{
			Figure $(2)$.
		};
	\end{pgfonlayer}
	\begin{pgfonlayer}{edgelayer}
		\draw(a1att.center) to (v1);
		\draw(a2att.center) to (v2);
		\draw[out=0, in=-120] (w1) to (divfst.center);
		\draw[out=0, in=120](w2) to (divsnd.center);
		\draw(divout.center) to (w);
	\end{pgfonlayer}
	\begin{pgfonlayer}{eqlayer}
		\draw[dashed, rounded corners] (-1.7, 0.2) rectangle (-0.6, 0.8);
		\draw[dashed, rounded corners] (-1.7, -0.8) rectangle (-0.6, -0.2);
			\draw[dashed, rounded corners] (.6, -0.3) rectangle (1.2, 0.3);
	\end{pgfonlayer}
\end{tikzpicture}
\end{center}
\begin{center}
\begin{tikzpicture}[baseline=(w)]
	\begin{pgfonlayer}{nodelayer}
		\node[style=small box](a) at (-1.8, 0){$a$};
		\node[style=none] (aatt) at (-1.5, 0){};
		\node[style=node](v) at (-0.9, 0){};
		\node[style=medium box](div) at (0, 0){$/$};
		\node[style=none](divfst) at (-0.3, -0.3){};
		\node[style=none] (divsnd) at (-0.3, 0.3){};
		\node[style=none] (divout) at (0.3, 0){};
		\node[style=node] (w) at (0.9, 0){};
		      \node [below=.5cm, align=flush center,text width=2cm] at (-0.9,-0.5)
		{
			Figure $(3)$.
		};
	\end{pgfonlayer}
	\begin{pgfonlayer}{edgelayer}
		\draw(aatt.center) to (v);
		\draw[out=-60, in=-180] (v) to (divfst.center);
		\draw[out=60, in=-180](v) to (divsnd.center);
		\draw(divout.center) to (w);
	\end{pgfonlayer}
	\begin{pgfonlayer}{eqlayer}
			\draw[dashed, rounded corners] (.6, -0.3) rectangle (1.2, 0.3);
				\draw[dashed, rounded corners] (-1.2, -0.3) rectangle (-.6, 0.3);
	\end{pgfonlayer}
\end{tikzpicture}
\qquad \qquad 
\begin{tikzpicture}[baseline=(w)]
	\begin{pgfonlayer}{nodelayer}
		\node[style=node](w1) at (-0.9, -0.5){};
		\node[style=small box](a2) at (-2.3, 0){$a$};
		\node[style=none] (a2att) at (-2.3, 0){};
		\node[style=node](v) at (-1.4, 0){};
		\node[style=node](w2) at (-0.9, 0.5){};
		\node[style=medium box](div) at (0, 0){$/$};
		\node[style=none](divfst) at (-0.3, -0.3){};
		\node[style=none] (divsnd) at (-0.3, 0.3){};
		\node[style=none] (divout) at (0.3, 0){};
		\node[style=node] (w) at (0.9, 0){};
		
		      \node [below=.5cm, align=flush center,text width=2cm] at (-1.15, -0.5)
		{
			Figure $(4)$.
		};
	\end{pgfonlayer}
	\begin{pgfonlayer}{edgelayer}
		\draw(a2att.center) to (v);
		\draw[out=0, in=-120] (w1) to (divfst.center);
		\draw[out=0, in=120](w2) to (divsnd.center);
		\draw(divout.center) to (w);
	\end{pgfonlayer}
	\begin{pgfonlayer}{eqlayer}
		\draw[dashed, rounded corners] (-1.7, -0.8) rectangle (-0.6, 0.8);
			\draw[dashed, rounded corners] (.6, -0.3) rectangle (1.2, 0.3);
	\end{pgfonlayer}
\end{tikzpicture}
\end{center}

\end{frame}


\begin{frame}{(labelled) e-hypergraphs}
	Let $\pbe$ and $\pbe_\Sigma$ be the restriction of $\pbc$ and $\pbc_\Sigma$  to the subcategories of e-hypergraphs. 
	
	\pause 
	\begin{theorem}
	$\egg$ is $\pbe$-adhesive and $\egg_\Sigma$ is $\pbe_\Sigma$-adhesive.
	\end{theorem}
\end{frame}

\begin{frame}{The key example}

\begin{center}
\begin{tikzpicture}[baseline=(w)]
        \begin{pgfonlayer}{nodelayer}
                \node[style=small box](a1) at (-1.8, -0.5){$a$};
                \node[style=none] (a1att) at (-1.5, -0.5){};
                \node[style=node](v1) at (-0.9, -0.5){};
                \node[style=small box](a2) at (-1.8, 0.5){$a$};
                \node[style=none] (a2att) at (-1.5, 0.5){};
                \node[style=node](v2) at (-0.9, 0.5){};
                \node[style=medium box](div) at (0, 0){$/$};
                \node[style=none](divfst) at (-0.3, -0.3){};
                \node[style=none] (divsnd) at (-0.3, 0.3){};
                \node[style=none] (divout) at (0.3, 0){};
                \node[style=node] (w) at (0.9, 0){};
                
                 \node [below=.5cm, align=flush center,text width=2cm] at (v1)
                {
                	Figure $(1)$.
                };
        \end{pgfonlayer}
        \begin{pgfonlayer}{edgelayer}
                \draw(a1att.center) to (v1);
                \draw(a2att.center) to (v2);
                \draw[out=0, in=-120] (v1) to (divfst.center);
                \draw[out=0, in=120](v2) to (divsnd.center);
                \draw(divout.center) to (w);
        \end{pgfonlayer}
        \begin{pgfonlayer}{eqlayer}
        	\draw[dashed, rounded corners] (-1.2, -0.8) rectangle (-.6, -0.2);
        	\draw[dashed, rounded corners] (-1.2, 0.2) rectangle (-.6, 0.8);
        	\draw[dashed, rounded corners] (.6, -0.3) rectangle (1.2, 0.3);
        \end{pgfonlayer}
\end{tikzpicture}
\qquad \qquad 
\begin{tikzpicture}[baseline=(w)]
	\begin{pgfonlayer}{nodelayer}
		\node[style=small box](a1) at (-2.3, -0.5){$a$};
		\node[style=none] (a1att) at (-2.3, -0.5){};
		\node[style=node](v1) at (-1.4, -0.5){};
		\node[style=node](w1) at (-0.9, -0.5){};
		\node[style=small box](a2) at (-2.3, 0.5){$a$};
		\node[style=none] (a2att) at (-2.3, 0.5){};
		\node[style=node](v2) at (-1.4, 0.5){};
		\node[style=node](w2) at (-0.9, 0.5){};
		\node[style=medium box](div) at (0, 0){$/$};
		\node[style=none](divfst) at (-0.3, -0.3){};
		\node[style=none] (divsnd) at (-0.3, 0.3){};
		\node[style=none] (divout) at (0.3, 0){};
		\node[style=node] (w) at (0.9, 0){};
		
		      \node [below=.5cm, align=flush center,text width=2cm] at (-1.15,-0.5)
		{
			Figure $(2)$.
		};
	\end{pgfonlayer}
	\begin{pgfonlayer}{edgelayer}
		\draw(a1att.center) to (v1);
		\draw(a2att.center) to (v2);
		\draw[out=0, in=-120] (w1) to (divfst.center);
		\draw[out=0, in=120](w2) to (divsnd.center);
		\draw(divout.center) to (w);
	\end{pgfonlayer}
	\begin{pgfonlayer}{eqlayer}
		\draw[dashed, rounded corners] (-1.7, 0.2) rectangle (-0.6, 0.8);
		\draw[dashed, rounded corners] (-1.7, -0.8) rectangle (-0.6, -0.2);
			\draw[dashed, rounded corners] (.6, -0.3) rectangle (1.2, 0.3);
	\end{pgfonlayer}
\end{tikzpicture}
\end{center}
\begin{center}
\begin{tikzpicture}[baseline=(w)]
	\begin{pgfonlayer}{nodelayer}
		\node[style=small box](a) at (-1.8, 0){$a$};
		\node[style=none] (aatt) at (-1.5, 0){};
		\node[style=node](v) at (-0.9, 0){};
		\node[style=medium box](div) at (0, 0){$/$};
		\node[style=none](divfst) at (-0.3, -0.3){};
		\node[style=none] (divsnd) at (-0.3, 0.3){};
		\node[style=none] (divout) at (0.3, 0){};
		\node[style=node] (w) at (0.9, 0){};
		      \node [below=.5cm, align=flush center,text width=2cm] at (-0.9,-0.5)
		{
			Figure $(3)$.
		};
	\end{pgfonlayer}
	\begin{pgfonlayer}{edgelayer}
		\draw(aatt.center) to (v);
		\draw[out=-60, in=-180] (v) to (divfst.center);
		\draw[out=60, in=-180](v) to (divsnd.center);
		\draw(divout.center) to (w);
	\end{pgfonlayer}
	\begin{pgfonlayer}{eqlayer}
			\draw[dashed, rounded corners] (.6, -0.3) rectangle (1.2, 0.3);
				\draw[dashed, rounded corners] (-1.2, -0.3) rectangle (-.6, 0.3);
	\end{pgfonlayer}
\end{tikzpicture}
\qquad \qquad 
\begin{tikzpicture}[baseline=(w)]
	\begin{pgfonlayer}{nodelayer}
		\node[style=node](w1) at (-0.9, -0.5){};
		\node[style=small box](a2) at (-2.3, 0){$a$};
		\node[style=none] (a2att) at (-2.3, 0){};
		\node[style=node](v) at (-1.4, 0){};
		\node[style=node](w2) at (-0.9, 0.5){};
		\node[style=medium box](div) at (0, 0){$/$};
		\node[style=none](divfst) at (-0.3, -0.3){};
		\node[style=none] (divsnd) at (-0.3, 0.3){};
		\node[style=none] (divout) at (0.3, 0){};
		\node[style=node] (w) at (0.9, 0){};
		
		      \node [below=.5cm, align=flush center,text width=2cm] at (-1.15, -0.5)
		{
			Figure $(4)$.
		};
	\end{pgfonlayer}
	\begin{pgfonlayer}{edgelayer}
		\draw(a2att.center) to (v);
		\draw[out=0, in=-120] (w1) to (divfst.center);
		\draw[out=0, in=120](w2) to (divsnd.center);
		\draw(divout.center) to (w);
	\end{pgfonlayer}
	\begin{pgfonlayer}{eqlayer}
		\draw[dashed, rounded corners] (-1.7, -0.8) rectangle (-0.6, 0.8);
			\draw[dashed, rounded corners] (.6, -0.3) rectangle (1.2, 0.3);
	\end{pgfonlayer}
\end{tikzpicture}
\end{center}

\end{frame}

\begin{frame}{Rewriting}
\[
	\hspace{.25cm}
	\xymatrix{        
		\begin{tikzpicture}[baseline=(w.base)]\begin{pgfonlayer}{nodelayer}
				\node[style=node](v1) at (-0.9, 0.3){};
				\node[style=node](v2) at (-0.9, -0.3){};
				\node[style=medium box] at (0, 0){$/$};
				\node[style=none](divfst) at (-0.3, 0.3){};
				\node[style=none](divsnd) at (-0.3, -0.3){};
				\node[style=none](divout) at (0.3, 0){};
				\node[style=node] (w) at (0.9, 0){};
			\end{pgfonlayer}        
			\begin{pgfonlayer}{edgelayer}
				\draw(v1) to (divfst.center);
				\draw(v2) to (divsnd.center);
				\draw(divout.center) to (w);
			\end{pgfonlayer}
			\begin{pgfonlayer}{eqlayer}
				\draw[dashed, rounded corners](-1.2, 0.6) rectangle (-0.6, -0.6);
					\draw[dashed, rounded corners] (.6, -0.3) rectangle (1.2, 0.3);
			\end{pgfonlayer}\begin{pgfonlayer}{background}
				\draw[color=white] (-1.5, 0.2) rectangle (1.5, -0.2);
			\end{pgfonlayer}
		\end{tikzpicture}
		\ar@{=>}[r] &
		\begin{tikzpicture}[baseline=(nil.center)]\begin{pgfonlayer}{nodelayer}
				\node[style=node](v1) at (-0.9, 0.3){};
				\node[style=node](v2) at (-0.9, -0.3){};
				\node[style=medium box] at (0, 0){$/$};
				\node[style=none](divfst) at (-0.3, 0.3){};
				\node[style=none](divsnd) at (-0.3, -0.3){};
				\node[style=none](divout) at (0.3, 0){};
				\node[style=node] (w) at (0.9, 0){};
				\node[style=small box] at (0, -1){$1$};
				\node[style=none](one) at (0.3, -1){};
				\node[style=node](z) at (0.9, -1){};
				\node[style=none](nil) at (0, -0.5){};
			\end{pgfonlayer}        
			\begin{pgfonlayer}{edgelayer}
				\draw(v1) to (divfst.center);
				\draw(v2) to (divsnd.center);
				\draw(divout.center) to (w);
				\draw(one.center) to (z);
			\end{pgfonlayer}
			\begin{pgfonlayer}{eqlayer}
				\draw[dashed, rounded corners](-1.2, 0.6) rectangle (-0.6, -0.6);
				\draw[dashed, rounded corners](0.6, 0.3) rectangle (1.2, -1.3);
			\end{pgfonlayer}\begin{pgfonlayer}{background}
				\draw[color=white] (-1.5, 0.2) rectangle (1.5, -0.2);
		\end{pgfonlayer}
\end{tikzpicture}
	}
\]
\begin{center}
                	The rule $x / x \Rightarrow 1$, graphically.
\end{center}
\[
	\pause
	\begin{tikzpicture}[baseline=(w)]
		\begin{pgfonlayer}{nodelayer}
			\node[style=small box](a) at (-1.8, 0){$a$};
			\node[style=none] (aatt) at (-1.5, 0){};
			\node[style=node](v) at (-0.9, 0){};
			\node[style=medium box](div) at (0, 0){$/$};
			\node[style=none](divfst) at (-0.3, -0.3){};
			\node[style=none] (divsnd) at (-0.3, 0.3){};
			\node[style=none] (divout) at (0.3, 0){};
			\node[style=node] (w) at (0.9, 0){};
				\node[style=small box] at (0, -1){$1$};
			\node[style=none](one) at (0.3, -1){};
			\node[style=node](z) at (0.9, -1){};
		\end{pgfonlayer}
		\begin{pgfonlayer}{edgelayer}
			\draw(aatt.center) to (v);
			\draw[out=-60, in=-180] (v) to (divfst.center);
			\draw[out=60, in=-180](v) to (divsnd.center);
			\draw(divout.center) to (w);
			\draw(one.center) to (z);
		\end{pgfonlayer}
		\begin{pgfonlayer}{eqlayer}
			\draw[dashed, rounded corners](0.6, 0.3) rectangle (1.2, -1.3);
				\draw[dashed, rounded corners] (-1.2, -0.3) rectangle (-.6, 0.3);
		\end{pgfonlayer}
	\end{tikzpicture}
	\qquad \qquad 
	\pause
	\begin{tikzpicture}[baseline=(w)]
		\begin{pgfonlayer}{nodelayer}
			\node[style=node](w1) at (-0.9, -0.5){};
			\node[style=small box](a2) at (-2.3, 0){$a$};
			\node[style=none] (a2att) at (-2.3, 0){};
			\node[style=node](v) at (-1.4, 0){};
			\node[style=node](w2) at (-0.9, 0.5){};
			\node[style=medium box](div) at (0, 0){$/$};
			\node[style=none](divfst) at (-0.3, -0.3){};
			\node[style=none] (divsnd) at (-0.3, 0.3){};
			\node[style=none] (divout) at (0.3, 0){};
			\node[style=node] (w) at (0.9, 0){};
			\node[style=small box] at (0, -1){$1$};
			\node[style=none](one) at (0.3, -1){};
			\node[style=node](z) at (0.9, -1){};
		\end{pgfonlayer}
		\begin{pgfonlayer}{edgelayer}
			\draw(a2att.center) to (v);
			\draw[out=0, in=-120] (w1) to (divfst.center);
			\draw[out=0, in=120](w2) to (divsnd.center);
			\draw(divout.center) to (w);
			\draw(one.center) to (z);
		\end{pgfonlayer}
		\begin{pgfonlayer}{eqlayer}
			\draw[dashed, rounded corners] (-1.7, -0.8) rectangle (-0.6, 0.8);
			\draw[dashed, rounded corners](0.6, 0.3) rectangle (1.2, -1.3);
		\end{pgfonlayer}
	\end{tikzpicture}
\]
\end{frame}

\begin{frame}{Cycling (with collapsing rules)}
\[
\hspace{.25cm}
\xymatrix{        
	\begin{tikzpicture}[baseline=(w.base)]\begin{pgfonlayer}{nodelayer}
			\node[style=node](v1) at (-0.9, 0.4){};
			\node[style=node](v2) at (-0.9, -0.4){};
			\node[style=medium box] at (0, 0){$\ast$};
			\node[style=none](divfst) at (-0.3, 0.3){};
			\node[style=none](divsnd) at (-0.3, -0.3){};
			\node[style=none](divout) at (0.3, 0){};
			\node[style=node] (w) at (0.9, 0){};
			\node[style=small box] at (-2.3, 0.3){$1$};
			\node[style=none](one) at (-2, 0.3){};
			\node[style=node](z) at (-1.4, 0.4){};
		\end{pgfonlayer}        
		\begin{pgfonlayer}{edgelayer}
			\draw(v1) to (divfst.center);
			\draw(v2) to (divsnd.center);
			\draw(divout.center) to (w);
			\draw(one.center) to (z);
		\end{pgfonlayer}
		\begin{pgfonlayer}{eqlayer}
			\draw[dashed, rounded corners](-1.7, 0.1) rectangle (-0.6, 0.7);
			\draw[dashed, rounded corners](1.2, -0.3) rectangle (0.6, 0.3);
			\draw[dashed, rounded corners] (-1.2, -0.7) rectangle (-.6, -0.1);
		\end{pgfonlayer}\begin{pgfonlayer}{background}
			\draw[color=white] (-1.5, 0.1) rectangle (1.5, -0.3);
		\end{pgfonlayer}
	\end{tikzpicture}
	\ar@{=>}[r] &	\hspace{12pt}
	\begin{tikzpicture}[baseline=(w.base)]\begin{pgfonlayer}{nodelayer}
			\node[style=node](v1) at (-0.9, 0.3){};
			\node[style=node](v2) at (-0.9, -0.3){};
			\node[style=medium box] at (0, 0){$\ast$};
			\node[style=none](divfst) at (-0.3, 0.3){};
			\node[style=none](divsnd) at (-0.3, -0.3){};
			\node[style=none](divout) at (0.3, 0){};
			\node[style=node] (w) at (0.9, 0){};
			\node[style=small box] at (-2.3, 0.3){$1$};
			\node[style=none](one) at (-2, 0.3){};
			\node[style=node](z) at (-1.4, 0.3){};
		\end{pgfonlayer}        
		\begin{pgfonlayer}{edgelayer}
			\draw(v1) to (divfst.center);
			\draw(v2) to (divsnd.center);
			\draw(divout.center) to (w);
			\draw(one.center) to (z);
		\end{pgfonlayer}
		\begin{pgfonlayer}{eqlayer}
			\draw[dashed, rounded corners](-1.7, 0.6) rectangle (-0.6, -0);
		\end{pgfonlayer}
		\begin{pgfonlayer}{background}
			\draw[color=white] (-1.5, 0.2) rectangle (1.5, -0.2);
		\end{pgfonlayer}
		\draw[dashed,rounded corners]
		([shift={(0,0.2)}]v2.north) --
		([shift={(-0.2,0)}]v2.west) --
		(-.9,-1.2)--(.9,-1.2)--([shift={(0.2,0)}]w.east)--
		([shift={(0,0.2)}]w.north)  -- ([shift={(-0.2,0)}]w.west)-- (0.7,-0.8)--(-0.7,-0.8)--([shift={(0.2,0)}]v2.east)-- cycle;
	\end{tikzpicture}
}
\]
\begin{center}
               	The rule $1 \ast x \Rightarrow x$, graphically.
\end{center}

\end{frame}

\begin{frame}{The whole picture}

Graphical structures, aplenty \ldots

\[\xymatrix@R=8pt@C=10pt{
& \pshyp_\mathcal{G} \ar[dd]|\hole \ar@/_.2cm/@{_{(}->}[rr] \ar@{_{(}->}[dl]
  &
    & \phyp_\mathcal{G} \ar[dd]|\hole \ar@/_.2cm/[ll]\ar@/_.2cm/@{_{(}->}[dr] \\
\shyp_\mathcal{G} \ar@{^{(}->}[rr] \ar[dd]&& \lmo_\mathcal{G}  \ar@{^{(}->}[rr] \ar[dd]
 & 
   & \hyp_\mathcal{G} \ar@/_.2cm/[ul] \ar[dd] \\
 & \pshyp \ar@/_.2cm/@{_{(}->}[rr]|\hole \ar@{^{(}->}[dl] 
   &
     & \phyp \ar@/_.2cm/[ll]|\hole\ar@/_.2cm/@{_{(}->}[dr] \\
\shyp \ar@{_{(}->}[rr] && \lmo  \ar@{_{(}->}[rr]&& \hyp \ar@/_.2cm/[ul]}
\]

\pause \medskip
$\lmo$: left-monogamous hypergraphs (ie term graphs)

$\shyp$: also right-monogamous (ie disconnected/separated graphs)

$\pshyp$ also pruned (ie no isolated nodes)

\end{frame}

\section{Conclusions}
\begin{frame}{Future works and questions.}

Exploit features of the DPO paradigm (ie~parallelism, causality) for EGGs update.\pause

\medskip How far can we move way from $\catname{Set}$?\pause 
\begin{itemize}
	\item Edges and nodes taken from a different category (hypergraphs are presheaves\ldots)
		 \pause 
	\item Combine categorical constructions to add features to EGGs (ie hierarchy of edges) %while retaining adhesivity.
\end{itemize}
\end{frame}



\begin{noheadlineframe}
	\huge
	Thank you for your attention!
\end{noheadlineframe}

\appendix 
\begin{noheadlineframe}[allowframebreaks]
	\frametitle{Bibliography}
	\printbibliography		
\end{noheadlineframe}



    
\end{document}

