\documentclass[english,11pt,a4paper]{article}
\usepackage[T1]{fontenc}
\usepackage{babel}
\usepackage{amsfonts}
\usepackage{cleveref}
\usepackage{fullpage}

\title{``EGGs are adhesive!''\\
	Answer to reviewers}
\author{R. Biondo, D. Castelnovo, F. Gadducci}


\begin{document}
	\maketitle
	
	
We wish to thank the reviewers for the time they have dedicated to our paper, and for the many useful suggestions. We corrected all typos indicated; below we explain how we have addressed the issues raised by the reviewers.

\section*{Reviewer 1}
\subsection*{Comments}
\begin{itemize}
\item  $91$:  Put this in a definition + add the definition of ``closed under decomposition'', since this is mentioned line $122$.

ANSWER: Since this notions are well known and we are recalling them only for the sake of completeness, we prefer to keep the present style, without using a definition box. We added the definition of closure under decomposition.

\item $108$: Add to the list that it is well-behaved w.r.t comma categories and certain subcategories.

ANSWER: We added the reference to subcategories. Since comma categories are slices that case is already covered.

\item $109$: It would be helpful to include precise theorem numbers when referring to [14,24], as this would make it easier for readers to verify the claims without unnecessary effort. In this case, the first item of Proposition 2.4 seem to correspond to [14, Theorem 4.15], which speaks of (weak) adhesive HLR categories. If that's your reference, you should add how this notion relates to M-adhesivity, as done in Remark 2.3 for adhesivity and quasiadhesivity.

ANSWER: Added the references to the relevant theorems or proposition. and the comparison with [14].

\item $112$: R:X -> A should be preserve *M*-pullbacks, shouldn't it? 

ANSWER: fixed.

\item $113$ and $116$: add diagrams if space permits.

ANSWER: we think that adding diagrams at lines 113 and 116 would not  significantly enhance clarity or understanding.

\item $129$: Why refer to strict Reg(sGraphs)-adhesive as quasiadhesive? Seems strange to use two naming conventions.

ANSWER: it is conventional to use the term quasiadhesive for categories which are strictly adhesive with respect to the class of regular monos. Other proposals have been made (e.g.~ in [18]) but we prefer to stick with the more used convention.

\item $159$: Why mention here that a proof of Lemma 2.14 can be found in the appendix, and not for Lemma 2.12 or Proposition 2.13? Moreover, it is already mentioned in the synopsis that missing proofs would be in the appendix.

ANSWER: there is a pointer to the appenix befor Lemma 2.12. Such pointer (and all the others) are there to provide the reader with a quick link to the correct proof in the appendix.

\item $160$: Surprising that the bottom is now supposed to be a pullback, instead of a pushout as in Proposition 2.13. Can any intuition be given for this change in the assumptions?

ANSWER: In Prop.~2.13 it is not assumed that the bottom face is a pushout, eve if this property can be deduced from $\mathcal{M}$-adhesivity. Notice, however that to Prop.~2.13 it should be enough to ask for stability of $\mathcal{M}$-pushouts: in such a case the bottom face of the cube is not guaranteed to be a pushout (we do not need such level of generality here). 

On the other hand, we need a result allowing us to present the kernel pair of $d$ as a pushout even without the hypothesis that the back face is a pullback. This is the content of Lem.~$2.14$ and the proof relies on some peculiar properties of the category of sets. We suspect that the proof carry on in any extensive and distributive category but, again, we do not need such a level of generality here.

Notice, by the way, that distributivity and extensivity are exactly the properties needed to build a Kleene star functor satisfying Prop.~3.1.

\item $167$: Give the type of f$^\star$.

ANSWER: Fixed.


\item $171$: Say explicitly that $\mathsf{lg}_X$ = $(!_X)^\star$.

ANSWER: this would be slightly imprecise since $1^\star$ is not $\mathbb{N}$, but only canonically isomorphic to it.

\item $162$ and $173$: Maybe mention that it is the definition of hypergraph **that you will work with in this paper**. There are multiple existing definitions.

ANSWER: while there exists multiple definitions of hypergraphs, it seems to us that this is well known and widely used, at least in the field of algebraic models of graph rewriting. We believe that it would be redundant to specify that this definition is the one used in the paper in which it appears.

\item $175$: Why call $V_G$ the set of nodes? Maybe denote it $N_G$ or call it the set of vertices, to be consistent.

ANSWER: we do not believe that this issue is crucial to the readability of understandability of the paper.

\item $176$: Please give the type of $\mathsf{prod}^\star$. Is it $\mathbf{Set} \to \mathbf{Set}$?. Also, split this sentence into two sentences.

ANSWER: Fixed


\item $180$: Corollary 3.7 basically says that $\mathsf{Mono}(\mathbf{Set}) \downarrow \mathsf{Mono}(\mathbf{Set}) = \mathsf{Mono}(\mathbf{Hyp})$, which is part of the (unwritten) proof of Corollary 3.6, yet this is not mentioned. The order of the results is not logical. Moreover, the fact that Corollary 3.7 follows from Corollary B.5 is not mentioned.

ANSWER: Fixed.

\item $180$: Was this result not previously known in the literature?

ANSWER: Indeed it was, we added a pointer to [14].

\item $194$: Did you forget to draw $v_5$ or is it on purpose?

ANSWER: It's a typo, $v_5$ is not needed.

\item $198$: Why not more precisely say presheaf categories instead of functor categories?

ANSWER: we changed the text.

\item $208$: The target function $\gamma_1$ is supposed to output a list in $1^\star$. Does it return the one element list [1]?

ANSWER: we removed the brackets since they were useless. Since we are identifying $1^\star$ with $\mathbb{N}$, $\gamma_1$ as codomain $\mathbb{N}$is precisely the element $1\in \mathbb{N}$.

\item $214$: Why have the two diagram squares been oriented differently that the ones in Definition 3.4? Moreover, it should be twice the set $V_\mathcal{H}^*$ in the bottom left corners, and it should be $\gamma_1$ for the right morphism in the bottom square.

ANSWER: Fixed

 \item $214$: It seems to me that the second diagram can first be explained in much simpler terms, such as "in labelled hypergraphs, all hyperedges must have a single target vertex", and then it could be restated as a factorisation result for $t_\mathcal{H}$. The current explanation is not clear enough.
 
 ANSWER: Fixed.
 
 \item $215$: What is $U_{\mathbf{X}}$? Is it $U_{\mathbf{Set}}$? Or maybe $U_{\mathbf{Hyp}}$ that gives the set of vertices? There has been a lot of forgetful functor introduced so far, and here is one more being defined, so it would help to have a description in words added to this sentence.
 
 ANSWER: $U_{\mathbf{X}}$ should be $U_{\mathbf{Hyp}}$. Fixed.
 
 \item $216$: What is $\Delta_{\mathbf{X}}$? Only $\Delta_{\mathbf{Hyp}}$ has been introduced so far, and from my understanding $\mathbf{X}$ (in bold) is a typo for $\mathbf{Set}$ or $\mathbf{Hyp}$. Please draw the diagram for $\Delta_\Sigma(X) = (?,!)$ to visualise and understand what it represents, and give additional explanations in words. 

ANSWER: same as above. $\Delta_\Sigma(X)$ is the discrete hypergraph on $X$ (the one with no edges) labeled in the unique possible way. In our opinion drawing a diagram of it would not enhance the clarity of the paragraph.

\item $217$: What is the definition of the functor $\Delta_\Sigma$ on morphisms? The proof of Proposition 3.17 is not straightforward and would be welcomed to be added.

ANSWER: The proof is omitted for lack of space and since it is neither particularly informative nor useful. Indeed, given $(h,k)\colon \mathcal{H}\to \mathcal{G}^\Sigma$,  a morphism $X\to U_{\Sigma}(h,k)$ is just an arrow from $X$ to the set of nodes of $\mathcal{H}$. Since the set of hyperedges of $\Delta_{\Sigma}(X)$ is empty, it can be extended in a unique way to a morphism of hypergraph which, moreover, respect the labeling (because on $\Delta_{\Sigma}(X)$ there is nothing to label).

\item $224-225$: What do you mean by "identifier"? Do you mean "label"? And what does "the latter" refer to? 

ANSWER: changed with "labels"

\item $233$: Mention that the objects of $\mathbf{TG}_\Sigma$ are precisely the term graphs.

ANSWER: Fixed.

\item $234-237$: Please add somewhere an example of a term graph.

ANSWER: added

\item $236$: Since every labelled hyperedge has a unique vertex as target, being a term graph simply means each hyperedge having a distinct target. This could have been said more explicitly.

ANSWER: That's true but it seems to us that this formulation is neither clearer nor more informative that ours.

\item $238$: What is the left adjoint $\Delta_{\mathbf{TG}_\Sigma}$? Is it equal to $\Delta_{\Sigma}$? If that's the case, it's worth it to say in one sentence that there is nothing to be checked in order to prove that the resulting labelled hypergraph is a term graph, as there are no hyperedges.

ANSWER: Added.

\item $247$: A little more explanation is required. Are you doing a proof by contradiction supposing a terminal object $1$ (with at most one hyperedge) and then showing that two hypergraphs $\mathcal{G}_a$ and $\mathcal{G}_b$, with one hyperedge each labeled $a$ and $b$ respectively, then the unique morphisms $\mathcal{G}_a \to 1$ and $\mathcal{G}_b \to 1$ lead to a contradiction about the label of the hyperedge in 1 (in the case 1 has an hyperedge, and with another argument in case $1$ has no hyperedge)? 

ANSWER: if $1$ has no hyperedge than no argument is needed since $\mathcal{G}_a$ does not have arrows into it. If $1$ has one hyperedge than the existence of arrows $\mathcal{G}_a\to 1$ and $\mathcal{G}_b\to 1$ entails that $a=b$, which is a contradiction. Alll the details are in the cited reference [10].

\item $259$: The wording is confused: "then its pushout along (f2,g2), then their pushout".

ANSWER: Fixed.

\item $263$: How does Corollary 3.27 follow from Proposition 2.4, Proposition 3.25, and Lemmma 3.26? This needs to be fleshed out. Does it use the fact that $\mathbf{Hyp}$ is adhesive (Corollary 3.6) and the subcategory property of Proposition 2.4?

ANSWER: $\mathbf{Hyp}_\Sigma$ is adhesive, $\mathbf{TG}_\Sigma$ is closed in it under pullbacks and pushout along monos which preserve input nodes. Since these are the regular monos Prop.~2.4(4) entails the thesis. We add (another) pointer to [10] for details.


\item $273$: Why is the notation $h_C$ and not $h_Q$?
ANSWER: Fixed.

\item $285$: Do you mean to say that proofs (pural) of Section 4 can be found in Appendix A.2?

ANSWER: This pointer refers to the proof of Prop.~4.4. There is another pointer some lines below for the proof of Lem,~4.8.

\item $293$: Please add a proof of Corollary 4.6, even if it's only a short one.

ANSWER: We would rather not. As stated Cor. 4.6 is an immediate consequence of Prop.~4.4 because a faithful right adjoint preserves and reflects monos. We have eliminated the reference to Rem.~4.3 as it was misleading.


\item $297$: It wouldn't hurt to have the diagram showing the two adjunctions in arrow.

ANSWER: in our opinion it would not enhance the clarity of the paper.

\item $301$: In item 2, the sentence wording uses once again the confusing structure of ending with a definition. Please restructure the sentence.

ANSWER: we do not find the original sentence confusing at all. We nonetheless rephrased it.

\item $301$: Why draw a different diagram here than on line 753?

ANSWER: on line 753 there the same exact diagram, just turned by 90 degree for typographical reasons. 

\item $306$: Why is this square not in the same orientation as the one on line 277?

ANSWER: Fixed, even if this change does not affect in the slightest way the readability of the paper. 

\item $307$: Be more precise about the sense in which you mean ``suitable", i.e., as the suitable class of monos $\mathcal{M}$ for which $\mathcal{M}$-adhesivity holds.

ANSWER: we find the meaning completely clear from the context. Moreover, the meaning of ``suitable'' is made explicit by the content of Lem.~4.11.

\item $321$: You could add a reference to Definition 3.13 for $\mathcal{G}^\Sigma$.

ANSWER: we find no need to do so: Def.~3.13 is just three pages before.


\item $325$: You could mention that $EqHyp_\Sigma$ can be seen as the comma category $T \downarrow$ ``Constant functor at $\mathcal{G}^\Sigma$". This would make the parallel with the definition of the category of labelled hypergraphs that was defined as a slice category. Moreover, you could then give a reference for the proof of Proposition 4.17, such as [MacLane Categories for the working mathematician, IV.1, Exercise 2, p86].

ANSWER: We rather not: what the reviewer asked us to mention is exactly the content of Prop.~4.17. In turn, we do not see any need to quote the exercise of MacLane's book since the proof Prop.~4.17 it's a complete triviality.

\item $327$: ``to the arrow" -> ``to an arrow"? And why is this arrow necessarily of the form (l,!)? Please provide more details

ANSWER:  no: as mentioned, $R$  is the right adjoint to $T$ so $l\colon T(\mathcal{H})\to \mathcal{G}^\Sigma$ corresponds to a unique arrow $\mathcal{H}\to R(\mathcal{G}_\Sigma)$. That this mate is exactly $(l, !_{V_{\mathcal{H}}})$ follows from the proof of Prop.~4.4 but it's also commpletely obvious given the definition of $\mathcal{G}^\Sigma$.


\item $332$: Please provide details on how invoking those Proposition/Lemma/Corollary really prove Proposition 4.18.

ANSWER: we do not think that these details should be provided because they are already contained in the  Proposition, Lemma and Corollaries quoted.  added a reference to Cor.~B.11 in which the details are spelled out even more explicitly.  Point $1$ follows immediately from Cor.~B.11 and Lem.~4.8 and point $2$ is simply the closure of adhesivity properties under the slice construction.


\item$343-344$: Please provide an example of a term graph with equivalence.

ANSWER: we do not think that there is a need to provide an explicit example. If the reviewer needs one, it can equip the term graph of Ex.~3.21 with the equality

\item $344$: ``Thus ... right": You meant to say that $S_\Sigma$ makes the diagram on the right commute, right? Moreover, please describe what $S_\Sigma$ does explicitly, and why it would make the diagram commute.

ANSWER: the action of $S_\Sigma$ is already in the text: it sends $(\mathcal{H},l)$ to $(T(\mathcal{H}),l)$. 

\item $345$: Please provide a proof of this claim.
ANSWER: we have already provided the proof citing Cor.~4.5 and Prop.~4.18.

\item $351$: Please turn this observation into a lemma and provide a detailed proof of it.

ANSWER: we added a reference to Cor.~4.19  and we believe that this is all that's needed. 


\item $367$: What does it mean concretely to ask for $q^\star \circ t\circ \pi_1= q^\star \circ t \circ \pi_2$? Please add a connection between the intuitive explanation given in lines 364-365 and the mathematical equation given here. 

ANSWER: we do not think that a further explanation is needed because the equation $q^\star \circ t\circ \pi_1= q^\star \circ t \circ \pi_2$ is \emph{exactly} the symbolic translation of ``whenever the relation identifies the source of two hyperedges, it identifies their targets too''.

\item $367$: Also, add a diagram corresponding to the equality, in order to visualise the types of each morphism.

ANSWER: this would be unnecessary and space consuming.

\item $368$: Why does going from ``hypergraph with equivalence" to ``e-hypergraph" result in the category names going ``EqHyp" to ``GEqHyp"? What does the ``G" mean?

ANSWER: notation changed to $\mathsf{e}-\mathbf{EqHyp}$


\item $369-370$: Please add an example of an e-hypergraph.
ANSWER: added an example


\item $374$: "is so" -> "is a regular mono"

ANSWER: fixed.

\item $383$: Does the use of the word ``context" in this sentence refer to its common meaning in graph rewriting?

ANSWER: no.


\item $391$: Shouldn't $\mathsf{id}_\mathbb{N}$ be $\mathsf{id}^\star_1$ instead? 

ANSWER: no, see the answers above about the isomorphism between $\mathbb{N}$ and $1^\star$.

\item $405$: Why the name ``EGG"? Why is this short for ``e-term graph"?

ANSWER: indeed the name ``EGG'', which comes from the already existing literature, is short for ``e-term graph'' or for ``e-graph''.

\item $412$: A detailed explanation and example of what lies precisely in the class $\mathcal{T}_\Sigma$ is necessary to appreciate the next adhesivity result, and figure out its potential applications.

ANSWER: we disagree on the fact that such an explanation is necessary: it is simply the restriction of $\mathcal{T}$ to the subcategory $\mathbf{EGG}$ and $\mathcal{T}$ has been already descripted in elementary terms.
 \end{itemize}
\subsection*{Bibliography}
\begin{itemize}
	\item 
\end{itemize}

\subsection*{Sources of theorems and definitions}
\begin{itemize}
	\item $109$: [14,24] (as mentioned earlier)
	
	ANSWER: added a pointer to the relevant results.
\end{itemize}


\section*{Reviewer 2}



	
	
\end{document}