\documentclass[a4paper,UKenglish,cleveref,pdftex,thm-restate,numberwithinsect]{lipics-v2021}
\usepackage{stmaryrd}
\usepackage{bbding}
%\Envelope
\newcommand{\cat}[1]{\ensuremath{\mathbf{#1}}}
\usepackage{amsthm,amsmath,amssymb,mathrsfs, dsfont}
\usepackage{hyperref, cleveref} %simboli matematici
\usepackage[all, cmtip]{xy}

\usepackage[utf8]{inputenc} % Input encoding - per caratteri particolari
%\usepackage[english]{babel} % Lingua principale inglese
\usepackage{graphicx} % Per includere immagini esterne
\usepackage[tickmarkheight=.5em,textwidth=\marginparwidth,textsize=small]{todonotes}
\usepackage{mathtools}
\usepackage{csquotes}


\usepackage{tikz-cd}
\usepackage{tikz}
\usetikzlibrary{decorations.markings}

% General Setting diagrams
%\usepackage{tikz-cd} %diagrammi
%\tikzcdset{row sep/normal=5em}
%\tikzcdset{column sep/normal=5em}
%\tikzcdset{every label/.append style = {font = \small}}

%\spnewtheorem*{notation}{Notation}{\bfseries}{}
%\spnewtheorem*{convention}{Convention}{\bfseries}{}



%funtori
\usepackage[usestackEOL]{stackengine}
\newcommand\functorop[1][l]{\csname#1functor\endcsname}
\newcommand\lfunctorop[3]{%
	\setbox0=\hbox{$#2$}%
	\kern\wd0%
	\ensurestackMath{\Centerstack[c]{#1\\ \mathllap{#2\;\,}\mathclap{\DownArrow}\\#3}}%
}		
\newcommand\rfunctorop[3]{%
	\setbox0=\hbox{$#2$}%
	\ensurestackMath{\Centerstack[c]{#1\\\mathclap{\UpArrow}\mathrlap{\,\;#2}\\#3}}%
	\kern\wd0%
}
\newcommand\functoropmapsto{\mathrel{\ensurestackMath{\Centerstack[c]{\longmapsto\\ \\\longmapsto}}}}
\setstackgap{L}{1.3\normalbaselineskip}
\newcommand\UpArrow{\rotatebox[origin=c]{90}{$\longrightarrow$\,}}
\newcommand\DownArrow{\rotatebox[origin=c]{-90}{$\longrightarrow$\,}}
\newcommand\functor[1][l]{\csname#1functor\endcsname}
\newcommand\lfunctor[3]{%
	\setbox0=\hbox{$#2$}%
	\kern\wd0%
	\ensurestackMath{\Centerstack[c]{#1\\ \mathllap{#2\;\,}\mathclap{\DownArrow}\\#3}}%
}
\newcommand\rfunctor[3]{%
	\setbox0=\hbox{$#2$}%
	\ensurestackMath{\Centerstack[c]{#1\\\mathclap{\DownArrow}\mathrlap{\,\;#2}\\#3}}%
	\kern\wd0%
}
\newcommand\functormapsto{\mathrel{\ensurestackMath{\Centerstack[c]{\longmapsto\\ \\\longmapsto}}}}
\setstackgap{L}{1.3\normalbaselineskip}

\newcommand{\lgh}{\mathsf{lg}}

\DeclareMathAlphabet{\mymathbb}{U}{BOONDOX-ds}{m}{n}

\newcommand{\Ob}{\mathcal{O}b}
\newcommand{\Hom}{\mathcal{H}om}
\newcommand{\Set}{\mathbf{Set}}
\newcommand{\Reg}{\mathcal{Reg}}
\newcommand{\Mono}{\mathcal{Mono}}
\newcommand{\initial}{\mymathbb{0}}
\newcommand{\terminal}{\mathds{1}}
\newcommand{\eg}[1]{\mathbf{EqGraph}_{\textbf {\textup{#1}}}}
%\newcommand{\egg}[1]{\mathbf{EGG}_{\textbf {\textup{#1}}}}

\makeatletter
\def\@citecolor{blue}%
\def\@urlcolor{blue}%
\def\@linkcolor{blue}%
\def\UrlFont{\rmfamily}
\def\orcidID#1{\smash{\href{http://orcid.org/#1}{\protect\raisebox{-1.25pt}{\protect\includegraphics{orcid_color.eps}}}}}
\makeatother



\def\R{\mathsf{R}}
\def\B{\textbf {\textup{B}}}
\def\C{\textbf {\textup{C}}}
\def\D{\textbf {\textup{D}}}
\def\X{\textbf {\textup{X}}}
\def\Y{\textbf {\textup{Y}}}
\def\E{\textbf {\textup{E}}}
\def\T{\textbf {\textup{1}}}
\def\A{\textbf {\textup{A}}}
\def\M{\mathcal{M}}
%categorie varie
\newcommand{\catname}[1]{\textbf{\textup{#1}}}
\newcommand{\lab}{\catname{LHyp}}
\newcommand{\hyp}{\catname{Hyp}}
\newcommand{\hyps}{\catname{Hyp}_{\Sigma}}
\newcommand{\EqHyp}{\catname{EqHyp}} %equivalence hypergraphs
\newcommand{\EqHyps}{\catname{EqHyp}_{\Sigma}}
\newcommand{\EqTG}{\catname{EqTG}}
\newcommand{\EqTGs}{\catname{EqTG}_{\Sigma}}
\newcommand{\gr}{\textbf{\textup{Graph}}}
\newcommand{\dgr}{\catname{SGraph}}
\newcommand{\dg}{\catname{DAG}}
\newcommand{\rt}{\mathsf{dcl_s}}
\newcommand{\rta}{\mathsf{dcl}}
\newcommand{\rtd}{\mathsf{dcl_{d}}}
\newcommand{\slice}[2]{(\catname{#1}\downarrow{#2})}
\newcommand{\tg}[0]{\catname{TG}_{\Sigma}}
\newcommand{\teg}[0]{\catname{TeGr}_{\Sigma}}
\newcommand{\sv}[0]{\mathsf{Sieves}}
\newcommand{\mono}[1]{\mathsf{Mon}(\catname{#1})}
\newcommand{\mo}[1]{{#1}_\mathsf{Mon}}
\newcommand{\pro}{\mathsf{prod}}
\newcommand{\spro}{\mathsf{ps}}
\newcommand{\prol}{\mathsf{lprod}}
\newcommand{\pred}[1]{{\downarrow}#1}
\newcommand{\colim}[0]{\mathrm{colim}}
\newcommand{\cod}{\mathsf{cod}}
\renewcommand{\sp}{\mathsf{sp}}
\renewcommand{\sup}{\mathsf{sup}}
\newcommand{\cow}[1]{\mathsf{cwd}({#1})}
\renewcommand{\inf}{\mathsf{inf}}
\newcommand{\dom}{\mathsf{dom}}
\newcommand{\dwnarrow}{\downarrow \hspace{-2pt}}
\newcommand{\Dwnarrow}{\Downarrow \hspace{-2pt}}
\newcommand{\egg}{\catname{EGG}}

\newcommand{\ari}{\mathsf{ar}}
\newcommand{\abs}[1]{\lvert #1\rvert}


\newcommand{\upstr}[1] { {#1}^{\uparrow }}
\newcommand{\comma}[2]{#1\hspace{1pt} {\downarrow}#2}
\newcommand{\cma}[2]{\mathcal{#1}\hspace{1pt} {\downarrow}\hspace{1pt} \mathcal{#2}}

\newcommand{\commentato}[1]{ {} }

%sommatoria e prodotto per xy

\usepackage{relsize}
\newcommand{\Sum}{\mathlarger{\sum}}
\newcommand{\Prod}{\mathlarger{\prod}}

\bibliographystyle{abbrv}

%frecce
\newcommand{\mor}{\mathsf{Mor}}
\newcommand{\mon}{\mathsf{Mono}}
\newcommand{\reg}{\mathsf{Reg}}
\newcommand{\mto}{\rightarrowtail}
\newcommand{\eto}{\twoheadrightarrow}
\newcommand{\id}[1]{\mathsf{id}_{#1}}



\title{On the adhesivity of EGGS}

%\titlerunning{????} %TODO optional, please use if title is longer than one line

%% Author with single affiliation.
%\author{Roberto Biondo \inst{1}{\small\Envelope}\and Davide Castelnovo\inst{1}\orcidID{0000-0002-5926-5615}
	%\and \\Fabio Gadducci\inst{1}\orcidID{0000-0003-0690-3051}
%}
%\institute{Dept.~of Computer Science, University of Pisa, Italy. 
	%\email{r.biondo@studenti.unipi.it},
	%\email{castelnovod@gmail.com},
	% \email{fabio.gadducci@unipi.it}
%}






\authorrunning{R.~Biondo, D.~Castelnovo, F.~Gadducci}


\begin{document}


\tikzset{->-/.style={decoration={
			markings,
			mark=at position #1 with {\arrow{>}}},postaction={decorate}}}

\maketitle 

\begin{abstract}
\todo{a very nice abstract}
\end{abstract}


\section{Introduction}
\todo{A very nice introduction}

\begin{itemize}
\item general on adhesive categories
\item general on the use of eggs
\end{itemize}

In the original definition, e-graphs are defined as term graphs with an additional notion of equivalence on nodes. 
Adopting the more modern presentation via string diagrams, they are tree with sharing of subtrees, and with an 
an equivalence $\equiv$ on nodes that is closed under xxx. In plain words, if $a$ and $b$ are two constants such that
$a \equiv b$, then $f(a) \equiv f(b)$ for any unary operator $f$.

\cite[Section~4.2]{DetlefsNS05}
\cite{WillseyNWFTP21}


\emph{Synopsis}
The paper has the following structure. 
In Section~\ref{sec:ade} we briefly recall 
the theory of
\emph{$\mathcal{M}$-adhesive categories}
and of \emph{kernel pairs}.
In Section~\ref{sec:hyper} we present the graphical structures of our interest, 
 (labelled, acyclic) hypergraphs and term graphs, and we provide a
functorial characterisation, which allows for proving their adhesivity properties.
This is expended in Section~\ref{hypereq} for describing hypergraphs 
and term graphs with equivalence and in Section~\ref{eggs} for capturing
their variants where the equivalences are closed with respect to operator application,
thus subsuming EGGs.
%
All the proofs have been moved to the appendices. 
%For the sake of completeness, 
%and in order to fix the notation, we prove all the results 
%recalled in the background section, besides those that are original 
%to our work.

\section{Facts about $\mathcal{M}$-adhesive categories and kernel pairs}\label{sec:ade}


This  section briefly recalls \emph{$\mathcal{M}$-adhesive categories} \cite{azzi2019essence,ehrig2012,ehrig2014adhesive,lack2005adhesive,heindel2009category}. 
%
%\begin{notation}
Given a category $\X$ we do not distinguish notationally between $\X$ and its class of objects, so
``$X\in \X$'' means that $X$ is an object of $\X$. We let $\mor(\X)$, $\mon(\X)$ and $\reg(\X)$ denote the class of all arrows, monos and regular monos of $\X$, respectively.  Given an object $X$, we  denote by $?_X$ the unique arrow from an initial object into $X$ and by $!_X$ that  unique arrow from $X$ into a terminal one.
%\end{notation}

\subsection{$\mathcal{M}$-adhesivity}\label{subsec:ade}
The key property of $\mathcal{M}$-adhesive categories is the \emph{Van Kampen condition}~\cite{brown1997van,johnstone2007quasitoposes,lack2005adhesive},
%
and for defining it we need some notions.
%and to define it we introduce some notions.  
Let  $\X$ be a category. A subclass $\mathcal{A}$ of $\mor(\X)$ is said to be

\parbox{11cm}{\begin{itemize}
	\item		\emph{stable under pushouts (pullbacks)} if for every pushout (pullback) square as the one aside, if $m \in \mathcal{A}$ ($n\in \mathcal{A}$) then $n \in \mathcal{A}$ ($m \in \mathcal{A}$);
		\item \emph{closed under composition} if $h, k\in \mathcal{A}$ implies $h\circ k\in \mathcal{A}$ whenever $h$ and $k$ are composable.
\end{itemize}}\hfill
\parbox{1cm}{
\xymatrix{A \ar[r]^{f} \ar[d]_{m}& B \ar[d]^{n} \\ C \ar[r]_{g} & D}}

	\begin{definition}
	
	Let $\mathcal{A}\subseteq \mor(\X)$ be a class of arrows in a category $\X$ and consider the cube below on the right. 

\vspace{-.25cm}
\parbox{9.5cm}{We say that the bottom square is an \emph{$\mathcal{A}$-Van Kampen square} if
	\begin{enumerate}
		\item it is a pushout square;
		\item 	whenever the cube above has pullbacks as back and left faces and the vertical arrows belong to $\mathcal{A}$, then its top face is a pushout 
		if and only if the front and right faces are pullbacks.
	\end{enumerate}} \hfill
	\parbox{3cm}{
	\xymatrix@C=10pt@R=6pt{&A'\ar[dd]|\hole_(.65){a}\ar[rr]^{f'} \ar[dl]_{m'} && B' \ar[dd]^{b} \ar[dl]_{n'} \\ C'  \ar[dd]_{c}\ar[rr]^(.7){g'} & & D' \ar[dd]_(.3){d}\\&A\ar[rr]|\hole^(.65){f} \ar[dl]^{m} && B \ar[dl]^{n} \\C \ar[rr]_{g} & & D }
}


	Pushout squares that enjoy only the ``if'' half of item (2) above are called \emph{$\mathcal{A}$-stable}. A $\mor(\X)$-Van Kampen square is called  \emph{Van
		Kampen} and a $\mor(\X)$-stable square  \emph{stable}.
\end{definition}

We can now define $\mathcal{M}$-adhesive categories.

\begin{definition}
	Let $\X$ be a category and $\mathcal{M}$ a subclass of
	$\mon(\X)$  including  all isomorphisms, closed under composition,  and stable under pullbacks and pushouts.  The category  $\X$ is said to be \emph{$\mathcal{M}$-adhesive} if
	\begin{enumerate}
		\item it has \emph{$\mathcal{M}$-pullbacks}, i.e.~pullbacks along arrows of $\mathcal{M}$;
		\item it has \emph{$\mathcal{M}$-pushouts}, i.e.~pushouts along arrows of $\mathcal{M}$;
		\item  $\mathcal{M}$-pushouts are $\mathcal{M}$-Van Kampen squares.
	\end{enumerate}
	%
	A category $\X$ is said to be \emph{strictly $\mathcal{M}$-adhesive}
	if $\mathcal{M}$-pushouts are Van Kampen. We write $m\colon X \mto Y$ to denote that an arrow $m\colon X\to Y$ belongs to $\mathcal{M}$.
\end{definition}



\begin{remark}
	\label{rem:salva}
	\emph{Adhesivity} and \emph{quasiadhesivity} 
	\cite{lack2005adhesive,garner2012axioms} coincide with strict
	$\mon(\X) $-adhesivity and strict $\reg(\X)$-adhesivity,
	respectively.
\end{remark}


$\mathcal{M}$-adhesivity is well-behaved with respect to  the construction of slice and functor categories \cite{mac2013categories}, as shown by the following theorems~\cite{ehrig2006fundamentals,lack2005adhesive}.

\begin{proposition}
	\label{thm:slice-functors}
	Let $\X$ be an (strict) $\mathcal{M}$-adhesive category. Then it holds
	\begin{enumerate}
		\item if $\Y$ is an (strict) $\mathcal{N}$-adhesive category $L\colon \Y\to \A$ a functor preserving $\mathcal{N}$-pushouts and $R\colon \X\to A$ one preserving pullbacks, then $\comma{L}{R}$ is (strictly) $\comma{\mathcal{N}}{\mathcal{M}}$-adhesive, where
		\[\comma{\mathcal{N}}{\mathcal{M}}:=\{(h,k) \in \mor(\comma{L}{R}) \mid h\in \mathcal{N}, k\in \mathcal{M}\}\]
		\item for every object $X$
		the categories $\X/X$  and $X/X$ are, respectively, (strictly) $\mathcal{M}/X$-adhesive and (strictly) $X/\mathcal{M}$-adhesive, where
		\[\mathcal{M}/X:=\{m\in \mor(\X/X) \mid m\in
		\mathcal{M}\} \hspace{5pt} X/\mathcal{M}:=\{m\in \mor(X/\X) \mid m\in \mathcal{M}\}\]
		\item for every small category $\Y$, the category $\X^\Y$ of
		functors $\Y\to \X$ is (strictly) $\mathcal{M}^{\Y}$-adhesive, where
		$\mathcal{M}^{\Y}:=\{\eta \in \mor(\X^\Y) \mid \eta_Y \in
		\mathcal{M} \text{ for every } Y\in \Y\}$;
		\item if $\Y$ is a full subcategory of $\X$ closed in it under pullbacks and $\mathcal{M}$-pushouts, then $\Y$ is (strictly) $\mathcal{N}$-adhesive for every class of arrows $\mathcal{N}$ of $\Y$ contained in $\mathcal{M}$ that is stable under pullbacks and pushouts, contains all the isomorphisms, and is closed under composition and decomposition.
	\end{enumerate} 
\end{proposition}

We will briefly list some examples of $\mathcal{M}$-adhesive categories.

\begin{example}
	\label{ex:adhesive}
	$\cat{Set}$ is adhesive, and, more generally, every topos is
	adhesive~\cite{lack2006toposes}. By the closure properties above, every presheaf $[\cat{X},\cat{Set}]$ is adhesive, thus the category
	$\cat{Graph} = [ E \rightrightarrows V, \cat{Set}]$ is adhesive
	where $E \rightrightarrows {V}$ is the two objects category with two
	morphisms $s,t \colon{E} \to {V}$. Similarly, various
	categories of hypergraphs can be shown to be adhesive, such as term
	graphs and hierarchical graphs~\cite{CastelnovoGM24}. Note that the category $\cat{sGraphs}$ of simple graphs, 
	i.e.~graphs without parallel edges, is
	$\reg{(\cat{sGraphs})}$-adhesive~\cite{BehrHK23} but not
	quasiadhesive.
\end{example}

\iffalse 
\begin{remark}\label{rem:deco}
We can point out an important property of strict $\mathcal{M}$-adhesive categories with pullbacks.  Consider the solid part of the cube aside, whose bottom case is an $\mathcal{M}$-pushout.
	
	\parbox{9.5cm}{	Given an arrow $d\colon X\to D$, we can present the object $X$ has a pushout: indeed, consider the following cube, in which all the three vertical squares are pullbacks.}
	\parbox{3cm}{
	\xymatrix@C=15pt@R=9pt{&V\ar[dd]|\hole_(.65){a}\ar[rr]^{v} \ar@{>.>}[dl]_{u} && Y \ar[dd]^{b} \ar@{>->}[dl]_{y} \\ Z  \ar[dd]_{c}\ar[rr]^(.7){z} & & X \ar[dd]_(.3){d}\\&A\ar[rr]|\hole^(.65){f} \ar@{>->}[dl]^{m} && B \ar@{>->}[dl]^{n} \\C \ar[rr]_{g} & & D }}

Now, notice that, since the front square is a pullback, then the dotted arrow $u\colon V\to Z$ exists. Moreover, the usual composition and decomposition property of pullbacks \cite{mac2013categories} entails that the left face of the cube so obtained is a pullback too, proving that $u$ is in $\mathcal{M}$ and that the top square is a pushout.

	Clearly if the arrow $p\colon X\to D$ is in $\mathcal{M}$, we can omit the assumptions of strictness and the existence of all pullbacks.
\end{remark}
\fi 


We can state some useful properties of $\mathcal{M}$-adhesive category.

\begin{restatable}{proposition}{regmono}\label{prop:regmono}
	If $\X$ is $\mathcal{M}$-adhesive then it holds
	\begin{enumerate}
		\item every $\mathcal{M}$-pushout square is also a pullback;
		\item every arrow in $\mathcal{M}$ is a regular mono.
	\end{enumerate}
\end{restatable}
%  [Proof in \Cref{regmono-proof}]


\subsection{Kernel Pairs and Regular Epimorphisms}

In this section we recall the definition and some properties of \emph{kernel pairs}.
%which will be useful in the next sections. 

\noindent 
\parbox{11cm}{
\begin{definition}
    A \emph{kernel pair} for an arrow $f\colon A \to B$ is an object $K_f$ together with two arrows $\pi^1_f, \pi^2_f\colon K_f \rightrightarrows A$, denoted as $(K_f, \pi^1_f, \pi^2_f)$, such that the square aside is a pullback.
\end{definition}}\hfill 
\parbox{2cm}{\xymatrix@R=15pt{K_f \ar[r]_{\pi^2_f} \ar[d]_{\pi^1_f}& A \ar[d]^f \\ A \ar[r]^{f}& B}}

\smallskip
\begin{remark}\label{prop:pairng_of_kernel_pairs_mono}
If $(K_f, \pi^1_f, \pi^2_f)$ is a a kernel pair for $f\colon X \to Y$ and a product of $X$ with itself exists, then the canonical arrow $\langle \pi^1_f, \pi^2_f\rangle \colon K_f \to X \times X$ is a mono.
\end{remark}

\begin{remark}\label{prop:kermono}
An arrow $m\colon M\to X$ is a mono if and only if it admits $(M, \id{M}, \id{M})$ as a kernel pair.
\end{remark}

Together with \Cref{lem:pb1}, the previous remarks allow us to prove the following result.

\begin{proposition}\label{cor:kermono}
	Let $f\colon X\to Y$ be an arrow and $m\colon Y\to Z$ a mono. If
	$(K_f, \pi_f^1, \pi_f^2)$ is a kernel pair for $f\colon X\to Y$, then it is also a kernel pair for $m\circ f$.
\end{proposition}

We explore the link between regular epis and kernel pairs.

\begin{restatable}{proposition}{epic}\label{prop:reg_epi_coeq}
    Let $e\colon X \to Y$ be a regular epi in a category $\X$ with a kernel pair $(K_e, \pi^1_e, \pi^2_e)$. Then, $e$ is the coequalizer of $\pi^1_e$ and $\pi^2_e$.
\end{restatable}
%[Proof in \Cref{epic-proof}]


\begin{restatable}{corollary}{natepi}\label{cor:reg_epi_components_reg_epi_nat_trans}
    Let $\X$ be a category with pullbacks and $\phi\colon F \dot\to G$ a natural transformation between functors $F, G: \D \rightrightarrows \X$. If $\phi_d$ is a regular epi for every $d$ in $\D$, then $\phi$ is a regular epi.
\end{restatable}
%[Proof in \Cref{natepi-proof}]

From the previous result we deduce that the class of regular epis is closed under colimits.
% as shown by the next lemma

\begin{restatable}{lemma}{epicol}\label{lemma:nat_trans_reg_epi_canonical_arrow_reg_epi}
    Let $F,G\colon \D\rightrightarrows \X$ be two diagrams, and suppose that $\X$ has all colimits of shape $\D$. Let $(X, \{x_d\}_{d \in \D})$ and $(Y, \{y_d\}_{d\in D})$ be the colimits of $F$ and $G$, respectively.  If $\phi\colon  F \to G$ is a natural transformation whose components are regular epis, then the arrow induced by $\phi$ from $X$ to $Y$ is a regular epi.
\end{restatable}
%[Proof in \Cref{epicol-proof}]

We now explore some properties of kernel pairs and their link with adhesive categories. 
The results below appear to be original, and we give their proofs in \Cref{kpp-proof}.

\begin{restatable}{lemma}{kpp}\label{lemma:kern_pairs_pres_pullbacks}
	Let $f\colon X \to Y$ and $g\colon Z \to W$ be two arrows admitting kernel pairs and suppose that the solid part of the three squares below is given. 
	%
	If the leftmost square is commutative, then there exists a a unique arrow $k_h: K_f \to K_g$ making the other two commutative.
	%
	Moreover, if the leftmost is a pullback, then also the other two are so.
%
	\[\xymatrix{X \ar[r]^{h}\ar[d]_{f}& Z \ar[d]^{g} & K_f \ar@{.>}[r]^{k_h} \ar[d]_{\pi^1_f}& K_g \ar[d]^{\pi^1_g} & K_f  \ar@{.>}[r]^{k_h} \ar[d]_{\pi^2_f}& K_g \ar[d]^{\pi^2_g}\\ Y \ar[r]_t& W & X \ar[r]_{h} & Z & X\ar[r]_{h} &Z}\]
	
\end{restatable}
%[Proof in \Cref{kpp-proof}]

The previous result allows us to deduce the following lemma in an $\mathcal{M}$-adhesive context.

\noindent
\parbox{7cm}{
\begin{restatable}{proposition}{mpo}\label{lem:mpo}
	Let $\X$ be a strict $\mathcal{M}$-adhesive category with all pullbacks, and suppose that in the cube aside the top face is an $\mathcal{M}$-pushout. Then the right square is a pushout.
\end{restatable}}\hfill 
\parbox{6cm}{\xymatrix@C=10pt@R=6pt{&A'\ar[dd]|\hole_(.65){a}\ar[rr]^{f'} \ar@{>->}[dl]_{m'} && B' \ar[dd]^{b} \ar@{>->}[dl]_{n'} & K_a\ar[rr]^{k_{f'}} \ar[dd]_{k_{m'}}&& K_b \ar[dd]^{k_{n'}} \\ C'  \ar[dd]_{c}\ar[rr]^(.7){g'} & & D' \ar[dd]_(.3){d}\\&A\ar[rr]|\hole^(.65){f} \ar[dl]^{m} && B \ar[dl]^{n}  & K_{c} \ar[rr]_{k_{g'}}&& K_d\\C \ar[rr]_{g} & & D }}

%[Proof in \Cref{mpo-proof}]
%
%\subsection{Factorization systems}

\begin{definition}[Factorization System]\label{def:fact_sys}
	Let $\X$ be a category, and let $\E$, $\M$ be two classes of arrows.
	Then, $(\E, \M)$ is a \emph{factorization system} if:
	\begin{enumerate}
		\item $\E$ and $\M$ are closed under composition with isomorphisms: if $f: X \to Y$ belongs to $\E$ (to $\M$), and $h: Y\to Z$ is an isomorphism, then $h \circ f$ belongs to $\E$ (to $\M$);
		\item every arrow $f: X \to Y$ of $\X$ admits an \emph{$(\E, \M)$-factorization}, that is, there are arrows $e \in \E$ and $m \in \M$ such that $f = m\circ e$;
		\item every $e \in \E$ has the \emph{left lifting property} with respect to every $m \in \M$: for every commutative square as the one below, with $e \in \E$ and $m \in \M$,
			there exists a unique $k: Y \to Z$ whitch fits in it.
			\[\xymatrix{
					X\ar[r]^{g} \ar[d]_e & Z \ar[d]^m \\ Y \ar@{.>}[ur]^k \ar[r]_f & V
			}\]
	\end{enumerate}
	A factorization system $(\E, \M)$ is \emph{proper} if every $e \in \E$ is epi and every $m \in \M$ is mono, while it is \emph{stable} is $\E$ is stable under pullbacks.
\end{definition}

\begin{remark}\label{rem:dual_fact_sys}
	It is immediate to notice that, given a factorization system $(\E, \M)$ on a category $\X$, then $(\M, \E)$ is a factorization system on $\X^{op}$, which is proper if $(\E, \M)$ is so, and it is stable if
	$\M$ is stable under pushouts in $\X$.
\end{remark}

\begin{remark}
	If $(\E, \M)$ is a factorization system for a category $\X$, then $\E$ is the class of all the arrows with the left lifting property with respect to every $m\in \M$.
	Indeed, if $f:X \to Y$ is an arrow with such a property with respect to every element of $\M$, it is possible to factorize it as $f = m \circ e$, with $m: M \to Y$ in $\M$ and 
	$e: X \to M$ in $\E$, obtaining the above situation by definition.
	\[\xymatrix{X \ar[r]^e\ar[d]_f & M \ar[d]^m \\ Y \ar[r]_{\id{Y}} \ar@{.>}[ur]^k & Y}\]
	From this, it is possible to prove that $k$ and $m$ are mutually inverses, hence $f = m\circ e$ belongs to $\E$ due to closure under isomorphisms.
\end{remark}

\begin{proposition}
	Let $(\E, \M)$ be a factorization system on $\X$. Then $\E$ contains every isomorphism, and it is closed under composition.
\end{proposition}

By \Cref{rem:dual_fact_sys}, we have also that $\M$ is closed under composition, and contains every isomorphism.

\begin{lemma}
	Let $(\E, \M)$ be a factorization system on $\X$. If $e:Y \to E$, $e': Y\to E'$ and $m: E \to X$, $m':E'\to X$ are arrows, respectively, in $\E$ and in $\M$ such that $e'\circ m' = e \circ m$,
	then there exists a unique isomorphism $\phi: E \to E'$ such that the diagram below commutes.
	\[\xymatrix{Y\ar[r]^{e'}\ar[d]_{e}&E'\ar[d]^{m'}\\E\ar[r]_{m}\ar@{.>}[ur]^{\phi}\ar[r]_{m}&X}\]
\end{lemma}

\begin{proof}
	Suppose the following situation.
	\[\xymatrix{X \ar[r]^e\ar[d]_{e'}&E\ar[d]^{m}&&X\ar[r]^{e'}\ar[d]_{e}&{E'}\ar[d]^{m'}\\{E'}\ar[r]_{m'}\ar@{.>}[ur]^{\phi}&Y&&E\ar[r]_{m}\ar@{.>}[ur]^{\psi}&Y}\]
	From the commutativity of the diagrams below, we obtain at once $\phi\circ\psi=\id{E'}$ and $\psi\circ\phi=\id{E}$.
	\[\xymatrix{
			X\ar[rr]^{e}\ar[dd]_{e}\ar[dr]_{e'}&&E\ar[r]^{m}\ar[dl]^{\psi}&Y\ar[dd]^{\id{Y}} &&& X\ar[rr]^{e'}\ar[dd]_{e'}\ar[dr]_{e}&&{E'}\ar[r]^{m}\ar[dl]^{\phi}&Y\ar[dd]^{\id{Y}}\\
							     & {E'}\ar[dl]_{\phi}\ar[drr]^{m'}& & &&&  & {E'}\ar[dl]_{\psi}\ar[drr]^{m'}& & \\
			E\ar[rrr]_{m} &&& Y &&& {E'} \ar[rrr]_{m'} &&& Y
	}\]
\end{proof}

The previous lemma allows us to deduce other useful properties of factorization systems.

\begin{proposition}
	Given a factorization system $(\E, \M)$ on a category $\X$, it holds that:
	\begin{enumerate}
		\item an arrow $f:X\to Y$ that factors as $m \circ e$ belongs to $\M$ if and only if $e$ is an isomorphism;
		\item $f\in \E$ and $f\in\M$ if and only if $f$ is an isomorphism;
		\item $\M$ is stable under pullbacks;
		\item if $(\E, \M)$ is proper, then $g\circ f \in \M$ implies $f \in \M$.
	\end{enumerate}
\end{proposition}

\todo{proof?}




\section{Hypergraphical structures}\label{sec:hyper}

In this section we briefly recall the notion \emph{$\X$-hypergraph}. It is necessary to have a monad $(-)^\star\colon \Set\to \Set$, also known as \emph{list monad},
sending a set to the free monoid on it \cite{sakarovitch2009elements,Wadler95} and playing a role analogous to the usual \emph{Kleene star}.
We recall some of its proprieties.

\begin{proposition}
	The following facts hold
	\begin{enumerate}
		\item for every set $X$ and $n\in \mathbb{N}$ there are arrows $v_{n}\colon X^n\to X^\star $ such that $(X^\star, \{v_{n}\}_{n\in \mathbb{N}})$ is a coproduct;
		\item for every arrow $f\colon X\to Y$, $f^\star$ is the coproduct of the family $\{f^n\}_{n\in \mathbb{N}}$;
		\item $(-)^\star$ preserves all \emph{connected limits} \cite{carboni1995connected}, in particular it preserves pullbacks and equalizers.
	\end{enumerate}
\end{proposition}

\begin{remark}\label{rem:mono}
	Preservation of pullbacks implies that $(-)^\star$ sends monos to monos.
\end{remark}

\begin{remark}\label[remark]{rem:length}Notice that $1^\star$ can be canonically identified with $\mathbb{N}$, thus for every set $X$ the arrow $!_\X\colon X\to 1$ induces a \emph{length function} $!^\star_X\colon X^\star \to \mathbb{N}$, which sends a word to its length.
\end{remark}

\subsection{The category of hypergraphs}

We open this section with the definition of hypergraphs and we show how to label them with an algebraic signature.  


\begin{definition}An \emph{hypergraph} is a 4-uple $\mathcal{G}:=(E_\mathcal{G}, V_\mathcal{G}, s_\mathcal{G}, t_\mathcal{G})$ made by two sets $E_\mathcal{G}$ and $V_\mathcal{G}$, called respectively the set of \emph{hyperedges} and \emph{nodes}, plus a pair of \emph{source} and \emph{target} arrows  $s_\mathcal{G}, t_\mathcal{G}\colon E_\mathcal{G}\rightrightarrows V_\mathcal{G}^\star$. A \emph{hypergraph morphism} $(E_\mathcal{G}, V_\mathcal{G}, s_\mathcal{G}, t_\mathcal{G})\to (E_\mathcal{H}, V_\mathcal{H}, s_\mathcal{H}, t_\mathcal{H})$ is a pair $(h,k)$ of functions $h\colon E_\mathcal{G}\to E_\mathcal{H}$, $k\colon V_\mathcal{G}\to V_\mathcal{H}$ such that the following diagrams commute
	\[\xymatrix{ E_{\mathcal{G}} \ar[d]_{h} \ar[r]^{s_{\mathcal{G}}}& V^\star_{\mathcal{G}}  \ar[d]^{k^\star}& E_{\mathcal{G}} \ar[r]^{t_{\mathcal{G}}} \ar[d]_{h} & V^\star_{\mathcal{G}} \ar[d]^{k^\star}  \\ E_{\mathcal{G}} \ar[r]_{s_{\mathcal{H}}} & V^\star_{\mathcal{H}} & E_{\mathcal{G}} \ar[r]_{t_{\mathcal{H}}} & V^\star_{\mathcal{H}} }\]
	We define $\hyp$ to be the resulting category.
\end{definition}

Let $\pro^\star$ be the functor sending $X$ to $X^\star\times X^\star$: we can get $\hyp$ as a comma category.

\begin{proposition}\label[proposition]{prop:com}
	$\hyp$ is isomorphic to $\comma{\id{\Set}}{\pro^\star}$
\end{proposition}

\begin{corollary}\label[corollary]{prop:hypadh}
	 $\hyp$ is an adhesive category.
\end{corollary}
\begin{proof}
	By hypothesis $(-)^\star$ preserves pullbacks, while $\pro$ is continuous by definition, thus the thesis follows from  \Cref{thm:slice-functors} and \Cref{prop:com}.   
\end{proof}

Another useful corollary of \Cref{prop:com} is the following one.

\begin{corollary}\label[corollary]{cor:monhyper}
A morphism $(h,k)$ is a mono in $\hyp$ if and only if both its components are injective functions.
\end{corollary}

\Cref{prop:left,prop:com} allow us to deduce immediately the following.

\begin{proposition}\label[proposition]{cor:left}  The forgetful functor $U_{\hyp}$ which sends an hypergraph $\mathcal{G}$ to its object of nodes has a left adjoint $\Delta_{\hyp}$.
\end{proposition}

\begin{example}Since the initial object of $\catname{Set}$ is the empty set,  $\Delta_{\Set}(X)$ is the hypergraph which has $X$ as set of nodes, $\emptyset$ as set of hyperedges, and $?_X$ as source and target function.
\end{example}

\todo{forse possiamo anche cancellare tutti gli esempi sotto}
\todo{FG ne lascerei uno, ripreso magari dalla intro e presentato come string diagram}

We can represent hypergraphs graphically. We will use dots to denote nodes and squares to denote hyperedges, the name of a node or of an hyperedge will be put near the corresponding dot or square. Sources and targets are represented by lines between dots and squares: the lines from the sources of an hyperedge will have an arrowhead in the middle pointing towards the hyperedge, while the lines to the targets will have arrowheads pointing to the target nodes.  We will decorate the arrow corresponding to the $i^{th}$ letter of a target or a source with a label $i$.


\begin{example}Take $V_{\mathcal{G}}$ to be be $\{v_1, v_2, v_3, v_4, v_5\}$ and $E_{\mathcal{G}}$ to be $\{h_1, h_2, h_3\}$. Sources and targets are given by:
	\[\begin{matrix}
		s_{\mathcal{G}}(h_1)\colon 2\to V_{\mathcal{G}}  & \begin{matrix}
			0 \mapsto v_1\\
			1\mapsto v_2
		\end{matrix} && s_{\mathcal{G}}(h_2)\colon 2\to V_{\mathcal{G}} & \begin{matrix}
			0 \mapsto v_3\\
			1\mapsto v_4 
		\end{matrix} && s_{\mathcal{G}}(h_3)\colon 1\to V_{\mathcal{G}} & 
		0 \mapsto v_5\\
		t_{\mathcal{G}}(h_1)\colon 2\to V_{\mathcal{G}} & \begin{matrix}
			0 \mapsto v_3\\
			1\mapsto v_4
		\end{matrix} && t_{\mathcal{G}}(h_2)\colon 2\to V_{\mathcal{G}} & 0\mapsto v_5 && t_{\mathcal{G}}(h_3)\colon 0\to V_{\mathcal{G}} &  t_{\mathcal{G}}(h_3)=?_{ V_{\mathcal{G}}} 
	\end{matrix}\]
	
	We can draw the resulting $\mathcal{G}$ as follows:
	\begin{center}\begin{tikzpicture}
			\node[circle,fill=black,inner sep=0pt,minimum size=6pt,label=above:{$v_1$}] (A) at (0,0) {};
			\node[circle,fill=black,inner sep=0pt,minimum size=6pt,label=above:{$v_2$}] (B) at (0,-1.5) {};
			\node[circle,fill=black,inner sep=0pt,minimum size=6pt,label=above:{$v_3$}] (C) at (3,0) {};
			\node[circle,fill=black,inner sep=0pt,minimum size=6pt,label=above:{$v_4$}] (D) at (3,-1.5) {};
			\node[circle,fill=black,inner sep=0pt,minimum size=6pt,label=above:{$v_5$}] (E) at (6,-0.75) {};
			\draw[rounded corners] (1.25, -1) rectangle (1.75, -0.5) {};
			\draw[->-=.5](4.75,-0.75)--(E)node[pos=0.5, above,font=\fontsize{7}{0}\selectfont]{$1$};
			\draw[->-=.5](E)--(7,-0.75)node[pos=0.5, above,font=\fontsize{7}{0}\selectfont]{$1$};
			\draw[rounded corners] (4.25, -1) rectangle (4.75, -0.5) {};
			\node at (4.5, -0.3){$h_2$};
			\node at (1.5, -0.3){$h_1$};
			\node at (7.25, -0.3){$h_3$};
			\draw[rounded corners] (7, -1) rectangle (7.5, -0.5) {};
			\draw(A)[->-=.5]..controls(0.5,0)and(1.2,-0.2)..(1.25,-0.6)node[pos=0.5, above,font=\fontsize{7}{0}\selectfont]{$1$};
			\draw(B)[->-=.5]..controls(0.5,-1.5)and(1.2,-1.3)..(1.25,-0.9)node[pos=0.5, below,font=\fontsize{7}{0}\selectfont]{$2$};
			
			\draw(C)[->-=.5]..controls(3.5,0)and(4.2,-0.25)..(4.25,-0.6)node[pos=0.5, above,font=\fontsize{7}{0}\selectfont]{$1$};
			\draw(D)[->-=.5]..controls(3.5,-1.5)and(4.2,-1.3)..(4.25,-0.9)node[pos=0.5, below,font=\fontsize{7}{0}\selectfont]{$2$};
			
			\draw[->-=.5](1.75,-0.9)..controls(1.8,-1.3)and(2.5,-1.5)..(D)node[pos=0.5, below,font=\fontsize{7}{0}\selectfont]{$2$};
			\draw[->-=.5] (1.75,-0.6)..controls(1.8,-0.25)and(2.5,0)..(C) node[pos=0.5, above,font=\fontsize{7}{0}\selectfont]{$1$};
		\end{tikzpicture}
	\end{center}
\end{example}
\begin{example}\label[example]{exa_2} Let $V_{\mathcal{G}}$ be as in the previous example and $E_{\mathcal{G}}=\{h_1, h_2, h_3\}$.	Then we define
	\[\begin{matrix}
		s_{\mathcal{G}}(h_1)\colon 0\to V_{\mathcal{G}}  & s_{\mathcal{G}}(h_1)=?_{V_\mathcal{G}} && s_{\mathcal{G}}(h_2)\colon 2\to V_{\mathcal{G}} & \begin{matrix}
			0 \mapsto v_1\\
			1\mapsto v_2
		\end{matrix}&& s_{\mathcal{G}}(h_3)\colon 2\to V_{\mathcal{G}} & \begin{matrix} 
			0 \mapsto v_1\\
			1\mapsto v_4	
		\end{matrix}\\
		t_{\mathcal{G}}(h_1)\colon 1\to V_{\mathcal{G}} & 
		0 \mapsto v_1 && 		t_{\mathcal{G}}(h_2)\colon 1\to V_{\mathcal{G}} & 0\mapsto v_3 &&   t_{\mathcal{G}}(h_3)\colon 1\to V_{\mathcal{G}} & 1\mapsto v_5
	\end{matrix}\]
	
	Now we can depict $\mathcal{G}$ as
	\begin{center}\begin{tikzpicture}
			\node[circle,fill=black,inner sep=0pt,minimum size=6pt,label=above:{$v_1$}] (A) at (0,0) {};
			\node[circle,fill=black,inner sep=0pt,minimum size=6pt,label=above:{$v_2$}] (B) at (0,-1.5) {};
			\node[circle,fill=black,inner sep=0pt,minimum size=6pt,label=above:{$v_3$}] (C) at (3,-0.75) {};
			\node[circle,fill=black,inner sep=0pt,minimum size=6pt,label=above:{$v_4$}] (D) at (3,-2.25) {};
			\node[circle,fill=black,inner sep=0pt,minimum size=6pt,label=right:{$v_5$}] (E) at (6,-1.5) {};
			\draw[->-=.5] (1.75,-0.75)--(C)node[pos=0.5, above,font=\fontsize{7}{0}\selectfont]{$1$};
			\draw[rounded corners] (1.25, -1) rectangle (1.75, -0.5) {};
			\draw[->-=.5] (4.75,-1.5)--(E)node[pos=0.5, above,font=\fontsize{7}{0}\selectfont]{$1$};
			\draw[->-=.5] (-1.5,0)--(A)node[pos=0.5, above,font=\fontsize{7}{0}\selectfont]{$1$};
			\draw[rounded corners] (4.25, -1.75) rectangle (4.75, -1.25) {};
			\node at (4.5, -1.05){$h_3$};
			\node at (1.5, -0.3){$h_2$};
			\node at (-1.75, 0.45){$h_1$};
			\draw[rounded corners] (-2, -0.25) rectangle (-1.5, 0.25) {};
			\draw[->-=.5] (A)..controls(0.5,0)and(1.2,-0.2)..(1.25,-0.6)node[pos=0.5, above,font=\fontsize{7}{0}\selectfont]{$1$};
			\draw[->-=.5] (B)..controls(0.5,-1.5)and(1.2,-1.3)..(1.25,-0.9)node[pos=0.5, below,font=\fontsize{7}{0}\selectfont]{$2$};
			
			\draw[->-=.5] (C)..controls(3.5,-0.75)and(4.2,-0.95)..(4.25,-1.35)node[pos=0.5, above,font=\fontsize{7}{0}\selectfont]{$1$};
			\draw[->-=.5] (D)..controls(3.5,-2.25)and(4.2,-2.05)..(4.25,-1.65)node[pos=0.5, below,font=\fontsize{7}{0}\selectfont]{$2$};
		\end{tikzpicture}
	\end{center}
\end{example}

\subsubsection{$\hyp$ as a category of functors}

Following \cite{bonchi2022string}, we can present $\hyp$ as a category of functor over a suitable category.

\begin{definition}Let $\catname{H}$ be the category such that
	\begin{itemize}
		\item the set of objects is $ (\mathbb{N}\times \mathbb{N}) \cup \{\bullet\}$;
		\item arrows are given by the identities $\id{k,l}$ and $\id{\bullet}$ and exactly $k+l$ arrows $f_i\colon (k,l)\rightarrow \bullet$, where $i$ ranges from $0$ to $k+l-1$;
		\item composition is defined by putting
		%\begin{equation*}
			$f_i=f_i\circ \id{k,l}$ and $f_i = \id{\bullet}\circ f_i$
		%\end{equation*}
		for every $f_i\colon (k,l)\rightarrow \bullet$.
	\end{itemize}
\end{definition}

The idea is that for every functor $F\colon \catname{H}\to \catname{Set}$ we can define
\[E_F:=\sum_{k,l\in \mathbb{N}}F(k,l)\]
%
Now, for every $k$, $l$, $i$ and $j$ in $\mathbb{N}$ with $i< k$ and $j< l$ we define $s^F_{k,l}\colon F(k,l)\to F(\bullet)^k$ and  $t^F_{k,l}\colon F(k,l)\to F(\bullet)^l$ as the unique arrows fitting in the diagrams below, where the vertical arrows are the projections
\[\xymatrix{F(k,l)  \ar@{.>}[r]^{s^F_{k,l}} \ar[dr]_{F(f_i)}& F(\bullet)^{k} \ar[d]^{\pi^F_{k,i}} & F(k,l) \ar@{.>}[r]^{t^F_{k,l}} \ar[dr]_{F(f_{k+j})} & F(\bullet)^{l} \ar[d]^{\pi^F_{l,j}} \\ & F(\bullet) && F(\bullet)}\]
%
In turn, these arrows allow us to consider
$s_F, t_F\colon E_F\rightrightarrows F(\bullet)^{\star}$ as the unique arrows fitting in the diagrams below, where the vertical arrows are coprojections
\[\xymatrix{F(k,l) \ar[d]_{a^F_{k,l}}  \ar[r]^{s^{F}_{k,l}}& F(\bullet)^{k} \ar[d]^{b^F_{k}} & F(k,l) \ar[d]_{a^F_{k,l}}  \ar[r]^{t^{F}_{k,l}}& F(\bullet)^{l} \ar[d]^{b^F_{l}}\\ E_F \ar@{.>}[r]_-{s_F}& F(\bullet)^\star & E_F \ar@{.>}[r]_-{t_F}& F(\bullet)^\star}\]

Let $\mathcal{G}_F$ be the resulting hypergraph. One can now show that  sending $F$ to $\mathcal{G}_F$ can be extended to an equivalence $\mathcal{G}_{-}\colon \Set^{\catname{H}}\to \hyp$ (see \cite{castelnovo2023thesis,CastelnovoGM24} for details).


\begin{proposition}
	$\hyp$ is equivalent to the category $\Set^{\catname{H}}$.
\end{proposition}

\subsubsection{Labelling hypergraph with an algebraic signature}\label{sssect:hyp_alg_sign}

Our interest for hypergraphs stems from their use as a graphical representation of algebraic terms. We thus need a way to label hyperedges with symbols taken from a signature.

\begin{definition}
An \emph{algebraic signature} $\Sigma$ is a pair $(O_\Sigma, \ari_\Sigma)$ given by a \emph{set of operations} $O_\Sigma$ and an \emph{arity function} $\ari_\Sigma\colon O_\Sigma \to \mathbb{N}$. 
%
We define the \emph{hypergraph $\mathcal{G}_\Sigma$ associated with $\Sigma$} taking $O_\Sigma$ as set of hyperedges, $1$ as set of nodes, so that $1^\star$ is $\mathbb{N}$, $\ari_\sigma$ as the source function and $\delta_1$ as target function, where $\delta_1$ picks the element $1$. The category $\hyps$ of \emph{algebraically labelled hypergraphs} is the slice category $\hyp/\mathcal{G}^\Sigma$.
\end{definition}


\begin{example}\label[example]{exa_3} Let $\Sigma=(O_\Sigma, \ari_\Sigma)$ be an algebraic signature in $\Set$. This simply amount to a set of \emph{operations} with an associated natural number, called \emph{arity}. 	For instance let $\Sigma_G$ be the signature of groups, then $\mathcal{G}^{\Sigma_G}$ can be depicted as
	\begin{center}
		\begin{tikzpicture}
			\node[circle,fill=black,inner sep=0pt,minimum size=6pt,label=above:{$v$}] (V) at (0,0) {};
			\node(E)at(-2, 0.4){$e$};
			\node(M)at(0, 2.15){$\cdot$};
			\node(I)at(2, 0.5){$(-)^{-1}$};
			\draw[->-=.5](-1.75,0)--(V)node[pos=0.5, above,font=\fontsize{7}{0}\selectfont]{$1$};
			\draw[->-=.5](V)..controls(-0.5,0.5)and(-0.8,1)..(-0.25,1.6)node[pos=0.5, right,font=\fontsize{7}{0}\selectfont]{$2$};
			\draw[->-=.5](V)..controls(-1,0.6)and(-1,1.1)..(-0.25,1.9)node[pos=0.5, left,font=\fontsize{7}{0}\selectfont]{$1$};
			\draw[->-=.5](0.25,1.75)..controls(0.8,0.8)and(0.5,0.5)..(V)node[pos=0.5, right,font=\fontsize{7}{0}\selectfont]{$1$};
			\draw[->-=.5](V)--(1.75,0)node[pos=0.5, above,font=\fontsize{7}{0}\selectfont]{$1$};
			\draw[->-=.5](2.25,0)..controls(3.5,0)and(2.5,-2)..(V)node[pos=0.5, below,font=\fontsize{7}{0}\selectfont]{$2$};
			\draw[rounded corners] (-2.25, -0.25) rectangle (-1.75, 0.25) {};
			\draw[rounded corners] (-0.25, 1.5) rectangle (0.25, 2) {};
			\draw[rounded corners] (2.25, -0.25) rectangle (1.75, 0.25) {};
		\end{tikzpicture}
	\end{center}
\end{example}

\Cref{cor:mono} and \Cref{thm:slice-functors} give us immediately an adhesivity result for $\hyp_{\Sigma}$ and a characterisation of monos in it.

\begin{proposition}\label[proposition]{prop:mono} Let $\Sigma$ be an algebraic signature. Then it holds
	\begin{enumerate}
		\item a morphism $(h,k)$ between two object of $\hyp_{\Sigma}$ is a mono if and only if $h$ and $k$ are injective functions;
		\item $\hyp_{\Sigma}$ is an adhesive category. 
	\end{enumerate}
\end{proposition}


\begin{remark}\label[remark]{rem:label}	
Let $\mathcal{H}=(E, V, s, t)$ be an hypergraph, by definition we know that $U_{\hyp}(\mathcal{G}^{\Sigma})$ is the terminal object $1$, so an arrow $\mathcal{H}\rightarrow \mathcal{G}^{\Sigma}$, is determined by a morphism $h\colon E_\mathcal{H}\to O_\Sigma$  making the two squares below commute (cfr.~ \Cref{rem:length}).
	\[\xymatrix{E_{\mathcal{H}} \ar[r]^{h} \ar[d]_{s_{\mathcal{H}}}& O_\Sigma \ar[d]^{\ari_\Sigma}& E_{\mathcal{H}} \ar[r]^{h} \ar[d]_{t_{\mathcal{H}}} & O_\Sigma \ar[d]^{\delta_1}\\ V^\star_{\mathcal{{H}}} \ar[r]_{\lgh_{V_{\mathcal{H}}}}& \mathbb{N} & V^\star_{\mathcal{{H}}} \ar[r]_{\lgh_{V_{\mathcal{H}}}} & \mathbb{N}}\]
%	
Let $v_n\colon V^n_\mathcal{H}\to  V^\star_{\mathcal{H}}$ be a coprojection.  The second diagram above entails that $t_{\mathcal{H}}$ factors via the inclusion $v_1\colon V_{\mathcal{H}}\to V^{\star}$ of words of length $1$, i.e. $t_{\mathcal{H}}=v_1\circ \tau_{\mathcal{H}}$ for some $\tau_{\mathcal{H}}\colon E_{\mathcal{H}}\to \tau_{\mathcal{H}}$.
\end{remark}

$\hyp_{\Sigma}$, has a forgetful functor $U_{\Sigma}\colon \hyp_{\Sigma}\to \X$ which sends $(h,k)\colon \mathcal{H}\to \mathcal{G}^{\Sigma}$ to $U_{\X}(\mathcal{H}$). Now, $U_{\X}(\mathcal{G}^{\Sigma})=1$ thus, for every object $X$, there is only one arrow $X\to U_{\X}(\mathcal{G}^{\Sigma})$. Define $\Delta_{\Sigma}(X)\colon \Delta_{\X}(X)\to \mathcal{G}^{\Sigma}$ as the transpose of this arrow. Explicitly, $\Delta_{\X}(X)=(0, X, ?_{X^\star}, ?_{X^\star})$ and $\Delta_\Sigma(X)$ is simply $(?_{O_\Sigma}, !_{X})$.

\begin{proposition} $U_\Sigma$
	has a left adjoint $\Delta_\Sigma$.
\end{proposition}
\begin{proof}Let $(h, !_{V_\mathcal{H}})\colon \mathcal{H}\to \mathcal{G}^{\Sigma}$ be an object of $\hyp_{\Sigma}$, and suppose that there exists $f\colon X\to U_{\Sigma}(\mathcal{H})$. Since, $U_{\Sigma}(\mathcal{H})=U_{\X}(\mathcal{H})$ and the identity is the unit of $\Delta_\hyp \dashv U_{\hyp}$, we get a morphism $(?_{E_{\mathcal{H}}},f)\colon \Delta_{\X}(X)\to \mathcal{H}$ of $\hyp$. And then the thesis follows since we have
	\[
	(h, !_{V_{\mathcal{{H}}}})\circ (?_{E_{\mathcal{H}}}, f)=(h\circ ?_{E_{\mathcal{H}}}, !_{V_{\mathcal{{H}}}}\circ f)=(?_{0_{\Sigma}}, !_{X})=\Delta_{\hyp}(X)\]
%and the thesis follow.   
\end{proof}

\todo{anche questo forse val la pena toglierlo}
\todo{presentarlo come string diagram e magari prendere la segnatura de un esempio egg}

We will extend our graphical notation of hypergraphs to labeled ones putting the label of an hyperedge $h$ inside its corresponding square.
\begin{example}\label[example]{lab_1}
	The simplest example is given by the identity $\id{\mathcal{G}^\Sigma}\colon \mathcal{G}^\Sigma\rightarrow \mathcal{G}^{\Sigma}$. If $\Sigma$ is the signature of groups $\Sigma_G$we get 
	\begin{center}
		\begin{tikzpicture}
			\node[circle,fill=black,inner sep=0pt,minimum size=6pt,label=above:{$\heartsuit$}] (V) at (0,0) {};
			\node(E)at(-2, 0.4){$e$};
			\node(M)at(0, 2.15){$\cdot$};
			\node(I)at(2, 0.75){$(-)^{-1}$};
			
			\node(E')at(-2, 0){$e$};
			\node(M')at(0, 1.75){$\cdot$};
			\node(I')at(2, 0){$(-)^{-1}$};
			\draw[->-=.5](-1.75,0)--(V)node[pos=0.5, above,font=\fontsize{7}{0}\selectfont]{$1$};
			\draw[->-=.5](V)..controls(-0.5,0.5)and(-0.8,1)..(-0.25,1.6)node[pos=0.5, right,font=\fontsize{7}{0}\selectfont]{$2$};
			\draw[->-=.5](V)..controls(-1,0.6)and(-1,1.1)..(-0.25,1.9)node[pos=0.5, left,font=\fontsize{7}{0}\selectfont]{$1$};
			\draw[->-=.5](0.25,1.75)..controls(0.8,0.8)and(0.5,0.5)..(V)node[pos=0.5, right,font=\fontsize{7}{0}\selectfont]{$1$};
			\draw[->-=.5](V)--(1.5,0)node[pos=0.5, above,font=\fontsize{7}{0}\selectfont]{$1$};
			\draw[->-=.5](2.5,0)..controls(4,0)and(2.5,-2)..(V)node[pos=0.5, below,font=\fontsize{7}{0}\selectfont]{$2$};
			\draw[rounded corners] (-2.25, -0.25) rectangle (-1.75, 0.25) {};
			\draw[rounded corners] (-0.25, 1.5) rectangle (0.25, 2) {};
			\draw[rounded corners] (2.5, -0.5) rectangle (1.5, 0.5) {};
		\end{tikzpicture}
	\end{center}
\end{example}

\begin{example}\label[example]{lab_2}
	Take again $\Sigma_G$ the signature of groups, then the hypergraph $\mathcal{G}$ of \Cref{exa_2} can be labeled defining
	\begin{align*}
		e=f(h_1) \quad \cdot=f(h_2)\quad \cdot=f(h_3)
	\end{align*}
	In this case we get the following picture
	
	\begin{center}\begin{tikzpicture}
			\node[circle,fill=black,inner sep=0pt,minimum size=6pt,label=above:{$v_1$}] (A) at (0,0) {};
			\node[circle,fill=black,inner sep=0pt,minimum size=6pt,label=above:{$v_2$}] (B) at (0,-1.5) {};
			\node[circle,fill=black,inner sep=0pt,minimum size=6pt,label=above:{$v_3$}] (C) at (3,-0.75) {};
			\node[circle,fill=black,inner sep=0pt,minimum size=6pt,label=above:{$v_4$}] (D) at (3,-2.25) {};
			\node[circle,fill=black,inner sep=0pt,minimum size=6pt,label=right:{$v_5$}] (E) at (6,-1.5) {};
			\draw[->-=.5] (1.75,-0.75)--(C)node[pos=0.5, above,font=\fontsize{7}{0}\selectfont]{$1$};
			\draw[rounded corners] (1.25, -1) rectangle (1.75, -0.5) {};
			\draw[->-=.5] (4.75,-1.5)--(E)node[pos=0.5, above,font=\fontsize{7}{0}\selectfont]{$1$};
			\draw[->-=.5] (-1.5,0)--(A)node[pos=0.5, above,font=\fontsize{7}{0}\selectfont]{$1$};
			\draw[rounded corners] (4.25, -1.75) rectangle (4.75, -1.25) {};
			\node at (4.5, -1.05){$h_3$};
			\node at (1.5, -0.3){$h_2$};
			\node at (-1.75, 0.45){$h_1$};
			\draw[rounded corners] (-2, -0.25) rectangle (-1.5, 0.25) {};
			\draw[->-=.5] (A)..controls(0.5,0)and(1.2,-0.2)..(1.25,-0.6)node[pos=0.5, above,font=\fontsize{7}{0}\selectfont]{$1$};
			\draw[->-=.5] (B)..controls(0.5,-1.5)and(1.2,-1.3)..(1.25,-0.9)node[pos=0.5, below,font=\fontsize{7}{0}\selectfont]{$2$};
			
			\draw[->-=.5] (C)..controls(3.5,-0.75)and(4.2,-0.95)..(4.25,-1.35)node[pos=0.5, above,font=\fontsize{7}{0}\selectfont]{$1$};
			\draw[->-=.5] (D)..controls(3.5,-2.25)and(4.2,-2.05)..(4.25,-1.65)node[pos=0.5, below,font=\fontsize{7}{0}\selectfont]{$2$};
			
			\node at (-1.75,0) {$e$};
			\node at (1.5,-0.75) {$\cdot$};
			\node at (4.5,-1.5) {$\cdot$};
		\end{tikzpicture}
	\end{center}
\end{example}


\subsection{Term Graphs}
\todo{a very nice introduction}

Let us start using labelled hypergraphs to define term graphs.

\begin{definition}\label[definition]{def:tg}\index{term graph}
	Given an algebraic signature $\Sigma$, we say that a labelled hypergraph $(l, !_{V_\mathcal{G}})\colon \mathcal{G}\to \mathcal{G}^{\Sigma}$ is a \emph{term graph} if $t_\mathcal{G}$ is mono. We define $\tg$ to be the full subcategory of $\hyp_{\Sigma}$ and denote by $I_\Sigma$ the inclusion. Restricting $U_\Sigma\colon \hyp_{\Sigma}\to \catname{Set}$ we get a forgetful functor $U_{\tg}\colon \tg\to \catname{Set}$.
\end{definition}


\begin{remark}\label[remark]{rem:mono2}By \Cref{rem:label}, we know that if $\mathcal{G}$ is a term graph then $t_{\mathcal{G}}=v_1\circ \tau_{\mathcal{G}}$, where $v_1$ is the coprojection of $V_{\mathcal{G}}$ into $V^\star_{\mathcal{G}}$.  Notice that since $t_{\mathcal{G}}$ is mono then $\tau_{\mathcal{G}}$ is mono too.
\end{remark}

We now examine some properties of $\tg$, in order to study its adhesivity properties.

\begin{proposition}\label[proposition]{term:left}The forgetful functor $U_{\tg}$ has a left adjoint $\Delta_{\tg}$.
\end{proposition}
\begin{proof}
	This follows noticing that $\Delta_{\Sigma}(X)$ is a term graph for every object $X$.
\end{proof}

We can list some categorical properties of $\tg$


\begin{proposition}
Let $\Sigma$ be an algebraic signature. Then it holds
\begin{enumerate}
	\item if  $(i,j)\colon \mathcal{H}\to \mathcal{G}$ is a mono between  $(l, !_{V_\mathcal{G}})\colon \mathcal{G}\to \mathcal{G}^{\Sigma}$ and $(l', !_{V_\mathcal{H}})\colon \mathcal{H}\to \mathcal{G}^{\Sigma}$ in $\hyp_\Sigma$ and the latter is a term graph, then also the former is in $\tg$
	\item $\tg$ has equalizers, binary products and pullbacks and they are created by $I_\Sigma$.
\end{enumerate}
\end{proposition}

\begin{remark}
	$\tg$ in general does not have terminal objects. 
	%Consider an algebraic signature in $\Set$. 
	Since $U_{\tg}$ preserves limits, if a terminal object exists it must have the singleton as set of nodes, therefore the set of hyperedges must be empty or a singleton. 
	Hence, for a counterexample, it suffices to take as signature the one given by two operations $a$ and $b$, both of arity $0$.
	$\tg$ is not an adhesive category, either. 
	In particular, as noted in e.g.~\cite{CastelnovoGM24}, 
	 it does not have pushouts along all monos. 
\end{remark}
	
\commentato{
\begin{remark}
	$\tg$ in general does not have terminal objects. Consider an algebraic signature in $\Set$. Since $U_{\tg}$ preserves limits, if a terminal object exists it must have the singleton as set of nodes, therefore the set of hyperedges must be empty or a singleton $\{h\}$. Hence, for a counterexaqmple it suffices to take as signature the one given by two operations $a$ and $b$, both of arity $0$; we have three term graphs with only one node $v$: $\Delta_{\tg{\Sigma}}(\{v\})$, $(l_a, !_{V_{\mathcal{G}}})\colon \mathcal{G}_a\to \mathcal{G}^{\Sigma}$ and $(l_b, !_{V_{\mathcal{G}}})\colon \mathcal{G}_b\to \mathcal{G}^{\Sigma}$.
	\begin{center}\begin{tikzpicture}
			
			\node[circle,fill=black,inner sep=0pt,minimum size=6pt,label=left:{$v$}] (V) at (5,0) {};
			\node[circle,fill=black,inner sep=0pt,minimum size=6pt,label=left:{$v$}] (U) at (3,0) {};
			\node at(5,1.25){$a$};	
			\node at(5,1.7){$h$};	
			
			\draw[->-=.5](5,1)--(V)node[pos=0.5, right,font=\fontsize{7}{0}\selectfont]{$1$};
			
			\draw[rounded corners] (4.75, 1) rectangle (5.25, 1.5) {};
			
			
			\node[circle,fill=black,inner sep=0pt,minimum size=6pt,label=left:{$v$}] (V) at (7,0) {};
			
			\node at(7,1.25){$b$};	
			\node at(7,1.7){$h$};	
			
			\draw[->-=.5](7,1)--(V)node[pos=0.5, right,font=\fontsize{7}{0}\selectfont]{$1$};
			
			\draw[rounded corners] (6.75, 1) rectangle (7.25, 1.5) {};
			
		\end{tikzpicture}
	\end{center}
	There are no morphisms in $\tg$ between the last two and from the last two to the first one, therefore none of them can be terminal.
\end{remark}

\begin{remark}
	$\tg$ is not an adhesive category. In particular it does not have pushouts along all monos. For instance, if we take the three term graphs of the previous remark, then have two arrows
	$(?_{\{h\}}, \id{\{v\}})\colon \Delta_{\tg}(\{v\})\to (l_a, !_{V_{\mathcal{G}_a}})$ and $(?_{\{h\}}, \id{\{v\}})\colon \Delta_{\tg}(\{v\})\to (l_b, !_{V_{\mathcal{G}_a}})$ which cannot be completed to a square. Indeed if $(q, !_{V_\mathcal{H}})\colon \mathcal{H}\to \mathcal{G}^\Sigma$ is another term graph with $(g_E, g_V)\colon (l_a, !_{V_{\mathcal{G}}})\to (q, !_{V_\mathcal{H}})$ and $(k_E, k_V)\colon (l_a, !_{V_{\mathcal{G}}})\to (q, !_{V_\mathcal{H}})$  such that 
	\[(g_E, g_V)\circ (?_{\{h\}}, \id{\{v\}}) = (k_E, k_V)\circ (?_{\{h\}}, \id{\{v\}})\]
	then $g_V=k_V$ and
	\[t_{\mathcal{H}}(g_E(h))=g^\star_V(t_{\mathcal{G}}(h))=g_V^\star(\delta_v)=k^\star_V(\delta_V)=k^\star_V(t_{\mathcal{G}}(h))=t_{\mathcal{H}}(k_E(h))\]
	so that we also have $g_E=k_E$, but then
	\[
	a=l_a(h)=q(g_E(h))=q(k_E(h))=l_b(h)=b\]
\end{remark}
}

\begin{definition}
	Let $(l, !_{V_{\mathcal{G}}})\colon \mathcal{G}\to \mathcal{G}^{\Sigma}$  be a term graph. A \emph{input node} is an element of $V_{\mathcal{G}}$ not in the image of $\tau_{\mathcal{H}}$.  A morphism $(f,g)$ between
	Let $(l, !_{V_{\mathcal{G}}})\colon \mathcal{G}\to \mathcal{G}^{\Sigma}$ and $(l, !_{V_{\mathcal{H}}})\colon \mathcal{H}\to \mathcal{G}^{\Sigma}$ in $\tg$, is said to \emph{preserve input nodes} if $g$ sends input nodes to input nodes.
\end{definition}

\todo{Se questo remark sotto non serve nel pezzo sulle equivalenze possiamo toglierlo}
\begin{remark}\label{prop:image}
	Suppose that $(f,g)\colon ((l, !_{V_{\mathcal{G}}}))\to (l', !_{V_{\mathcal{H}}})$ preserves input nodes. Then  if $\tau_{\mathcal{H}}(h)=g(v)$ for some $v\in V_{\mathcal{G}}$ then $h$ belongs to the image of $f$. Indeed, by hypothesis $v$ must be in the image of $\tau_{\mathcal{G}}$ and so there exists $k$ such that $\tau_{\mathcal{G}}(k)=v$. But then $\tau_{\mathcal{H}}(f(k))=g(v)$ and we can conclude that $f(k)=h$.
\end{remark}

Preservation of inputs characterizes regular monos in $\tg$.

\begin{proposition}\label[proposition]{lem:reg} Let $(i,j)$ be a mon between two term graphs  $(l, !_{V_{\mathcal{G}}})\colon \mathcal{G}\to \mathcal{G}^{\Sigma}$ and  $(l', !_{V_{\mathcal{H}}})\colon \mathcal{H}\to \mathcal{G}^{\Sigma}$. Then it is a regular mono if and only if it preserves the input nodes.
\end{proposition}

This characterization, in turn, provides us with the following result \cite{CastelnovoGM24,castelnovo2023thesis}. 

\begin{lemma}\label[lemma]{prop:push} Consider three term graphs $(l_0, !_{V_\mathcal{G}})\colon \mathcal{G}\to \mathcal{G}^{\Sigma}$, $(l_1, !_{V_\mathcal{H}})\colon \mathcal{H}\to \mathcal{G}^{\Sigma}$ and $(l_2, !_{V_\mathcal{K}})\colon \mathcal{K}\to \mathcal{G}^{\Sigma}$. Given $(f_1, g_1)\colon (l_0, !_{V_\mathcal{G}})\to (l_1, !_{V_\mathcal{H}})$, $(f_2, g_2)\colon (l_0, !_{V_\mathcal{G}})\to (l_2, !_{V_\mathcal{K}})$, if $(f_1, g_1)$ is a regular mono, then its pushout along $(f_2, g_2)$,  then their pushout $(p, !_{V_{\mathcal{P}}})\colon \mathcal{P}\to \mathcal{G}^{\Sigma}$ in $\hyp_{\Sigma}$ is a term graph too.
\end{lemma}


\Cref{thm:slice-functors}, \Cref{lem:reg} and \Cref{prop:push} allow us to recover the following result, previously proved by direct computation in \cite[Thm.~4.2]{CorradiniG05}.
\begin{corollary}
	The category $\tg$ is quasiadhesive.
\end{corollary}



\section{Hypergraphs with equivalences}

\begin{definition}
	A \emph{hypergraph with equivalence} $\mathcal{G} = (E_\mathcal{G}, V_{\mathcal{G}}, C_\mathcal{G}, s_\mathcal{G}, t_\mathcal{G}, q_\mathcal{G})$ is a 6-tuple such that $\mathcal{G} = (E_\mathcal{G}, V_{\mathcal{G}}, s_\mathcal{G}, t_\mathcal{G})$ is a hypergraph, $C_\mathcal{G}$ is the object of \emph{equivalence classes} and $q_{\mathcal{G}}: V_{\mathcal{G}}\to C_{\mathcal{G}}$ is an epimorphism called \emph{quotient map}. A morphism $h:\mathcal{G\to H}$ is a triple $(h_E, h_V, h_C)$ such that the following diagrams commute.
	\[\xymatrix{
		{E_\mathcal{G}}\ar[r]^{s_\mathcal{G}}\ar[d]_{h_E} & {V_{\mathcal{G}}^\star}\ar[d]^{h_V^\star} & {E_\mathcal{G}}\ar[r]^{t_\mathcal{G}}\ar[d]_{h_E} & {V_{\mathcal{G}^\star}}\ar[d]^{h_V^\star} \\
		{E_\mathcal{H}}\ar[r]^{s_\mathcal{H}} & {V_{\mathcal{H}}^\star}	 			& {E_\mathcal{H}}\ar[r]^{t_\mathcal{H}} & {V_{\mathcal{H}^\star}} \\
						      & {V_\mathcal{G}}\ar[r]^{q_\mathcal{G}}\ar[d]_{h_V} & {C_{\mathcal{G}}} \ar[d]^{h_C} \\
						      & {V_\mathcal{H}}\ar[r]^{q_\mathcal{H}} & {C_\mathcal{H}}
	}\]
	The category of hypergraphs with equivalences and their morphisms is denoted $\EqHyp$.

\end{definition}

\begin{remark}\label{rem:eqhyp_morphs}
	Morphisms of hypergraphs with equivalences are uniquely determined by the first two components. That is, if $h_1 = (h_E, h_V, f)$ and $h_2 = (h_E, h_V, g)$ are two morphisms $\mathcal{G \to H}$, then we have
	\[\xymatrix{
			{V_\mathcal{G}} \ar[r]^{h_V}\ar[d]_{q_\mathcal{G}} & V_\mathcal{H} \ar[d]^{q_\mathcal{H}} & V_\mathcal{G}\ar[l]_{h_V}\ar[d]^{q_\mathcal{G}}\\
			C_{\mathcal{G}}\ar[r]_{f} & C_{\mathcal{H}} & C_{\mathcal{G}}\ar[l]^{g}
	}\]
	Hence,
	\begin{align*}
		f \circ q_\mathcal{G} &= q_\mathcal{H}\circ h_V \\ &=g\circ q_\mathcal{G}
	\end{align*}
	Since $q_\mathcal{G}$ is epi, we obtain $f = g$.
\end{remark}

$\EqHyp$ has a forgetful functor $U_{\EqHyp}:\EqHyp \to \Set$, which sends each $\mathcal{G} = (E_\mathcal{G}, V_{\mathcal{G}}, C_\mathcal{G}, s_\mathcal{G}, t_\mathcal{G}, q_\mathcal{G})$ into $V_\mathcal{G}$, and each $h = (h_E, h_V, h_C)$ onto $h_V$. 

\begin{proposition}
	$U_\EqHyp$ has a left adjoint $\Delta_{\EqHyp}: \Set \to \EqHyp$.
\end{proposition}

\begin{proof}
	For each set $X$, define $\Delta_\EqHyp(X):= (\emptyset, X, \{\bullet\}, ?_X, ?_X, !_X)$. Consider now $h: \Delta_{\EqHyp}(X) \to \mathcal{H}$.
	\[\xymatrix@C=2.3cm{
			\Delta_{\EqHyp}(X) \ar@{.>}[d]_{\Delta_\EqHyp(f)} \ar[dr]^{h} & \\
			\Delta_\EqHyp(U_{\EqHyp}(\mathcal{H})) \ar[r]_{\epsilon_{\mathcal{H}}} & \mathcal{H}
	}\]
	Where $\Delta_\EqHyp(U_{\EqHyp}(\mathcal{H})) = (\emptyset, V_\mathcal{H}, \{\bullet\}, ?_{V_\mathcal{H}}, ?_{V_\mathcal{H}}, !_{V_\mathcal{H}})$ and $\epsilon_{\mathcal{H}} = (?_{E_\mathcal{H}}, id_{V_\mathcal{H}}, g)$.
	Note that, since $\Delta_{\EqHyp}(X)$ has the empty set as object of edges, $h_E = ?_{E_\mathcal{H}}$, then, the unique arrow that fits in the diagram is $\Delta_{\EqHyp}(f) = (?_{E_\mathcal{H}}, h_V, id_{\{\bullet\}})$.

\end{proof}




\subsection{Labeled Hypergraphs with Equivalences}

As we have done in \Cref{sssect:hyp_alg_sign}, we can define the category of hypergraphs with equivqlence labeled over an algebraic signature.

\begin{definition}
	Let $\Sigma = (O_\Sigma, \ari_\Sigma)$ be an algebraic signature, and let $\mathcal{G}^{\Sigma}$ the hypergraph associated to $\Sigma$.
	Then, the \emph{hypergraph with equivalence} associated to $\Sigma$ is $L(\mathcal{G}^\Sigma)$, and the category of hypergraphs with equivalence labeled over $\Sigma$ is the 
	slice category $\EqHyp_\Sigma := \EqHyp/L(\mathcal{G}^\Sigma)$.
\end{definition}

By \Cref{thm:slice_functors}, we can deduce the following.

\begin{proposition}
	$\EqHyp_\Sigma$ is $\reg (\EqHyp_\Sigma)$-adhesive.
\end{proposition}




\subsubsection{Term Graphs with Equivalences}

\begin{definition}
	Let $\Sigma$ be an algebraic signature.
	A labeled hypergraph with equivalence $l: \mathcal{G} \to L(\mathcal{G}^\Sigma)$ is a \emph{term graph with equivalence} if $t_{\mathcal{G}}$ is mono.
	We define category of term graphs with equivalence over $\Sigma$, denoted $\EqTG_{\Sigma}$, as the full subcategory of $\EqHyp_\Sigma$,
	and the corresponding inclusion functor $I_{\EqTG_\Sigma}$.
\end{definition}

\begin{proposition}
	If $l: \mathcal{G \to G}^\Sigma$ is a term graph, then $L(l)$ is a term graph with equivalence.
\end{proposition}


\section{EGGs}

\todo{introduction}

\begin{definition}
	Let $\mathcal{G} = (E, V, C, s, t, q)$ be a hypergraph with equivalence, and let $(S, \pi_1, \pi_2)$ be the kernel pair of $q^\star \circ s$.
	Then, $\mathcal{G}$ is an \emph{e-hypergraph} whenever $q^\star \circ t \circ \pi_1 = q^\star \circ t \circ \pi_2$.
	$\egg$ is the full subcategory of $\EqHyp$ where objects are e-hypergraphs, and $I: \egg \to \EqHyp$ is the inclusion functor.
\end{definition}

\begin{lemma}
	$\egg$ has all limits, and $I$ preserves them.
\end{lemma}

\begin{proof}
        Let $D: \cat{I} \to \egg$ be a diagram, with $D(i) = (A_i, B_i, C_i, s_i, t_i, q_i)$, let $(U_i, u_1^i, u_2^i)$ be the kernel pair of $q_i\circ s_i$.
	Let now be $(A, B, C, s, t, q)$, togehter with projections $(\pi_E^i, \pi_V^i, \pi_C^i)_{i \in \cat I}$ the limit of $I \circ D$,
	let $(U, u_1, u_2)$ be the kernel pair of $q\circ s$ and let $(L, (l_i)_{i \in \cat I})$ be the limit of $K \circ I \circ D$.
        By construction (proof of \Cref{prop:eqhyp_complete}), there exists a monomorphism $m: C \to L$ such that $\pi_C^i = l_i \circ m$. Notice that
        \begin{align*}
                q_i^\star \circ s_i\circ \pi^i_E\circ u_1      &= q_i^\star\circ (\pi^i_V)^\star \circ s\circ u_1\\
                                                        &= (\pi_C^i)^\star\circ q^\star\circ s\circ u_1\\
                                                        &=(\pi_C^i)^\star\circ q^\star\circ s\circ u_2\\
                                                        &= q_i^\star \circ s_i \circ \pi_E^i \circ u_2
        \end{align*}
        Then, for each $i$, there exists an arrow $a_i:U\to U_i$ making the following diagram to commute
	\[
                \xymatrix{
			U \ar[r]^{u_1} \ar[d]_{u_2} \ar@{.>}[dr]_{a_i} & A \ar[drr]^{\pi_E^i} && \\
			A \ar[ddr]_{\pi_E^i} & U_i \ar[rr]_{u_1^i} \ar[dd]^{u_2^i}&& A_i \ar[d]^{s_i} \\
					     & & & B_i \ar[d]^{q_i}\\
					     & A_i \ar[r]_{s_i} & B_i \ar[r]_{q_i}& C_i
                }
        \]
        We have then
        \begin{align*}
                l_i^\star\circ m^\star \circ q^\star \circ t \circ u_1    &= q_i^\star\circ (\pi_V^i)^\star \circ t \circ u_1 \\
                                                        &= q_i^\star \circ t_i \circ \pi_E^i \circ u_1 \\
                                                        &= q_i^\star \circ t_i \circ u_1^i \circ a_i \\
                                                        &= q_i^\star \circ t_i \circ u_2^i \circ a_i \\
                                                        &= q_i^\star \circ t_i \circ \pi_E^i \circ u_2 \\
                                                        &= q_i^\star \circ (\pi_V^i)^\star \circ t \circ u_2 \\
                                                        &= l_i^\star \circ m^\star \circ q^\star \circ t \circ u_2
        \end{align*}
        By universal property of limits, we have that \- $m^\star\circ q^\star \circ t \circ u_1 = m^\star \circ q^\star \circ t \circ u_2$, and, since $m$ is mono, $q^\star \circ t \circ u_1 = q^\star \circ t \circ u_2$, hence the thesis.
\end{proof}

\begin{corollary}
	$I$ creates limits.
\end{corollary}

\begin{corollary}
	$h: \mathcal{G \to H}$ is a regular mono in $\egg$ if and only if it is a regular mono in $\EqHyp$.
\end{corollary}

\begin{lemma}
	Consider the following pushout square in $\EqHyp$.
	\[\xymatrix{\mathcal{G}_1 \ar[r]^{h}\ar[d]_{m}&\mathcal{G}_2\ar[d]^{n}\\\mathcal{G}_3\ar[r]_{k}&\mathcal{P}}\]
with $m$ regular mono. If $\mathcal{G}_1$, $\mathcal{G}_2$ and $\mathcal{G}_3$ are e-hypergraphs, then $\mathcal{P}$ is an e-hypergraph too, and $n$ is regular mono.
\end{lemma}

\begin{proof}
	Let $\mathcal{P} = (A, B, C, s, t, q)$, $(K_i, \pi_i^1, \pi_i^2)$ the kernel pair of $q_i^\star \circ s_i$, and let $(U, u_1, u_2)$ the kernel pair of $q^\star \circ s$ .
	Consider then the following situation.
	\[\xymatrix@C=10pt@R=6pt{&A_1\ar[dd]|\hole_(.65){q_1^\star \circ s_1}\ar[rr]^{h_E} \ar[dl]_{m_E} && A_2 \ar[dd]^{q_2^\star \circ s_2} \ar[dl]_{k_E} \\ A_3 \ar[dd]_{q_3^\star \circ s_3}\ar[rr]^(.7){k_E} & & A \ar[dd]_(.3){q^\star \circ s}\\&{C_1^\star}\ar[rr]|\hole^(.65){h_C^\star}\ar[dl]^{m_C^\star} && {C_2^\star} \ar[dl]^{n_C^\star} \\{C_3^\star} \ar[rr]_{k_C^\star} & & C}\]
	Since $m$ is regular mono, $m_C$ is mono ({\color{red}{inserire citazione}}). Then, by adhesivity of $\Set$, the bottom face is a Van Kampen square, hence a pushout. Therefore, by ({\color{red}{citazione}}), the square below is a pushout.
	\[\xymatrix{K_1\ar[r]^{f_k}\ar[d]_{f_m}&K_2\ar[d]^{f_n}\\K_3\ar[r]_{f_k}&U}\]
       Computing, we have
        \[
                \begin{split}
                        q^\star \circ t \circ u_1 \circ f_n &= q^\star \circ t \circ n_E \circ \pi_2^1 \\ &= n_C^\star \circ q_2^\star \circ s_2 \circ \pi_2^1 \\&= n_C^\star \circ q_2^\star \circ s_2 \circ \pi_2^2 \\&=q^\star \circ t \circ u_2 \circ f_n 
                        \end{split}
                        \qquad
                        \begin{split} q^\star \circ t \circ u_1 \circ f_k &= q^\star \circ t \circ k_E \circ \pi_3^1 \\ &= k_C^\star \circ q_3^\star \circ s_3 \circ \pi_3^1 \\&= k_C^\star \circ q_3^\star \circ s_3 \circ \pi_3^2 \\&=q^\star \circ t \circ u_2 \circ f_k
                        \end{split}
        \]
        By universal property of pushouts, we deduce $q^\star \circ t \circ u_1 = q^\star \circ t \circ u_2$, and the thesis follows.
\end{proof}

By direct application of \Cref{thm:slice-functors}, we can conclude what follows.

\begin{corollary}
	$\egg$ is $\reg(\egg)$-adhesive.
\end{corollary}


\section{Conclusions and further works}
The aim of our paper was to extend the theory of adhesive categories in order to include EGGS, 
an up-and-coming formalism for program optimisation and synthesis via a compact representation and 
efficient implementation of equality saturation.
%
To do so, we revisited and generalised the notions of hyper-graph and term graph with equivalence, and
we extended it in order to capture EGGS as term graphs satisfying a suitable closure property.

Our result opens two threads of research. The first is to use the quasi-adhesivity of EGGs to
model their rewriting via the double-pushout (DPO) approach. This seems now easy, since 
the rules adopted in the literature of EGGs appears to be span of regular monos, and such rules
perfectly fit the mold of rewriting on $\mathcal{M}$-adhesive categories. 
For example, the equivalence $x \div x = 1$, from the introductory example in~\cite{xxx}, 
can be modelled as the rule
\[to draw DPO rule\]

It still needs to be investigated what parallelism and termination, the key properties for DPO rewriting
on adhesive categories, 
mean in the context of EGGs. More interestingly, another venue for development is using the adhesive 
machinery to extend the EGGs formalism. In fact, most of the results
presented here for hyper-graphs can be generalised to hierarchical hyper-graphs, that is, 
hypergraphs with a hierarchy (a partial order) among 
edges that is useful for adding structural information, such as encapsulation and sandboxing~\cite{xxx}.

Finally, we need to draw a comparison with an alternative categorical presentation for EGGs 
advanced in~\cite{ghica}. The proposal is quite different from our own. Simplifying, the key is to equip categories 
of trees with a lattice on hom-sets. 
It seems that such proposal generalises our own, 
even if at the expenses of a more complex machinery.
%Instead of going through 
%such an enrichment, we chose to obtain closure via suitable requirements on morphisms, even if it 
%seems that the former generalises the latter, at the expenses of a more complex machinery.

\bibliography{biblio}

\appendix

\section{Omitted proofs}

This section contains the proofs which are omitted from the main body of the paper. 
%
We begin recalling  a well-known fact about composition and decomposition of pullbacks \cite{mac2013categories}.

\noindent
\parbox{10cm}{
\begin{lemma}\label{lem:pb1}
	Let $\X$ be a category, and consider the diagram aside, in which the right square is a pullback. Then the whole rectangle is a pullback if and only if the left square is one.
\end{lemma}} 
	\parbox{4cm}{
	\xymatrix@R=18pt{X \ar[d]_{a} \ar[r]^{f}& \ar[r]^{g} Y \ar[d]^{b}& Z \ar[d]^{c}\\ A \ar[r]_{h}& B \ar[r]_{k}& C}}
\commentato{
\begin{proof}
	$(\Rightarrow)$ Let $q_1\colon Q\to Y$ and $q_2\colon Q\to A$ be two arrows such that $b\circ q_1=h\circ q_2$, if we compute we get
	\[c\circ g\circ q_1  =k\circ b\circ q_1=k\circ h\circ q_2\]
	\parbox{9cm}{Thus by the pullback property of the whole rectangle we get the dotted $l$ in the diagram on the side. 
	All we have to prove is that $f\circ l=q_1$. By construction we know that $g\circ f\circ l= g\circ q_1$, while we also have}
	\hfill \parbox{4cm}{\xymatrix@R=18pt{Q \ar@/_.3cm/[dr]_{q_2}\ar@/^.4cm/[rrr]^{g\circ q_1} \ar@{.>}[r]_{l} & X \ar[d]_{a} \ar[r]_{f}& \ar[r]_{g} Y \ar[d]^{b}& Z \ar[d]^{c}\\ & A \ar[r]_{h}& B \ar[r]_{k}& C}}
	\[	b\circ f\circ l = h\circ a\circ l= h\circ q_2=b\circ q_1\]
	and we can conclude since the right square in the original diagram is a pullback.
	
	For uniqueness: if $l'\colon Q\to X$ is such that $f\circ l'= q_1$ and $a\circ l'=q_2$
	then  $g\circ f \circ l' = g\circ q_1$ and we can conclude applying the pullback property of the outer rectangle.
	
	\smallskip \noindent 
	\parbox{9cm}{
	$(\Leftarrow)$ Take two arrows $q_1\colon Q\to Z$ and $q_2\colon Q\to A$ such that $c\circ q_1=k\circ h \circ q_2$
	We can apply the pullback property of the right square to get the dotted $q\colon Q\to Y$ in the following 

	Now, by construction we have $b\circ q =h\circ q_2$
	and thus, since the left square is a pullback, we get also a unique $l\colon Q\to X$ such that  $f\circ l = q$ and $a \circ l = q_2$
	but then we clearly have} \hfill 
	\parbox{6cm}{\vspace{7pt}
		\xymatrix@R=18pt{Q \ar@/_.3cm/[ddr]_{q_2}\ar@/^.5cm/[drrr]^{ q_1} \ar@{.>}@/^.3cm/[drr]^(.6){ q} \ar@{.>}[dr]^{l}  \\& X \ar[d]_{a} \ar[r]^{f}& \ar[r]^{g} Y \ar[d]_{b}& Z \ar[d]^{c}\\ & A \ar[r]_{h}& B \ar[r]_{k}& C}}
	\[	g\circ f \circ l = g\circ q = q_1	\] 
	
	We are left with uniqueness. Let $l'\colon Q\to X$ be another arrow such that $q_1 = g\circ f \circ l'$ and $q_2=a\circ l'$,  then we must also have
	\[
	b\circ f \circ l'= h\circ a \circ l' = h\circ q_2= b\circ q\]
	which implies $f\circ l' = q$, from which $l=l'$ follows.
\end{proof}
}
%
\commentato{
\subsection{Proofs for \Cref{sec:ade}}
\regmono*
\begin{proof}\label{regmono-proof}
	\begin{enumerate}
		\item  Consider the following cube in which the bottom face is an $\mathcal{M}$-pushout.
		\[\xymatrix@C=15pt@R=9pt{&A\ar[dd]|\hole_(.65){\id{A}}\ar[rr]^{g} \ar[dl]_{\id{A}} && B \ar[dd]^{\id{B}} \ar[dl]_(.6){\id{B}} \\ A  \ar@{>->}[dd]_{m}\ar[rr]^(.65){g} & & B \ar@{>->}[dd]_(.3){n}\\&A\ar[rr]|\hole^(.65){g} \ar@{>->}[dl]^{m} && B \ar@{>->}[dl]^{n} \\C \ar[rr]_{f} & & D}\]
		By construction the top face of the cube is a pushout and the back one a pullback. The left face is a pullback because $m$ is mono, thus the Van Kampen property yields that the front and the right faces are pullbacks too and the thesis follows.
		\item Let $m\colon X\mto Y$ be an arrow in $\mathcal{M}$, we can then take its pushout along itself, which, by the previous point, is also a pullback.
		\[\xymatrix{X \ar@{>->}[r]^{m} \ar@{>->}[d]_{m}& Y \ar@{>->}[d]^{h}\\ Y \ar@{>->}[r]^{k} & Z}\]
		It is now immediate to see that $m$ is the equalizer of $h$ and $k$.   \qedhere
	\end{enumerate}
\end{proof}
}

\kpp*
\begin{proof}\label{kpp-proof}
Computing, we have
\[g \circ h \circ \pi_f^1 =  t \circ f \circ \pi_f^1     =  t \circ f \circ \pi_f^2     =  g \circ h \circ \pi_f^2\]

Therefore the existence  and uniqueness of the wanted $k_h$ follows at once from the the universal property of $K_g$ as the pullback of $g$ along itself.

To prove the second half of the thesis, we can notice that, by \Cref{lem:pb1},  two rectangles below are pullbacks.
\[\xymatrix{K_f \ar[r]^{\pi^2_f} \ar[d]_{\pi^1_f} & X \ar[r]^{h} \ar[d]_{f} & Z \ar[d]^{g} &  K_f \ar[r]^{\pi^1_f} \ar[d]_{\pi^2_f} & X \ar[r]^{h} \ar[d]_{f} & Z \ar[d]^{g}\\
	X \ar[r]_{f}& Y \ar[r]_{t} & W & X \ar[r]_{f} & Y  \ar[r]_{t}& W}\]

But then the following ones are pullbacks too.
\[\xymatrix{	K_f  \ar@/^.4cm/[rr]^{h\circ\pi^2_f}\ar[r]_{k_h} \ar[d]_{\pi^1_f}& K_g \ar[d]_{\pi^1_g}  \ar[r]_{\pi^2_g}& Z \ar[d]_{g} & K_f  \ar@/^.4cm/[rr]^{h\circ\pi^2_f}\ar[r]_{k_h} \ar[d]_{\pi^1_f}& K_g \ar[d]_{\pi^1_g}  \ar[r]_{\pi^2_g}& Z \ar[d]_{g} \\
	X \ar[r]_{h} \ar@/_.4cm/[rr]_{t\circ f}& Y \ar[r]_{g} & W & X \ar[r]_{h} \ar@/_.4cm/[rr]_{t\circ f}& Y \ar[r]_{g} & W}\]

The thesis now follows again by \Cref{lem:pb1}.   
\end{proof}

\noindent
\parbox{7cm}{\mpo*}\parbox{6cm}{\xymatrix@C=10pt@R=6pt{&A'\ar[dd]|\hole_(.65){a}\ar[rr]^{f'} \ar@{>->}[dl]_{m'} && B' \ar[dd]^{b} \ar@{>->}[dl]_{n'} & K_a\ar[rr]^{k_{f'}} \ar[dd]_{k_{m'}}&& K_b \ar[dd]^{k_{n'}} \\ C'  \ar[dd]_{c}\ar[rr]^(.7){g'} & & D' \ar[dd]_(.3){d}\\&A\ar[rr]|\hole^(.65){f} \ar[dl]^{m} && B \ar[dl]^{n}  & K_{c} \ar[rr]_{k_{g'}}&& K_d\\C \ar[rr]_{g} & & D }}

\noindent 
\parbox{4cm}{\xymatrix@C=10pt@R=6pt{&K_a\ar[dd]|\hole_(.65){\pi^1_a}\ar[rr]^{k_{f'}} \ar@{>->}[dl]_{k_{m'}} && K_n\ar[dd]^{\pi^1_b} \ar@{>->}[dl]_{k_{n'}} \\ K_c  \ar[dd]_{\pi^1_c}\ar[rr]^(.7){k_{g'}} & & K_d \ar[dd]_(.3){\pi^1_d}\\&A'\ar[rr]|\hole^(.65){f'} \ar@{>->}[dl]^(.4){m'} && B' \ar@{>->}[dl]^{n'}  \\C' \ar[rr]_{g'} & & D' }} \hfill \parbox{9cm}{\begin{proof}\label{mpo-proof}
	By  \Cref{prop:regmono}  we know that the the top face of the original cube is a pullback. Thus \Cref{lemma:kern_pairs_pres_pullbacks} entails that in the following cube the vertical faces are pullbacks.
	The thesis now follows from strict $\mathcal{M}$-adhesivity.
	\end{proof}}
		

\commentato{
\epic*
\noindent 
\parbox{10cm}{
\begin{proof}\label{epic-proof}
	By hypothesis, there exists a pair $f, g\colon Z \rightrightarrows X$ of which $e$ is the coequalizer.
	 Since $e \circ f = e \circ g$ we get the dotted arrow fitting in the digram aside.

	\parbox{\textwidth}{Let now $h: Z \to V$ be an arrow such that $h \circ \pi_1 = h \circ \pi_2$, then
	\[	h \circ f = h \circ \pi_1 \circ k 
	= h \circ \pi_2 \circ k 
	= h \circ g\]
	and thus there exists a unique $l: Y \to V$ such that $l \circ e = h$. \qedhere}
\end{proof}}\hfill
\parbox{9cm}{\vspace{-2cm}\xymatrix{Z \ar@{.>}[r]_k \ar@/_.3cm/[dr]_{g} \ar@/^.4cm/[rr]^{f} & K_e \ar[r]_{\pi^{1}_e}  \ar[d]_{\pi^2_e}& X \ar[d]^{e}\\ & X \ar[r]_{e} & Y}} 

\natepi*
\begin{proof}\label{natepi-proof}
	Let $(K_i, \pi_d^1, \pi_d^2)$ be the kernel pair of $\phi_d$ for each object $d$ in $\D$. Given an arrow $\alpha: d \to d'$ of $\D$, we have
	\[\phi_{d'} \circ F(\alpha) \circ \pi_d^1 = G(\alpha) \circ \phi_d \circ \pi_d^1 
	= G(\alpha) \circ \phi_d\circ \pi_d^2 
	= \phi_{d'} \circ F(\alpha) \circ \pi_d^2
	\]
	
	Thus, the solid part of the diagram below commutes, yielding the dotted arrow $K(\alpha)$.
	
	\noindent 
\parbox{5cm}{\xymatrix{K_d \ar[r]^{\pi^1_d} \ar[d]_{\pi^2_d} \ar@{.>}[dr]^{K(\alpha)}& F(d) \ar[dr]^{F(\alpha)} \\ F(d) \ar[dr]_{F(\alpha)} & K_{d'} \ar[r]^{\pi^1_{d'}} \ar[d]_{\pi^2_{d'}}& F(d') \ar[d]^{\phi_{d'}}\\ & F(d') \ar[r]_{\phi_{d'}} & G(d')}} \hfill \parbox{9cm}{In this way, we get a functor $K\colon \D\to \X$ mapping $d$ to $K_d$ and each arrow $\alpha$ onto $K(\alpha)$. Indeed, notice that $K(\id{d}) \colon  K_d \to K_d$ is the arrow such that
\begin{align*}
\pi_d^1 \circ K(\id{d})&=F(\id{d})\circ \pi^1_d=\id{F(d)}\circ \pi^{1}_d=\pi^1_d\\
\pi_d^2 \circ K(\id{d})&=F(\id{d})\circ \pi^2_d=\id{F(d)}\circ \pi^{2}_d=\pi^2_d
\end{align*}
Thus, by the universal property of pullbacks, $K(\id{d}) = \id{K_d}$.
}

	Let now $\alpha \colon a \to b$ and $\beta\colon b \to c$ be two arrows in $\D$,  computing, we have
	\begin{align*}
		\pi_c^1 \circ K(\beta \circ \alpha) &
		= F(\beta) \circ F(\alpha) \circ \pi_a^1 
		= F(\beta) \circ \pi_b^1 \circ K(\alpha) 
		= \pi_c^1 \circ K(\beta) \circ K(\alpha)\\
			\pi_c^2 \circ K(\beta \circ \alpha) &
		= F(\beta) \circ F(\alpha) \circ \pi_a^2 
		= F(\beta) \circ \pi_b^2 \circ K(\alpha) 
		= \pi_c^2 \circ K(\beta) \circ K(\alpha)
	\end{align*}
	Allowing us to conclude that $K(\beta \circ \alpha) = K(\beta) \circ K(\alpha)$, proving the functoriality of $K$.
	
	
	Hence,  by construction we have two natural transformations $\pi^1, \pi^2\colon  K \rightrightarrows F$. By \Cref{prop:reg_epi_coeq}, every component $\phi_d$ is the coequalizer of $\pi_d^1, \pi_d^2\colon K(d) \rightrightarrows F$, and so $\phi$ is the coequalizer of $\pi^1$ and $\pi^2$.   
\end{proof}


\epicol*

\begin{proof}\label{epicol-proof}
	By \Cref{cor:reg_epi_components_reg_epi_nat_trans}, we know that $\phi\colon F \to G$ is a regular epi, so that there is a functor $E\colon \D\to \X$ and $\eta, \theta\colon E \rightrightarrows F$ such that $\phi$ is the coequalizer of $\eta$ and $\theta$. Let now $(P, \{p_d\}_{d \in \D})$ be the colimit of $E$ and consider the unique arrows $a, b\colon P \rightrightarrows X$ fitting in the squares below
\medskip

\noindent 
	\parbox{4cm}{\xymatrix{E(d) \ar[r]^{p_d} \ar[d]_{\eta_d}& P \ar@{.>}[d]^{a} &E(d) \ar[d]_{\theta_d}
		 \ar[r]^{p_d}& P \ar@{.>}[d]^{b}\\ F(d) \ar[r]_{x_d} & X & F(d) \ar[r]_{x_d} & X}} \hfill	\parbox{8cm}{
	We want to show that $\phi$ coequalizes $a$ and $n$. Let thus $h\colon X \to Z$ be an arrow such that $h \circ a = h \circ b$. Then, for every $d$, 
	we have
	\[h \circ x_d \circ \eta_d = h \circ a \circ p_d 
	= h \circ b \circ p_d 
	= h \circ x_d \circ \theta_d\]}
	
	\noindent
	Thus, there is $h_d: G(d) \to Z$ such that $h\circ x_d = h_d \circ \phi_d$. It is now easy to see that $(Z, \{h_d\}_{d \in \D})$ is a cocone on $G$. Suppose $\alpha: d \to d'$ is an arrow of $\D$, then
	\[h_d\circ \phi_d=h\circ y_d=h\circ y_{d'}\circ F(\alpha)=h_{d'}\circ \phi_{d'}\circ F(\alpha)=h_{d'}\circ G(\alpha)\circ \phi_d\]
		By the hypothesis $\phi_d$ is regular epi for each $d$ and so we can conclude that $h_d = h_{d'} \circ G(\alpha)$.
	
	Therefore, we have an arrow $k\colon Y \to Z$ such that $k \circ y_d = h_d$. But then
	\[
	k \circ \phi \circ x_d = k \circ y_d \circ \phi_d = h_d \circ \phi_d = h\circ x_d\]
	Showing that $k \circ \phi = h$.
	
	For the uniqueness, let $k': Y \to Z$ be another arrow such that $k' \circ \phi = h$. Then we have
	\[k'\circ y_d\circ \phi_d=k'\circ \phi \circ x_d=h\circ x_d=h_d\circ \phi_d\]
	
	Since $\phi_d$ is a regular epi, we have $k' \circ y_d = h_d$ allowing us to conclude.
\end{proof}
}

\section{Some properties of comma categories}
In this section we will briefly recall the definition of the comma category \cite{mac2013categories} associated to two functors and some of its properties.
\begin{definition}\index{category!comma -}
	Let $L\colon \A\to \X$ and  $R\colon \B\rightarrow \X$ be two functors with the same codomain, the \emph{comma category} $\comma{L}{R}$ is the category in which
	\begin{itemize}
		\item objects are triples $(A, B, f)$ with $A\in \A$, $B\in \B$, and $f\colon L(A)\rightarrow R(B)$; 
		\item a morphism $(A, B, f)\rightarrow (A', B', g)$ is a pair $(h, k)$ with $h\colon A\rightarrow A'$ in $\A$ and $k\colon B\rightarrow B'$ in $\B$ such that the following diagram commutes
		\[\xymatrix@C=25pt{L(A) \ar[r]^{L(h)} \ar[d]_{f} & L(A') \ar[d]^{g}\\ R(B) \ar[r]_{R(k)}& R(B')}\]
	\end{itemize}
\end{definition} 
We have two forgetful functors 	$U_L\colon \comma{L}{R}\to \A$ and $U_R\colon \comma{L}{R} \to \B$ given, respectively by
\[
\begin{split}
	\functor[l]{(A,B, f)}{(h,k)}{(A', B', g)}
	& \functormapsto
	\rfunctor{A}{h}{A'}
\end{split}\quad 
\begin{split}
	\functor[l]{(A,B, f)}{(h,k)}{(A', B', g)}
	& \functormapsto
	\rfunctor{B}{k}{B'}
\end{split}
\]


Given $L\colon \A\to \X$ and $R\colon \B\to \X$, we can also consider their duals $L^{op}\colon \A^{op}\to \X^{op}$ and $R^{op}\colon \B^{op}\to \X^{op}$.  An arrow $f\colon L(A)\to R(B)$ in $\X$ is the same ting as an arrow $f\colon R^{op}(B)\to L^{op}(A)$ in $\X^{op}$, thus $\left(\comma{L}{R}\right)$ and $\comma{R^{op}}{L^{op}}$ have the same objects. Moreover, the commutativity in  $\X$ of the square
\[\xymatrix@C=25pt{L(A) \ar[r]^{L(h)} \ar[d]_{f} & L(A') \ar[d]^{g}\\ R(B) \ar[r]_{R(k)}& R(B')}\]
is tantamount to the commutativity in $\X^{op}$ of the square 
\[\xymatrix@C=25pt{ R(B') \ar[d]_{g} \ar[r]^{R(k)}& R(B) \ar[d]^{f}\\ L(A') \ar[r]_{L(h)}  & L(A) }\]
Summing up we have just proved the following fact.
\begin{proposition}\label[proposition]{prop:dual}
	$(\comma{L}{R})^{op}$ is equal to	$\comma{R^{op}}{L^{op}}$, moreover $U^{op}_L=U_{L^{op}}$ and $U^{op}_R=U_{R^{op}}$.
\end{proposition}


\begin{lemma}\label[lemma]{colim}
	Let $L\colon \A\to \X$ and $R\colon \B\to \X$ be functors and $F\colon \D\to\comma{L}{R}$ be a diagram such that $L$ preserves colimits along $U_L\circ F$. Then the family $\{U_L, U_R\}$ jointly creates colimits of $F$ (see \cite{castelnovo2023thesis,CastelnovoGM24}).
\end{lemma}
\begin{proof}
	Suppose that $U_L\circ F$ and $U_R\circ F$ have colimiting cocones  $\left(A, \{a_D\}_{D\in \D}\right)$ and $\left(B, \{b_D\}_{D\in \D}\right)$ respectively. By hypothesis $\left(L(A), \{L\left(a_D\right)\}_{D\in \D}\right)$ is colimiting for $L\circ U_L\circ F$. Now, if we define
	\[F(D):=(A_D, B_D, f_D)\]
	then we have arrows $R(a_i)\circ f_D\colon L(A_D)\to R(B)$ that forms a cocone on $L\circ U_L\circ F$: if $d\colon D\to D'$ is an arrow in $\D$ then $F(d)$ is an arrow in $\comma{L}{R}$ and so
	\begin{align*}
		R\left(b_{D'}\right)\circ f_{D'}\circ L(U_L(F(d)))&=R\left(b_{D'}\right)\circ R\left(U_R\left(F(d)\right)\right)\circ f_D\\&=R\left(b_{D'}\circ U_R\left(F(d)\right)\right)\circ f_D\\&=R\left(b_D\right)\circ f_D
	\end{align*}
	Thus there exists $f\colon L(A)\rightarrow R(B)$ such that
	\[\xymatrix@C=35pt{ L(A_D) \ar[r]^{L(a_D)} \ar[d]_{f_D}& L(A) \ar[d]^{f}\\ R(B_D) \ar[r]_{R(b_D)} &R(B)}\]
	Notice that $f$ is the unique arrow in $\X$ wich makes $\left(a_D, b_D\right)$ an arrow $\left(A_D, B_D, f_D\right)\to \left(A, B, f\right)$ of $\comma{L}{R}$. If we show that $\left((A, B, f), \left\{(a_D, b_D)\right\}_{D\in \D}\right)$ is colimiting for $F$ we are done.
	
	First of all, let us show that it is a cocone. Given $d\colon D\to D'$ in $\D$ we have:
	\begin{align*}
		\left(a_{D'}, b_{D'}\right)\circ F(d)&=	\left(a_{D'}, b_{D'}\right)\circ \left(U_L(F(d)), U_R(F(d))\right)\\&=\left( a_{D'}\circ U_L(F(d)),  b_{D'}\circ U_R(F(d)) \right)\\&=\left(a_D, b_D\right)
	\end{align*}
	For the colimiting property, let $\left((X, Y, g), \left\{\left(x_D, y_D\right)\right\}_{D\in \D}\right)$ be another cocone on $F$. In particular $\left(X, \left\{x_D\right\}_{D\in \D}\right)$ and $\left(Y, \left\{y_D\right\}_{D\in \D}\right)$ are cocones on $U_L\circ F$ and $U_R\circ F$ respectively, so we have uniquely determined arrows $x\colon A\rightarrow X$ and $y\colon B\rightarrow Y$ such that 
	\[x\circ a_D= x_D \qquad y\circ b_D= y_D\]
	Let us show that $(x,y)$ is an arrow of $\comma{L}{R}$. Given $D\in \D$ we have
	\begin{align*}
		R(y)\circ f\circ L(a_D)&=R(y)\circ R\left(b_D\right)\circ f_D\\&=R\left(y\circ b_D\right)\circ f_D\\&=R\left(y_D\right)\circ f_D\\&= g\circ L\left(x_D\right)\\&=g\circ L\left(x\circ a_D\right)\\&=g\circ L(x)\circ L\left(a_D\right)
	\end{align*}
	from which it follows  that the following diagram commutes.
	\[\xymatrix@C=35pt{L(A)\ar[d]_{f} \ar[r]^{L(x)} & X \ar[d]^{g}\\ R(B) \ar[r]_{R(y)}& Y}\]
	This shows that $\left((A, B, f), \left\{(a_D, b_D)\right\}_{D\in \D}\right)$ is colimiting for $F$ and the thesis follows.    
\end{proof} 


\Cref{prop:dual} and \Cref{colim} now yields the following.
\begin{corollary}\label[corollary]{lim} The family $\{U_L, U_R\}$ jointly creates limits along every diagram $F\colon \D\to \comma{L}{R}$ such that $R$ preserves the limit of $U_R\circ I$.
\end{corollary}

We can use \Cref{lim} to characterize monos in comma categories. 
\begin{corollary}\label[corollary]{cor:mono}
	If $R$ preserves pullbacks then an arrow $(h,k)$ in $\comma{L}{R}$ is mono if and only if both $h$ and $k$ are monos.
\end{corollary}
\begin{proof}
	$(\Rightarrow)$  If $(h,k)\colon (A,B,f)\to (A', B', g)$ is a mono then the following square is a pullback in $\comma{L}{R}$
	\[\xymatrix@C=50pt{(A, B, f)  \ar[r]^{\id{(A,B,f)}} \ar[d]_{\id{(A,B,f)}}& (A, B, f)  \ar[d]^{(h,k)} \\ (A, B, f)  \ar[r]_{(h,k)}& (A', B', g) }\]
	Using \Cref{lim} we deduce that the following two squares are pullbacks in $\A$ and $\B$.
	\[\xymatrix{A \ar[r]^{\id{A}} \ar[d]_{\id{A}} & A \ar[d]^{h} & B \ar[r]^{\id{B}} \ar[d]_{\id{B}} & B  \ar[d]^{k}\\ A \ar[r]_{h} & A' & B \ar[r]_{k} & B'}\]
	From which it follows that $h$ and $k$ are monos.
	
	\smallskip\noindent 
	$(\Leftarrow)$ Since $h$ and $k$ are monos then we have two pullback squares 	
	\[\xymatrix{A \ar[r]^{\id{A}} \ar[d]_{\id{A}} & A \ar[d]^{h} & B \ar[r]^{\id{B}} \ar[d]_{\id{B}} & B  \ar[d]^{k}\\ A \ar[r]_{h} & A' & B \ar[r]_{k} & B'}\]
	By \Cref{lim} this implies that 
	\[\xymatrix@C=50pt{(A, B, f)  \ar[r]^{\id{(A,B,f)}} \ar[d]_{\id{(A,B,f)}}& (A, B, f)  \ar[d]^{(h,k)} \\ (A, B, f)  \ar[r]_{(h,k)}& (A', B', g) }\]
	is a pullback in $\comma{L}{R}$ and we are done.   
\end{proof}

We end this section pointing out another useful fact,  showing that in some cases we can guarantee  the existence of a left adjoint to $U_R$. 

\begin{proposition}\label[proposition]{prop:left}
	If $\A$ has initial objects and $L$ preserves them then the forgetful functor $U_R\colon \comma{L}{R}\to \B$ has a left adjoint $\Delta$.
\end{proposition}
\begin{proof} For an object $B\in \B$ we can define $\Delta(B)$ as $(0, B, ?_{B})$, where $0$ is an initial object in $\A$ and $?_{R(B)}$ is the unique arrow $L(0)\to R(B)$. Consider $\id{B}\colon B\to U_R(\Delta(B))$ be the identity, and suppose that a $k\colon B\to U_R(A, B', f)$ in $\B$ is given. By initiality of $0$, there is only one arrow $?_A\colon 0\to A $ in $\A$ and, since $L$ preserves initial objects, the following square commutes.
	\[\xymatrix@C=30pt{L(0) \ar[r]^{L(?_A)} \ar[d]_{?_{R(B)}} & L(A) \ar[d]^{f}\\ R(B) \ar[r]_{R(k)}& R(B')}\]
	Thus $(h,k)$ is the unique morphism $\Delta(B)\to (A, B', f)$	such that $U_R(h,k)=k$.   
\end{proof}
Dualizing we get immediately the following.
\begin{corollary}If $\B$ has terminal objects preserved by $R$ then $U_L\colon \comma{L}{R}\to \A$ has a right adjoint.
\end{corollary}

\subsection{Slice categories}

This section is devoted to recall some basic facts about the so called \emph{slice categories}.
\begin{definition}\index{category!slice -}
	Let $X$ be an object of a category $\X$, we will define the following two categories.
	
	\begin{itemize}
		\item The \emph{ slice category over $X$} is the category $\X/X$ which has as objects arrows $f\colon Y\to X$ and  in which an arrow $h\colon f\to g$ is $h\colon Y\to Y'$ in $\X$ such that the following triangle commutes.
		\[\xymatrix@C=15pt{Y \ar[dr]_{f} \ar[rr]^{h}&& Y' \ar[dl]^{g}\\ & X}\] 
		\item  	 Dually, the \emph{ slice category under $X$} is the category $X/\X$ in which objects are arrows $f\colon X\to Y$  with domain $X$ and a morphism $h\colon  f\to g$ is an arrow of $\X$ fitting in a triangle as the one below.
		\[\xymatrix@C=15pt{&X \ar[dr]^{g}\ar[dl]_{f}\\ Y  \ar[rr]_{h}&& Y'}\] 
	\end{itemize}
\end{definition}

\begin{remark} For every $X\in\ X$ we have forgetful functors
	\[\begin{split}
		\dom_X&\colon \X/X\to \X\\
		\functor[l]{f}{h}{g}
		&\functormapsto
		\rfunctor{\dom(f) }{h}{\dom(g)}
	\end{split}\qquad \begin{split}
		\cod_X&\colon X/\X\to \X\\
		\functor[l]{f}{h}{g}
		&\functormapsto
		\rfunctor{\cod(f) }{h}{\cod(g)}
	\end{split}\]
\end{remark}


We can realize the slice over and under an object $X\in \X$ as comma categories.

\begin{proposition}\label[proposition]{prop:commaapp}
	For every object $X$ in a category $\X$, if $\delta_X\colon \T\to \X$ is  the constant functor of value  $X$ from the category with only one object $*$, then $\X/X$ and $X/\X$ are isomorphic to, respectively,  $\comma{\id{X}}{\delta_X}$ and $\comma{\delta_X}{\id{X}}$ .
\end{proposition}
\begin{proof} Define functors $F_1\colon \comma{\id{X}}{\delta_X}\to \X/X$ and $G_1\colon \X/X\to \comma{\id{X}}{\delta_X}$ as follows
	\[	\begin{split}
		\functor[l]{(Y, *,  f)}{(h, \id{*} )}{( Y', *,  g)}
		\functormapsto
		\rfunctor{f }{h}{g}
	\end{split} \qquad \begin{split}
		\functor[l]{f }{h}{g}
		\functormapsto
		\rfunctor{(\dom(f), *, f)}{(h, \id{*} )}{(\dom(g), *, g)}
	\end{split}\]
	Similarly, we have $F_2\colon \comma{\delta_X}{\id{X}}\to X/\X$ and $G_2\colon X/\X\to \comma{\delta_X}{\id{X}}$
	\[	\begin{split}
		\functor[l]{(*, Y, f)}{(\id{*},h )}{(*, Y', g)}
		\functormapsto
		\rfunctor{f }{h}{g}
	\end{split} \qquad \begin{split}
		\functor[l]{f }{h}{g}
		\functormapsto
		\rfunctor{(*, \cod(f), f)}{(\id{*}, h )}{(*, \cod(g), g)}
	\end{split}\]
	It is now obvious to see that $F_1,G_1$ and $F_2, G_2$ are pairs of inverses.   
\end{proof}

A straightforward application of \Cref{lim} now yields the following.
\begin{corollary}\label[corollary]{cor:pbapp}
	If $\X$ has pullbacks, then for every object $X$, the slice $\X/X$ has pullbacks too.
\end{corollary}

In a category $\cat X$ with pullbacks, each $f: X \to Y$ induces a functor $f^*: \cat X/Y \to \cat X/X$,
which sends each morphism $a: A \to Y$ of $\cat X$ onto its pullback along $f$, $p_a$, and each morphism $h: a \to b$ onto the unique arrow from the pullback of $a$ along $f$ to the pullback of $b$ along $f$.

Then, we have the following result.

\begin{proposition}
	Let $\cat X$ be a category with pullbacks, $R: \cat{Y \to X}$ be a functor and $L: \cat{X \to Y}$ be its left adjoint, and $\eta$ the unit of the adjunction.
	Then, each object $X$ of $\cat X$ induces an adjoint pair of functors $L_X: \cat X/X \to \cat Y/L(X)$, $R_X: \cat Y/L(X) \to \cat X/X$, 
	where $L_X$ is the obvious functor, and $R_X$ is the composite $(\eta_X)^* \circ R$.
\end{proposition}

\begin{proof}
	Let $f: L(l) \to b$ be a morphism of $\cat Y/L(X)$, where $l: A \to X$ in $\cat X$ and $b: B \to L(X)$ in $\cat Y$, as shown below.
	\[\xymatrix{L(A)\ar[rr]^{f}\ar[dr]_{L(l)}&&B\ar[dl]_{b}\\&L(X)}\]
	Then, we have the following situation in $\cat X$.
	\[\xymatrix{
			A \ar[rr]^{\eta_A}\ar[dr]_{l} \ar@/^2.0pc/@[black][rrrr]^{g} && R(L(A)) \ar[rr]^{R(f)} \ar[dr]_{R(L(l))} && R(B) \ar[dl]^{R(b)}\\
						      & X \ar[rr]_{\eta_X} && R(L(X))
	}\]
	Where $g$ is the adjunct of $f$.
	Consider now the pullback $P$ of $R(p)$ along $\eta_X$, as shown below.
	\[\xymatrix{
		A \ar@/^1.0pc/[drr]^{g}\ar@/_2.0pc/[ddr]_{l} \ar@{.>}[dr]^{v}\\&P\ar[r]^{q}\ar[d]_{\eta_X^*(R(b))}&R(B)\ar[d]^{R(b)}\\&X \ar[r]_{\eta_X}&R(L(X))
	}\]
	By universal property of pullbacks, the two diagrams express the same morphism $g$.
	Hence, we can rewrite the first diagram as follows.
	\[\xymatrix{
			L(A)\ar[r]^{L(g)}\ar[d]_{L(l)}\ar@/^2.0pc/[rr]^{f}&(L(R(B)))\ar[d]_{L(R(b))}\ar[r]^{\epsilon_B}&B\ar[d]^{b}\\
			L(X)\ar[r]_{L(\eta_X)}\ar@/_2.0pc/[rr]_{\id{L(X)}}&L(R(L(X)))\ar[r]_{\epsilon_{L(X)}}& L(X)
	}\]
	where $\epsilon$ is the counit of the adjunction.
	This descriveb the action of functors and thus the adjunction, obtained considering the isomorphism on hom-sets of the categories.

\end{proof}


%\section{Comparison with Ghica}
%\todo{Lo vogliamo fare?}


\end{document}

